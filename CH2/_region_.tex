\message{ !name(chapter2name.tex)}
\message{ !name(chapter2name.tex) !offset(-2) }
\chapter{Theories of Gamma-Ray Bursts}
\label{ch:theoryGRB}

\section{The Fireball Model}

I still need to be written.

\subsection{}


\section{Particle Acceleration Process}

\subsection{Fermi Process and Internal Shocks}

\subsection{Magnetic Reconnection}


\section{Emission Mechanisms}

\subsection{Synchrotron}

\subsection{Inverse Compton}





% \section{A Word about Figures}

% In \LaTeX, commands with mandatory arguments are followed by curly
% brackets \verb|{like these}|.  Sometimes, a command will have
% optional arguments too---they are indicated by square brackets,
% \verb|[like these]|.

% In the book class, the commands that place ``things'' in the table
% of contents (TOC), list of figures (LOF), and list of tables (LOF)
% have both optional and mandatory arguments.  For example, with a
% figure, the optional argument is the ``title'' of the figure that
% would be placed in the LOF, and the mandatory title is what would be
% placed below the figure (\ie, in the caption).  This is done,
% sometimes, if a figure has a very long caption.  However, UAH
% requires that the LOF (and TOC and LOT) have the same
% captions/titles as appear in the text---so, normally, the optional
% arguments are not used.  But, I'm bringing this up because, in some
% rare cases (\eg, if you want to force a ``line break'' in the TOC),
% the optional arguments can come in handy, even if the text is the
% same.  Make sense?  Good.

% But please note---in the absence of the optional argument, \LaTeX
% will put whatever is in the mandatory argument into the TOC, LOF, or
% LOT (depending if it's a chapter title, figure title, or table
% title).  So, in most cases, you'll only need the mandatory argument,
% like \verb|\caption{Caption for the figure.}|, which will be printed
% as both the caption and in the LOF.

% Almost forgot to mention---to cross-reference a figure, use the
% following command \verb|\figureref{fig:thisFigLabel}|.  The same
% goes for tables, chapters, equations, sections, and so on.  (Except,
% that obviously, the command are \verb|\tableref{}|,
% \verb|\chapterref{}|, \verb|\equationref{}|, and
% \verb|\sectionref{}|, respectively.)

% \begin{figure}
%     \cfig{2}{blank.eps}{3.5}
%     \caption{Caption for the figure.}
%     \label{fig:thisFigLabel}
% \end{figure}

\message{ !name(chapter2name.tex) !offset(-71) }
