\chapter{Theories of Gamma-Ray Bursts}
\label{ch:theoryGRB}
\begin{chapterquote}{Good Luck}
The stars were exploding,\\
One by one as they flicker and they fall
\end{chapterquote}

\section{The Fireball Model}
\label{sec:fbm}
Gamma-ray Bursts are believed to originate from the release of a
massive amount of energy, $E_0$ in a small region of space, $ r_0$
leading to a relativistically expanding fireball
\cite{Cavallo:1978,Goodman:1986,Paczynski:1986}. The current model, as
detailed below, includes the formation of the GRB jet and its evolution, as
well as the charged particle acceleration processes that occurs and the
radiation generated by those particles.

\subsection{Jet Dynamics}
\begin{figure}[h]
  \centering
  \cfig{2}{gammaEvo.pdf}{4}
  \caption{The expansion phases and radii of an expanding GRB jet.}
\end{figure}


\subsubsection{Initial Explosion Energetics}
Assume that an explosion of a stellar mass object deposits an energy
$E_0 \approx 10^{50}-10^{52}$ ergs into a volume $\frac{4}{3}\pi
r_0^2$ \cite{Kobayashi:1999}. Taking the progenitor of the GRB to be
the collapse of either a super-massive star or the coalescence of two
compact objects, $r_0$ can be assumed to be the within an order of
magnitude of the Schwarzschild radius of a black hole. This explosion
occurs over a timescale of a few seconds. The Eddington luminosity,
\begin{equation}
  \label{eq:eddington}
  L_E\;\equiv\;4\pi G M m_p c/\sigma_T\;=\; 1.25\;\times\;10^{38}(M/M_{sun})\;erg\;s^{-1} %%% Fix this
\end{equation}
is the maximum luminosity at which radiation pressure and self-gravity
form a hydrostatic equilibrium \cite{rybicki:1979}. For the energies and
timescales above, the Eddington luminosity is greatly exceeded and radiation
pressure forces an expansion of the fireball.

The velocity of the expansion is governed by the specific entropy of
the fireball:
\begin{equation}
  \eta\;\equiv\;\frac{E_0}{M_0 c^2},
\end{equation}
where $M_0$ is the initial baryonic mass of the jet. This specific
entropy of the jet will be the maximum Lorentz expansion can obtain
(see \sectionref{sec:eoj}). If the fireball contains a high fraction
of baryonic mass, i.e., $\eta\approx\;1$, then the expansion will be
subrelativistic corresponding to a Sedov-Taylor expansion
\cite{Sedov:1946,Taylor:1950}. Conversely, a low baryon load will
cause a relativistic expansion. Observational evidence for
relativistic expansion exists due to the presence of high-energy
{\gray}s. For high photon densities, the photon-photon collisional
cross-section is large for high-energy photons. It is however, a
function of photon propagation angle. If the fireball expands
relativistically, the photons will be beamed which reduces the
photon-photon annihilation cross-section due to the small collisional
angle of the nearly parallel photons.. Therefore, the measurement of
high-energy {\gray}s provides strong evidence for relativistic
expansion and the baryon load is assumed to be small, i.e., $\eta\gg
1$.

An explosion which imparts an energy $E_0$ into a mass
$M_0\ll\frac{E_0}{c^2}$ within a radius $r_0$ results and a
relativistically expanding fireball is formed
\cite{Blandford:1976}. This kinetic energy input energizes the
particles in the outflow to have initial random Lorentz factors
$\gamma_0=\eta$. There are three relevant phases of the jet that will
be focused on; corresponding to specific radii of the outflow. These
include the acceleration phase, which occurs before the saturation
radius, $r_s$, the thermalization phase, corresponding with the
photospheric radius, $r_{ph}$, and the optically-thin coasting phase,
where non-thermal radiation is likely to occur at $r_{nt}$.

\subsubsection{Acceleration of the Jet}
\label{sec:eoj}
Due to the relativistic velocities and initial high densities
involved, the fireball initially expands adiabatically. Adiabatic
expansion occurs when a volume of fluid expands without heat exchange
with its environment, typically when the expansion happens rapidly
and/or when the fluid is thermally insulated from the surrounding
environment. The deposit of energy into the fluid is rapid and the
early stages of the jet are extremely dense, sufficiently insulating
the plasma. From thermodynamics, we know that a relativistically
expanding gas dominated by radiation has an adiabatic index of
$\gamma_{\rm a}=4/3$. It is convenient to work in the rest or comoving
frame of the jet where the volume expands isotropically and then
Lorentz boost into the observer frame at the end. I will use a prime
to indicate quantities in the comoving frame ($x^{\prime}$) from this
point forward. The comoving volume, $V^{\prime}$, and comoving
temperature, $T^{\prime}$ are related via $T^{\prime} \propto
V^{\prime 1-\gamma_{\rm a}}$, from simple thermodynamics. In the
comoving frame, $V^{\prime} \propto r^3$ \cite{Meszaros:1993} which
leads to $T^{\prime} \propto r^{-1}$. The particles in the outflow
have random Lorentz factors $\gamma^{\prime} \propto T^{\prime}$
giving their radial evolution as $\gamma^{\prime} \propto r^{-1}$ as
the jet expands. Therefore, we see that after the initial input of
energy, $E_0$, into the particles, the particles will cool during the
expansion and that energy accelerates the jet.

The bulk Lorentz factor, $\Gamma$, of the outflow must increase to
balance the cooling of the particles. To conserve energy, the kinetic
energy per particle and the bulk Lorentz factor must balance each
other, i.e.,  $\gamma\Gamma = {\rm constant}$ which instantly yields the
evolution of the fireball's bulk expansion velocity, $\Gamma \propto
r$. The acceleration of the fireball continues until $\Gamma = \eta =
E_0/{M_0 c^2}$. This is because there is no more energy is available
for expansion. The
outflow is said to saturate at this point and the radius is referred
to as the saturation radius, $r_{\rm s}$, after which $\Gamma$ coasts
at a constant value \cite{Paczynski:1986,Goodman:1986,Shemi:1990}.

\subsubsection{The Photosphere}
The energies and volumes involved in the fireball scenario imply a
high particle and photon density initially. These high densities imply
that the outflow is optically thick for some part of its evolution
\cite{Paczynski:1986,Goodman:1986,Shemi:1990}. Initially, the plasma
contains a large number of $e^{\pm}$ pairs that are in thermodynamic
equilibrium with the photons. These drop out of equilibrium very deep
within the jet. These pairs could lead to a photosphere but only if
$\Gamma\gg1000$. When these pairs "freeze-out" of the plasma, there
number is much less than pairs coupled to baryons in the flow
\cite{Goodman:1986}.  The baryons in the fireball carry $e^{\pm}$
pairs with them which serve as scattering centers for the
photons. Additionally, pairs can be produced by photon collisions
which can contribute to the overall opacity. If we consider the
outflow to be a wind of varying density as a function of radius then
the opacity properties can be assessed using relativistic fluid
equations \cite{Paczynski:1990}. The flow has a dimensionless entropy
$\eta=L/\dot{M}c^2$ and baryon density
\begin{equation}
  \label{eq:pdensity}
  n^{\prime}_p\;=\;\dot{M}/4 \pi r^2 m_p c \Gamma 
\end{equation}
which can be rewritten
\begin{equation}
  \label{eq:pdensity2}
   n^{\prime}_p\;=\;L/ 4 \pi r^2 m_p c^3 \eta \Gamma.
\end{equation}
The majority of the electrons in the flow at this point are those
which are are coupled to the baryons, i.e.,
$n^{\prime}_p=n^{\prime}_e$. To calculate the optical depth, the
electron density multiplied by the Thomson electron-photon scattering
cross-section, $\sigma_{\rm T}$, must be integrated along the line of
sight. The optical depth as a function of radius is
\cite{Paczynski:1990}:
\begin{equation}
  \tau(r)\;=\;\int_r^{\infty}n^{\prime}_e \sigma_{\rm T}\left(\frac{1-\beta_v}{1+\beta_v}\right)^{1/2}dr^{\prime}\;\simeq\;n^{\prime}_e \sigma_{\rm T}(r/2 \Gamma).
\end{equation}
Here, $\beta_v=v/c$ is the dimensionless velocity of the outflow. The
fireball becomes optically thin with $\tau\approx1$ (but see
\cite{peer:2008} for cases where the photosphere is not a constant
surface) at the photospheric radius, $r_{\rm ph}$. Solving for $r_{\rm
  ph}$ by setting $\tau=1$ yields
\begin{equation}
  \label{eq:rph}
 r_{\rm ph}\;=\;\frac{ L \sigma_{\rm T}}{8 \pi m_p c^3 \eta \Gamma^2}.
\end{equation}

The photosphere can occur above or below the saturation radius
depending on the value of $\eta$. High values of $\eta$ force the
photosphere below $r_s$. This work will focus on values of $\eta$ such
that $r_s<r_{ph}$. The temperature evolution in this region is simply
\cite{Meszaros:1993}
\begin{equation}
  \label{eq:ktevo}
  T^{\prime}\propto \left(\dover{r}{r_s}\right)^{-2/3}.
\end{equation}
When the jet becomes optically thin at $r_{ph}$, the photon emission
should be in the form of a blackbody.


\subsection{Non-Thermal Emission Region}
\label{sec:nter}
The vast majority of GRB spectra are observed to be non-thermal
\cite{Goldstein:2012}. Details of these observations will be discussed
in \sectionref{sec:spectrum}. The source and process of this
non-thermal emission is yet to be fully understood. To understand this
problem, not only do the radiative processes that produce the
$\gamma$-ray emission have to be identified, but also the mechanisms
that accelerated the emitting particles to non-thermal energies. For
this work, it will be assumed that the non-thermal emission occurs in
an optically thin regime, i.e. when $r>r_{ph}$. There are two
macrophysical proceess that serve to extract kinetic energy of the
outflow and convert it into radiation at these radii: internal shocks,
and external shocks. Magnetic reconnection is another process that can
extract energy from the flow, but it extracts magnetic energy and will
be discussed in \sectionref{sec:MR}.


\subsubsection{Internal  Shocks}
\label{sec:nter:is}
The observed non-thermal shape of the broadband emission of GRBs
implies a non-thermal distribution of the emitting electrons. The
commonly-invoked method for generating non-thermal electron
distributions is the Fermi process by which electrons are accelerated
to high non-thermal energies. The details of the particle acceleration
are discussed below in \sectionref{sec:pa::fp}. However, it is important to
discuss the properties of the jet that can lead to particle
acceleration. If it is assumed that random variations in $\Gamma$ form
in the early stages of the jet then these variations can lead to
internal shocks \cite{Rees:1994,Daigne:1998}. These variations form
from random changes in the mass loss rate, $\dot{M}$, and are of order
unity, i.e., $\Delta \dot{M}/\dot{M}\sim 1$ . The variations will be spatially separated by $c t_v$, where
$t_v$ is the variability timescale of the central engine. These stratified portions of the
wind or \emph{shells} will catch up with one another at a radius
$r_{nt} \sim c t_v \Gamma^2$. This can occur above or below $r_{ph}$
depending on the internal properties of the jet. For $r_{nt}<r_{ph}$
energy will be dissipated into the thermalizing electrons. This
scenario will be further detailed in \sectionref{sec:subpht}.

To understand the properties of internal shocks occurring above the
photosphere, consider two shells denoted by their masses and bulk
Lorentz factors $m_{\rm s}$, $\Gamma_{\rm s}$ and $m_{\rm f}$,
$\Gamma_{\rm f}$ where $\Gamma_{\rm f} > \Gamma_{\rm s}$. When the
faster shell catches up with the slower shell and inelastic collision
occurs creating a merged shell with a resulting Lorentz factor

\begin{equation}
\label{eq:intshock}
\Gamma_{\rm m} \sim \sqrt{\frac{m_{\rm f} \Gamma_{\rm f} + m_{\rm s} \Gamma_{\rm s}}{{m_{\rm f} /\Gamma_{\rm f}}+m_{\rm s}/ \Gamma_{\rm s}}}
\end{equation}
by way of conservation of kinetic energy and momentum
\cite{Daigne:1998}. The internal energy available for the electrons to
radiate or be accelerated to high-energies is given by the difference
of the kinetic energy before and after the collision:
\begin{equation}
  \label{eq:intEne}
  E_{\rm int}=m_{\rm f} c^2 (\Gamma_{\rm f}-\Gamma_{\rm m})+m_{\rm s}c^2(\Gamma_{\rm s}-\Gamma_{\rm m}).
\end{equation}
%%%% MORE HERE!!

While internal shocks present a viable option for extracting energy
from the outflow for radiation, several drawbacks exist that have
yet to be resolved. Foremost, internal shocks are extremely
inefficient at converting bulk energy into radiation. The efficiency
of the process is given by
\begin{equation}
  \label{eq:eff}
  {\rm eff} = 1 - \dover{(m_{\rm f} + m_{\rm s})}{\sqrt{m_{\rm f}^2+m_{\rm s}^2 +m_{\rm f}m_{\rm s}\left( \dover{\Gamma_{\rm f}}{\Gamma_{\rm s}} +  \dover{\Gamma_{\rm s}}{\Gamma_{\rm f}}   \right)}}
\end{equation}
which is only on the order of 5-20\% for a typical collision speed
\cite{Daigne:1998,Guetta:2001}. This range of efficiencies can be
modified by changing the value of $r_{nt}$ or more directly, the
initial variation in $\Gamma$ \cite{Spada:2000,Beloborodov:2003}.  The
extremely high luminosities observed in GRBs require an immense amount
of radiation to be generated and the low efficiency of the internal
shock process places severe limits on the radiation processes that
must occur.

\subsection{External Shocks}


An external shock occurs when the outflow material in the jet collides
with the external interstellar medium (ISM) or the layers of material
previously blown off by the progenitor star with density $\rho_{\rm
  ext}$. The external matter is swept up by the jet producing a blast
wave \cite{Rees:1992}. A shock is formed at the velocity discontinuity
and propagates forward with a Lorentz factor $2^{1/2} \Gamma$
\cite{Blandford:1976}.  The shock becomes important when the amount of
energy in swept up material is roughly equal to the energy of the
outflow:
\begin{equation}
  \label{eq:extShockE}
  E_0 \sim \dover{4 \pi}{3} \rho_{ext} \Gamma^2 r_{dec}^3,
\end{equation}
at a deceleration radius $r_{dec}$. The bulk velocity of the jet will
decrease to half its original value at this point and have swept up a
mass approximately $M_{ext}\sim M_0/\Gamma$ \cite{Rees:1992}. The
deceleration causes a reverse shock to form and propagate back into
the jet which dissipates energy to the particles in the outflow. This energy
is free to be radiated in the form of a GRB via synchrotron or
inverse-Compton processes.

External shocks have been shown to be not viable for generating the
prompt non-thermal radiation in GRBs due to several factors. Namely,
external shocks cannot recreate the short-timescale variability
observed in GRB lightcurves \cite{Sari:1997}. They are briefly
reviewed here for completeness.

% It has been suggested that a part of the prompt emission could be due
% to emission from an external shock \cite{}. The emission is in the
% form of the so-called early afterglow. The afterglow is x-ray emission
% that occurs after the prompt phase and can last for several days \cite{}. The
% source of afterglow emission is almost certainly an external shock \cite{}.



\section{Particle Acceleration Processes}
As noted in \sectionref{sec:nter}, the observed non-thermal shape of
GRB spectra require the acceleration of the radiating particles to
non-thermal energies. The two main physical processes that can
accelerate these particle in GRB environs are the Fermi process and
magnetic reconnection.  While the Fermi process has been studied and
simulated extensively, the framework of magnetic reconnection in the
context of GRBs is still poorly understood and its application is
typically only to alleviate problems arising from the Fermi
acceleration. Here the two processes are reviewed.
  


\subsection{The Fermi Process}
\label{sec:pa::fp}
First posed by Enrico Fermi \cite{Fermi:1949}, the Fermi process's use
in high-energy astrophysics is ubiquitous. The mechanism is the
central theoretical underpinning of cosmic-ray generation and was
fully developed in the 1970's
\cite{Axford:1981,Krymskii:1977,Blandford:1978,Bell:1978b,Bell:1978a}. The
theory was extended to the ultra-relativistic regime in the 1980's
\cite{Kirk:1989,Kirk:1987} and has been theorized as a viable
mechanism to accelerate charges in GRB outflows. Two forms of the
process exist, first- and second-order acceleration, referring to the
order of the particle velocity ($u$ and $u^2$) to which the energy
gain is proportional.

\subsubsection{First-Order Acceleration}
First order Fermi acceleration is a process by which fast charges in a
shock wave cross the shock boundary. A velocity discontinuity exists
between a fast and slow region of material, referred to as the upstream
and downstream regions respectively (see \figureref{fig:fermiaccel}).
\begin{figure}[h]
  \centering
  \cfig{2}{fermiaccel.pdf}{4}
  \caption{An illustration of the
    first-order Fermi process. Charges are reflected between the up
    and downstream regions, gaining energy proportional to $u$. }
\label{fig:fermiaccel}
\end{figure}
The relative velocity of the two regions is $u=\beta_v c$.  Assume
that the charges in the up and downstream fluid are initially in
thermal distribution. Suprathermal charges (charges with a high
velocity compared with the bulk of the distribution i.e., those in the
exponential tail of the distribution) can escape ahead of the shock
front, where they see converging scattering centers in the form of
other charges with a velocity relative to the escaped charge. These
escaped charges will be reflected back across the shock boundary by
the scattering centers where they will once again see converging
scattering centers in the new region. This cycle can occur many times,
each resulting in a systematic energy gain of $u/c$. Therefore,
first-order Fermi acceleration can result in particles gaining a
substantial amount of energy compared to their original thermal
distribution \cite{longair}. Relativistic, first-order acceleration
works generally in the same manner and results in charges being
energized into a high-energy power-law. It has been shown that the
index of the electron energy power-law is $\delta \approx 2.2 -2.3$
\cite{Ostrowski:2002,Waxman:1997,Bednarz:1998,Achterberg:2001}.

A key problem in first-order acceleration is how to have enough
suprathermal particles serving as a pool for acceleration to
sufficiently populate the observed non-thermal distribution. This is
known as the injection problem. In addition, the question of whether
shocks can generate in situ the conditions necessary for acceleration
to occur. Monte carlo simulations and analytic simplifications of both
relativistic and non-relativistic shock acceleration have relied on
placing shock conditions and magnetic fields in by hand to validate
the process
\cite{Baring:2012,Baring:2011,Baring:1995,Ellison:1990,Ellison:2004}.
The success of these efforts begs the question of whether the shock
conditions can be self generated.  Recent numerical simulation work
has attempted to address these problems and confirms that first-order
Fermi acceleration is possible in realistic, relativistic shocks
\cite{Spitkovsky:2008}.
\begin{figure}[h]
  \centering
  \cfig{2}{slowE.pdf}{4}
  \caption{Post-shock electron distribution consisting of a
    relativistic Maxwellian and a high-energy power-law tail.}
\end{figure}
The general form of the post-shock particle distribution is that of
relativistic Maxwellian with a high-energy power-law tail
\cite{Baring:2012,Baring:2011,Baring:1995,Ellison:1990,Ellison:2004,Spitkovsky:2008}:
\begin{equation}
  n_e(\gamma )\; =\; n_{0} \biggl\lbrack\;
  \Bigl( \dover{\gamma}{\gamth} \Bigr)^2\,
  e^{-\gamma/\gamth } + \epsilon \,
  \Bigl( \dover{\gamma}{\gamth} \Bigr)^{-\delta}\,
  \Theta \Bigl( \dover{\gamma}{\gammaMin} \Bigr)\, \biggr\rbrack\, .
  \label{eq:elec_dist}
\end{equation}
This distribution is crucial to this work because it serves as the
foundation for the physical modeling of GRB spectra that will be
applied to data.
\subsubsection{Second-Order Acceleration}
Second-order Fermi acceleration is a process by which charges are
reflected off magnetic turbulence, which can be conceptualized as
magnetic clouds moving in random directions with a mean velocity
$u=\lvert\vec{u}\rvert=\beta_v c$. The overall effect of these random
scatterings is a stochastic gain in energy proportional to $u^2$. The
energy gain is much smaller than the first-order process, but is still
relevant to GRB particle acceleration as an attempt to solve the
so-called fast-cooling problem (see \sectionref{sec:spec:lod,sec:ascp}). The process is
also much slower than first-order acceleration.

\subsection{Magnetic Reconnection}
\label{sec:MR}
While extensive analytic and numerical studies of the Fermi process
have had some success in attempting to explain the non-thermal spectra
of GRBs, particle acceleration via magnetic reconnection remains a
plausible theoretical avenue as well
\cite{zhang:2011,Meszaros:1997b,Meszaros:1994,Thompson:1994,Usov:1994}.
The progenitor of a GRB can have a strong initial magnetic field,
e.g., as in the collision of two highly-magnetized neutron stars. If
it is assumed that the GRB jet ejects a highly-magnetized or
Poynting-flux-dominated (PDF) outflow, then the observed $\gamma$-ray
emission could result from the release of this magnetic energy when it
is transferred to charged particles in the jet and then radiated away.

The main process for releasing stored magnetic energy is via magnetic
reconnection. The details of magnetic reconnection are not well
understood; however, the basic mechanism occurs when two magnetic
regions of opposite orientation approach each other and then reconnect
via the so-called Sweet-Parker process
\cite{Parker:1957,Sweet:1958}. This classical scenario occurs over a
timescale that is far too slow to explain the rapid variability seen
in GRB events. Recently, a new mechanism was proposed, and verified by
numerical simulations, in which reconnection can occur over a rapid
timescale in the presence of magnetic turbulence
\cite{Kowal:2009,Lazarian:1999}. The release of this magnetic energy
serves to accelerate particles non-thermally as well as to
introduce turbulence into the surrounding magnetic field.  This in
turn can serve to accelerate particles via the second-order Fermi
process.


\section{Emission Mechanisms}
\label{sec:emissionMech}
While it is critical to understand the processes that govern the
dynamics of the jet and energize the electrons, ultimately the
radiative mechanisms that produce the observed radiation are key to
understanding the observed $\gamma$-ray flux. A review of the relevant
non-thermal radiative processes follows. These include synchrotron and
inverse-Compton radiation.

\subsection{Synchrotron}
\label{sec:synctheory}

Electrons in the presence of a magnetic field ($\vec{B}$) are accelerated by the Lorentz force:
\begin{equation}
\label{eq:FL}
  \vec{F_L}=\dover{d}{dt}(\gamma_e m \vec{v})=q\left(\vec{E} +\dover{1}{c} \vec{v}\times \vec{B} \right)
\end{equation}
where $\gamma_e$ is the Lorentz factor of the electron
\cite{rybicki:1979,jackson:1998}. For the plasmas considered in this
work, the electric field ($\vec{E}$) is typically shorted out and
therefore $\vec{E}=0$. The force of the magnetic field on the particle
causes a gyration.  Assume that the electrons lose little energy per
gyration implying $\gamma_e\neq \gamma_e(t)$. With these assumptions
\equationref{eq:FL} can be rewritten
 \begin{equation}
   \label{eq:FL2}
   \dover{d \vec{v}}{dt}=\omega_B\left( \vec{v} \times \dover{\vec{B}}{B} \right)
 \end{equation}
where
\begin{equation}
  \label{eq:cycfreq}
  \omega_B = \frac{qB}{\gamma_e m_e c}.
\end{equation}
Accelerated charges emit radiation with power, P, calculated via the Lamour formula:
\begin{equation}
  \label{eq:Psync}
  P=\dover{2 q^2}{3 c^3}\gamma_e^2\omega_B^2v^2=\dover{4}{3}\sigma_{\rm T}c\beta_v^2\gamma_e^2\dover{B^2}{8 \pi}.
\end{equation}
This radiation
is called synchrotron radiation. 

In GRB studies, synchrotron radiation is theorized to be emitted by
the distribution of electrons, $n_e$, that have been accelerated in
the jet by one of the processes described above. To calculate the
synchrotron energy flux, $F_{\nu}$ (erg s$^{-1}$ cm$^{-2}$), emitted
by a distribution of electrons ($n_e(\gamma)$), the single particle
emissivity of synchrotron radiation \cite{rybicki:1979},

\begin{equation}
  \mathcal{F}\left(w\right) \; =\; w \int_w^{\infty } K_{5/3}(x) \, dx,
  \label{eq:synch_func}
\end{equation}
expressed here in dimensionless form must be convolved with $n_e(\gamma)$ over all energies, i.e., 
\begin{equation}
  F_{\nu}(\mathcal{E})\; \propto\; \int_1^{\infty} n_e(\gamma ) \, 
  \mathcal{F} \left( \dover{\mathcal{E}}{E_c}\right) \, d\gamma\quad .
  \label{eq:synch_flux}
\end{equation}
The quantity $E_c$ is called the characteristic energy of emission and
is defined
\begin{equation}
  \label{eq:wc}
  E_c(\gamma) = \frac{3 \gamma^2q \hbar B \sin\alpha}{2 m_e c}.
\end{equation}
With \equationref{eq:Psync,eq:synch_flux,eq:wc} the full synchrotron
spectrum can be derived for any distribution of electrons. Of
particular interest  due to its prevalence in nature and the fact
that Fermi acceleration generates a power-law
(see \sectionref{sec:pa::fp}), is the synchrotron spectrum from a power-law
distribution of electrons. Let
\begin{equation}
  \label{eq:plelec}
  n_e^{pl}=n_0(\delta -1)\gamma_{\rm min}^{\delta-1}\gamma^{\delta}, \gamma_{\rm min} \leq \gamma
\end{equation}
where $\delta$ is the electron spectral index of the power-law. The $\Fv$ synchrotron spectrum of from this distribution can be described asymptotically described by
\begin{equation}
  \label{eq:plsync}
  F_{\nu}(\mathcal{E})\propto\left\{
     \begin{array}{lr}
       \mathcal{E}^{1/3} & : \mathcal{E}\leq E_{\rm p}\\
       \mathcal{E}^{-\frac{1-\delta}{2}} & : \mathcal{E}>E_{\rm p}
     \end{array}.
   \right.
\end{equation}
with a peak energy calculated from \equationref{eq:wc},
\begin{equation}
  \label{eq:synchPeak}
  E_{\rm p} = \frac{3 \gamma_{\rm min}^2 q \hbar B \sin{\alpha}}{2 m_e c}.
\end{equation}
The 1/3 index below the peak of the spectrum holds for the majority of
physical electron distributions. However, electrons will lose a
significant amount of their energy via synchrotron radiation. This
cooling can alter the initial $n_e(\gamma)$ and therefore we now
consider the synchrotron spectrum from electrons that have been
significantly cooled.

The cooling of the electron distribution is inevitable when there is
no source of heating because synchrotron is a highly efficient
radiator. Adapting from the derivation of \cite{Burgess:2013}
(hereafter B13), assume that electrons distributed as in
\equationref{eq:plelec} are injected via and acceleration process into
a region where they are allowed to cool by the emission of synchrotron
radiation. The cooling of the electrons is governed by the continuity
equation \cite{Blumenthal:1970}
\begin{equation}
  \label{eq:komp}
  \frac{\partial n_e(\gamma,t)}{\partial t}+\frac{\partial}{\partial \gamma}[\dot{\gamma}n_e(\gamma,t)]+\frac{n_e(\gamma,t)}{t_{esc}}\;=\;Q_e(\gamma)
\end{equation}
where $n_e/t_{esc}$ represents the loss of particles from the emission
region from which we can define the maximal cooling scale
$\gamma/\dot{\gamma}\;\sim\;t_{esc}$ and $Q_e$ is the injection term. This corresponds to a Lorentz
factor, $\gamma_{\rm cool}$, below which cooling shuts off. For very high
electron Lorentz factors, $\gamma/\dot{\gamma}\;\ll\;t_{esc}$ and
therefore the loss term can be neglected, i.e., the dynamical
timescale is much longer than the radiative timescale. With this
assumption, we can simplify \equationref{eq:komp} to become
time independent. The resulting electron distribution can then easily
be solved
\begin{equation}
  \label{eq:simpKomp}
  n_e(\gamma,t)\;\approx\;\frac{1}{\dot{\gamma}}\int_{\gamma}^{\infty}Q_e(\gamma')d\gamma'.
\end{equation}
Substituting in the synchrotron cooling rate,

\begin{equation}
  \label{synchCool}
  \dot{\gamma}\;=\;-\frac{\pi}{3}\frac{r_0c}{r_g^2}\gamma^2,
\end{equation}
where $r_g\;=\;m_ec^2/(eB)$ and $r_e\;=\;e^2/(m_ec^3)$, \equationref{eq:simpKomp} yields the synchrotron-cooled broken
power-law distribution of electrons (see \figureref{fig:fastE})

\begin{equation}
  \label{eq:necool}
  n_e^{cool}\;\propto\;\frac{q_e\gammaMin}{\gamma^2}\;\min\left\{\left(\frac{\gamma}{\gammaMin}\right)^{-(\delta-1)},1\right\}\;,\;\gamma_{cool}\leq\gamma.
\end{equation}

Convolving this distribution with the single-particle synchrotron
emissivity yields an energy flux spectrum
\begin{equation}
  \label{eq:coolSynch}
   F_{\nu}^{\rm cool}(\mathcal{E})\propto\left\{
     \begin{array}{lr}
       \mathcal{E}^{1/3} & : \mathcal{E}\leq E_{\rm p}\\
       \mathcal{E}^{-1/2} & : E_{cool}<\mathcal{E}\leq E_{\rm p}\\
       \mathcal{E}^{-\delta/2} & : \mathcal{E}>E_{\rm p}{\rm p}
     \end{array}.
   \right.
\end{equation}
Here, $E_{\rm cool}$ is cooling energy of the energy flux spectrum corresponding to $\gamma_{\rm cool}$, i.e., 
\begin{equation}
  \label{eq:Ec}
  E_{\rm cool}= \frac{3 \gamma_{\rm cool}^2 q \hbar B \sin{\alpha}}{2 m_e c}.
\end{equation}

\begin{figure}[t]

  \centering
  \cfig{2}{fastE.pdf}{4}
  \caption{The resulting
    electron distribution resulting from the synchrotron cooling of a
    shock injected power-law electron distribution. The distance
    between $\gamma_{\rm cool}$ and $\gamma_{\rm min}$ is dependent on
    the amount of time the electrons have been allowed to cool.}
\label{fig:fastE}
\end{figure}

\begin{figure}[ht]
  \centering
  \cfig{2}{sync.pdf}{4}
  \caption{The energy flux spectrum of synchrotron  ({\emph red}) and fast-cooling synchrotron ({\emph blue}).}
\end{figure}

For this distribution of electrons (\equationref{eq:necool}), the
low-energy spectral index of the $\gamma$-ray emission is 1/2, which
is steeper than that of the un-cooled electrons. This difference in
emitted spectral indices will be crucial for interpreting the
observations of GRB spectra in order to determine models of emission.





\subsection{Inverse Compton}

The process of Compton scattering occurs when a photon inelastically
scatters with an relativistic electron and changes energy and
direction to conserve momentum \cite{rybicki:1979}. The inverse of
this process occurs when the electron has a high kinetic energy
compared to the scattering photon and some of this energy is transfer
ed to the photon. This process is called inverse Compton
scattering. This mechanism is of importance to GRBs because it is
possible that low-energy seed photons of energy $\mathcal{E}_s$ present in
the outflow from thermal emission are scattered to $\gamma$-ray
energies by electrons concurrently existing in the outflow. The seed
photons could even come from the synchrotron radiation generated by
the electrons themselves. This is called synchrotron self-Compton
emission.

Considering only inverse Compton in optically thin environs, the $\Fv$
spectrum can be calculated by convolving the single-particle
scattering kernel with the given electron distribution
\cite{rybicki:1979,Baring:2004}:
\begin{equation}
  \label{eq:ic}
  F_{\nu}^{\rm IC}(\mathcal{E}) = \dover{3 \mathcal{E}}{4}\sigma_{T}\int_0^{\infty}d\mathcal{E}_s n_{\rm ph}(\mathcal{E}_s)\int_{\gamma_{\rm min}}^{\infty}\gamma n_e(\gamma)\dover{f(z)}{4 \gamma^2 \mathcal{E}_s}
\end{equation}
where $n_{\rm ph}$ is the ambient photon field, and the function
\begin{equation}
  \label{eq:fz}
  f(z)=(2z+\ln z+z+1-2z^2)\Theta(z),\;\;z=\dover{\mathcal{E}}{4 \gamma^2 \mathcal{E}_s}
\end{equation}
is the angle averaged scattering kernel and the Heaviside step
function $\Theta(z)=1$ for $0\le z \le 1$ and $\Theta(z)=0$
otherwise. The high-energy portion of the spectrum is related to the
index of the electron power-law the same as with synchrotron
emission. However, the low-energy portion of the spectrum can be much
harder than with synchrotron, depending on the seed photon
distribution. For a mono-energetic photon source scattering on a
power-law electron source, the asymptotic energy flux spectrum can be
written
\begin{equation}
  \label{eq:icflux}
   F_{\nu}(\mathcal{E})\propto\left\{
     \begin{array}{lr}
       \mathcal{E} & : \mathcal{E}\ll \mathcal{E}_s \gamma_{min}^2\\
       \mathcal{E}^{-\frac{1-\delta}{2} }& : \mathcal{E}\gg \mathcal{E}_s \gamma_{min}^2
     \end{array}.
   \right.
\end{equation}
Inverse Compton emission of GRBs is attractive because no magnetic
field is required to be present in the jet, unlike synchrotron
emission. However, if the source of GRB non-thermal emission in the 10
keV - 10 MeV range is from inverse-Compton emission, then there should
be no observable synchrotron emission at optical wavelengths. This
contradicts many observations \cite{}.

If a magnetic field is present in the outflow, the electrons will
radiate synchrotron photons that the electrons can ``upscatter'' to
higher energies. The so-called synchrotron self-Compton emission has
been considered as viable source of GRB prompt emission
\cite{Freedman:2001,Spada:2000}. The difference between self-Compton
emission and mono-energetic inverse-Compton emission is the low-energy
spectral index which is 1/3, similar to synchrotron emission. However,
self-Compton and primary synchrotron emission differ in that the self-Compton spectrum has a much broader spectral curvature.


\subsection{Sub-Photospheric Dissipation}
\label{sec:subpht}
The possibility for shocks to form below the photosphere arises from
the fact that the range of outflow parameters allows for
$r_{nt}<r_{ph}$ \cite{Peer:2005}. To illustrate, assume that energy
dissipation occurs at $r>r_s \equiv \eta r_0$. The minimum radius that internal shocks can form is
\begin{equation}
\label{eq:ri}
r_i\approx 2\Gamma r_s.
\end{equation}
Assuming $L=10^{52}$ erg, and $r_0 = 6\times10^{6}$ cm corresponding
to the last stable orbit of a stellar mass black hole. Using
\equationref{eq:rph,eq:ri}, \figureref{fig:subph} shows the allowed
values of $r_{\rm ph}$ and $r_i$.
\begin{figure}[t]
  \centering
  \cfig{2}{subph}{4}
  
  
  \caption{The allowed radii of internal shocks as a function of
    $\Gamma$ are shown in the purple shaded region with the values of
    $r_{\rm ph}$ indicated by the red line. It is evident that shocks
    may form below the photosphere.}
\label{fig:subph}
\end{figure}
This implies that internal shocks can form and dissipate energy into
the electrons before the jet becomes optically thin. In this scenario,
the electrons are heated by the dissipated energy and a balance
between Compton and inverse-Compton scattering occurs. If a low-energy
photon component such as a blackbody exists in this region of the
outflow, it will be upscattered to higher energies by the electrons
that have been heated by the dissipation. This component could explain
the observed non-thermal emission in GRBs. However, this model has yet
to be able to explain the lightcurves and variability present in the
data.




%%% Local Variables: 
%%% mode: latex
%%% TeX-master: "../thesis"
%%% End: 
