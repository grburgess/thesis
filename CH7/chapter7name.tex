\chapter{Spectral Analysis of {\it Fermi} GRBs with Fast and Slow-Cooled Synchrotron Photon Models}
\label{ch:pap2}
\begin{chapterquote}{Dr. Dog}
It's like that old black hole,\\
no matter how you try,\\
you set out each day\\
 never to arrive
\end{chapterquote}




With the validity of using physical models to directly fit GRB
spectral data established, the analysis can be expanded to a larger
sample to allow for a categorical analysis from which physical
implications can be derived. An important question not addressed in
\cite{Burgess:2012} (hereafter B12) is the fast-cooling problem. The spectrum of GRB
090820A was successfully fit with a slow-cooling model. However, GRBs
are expected to be in the fast cooling regime. To address this a
fast-cooling model must be tested along with the slow-cooling
model. The analysis of \cite{Burgess:2013} focuses on these questions
as well as looking at the physical implications of the spectral fit
parameters. The following is an adaptation of \cite{Burgess:2013}.

\begin{centering}

Authors

J.~M.~Burgess,
R.~D.~Preece,
V.~Connaughton,
M.~S.~Briggs,
A.~Goldstein,
P.~N. Bhat,
J.~Greiner,
D.~Gruber,
A.~Kienlin,
C.~Kouveliotou,
S.~McGlynn,
C.~A.~Meegan,
W.~S.~Paciesas,
A.~Rau,
S.~Xiong

M.~Axelsson,
M.~G.~Baring,
C.~D.~Dermer,
S.~Iyyani,
D.~Kocevski,
N.~Omodei,
F.~Ryde,
G.~Vianello

\end{centering}

\hspace{2 cm}

\section{Model Spectral Components}
\label{sec:model}



In the fireball model of GRBs, the majority of the flux is
theoretically expected to be in the form of thermal emission coming
from the photosphere of the jet. However, nearly all of the low-energy
indices implied by GRB spectral analysis with Band-function spectral
inputs have $\alpha<+1$, i.e. too soft to be thermal -- see
  for example \cite{Goldstein:2012} for the BATSE database.
This points to a non-thermal emission process for most
GRBs. Multi-spectral component analysis of {\it Fermi} GRBs has shown
that while the majority of the emission is non-thermal, a small
fraction of the energy radiated apparently originates from a blackbody
component \cite{Guiriec:2010,Axelsson:2012}. In B12, a blackbody
component was also identified when the non-thermal emission was fit
with a synchrotron photon model. This combination of blackbody and
non-thermal emission was predicted by \cite{Meszaros:2000}.
In this paper, before proceeding to the details of the
  analysis, we first review the synchrotron model (see B12), and also
several observable relations of the blackbody component. These
components are then implemented into a fitting program which directly
convolves the physical models with the GBM detector response to
compare with observations.



In principle, there are six spectral parameters that can be
constrained by the fits: $n_0$, $E_{*}$, $\delta$, $\epsilon$,
$\gamth$, and $\gammaMin$; however, we fix $\gamth$, $\gammaMin$, and
$\epsilon$ due to fitting correlations as explained below. The
parameter $E_{*}$ scales the energy of the fit and is linearly related
to the Band function's E$_{\rm p}$. Numerical simulations of particle
acceleration at relativistic shocks have shown \cite{Baring:2004}
that the non-thermal population is generated directly from the thermal
one. To match these circumstances, we set the ratio of $\gamth$ and
$\gammaMin$ to be $\sim$3, following \cite{Baring:2004}.  The
parameters $E_*$, $\gamth$, and $\gammaMin$ all directly scale the
peak energy of the spectrum but do not alter its shape and thus cannot
be independently determined. For this reason we chose values of
$\gamth=300$ and $\gammaMin=900$ for all fits and left $E_*$ free to
be constrained from the fit.

As shown in the \appendixref{ch:pap2app}, such parameter values are on the outer edge
for what is allowed energetically.  For these parameters, the flow is
strongly Poynting flux/magnetic-field energy dominated in order that
electrons with $\gamma\sim 300$ -- 900 can produce radiation in the
MeV regime.  Magnetized jet models are advantageous for energy
dissipation through magnetic reconnection, to produce short timescale
variability, and to accelerate ultra-high energy cosmic ray (see
\appendixref{ch:pap2app}). In fact, a wide range of parameter values with much larger
electron Lorentz factors and smaller magnetic fields are possible. In
weak magnetic-field models, a strong self-Compton component and
$\gamma\gamma$ opacity effects can make a cascade that modifies the
standard emission spectrum of GRBs. By considering a strongly
magnetically dominated model, these issues can be neglected.

For the chosen parameters, the system is always in the strongly cooled
regime (see \appendixref{ch:pap2app}). Nevertheless, we adopt the expression,
\equationref{eq:elec_dist}, to approximate an electron spectrum in the
slow-cooling regime.  The parameter $\epsilon$, corresponding to the
relative amplitude between the thermal and non-thermal portions of the
electron distribution, was not easily constrained in the fitting
process used in B12 and produced small non-physical discontinuities in
the electron distribution, as pointed out in
\cite{Beloborodov:2012}. Therefore, here we numerically fix this
parameter to the small value of $(\gammaMin/\gamth)^2\times
\exp(-\gammaMin/\gamth)$, so that there is no discernible
discontinuity between the thermal and non-thermal parts of the
distribution. The thermal component helps smooth out the spectral
structure at $E_{\ast}$, but does not alter the asymptotic index of
$\alpha = 2/3$ realized for synchrotron emission from populations with
lower bounds to their particle energies. After these simplifications,
three shape parameters remain free: $E_{*}$, $\delta$ and $n_0$, which
corresponds to the amplitude. Compared with the Band function's four
fit parameters this model is simpler yet tied to actual physical
processes.


Even though we examine a model with $\gamma_{th} = 300$ and
$\gamma_{min} = 900$, the spectral fitting is insensitive to the exact
value of the product $\Gamma B\gamma^2$ provided that the constraints
discussed in \appendixref{ch:pap2app} are satisfied. Even in a strong cooling regime
defined by these low assumed values of $\gamma_{th}$ and $\gamma_{min}
$, second-order processes in GRBs \cite{Waxman:1995,Dermer:2001},
which can become more important than first-order processes in
relativistic shocks \cite{2009herb.book.....D}, allow us to consider a
model that is effectively slowly cooled. Simulations typically have
more parameters than our current model
\cite{Peer:2004,Asano:2009,Daigne:2009}, and constraining those models
via spectral templates using data from {\it Fermi} may be difficult.



\subsection{Blackbody Component}
\label{sec:model:bb}
The pure fireball scenario for GRB emission predicts that most of the
flux is from thermal emission
\cite{Goodman:1986,Paczynski:1986}. This is because as the jet
becomes optically thin at some photospheric radius, $r_{ph}$, it
releases radiation that has undergone many scatterings with the
optically thick electrons below the photosphere. We model this
emission as a blackbody
\begin{equation}
  \label{eq:blackbody}
  F(\mathcal{E})\;=\;A \mathcal{E}^3 \frac{1}{ e^{\frac{\mathcal{E}}{kT}}-1}
\end{equation}
where A is the normalization and kT scales the energy of the
function. This is simplified thermal emission from the photosphere
that does not take into account the effects of relativistic broadening
that can produce a multi-color blackbody emerging from the photosphere
\cite{Beloborodov:2010,Ryde:2010,Peer:2011}. \cite{Ryde:2006} showed
that if it is assumed that the thermal component is emanating from the
photospheric radius of the jet, several properties about the blackbody
component are derivable. The cooling behavior is well predicted for a
thermal component. The temperature of the blackbody should decay as
$T\propto r_{ph}^{-2/3}$. If $\Gamma$ is assumed to remain constant
during the coasting phase of the jet then it can be shown that the
temperature should decay as $T \propto t^{-2/3}$ in time.  It has been
found observationally that the evolution of kT often follows a broken
power-law trend with the index below the break averaging to $\sim
-$2/3 \cite{Ryde:2009}. Finally, a true blackbody has a well defined
relation between energy flux and temperature:
\begin{equation}
   \label{eq:sbl}
  F_{BB}=N\sigma_{sb} T^4
\end{equation}
where $N$ is a normalization related to the transverse size of the
emitting surface, $r_{ph}/\Gamma$
\cite{Ryde:2005,Ryde:2009,Iyyani:2013}, and $\sigma_{sb}$ is the
Stephan-Boltzmann constant.


The photospheric radius and the transverse size of the photospheric
emitting region are also of great importance to understanding the geometry
and energetics of GRBs. In \cite{Ryde:2009}, a parameter
$\mathcal{R}$ 
\begin{equation}
  \label{eq:scR}
  \mathcal{R}(t)\;\equiv\;\left(\frac{F_{BB}(t)}{\sigma_{sb}T(t)^4} \right)^{1/2} \propto r_{ph}.
\end{equation}
is used to track the outflow dynamics of the burst (see \figureref{fig:grbjet2} for a conceptual view of $\mathcal{R}$). The connection
between $\mathcal{R}$ and N from \equationref{eq:sbl} is established
by noting that $N = \mathcal{R}^2$. Thereby, only if N is constant
would we expect to recover the relation established in
\equationref{eq:sbl}.  Several BATSE GRBs were found to have a
power-law increase of $\mathcal{R}$ with time. However, the connection
between $\mathcal{R}$, T, and $F_{BB}$ was difficult to establish
because the error on the data points was large. Understanding these
connections is essential to unmasking the structure and temporal
evolution of GRB jets.

%% Observations
\section{Time Resolved Analysis}

\label{sec:observe}

\subsection{Summary of Technique}
The GRBs in our sample were selected based on two criteria: large peak
flux and single-peaked, non-overlapping temporal structure. The GRBs
were binned temporally in an objective way described in
\sectionref{sec:tbin} and spectral fits were performed on each time
bin using four different photon models (Band, Band+blackbody,
synchrotron, synchrotron+blackbody). When fitting synchrotron we
compared the fits of slow-cooling and fast-cooling synchrotron. In
many cases the fits from fast-cooling synchrotron completely
failed. From these spectral fits a photon flux lightcurve was
generated for each component and fitted with a pulse model to
determine the decay phase of the pulse. We describe each step in the
following subsections.

\subsection{Sample Selection}
To fully constrain the parameters of the fitted models, we selected
GRBs with a requirement that the peak flux be greater than 5 photons
s$^{-1}$ cm$^{-2}$ between 10 keV and 40 MeV. It is important for our
GRBs to have a simple, single-peaked lightcurve structure to avoid the
overlapping of different emission episodes. This facilitates the
identification of distinct evolutionary trends in the physical
parameters for the emission region. While we cannot be sure that a
weaker emission episode does not lie beneath the main peak, the bursts
we selected have no significant additional peak during the rise or
decay phase of the pulse. These two cuts left us a sample of eight
GRBs: GRB 081110A \cite{GRB081110A}, GRB 081224A \cite{GRB081224A},
GRB 090719A \cite{GRB090719A}, GRB 090809B, GRB 100707A \cite{
  GRB100707A}, GRB 110407A, GRB 110721A, \cite{GRB110721A} and, GRB
110920A (\figureref{fig:lc}). GRB 081224A and GRB 110721A were both
analyzed including the new LAT Low-Energy (LLE) data that provides a
high effective area above 30 MeV for the analysis of short-lived
phenomena, thanks to a loosened set of cuts with respect to LAT
standard classes \cite{Vero:2010, Ack:2012a}. This data selection
bypasses the typical photon classification \cite{Ackerman:2012} tree
and includes events that would normally be excluded but can be
selected temporally when the signal to background rate is high, such
as with GRBs. GRB 081224A had very little data above 30 MeV but the
LLE data helped to constrain the spectral fits. From this sample, five
GRBs (GRB 081224A, GRB 090719A, GRB 100707A, GRB 110721A, and GRB
110920A) had blackbody components that were bright enough to analyze
(\tableref{tab:grbs}).


\begin{figure}

 \centering



  \subfigure[]{

    \label{fig:lc:a}
    \cfig{7}{lc3}{2.5}}\subfigure[]{
    \label{fig:lc:b}
    \cfig{7}{lc4}{2.5}}
  \subfigure[]{
    \label{fig:lc:c}
    \cfig{7}{lc5}{2.5}}\subfigure[]{
    \label{fig:lc:d}
    \cfig{7}{lc6}{2.5}}
\subfigure[]{
    \label{fig:lc:e}
    \cfig{7}{lc7}{2.5}}\subfigure[]{
    \label{fig:lc:f}
    \cfig{7}{lc8}{2.5}}
\subfigure[]{
    \label{fig:lc:g}
    \cfig{7}{lc9}{2.5}}\subfigure[]{
    \label{fig:lc:h}
    \cfig{7}{lc10}{2.5}}

  \caption{The energy flux lightcurves of the synchrotron component
    for the entire sample (\emph{black} curve). The integration range
    is from 10 keV - 40 MeV for all GRBs except GRB 081224A and GRB
    110721A which are from 10 keV - 300 MeV. Superimposed is the
    slow-cooled synchrotron $\Ep$ (\emph{red} curve) demonstrating the
    hard to soft evolution of the bursts. }

  \label{fig:lc}

\end{figure}


\begin{table}

\centering
\scriptsize
\begin{tabular}{c | c c c c}



GRB & Peak Flux (p/s/cm$^2$) & Duration Analyzed (s) & Blackbody Component & LLE data\\ 

\hline \hline

GRB 081110A & 20.88  &  4.61 &   \text{\sffamily X}         & \text{\sffamily X}\\ 

GRB 081224A & 17.11  & 18.36  & $\checkmark$ & $\checkmark$\\

GRB 090719A & 26.52   & 30.09  & $\checkmark$ & \text{\sffamily X}\\

GRB 090809B & 18.36  & 14.64  & \text{\sffamily X} & \text{\sffamily X}\\

GRB 100707A & 18.77  & 22.39  & $\checkmark$ & \text{\sffamily X}\\

GRB 110407A & 15.6  & 20.48  & \text{\sffamily X} & \text{\sffamily X}\\

GRB 110721A & 29.82   & 12.7  & $\checkmark$ & $\checkmark$\\

GRB 110920A & 8.08   & 238.29 & $\checkmark$ & \text{\sffamily X} \\



\end{tabular}

\caption{The GRBS in our sample. The peak fluxes were taken from the brightest bin of each GRB with a duration determined by Bayesian blocks.}

\label{tab:grbs}

\end{table}



\subsection{Spectral Analysis}
\label{sec:specan}
For spectral fits we used the RMFIT
ver4.1\footnote{http://fermi.gsfc.nasa.gov/ssc/data/analysis/user/}
software package developed by the GBM team. Fitting the synchrotron
photon model requires a custom module developed and used in B12. Each
time bin was fit with one of the four spectral models mentioned
above. We fit the physical models to compare the validity of each one
against the other and the Band function to try and understand how the
Band function parameters correlate with the best fit physical
model. If the addition of blackbody component did not make a
significant improvement of at least 10 units of C-stat
\cite{Arnaud:2011} for any time bins of a particular GRB, then we did
not include the blackbody component in the analyzed fits for that
burst. However,
near the end of the prompt emission in some GRBs, the blackbody
component becomes weak but has spectral evolution consistent with more
significant time bins in the burst. The spectral parameters of the
blackbody in those bins were included even though they contributed
large error bars to some quantities. We checked with simulations that
this cut was sufficient to identify a significant addition of a
blackbody to the fit model.



\section{Results}
\label{sec:results}
\subsection{Test of Slow-Cooling Synchrotron}
\label{sec:results:scs}

In nearly all cases the synchrotron or synchrotron+blackbody model
produced a fit with a comparable or better C-stat than the Band
function. The GRBs exhibited hard to soft spectral evolution (
\figureref{fig:specEvo}) for both components. From these fits we can
derive several interesting properties of the bursts. The results of
fitting the non-thermal part of each time bin in our sample with
slow-cooled synchrotron indicate that this model can indeed fit the
data well. The spectral parameters are summarized
in \appendixref{ch:fitparams}.
\begin{figure}[h]

  \centering

 \cfig{7}{spectrum}{5}

 \caption{The spectral evolution of GRB 081224A is an example of the
   typical evolution observed for the entire sample.  The synchrotron
   (from \emph{light blue} to \emph{dark blue}) and blackbody (from
   \emph{yellow} to \emph{red}) both evolve from hard to soft peak
   energies with time. For this GRB, the high-energy power-law
   corresponding to the electron spectral index does not evolve
   significantly over the duration of the burst.}

  \label{fig:specEvo}

\end{figure}
The C-stat fit statistic per degree of freedom was at or
near 1 for most time bins. The spectral fit residuals cluster around
zero, with no deviations at low-energy that might indicate the
presence of an additional power-law component
(\figureref{fig:counts}). The residuals are below 4$\sigma$ for the
entire energy range. 
\begin{figure}

 \centering

 \subfigure[]{
   \label{fig:counts:a}
  \cfig{7}{counts}{3.2}
}\subfigure[]{
   \label{fig:counts:b}
   \cfig{7}{counts-1001}{3.}
}





\caption{A time bin of GRB 110721A (left panel) and GRB 100707A (right
  panel) demonstrating typical count spectra from the sample. Two
  extreme cases are shown: a subdominant and dominant blackbody
  component. The response has been convolved with synchrotron
  (\equationref{eq:synch_flux}) and a blackbody to produce counts. The
  residuals from the fits indicate that the model is fitting the data
  well.}

\label{fig:counts}

\end{figure}
As an example, the fit C-stats for GRB 100707A
and GRB 110721A are shown in \tableref{tab:grb1c,tab:grb2c}
respectively. They show that the slow-cooled synchrotron model fits
the data as well as the Band function when a blackbody is included in
both cases. The fit C-stats for fast-cooled synchrotron are shown for
GRB 110721A to compare all three non-thermal models. These results
imply that slow-cooled synchrotron is a viable model for GRB prompt
emission. We cannot claim it provides a better fit to the data than
other untested models and we will investigate and compare other
physical models in future work.

\begin{table}[h!]

\centering

\begin{tabular}{c | c c c c}



Time Bin & Band C-Stat & Band+BB C-Stat & Synchrotron C-stat & Synchrotron+BB C-stat \\ 

\hline \hline

-0.2-0.2 & 453 & 450 & 485 & 455 \\ 



0.2-0.8 & 374 & 362 & 695 & 364 \\ 



0.8-2.4 & 427 & 405 & 2546 & 487 \\ 



2.4-3.0 & 408 & 400 & 903 & 413 \\ 



3.0-4.3 & 431 & 415 & 1214 & 430 \\ 



4.3-5.7 & 397 & 363 & 791 & 399 \\ 



5.7-7.2 & 411 & 390 & 598 & 422 \\ 



7.2-12.8 & 488 & 414 & 829 & 447 \\ 



12.8-22.2 & 558 & 412 & 594 & 423 \\ 





\end{tabular}



\caption{The time resolved C-Stat values for GRB100707A show that while the Band function and synchrotron models combined with a blackbody function both fit the data well, the non-thermal functions fit the data very differently when not combined with a blackbody. Specifically, where the blackbody is the brightest (\figureref{fig:fluxComp:c,fig:fluxComp:d} intervals 2, 3, and 4) the Band function alone fits the data acceptably while the synchrotron model alone fits the data poorly. This shows that the flexibility of the Band function can mask the need for the blackbody component. The Band+blackbody fits actually fit the data better when the blackbody is very bright in this case. This is most likely due to the blackbody function (\equationref{eq:blackbody}) used is simplified and the actual emission may be broadened due to beaming effects that are only important to the fit when the blackbody is bright and the synchrotron fit is used. The Band function makes up for these effects by having a harder $\alpha$. We tested using an exponentially cutoff power-law combined with the synchrotron model and the fits were as good as those with the Band function. We will examine the use of a more realistic photosphere model in future work.}

\label{tab:grb1c}

\end{table}











\begin{table}[h!]

\centering
\scriptsize
\begin{tabular}{c | c c c c c c}



Time Bin & Band C-Stat & Band+BB C-Stat & Synchrotron C-stat & Synchrotron+BB C-stat & Fast C-stat & Fast+BB C-stat \\ 

\hline \hline

-0.07-0.08 & 640 & 640 & 673 & 673 & 709 & 709 \\ 

0.08-0.48 & 690 & 690 & 704 & 704 & 1088 & 1088 \\ 

0.48-1.28 & 709 & 668 & 688 & 670 & 1654 & 957 \\ 

1.28-2.78 & 887 & 761 & 838 & 770 & 1646 & 1041 \\ 

2.78-3.78 & 678 & 642 & 655 & 643 & 797 & 666 \\ 

3.78-5.88 & 648 & 631 & 677 & 634 & 694 & 660 \\ 

5.88-7.63 & 729 & 728 & 733 & 721 & 773 & 724 \\ 

7.63-12.63 & 932 & 693 & 693 & 692 & 756 & 698 \\ 
\end{tabular}



\caption{The C-stat values for GRB 110721A. The significance of the addition of the blackbody is not as large as with GRB 100707A (\tableref{tab:grb1c}) due to the weakness of the blackbody component. The fits for fast-cooled synchrotron are included to demonstrate the poor quality fits that are obtained both with fast-cooling synchrotron and fast-cooling synchrotron with a blackbody.}

\label{tab:grb2c}

\end{table}


An important parameter constrained in these fits is the electron
index, $\delta$, of the accelerated power-law. The canonical value for
diffusive acceleration at ultra-relativistic, parallel shocks is
$\delta$=2.2 \cite{Kirk:1987,Kirk:2000}. The distribution of
constrained $\delta$'s (\figureref{fig:index}) is broad and centered
around $\delta=5$ (i.e., $\beta = 3$).
\begin{figure}[h!]

 \centering

  \cfig{7}{index}{4}

  \caption{The distribution of electron indices from the slow-cooling
    synchrotron fits. Only indices that were constrained are
    plotted. The distribution is broad but centered at $\delta=$5
    which is much steeper than expected from simple relativistic shock
    acceleration.}

  \label{fig:index}

\end{figure}
This steep index could provide
clues for the structure and magnetic turbulence spectrum of the
shocks. \cite{Baring:2006}, \cite{Ellison:2004} and \cite{Baring:2012}
show that shock speed, obliquity, and turbulence all have a strong
effect on the electron spectral index of the accelerated
electrons. Steeper indices correspond to increasing shock obliquity in
superluminal shocks. Fit models which are built from the electron
distribution such as the one used in this work enable a direct
diagnostic of the GRB shock structure.






\subsection{Test of Fast-Cooling Synchrotron}
\label{sec:fast}
In order to see if any spectra were consistent with the fast-cooling
synchrotron spectrum, we implemented a fast-cooled synchrotron model
where the electrons were distributed according to the broken power-law
in \equationref{eq:necool}. These apply to the ``undisturbed plasma''
outside the shock acceleration/injection zone. As with the
slow-cooling fits, $\gammaMin$ was held fixed to 900. Several spectra
were tested and all resulted in very poor fits regardless of whether
the low-energy index found with the Band function was much harder than
$-$3/2 (\figureref{fig:fastS} and \tableref{tab:fast}). This is due to
the broad spectral curvature of the fast-cooled spectrum around the
$\nu F_{\nu}$ peak. 
\begin{figure}

 \centering

  \cfig{7}{fast}{5}

  \caption{The fast-cooled synchrotron fits are poor for nearly all of our
    sample because none of the spectra have a low-energy
    index as steep as $-$3/2. Therefore the fast-cooled synchrotron
    spectrum is too broad around the $\vFv$ peak as shown in
    this example spectrum.}

  \label{fig:fastS}

\end{figure}


\begin{table}[h!]
\scriptsize
\centering

\begin{tabular}{c | c | c c | c}



Time Bin & Band $\alpha$ & Slow-cooled Synchrotron C-stat & Fast-cooled Synchrotron C-stat & $\Delta_{\rm C-stat}$ \\ 

\hline \hline



-5.38-2.82 & -0.9 & 523 & 599 & 76 \\ 



2.82-3.84 & -0.7 & 507 & 604 & 97 \\ 



3.84-4.86 & -0.8 & 506 & 596 & 90 \\ 



4.86-6.91 & -1.0 & 534 & 626 & 92 \\ 



6.91-9.98 & -1.1 & 591 & 639 & 48 \\ 



9.98-15.1 & -1.5 & 494 & 494 & 0 \\ 



\end{tabular}

\caption{For each time bin of GRB 110407A we examine the C-stat value of synchrotron and fast-cooled synchrotron. This GRB did not have a blackbody in its spectrum. While the Band function and slow-cooling synchrotron fit the spectrum well, fast-cooling synchrotron does not fit the spectrum unless Band $\alpha=-1.5$. In this case, the fast-cooled synchrotron peak energy was very unconstrained due to the curvature of the data being narrower than the photon model's curvature.}

\label{tab:fast}

\end{table}

The broken power-law nature of the electron
distribution is smeared out by the synchrotron kernel and cannot fit
the typical curvature of the GBM data. In fact, the fast-cooling
synchrotron spectrum has a spectral index of $-$2/3 below the
$\gamma_c$ which we have fixed at 1. The fitting algorithm increased
$E_*$ to high values to align the $-$2/3 index with the data, which
resulted in poor fits (\figureref{fig:fastComp}). Even when a
blackbody is present in the bursts, fast-cooled synchrotron is not a
good fit to the non-thermal part of the spectrum
(\tableref{tab:grb2c}). The lack of GRBs with low-energy indices as
steep as $-$3/2, additionally disfavors fast-cooled synchrotron as the
non-thermal emission component in GRB spectra.
\begin{figure}[h!]

 \centering

  \cfig{7}{fastComp}{5}

  \caption{An example time bin of GRB 110407A comparing the fitted
    $\vFv$ spectra of the Band Function (\emph{green}),
    slow-cooled synchrotron (\emph{red}), and fast-cooled synchrotron
    (\emph{blue}). While the Band function and slow-cooled synchrotron
    fits resemble each other, the fast-cooled synchrotron fit is only
    able to fit the low-energy part of the spectrum. Because
    fast-cooled synchrotron has an index of $-$2/3 below the cooling
    frequency, the fitting engine pushes the value of E$_*$ very high
    to fit the low-energy part of the spectrum resulting in a $-$3/2
    index near the $\vFv$ peak. The high-energy power-law of
    the fast-cooling synchrotron spectrum is pushed out of the data
    energy window.}

  \label{fig:fastComp}

\end{figure}

\begin{table}[h!]
\centering
\begin{tabular}{c | c c c}

Time Bin & Band-BB {\dcstat} & Synchrotron-BB {\dcstat} & Fast-BB {\dcstat} \\ 
\hline \hline

-0.07-0.08 & 0 & 0 & 0 \\ 

0.08-0.48 & 0 & 0 & 0 \\ 

0.48-1.28 & 41 & 18 & 697 \\ 

1.28-2.78 & 126 & 68 & 605 \\ 

2.78-3.78 & 36 & 12 & 131 \\ 

3.78-5.88 & 17 & 43 & 34 \\ 

5.88-7.63 & 1 & 12 & 49 \\ 

7.63-12.63 & 239 & 1 & 58 \\ 


\end{tabular}

\caption{ The {\dcstat} values for GRB 110721A tell a different story than GRB 100707A (\tableref{tab:grb1dc}), though both GRBs show a significant improvement in the fit when a blackbody is included. Even thought the fast-cooled fits showed extreme improvement with the inclusion of a blackbody, the fits are still poor compared with the slow-cooled model (See \tableref{tab:grb2c}).}
\label{tab:grb2dc}
\end{table}


\subsection{Synchrotron vs. Band}
\label{sec:results:bvs}
The Band function has been used in the literature as a proxy for
distinguishing among non-thermal emission mechanisms. The predicted
non-thermal emission of GRBs is typically characterized as a
smoothly-broken power-law with the high-energy spectral index related
to the index of accelerated electrons and the low-energy index related
to the radiative emission process. Therefore, fitting a Band function
to the emission spectrum of a GRB {\em should} serve as a diagnostic
of the radiative process responsible for the
emission. \cite{preece:1998} examined the BATSE GRB catalog and
looked at the distribution low-energy indices from Band function
fits. They found that the distribution peaked at $\alpha\approx -1$
and that 1/3 of the fitted spectra had low-energy indices too hard for
synchrotron radiation. The assumption is that the Band function's
shape approximates synchrotron but has an added degree of freedom in
the low-energy index. However, the Band function has a broader range
of curvatures around the $\vFv$ peak allowing it the possibility to
deviate from the shape of synchrotron above and around the $\vFv$
peak. The synchrotron $\vFv$ peak is $\propto\gammaMin^2 B\propto
E_*$, leading to the relation between Band and synchrotron models
E$_{\rm p}\propto E_*$. This relationship is easily recovered from our
sample (\figureref{fig:EpEc}). 
\begin{figure}[h!]

  \centering

  \cfig{7}{EpEc}{4}

  \caption{Derived values of the parameter E$_{\rm p}$ (obtained using
    the Band function to fit GRB spectra), versus $E_*$ (obtained using
    an optically-thin non-thermal synchrotron to fit GRB spectra).}

  \label{fig:EpEc}

\end{figure}
Direct comparison of the quality of
the fits using Band and the synchrotron model is not the goal of this
study. Both Band and the synchrotron model fit the data well with
their respective fit residuals not deviating more than 4$\sigma$ and
centered around zero (\figureref{fig:counts}). It is important to
stress that the questions being asked are does the synchrotron model
fit the data?  and, what temporal evolution do the synchrotron
parameters undergo?

For all GRBs in our sample that include both a blackbody and
non-thermal component we compare the photon flux (photons s$^{-1}$
cm$^{-2}$) lightcurves (integrated from 10 keV - 40 MeV) derived from
synchrotron fits with those derived from Band fits (
\figureref{fig:fluxComp}). It is seen that while both methods recover
the same total flux, the flux from the individual components is much
better constrained when using the synchrotron model for the
non-thermal component. This is due to the pliability of the Band
function below E$_{\rm p}$ that is not afforded to the synchrotron
model.
\begin{figure}

  \centering

   \subfigure{
    \label{fig:fluxComp:a}
    \cfig{7}{10fc}{2.3}}\subfigure{
    \label{fig:fluxComp:b}
    \cfig{7}{11fc}{2.3}}
  \subfigure{
    \label{fig:fluxComp:c}
    \cfig{7}{0fc}{2.3}}\subfigure{
    \label{fig:fluxComp:d}
    \cfig{7}{1fc}{2.3}}
\subfigure{
    \label{fig:fluxComp:e}
    \cfig{7}{5fc}{2.3}}\subfigure{
    \label{fig:fluxComp:f}
    \cfig{7}{6fc}{2.3}}

  \caption{A subset of flux lightcurves illustrating both the temporal
    structure of the different components and the advantages of using a
    physical model to deconvolve the detector response. The left
    column contains the lightcurves using synchrotron (\emph{blue
      thick line}) and blackbody (\emph{red thin line}) while the
    right column contains the lightcurves made from using the Band
    function (\emph{green thick line}) and blackbody (\emph{red thin
      line}). The total flux lightcurve (\emph{black dotted line}) of
    both approaches are the same. The components have a very simple
    and constrained evolution when using synchrotron as the
    non-thermal component. This is potentially indicative that
    synchrotron is the actual emission mechanism and the response is
    being properly deconvolved. In contrast, the lightcurves where the
    Band function is used have large errors and the blackbody does not
    have a consistent evolution.}

  \label{fig:fluxComp}

\end{figure}



The C-stat fit values for the synchrotron model loosely correlate with
the value of Band $\alpha$ found by fitting the same interval with the
Band function. When Band $\alpha$ was much harder than zero, the
synchrotron fit was poor and typically required adding blackbody to
fit the data. The flexibility of the Band function with its low-energy
power law creates the possibility that the index alpha of that power
law will not accurately measure the true slope if E$_{\rm p}$ is too
close to the low-energy boundary of GBM data. Simulated spectra using
the Band function were created with a grid in both Band $\alpha$ and
E$_{\rm p}$ to ensure that low values of E$_{\rm p}$ do not affect the
reconstruction of Band $\alpha$ in our fits. It was found that Band
$\alpha$ could be accurately measured when E$_{\rm p}$ was as low as
$\sim$20 keV. While the asymptotic value of synchrotron is $-$2/3,
fitting the photon model with an empirical function like Band with a
slightly different curvature could result in measured low-energy
indices that are different. To measure this effect, simulated
synchrotron spectra with different $E_*$ were fit with the Band
function. The Band $\alpha$ showed a slight dependence on the
synchrotron peak; moving to softer values for lower $E_*$. The
distribution of fitted Band $\alpha$ values from these simulations
centered around $-$0.81 $\pm$ 0.1, a slightly softer value than $-$2/3
which may explain the clustering of Band $\alpha$ at $-$0.82 in the
GBM spectral catalog \cite{Goldstein:2012} if a majority of the
non-thermal spectra are the result of synchrotron emission.



\subsection{High-Energy Correlations}
\label{sec:results:hec}
There is a well-known spectral evolution in GRB pulses of $E_{\rm
  peak}$ evolving from hard to soft (see
\figureref{fig:lc,fig:specEvo}). This leads to two time-resolved
correlations between hardness (measured as E$_{\rm p}$) and flux
\cite{golenetskii:1983,Liang:1996,Ghisellini:2010}. \cite{Liang:1996}
(hereafter LK96) showed that the hardness intensity correlation (HIC)
which relates the instantaneous energy flux $F_{E}$ to spectral
hardness and can be defined as
\begin{equation}
  F_E\;=\;F_0\left(\frac{E_{\rm p}}{E_{{\rm p},0}} \right)^q \;,
\end{equation}
where $F_0$ and E$_{\rm p,0}$ are the initial values at the start of the
pulse decay phase and $q$ is the HIC index. \cite{Ryde:2001} found
that 57\% of a sample of 82 BATSE GRBs were consistent with this
relation. The second relation is the hardness-fluence correlation
(HFC) which relates hardness to the time-running fluence of the
GRB. Time-running fluence, $\Phi(t)$, is defined as the cumulative,
time-integrated flux of each time bin in a GRB. The HFC is expressed
as
\begin{equation}
  \label{eq:hfc}
  E_{\rm p}\;=\;E_{\rm p,0}e^{-\Phi(t)/\Phi_0}\;,
\end{equation}
where $\Phi_0$ is the decay constant. LK96 noted that this equation is
similar to the form of a confined radiating plasma. This should not be
the case for optically-thin synchrotron. Upon differentiating
\equationref{eq:hfc} it becomes apparent that the change in hardness
is nearly equal to the energy density:
\begin{equation}
  \label{eq:plasma}
  -\frac{dE_{\rm p}}{dt}\;=\;-\frac{F_{\nu}E_{\rm p}}{\Phi_0}\;\approx\;-\frac{F_E}{\Phi_0}.
\end{equation}
The HFC could be the result of a confined plasma with a fixed number
of particles cooling via $\gamma$-radiation as proposed by LK96. Since
these relations are only applicable during the decay phase of a pulse
the value of $T_{max}$ (the time of the peak flux) from the pulse fit
of each GRB is used as the initial point for $F_0$ and $E_{\rm p,0}$.

The use of a hardness indicator is somewhat ambiguous. Historically,
the ratio of counts in low and high-energy channels was used as a
hardness measure. This has an advantage of being model-independent but
suffers from the lack of information associated with the instrument
response. High-energy photons can scatter in the detector and not
deposit their full energy thereby artificially lowering the hardness
ratio. LK96 used the Band function E$_{\rm p}$ to compute hardness,
which as a deconvolved quantity is less instrument-dependent but
introduces a model dependence. We take this approach for both the Band
and synchrotron model fluxes. For synchrotron we use the $E_*$
parameter as our hardness indicator. This is justified by the
relationship between $E_*$ and E$_{\rm p}$ (see
\sectionref{sec:results:bvs} and \figureref{fig:EpEc}).


We compute the HIC and HFC for the synchrotron fits for each GRB in
our sample (\figureref{fig:Epcor} and \tableref{tab:cor}). All the
GRBs seemed to follow the HIC to some extent. We find that the HIC
index for $E_*$ ranges between $\approx$1-2. When using the Band
function E$_{\rm p}$ as a hardness indicator, it is expected that
E$_{\rm p} \propto {\rm L}^{1/2} \propto F_{E}^{1/2}$, which follows
from synchrotron theory, supposing that only the $\Gamma$ factor
changes while the internal properties remain (however unlikely) the
same \cite{Ghisellini:2010}. Decay behavior due to light travel-time
effects of a briefly illuminated relativistic spherical shell varies
according to $E_{p}\propto L^{1/3}$, that is, $q = 3$
\cite{Kumar:2000,Dermer:2004,Genet:2009}, whereas GRB observations
here show $L\propto F \propto E_{p}^{1.1}$ -- $E_{p}^{2.3}$
(\tableref{tab:cor}).  Evolution of internal parameters that would
explain the observed correlations is an open question. The synchrotron
fits seem to obey the HFC fairly well. Owing to the large errors in
the Band flux, fits with synchrotron are more consistent with the HFC
and HIC than those with Band. The deviations of the data from the
expected synchrotron HIC may be due to the fact that there are
overlapping pulses under the main emission that alter the decay
profile. In addition, the use of Bayesian blocks to select time bins
ignores spectral evolution. If bins with very different E$_{\rm p}$
are combined then it could affect the the HIC and HFC data.

\begin{figure}[h!]

  \begin{center}
    \subfigure{
      \label{fig:Epcor:a}
      \cfig{7}{flux}{4.2}
    }    \subfigure{
      \label{fig:Epcor:b}
      \cfig{7}{fluence}{4.2}
    }
  \end{center}  

  \caption{The non-thermal emission of all of the bursts in the sample
    loosely follow $F_E$-$E_p$ and $E_p$-fluence relations. See 
    \tableref{tab:cor} for the numerical results. }

  \label{fig:Epcor}

\end{figure}


\begin{table}[h!]
\centering
\begin{tabular}{c | c c c c}
GRB & Flux Index $q$ & $\chi^2_{red}$ & $\Phi_0$ & $\chi^2_{red}$ \\ 
\hline \hline
GRB 081110A & 2.32$\pm$0.4 & 0.6 & 97$\pm$23 & 0.4 \\ 

GRB 081224A & 1.74$\pm$0.1 & 1.5 & 253$\pm$23 & 0.3 \\ 

GRB 090719A & 1.14$\pm$0.07 & 0.98 & 245$\pm$17 & 1.2\\

GRB 09080B & 1.58$\pm$0.05 & 8.0 & 188$\pm$9 & 1.0 \\ 

GRB 100707A & 1.04$\pm$0.02 & 1.2 & 444$\pm$24 & 7.3 \\ 

GRB 110407A & 1.72$\pm$0.20 & 0.5 & 214$\pm$32 & 4.2 \\ 

GRB 110721A & 1.08$\pm$0.03 & 14.4 & 269$\pm$13 & 15.4 \\ 

GRB 110920A & 1.37$\pm$0.06 & 0.5 & 669$\pm$33 & 1.2 \\ 



\end{tabular}
\caption{Sample correlations for both flux and fluence for the synchrotron component.}
\label{tab:cor}
\end{table}

%%



\subsection{Blackbody component}
\label{sec:results:bb}
For most of the spectra in our sample, the blackbody's $\vFv$ peak is
below the $\vFv$ peak of the non-thermal component. There is sometimes
a much larger change in C-stat between fits with synchrotron and
synchrotron+blackbody than those of Band and Band+blackbody owing to
the fact that the Band function has more freedom in the shape below
E$_{\rm p}$ (\tableref{tab:grb1dc,tab:grb2dc}). Simulations of both
Band and slow-cooling synchrotron were used to find the significance
of adding a blackbody to the spectrum. As an example, the time bin
covering 0.8 s to 2.4 s is examined here. The \dcstat between the fit
with the Band function and the fit with Band and the blackbody is 22
while \dcstat using the synchrotron model is 2059. 10000 simulations
of each model were created as described in \sectionref{sec:sigtest}
and fit with $H_0$ and $H_1$. For the Band function a p-value of
$4\times 10^{-4}$ was obtained while the p-value of the Synchrotron
fits was not obtainable due to CPU time constraints but is clearly
smaller than that of the Band function. Therefore, the addition of the
blackbody is significant.

\begin{figure}[h!]
  \centering
  \subfigure{\cfig{7}{banddcstat}{3.2}}\subfigure{\cfig{7}{bandfracFig}{3.2}}
  \subfigure{\cfig{7}{syncdcstat}{3.2}}\subfigure{\cfig{7}{syncfracFig}{3.2}}
  \caption{The \dcstat distribution and cumulative distribution from
    the Band+blackbody (\emph{top}) and synchrotron+blackbody
    (\emph{bottom}) simulations. Two p-value levels are shown in the
    cumulative plots indicating the classical 1$\sigma$ and 2$\sigma$
    significance levels. The blue line indicates the fraction of the
    distribution below the fitted \dcstat value and the red line
    indicates the fraction above it. It can be seen that the \dcstat
    value from the synchrotron fit is far out in the tail of the
    distribution.}
  \label{fig:sigtest}
\end{figure}


\begin{table}[h]
\centering
\begin{tabular}{c | c c}

Time Bin & Band-BB {\dcstat} & Synchrotron-BB {\dcstat} \\ 
\hline \hline

-0.2-0.2 & 3 & 30 \\ 

0.2-0.8 & 12 & 331 \\ 

0.8-2.4 & 22 & 2059 \\ 

2.4-3.0 & 8 & 490 \\ 

3.0-4.3 & 16 & 784 \\ 

4.3-5.7 & 34 & 392 \\ 

5.7-7.2 & 21 & 176 \\ 

7.2-12.8 & 74 & 382 \\ 

12.8-22.2 & 146 & 171 \\ 

\end{tabular}

\caption{The {\dcstat} between the Band function and synchrotron model fits with and without the inclusion of a blackbody for GRB 100707A. The blackbody has a significantly larger impact on the fit when included with the synchrotron model.}
\label{tab:grb1dc}
\end{table}



\begin{table}[h]
\centering
\begin{tabular}{c | c c c}

Time Bin & Band-BB {\dcstat} & Synchrotron-BB {\dcstat} & Fast-BB {\dcstat} \\ 
\hline \hline

-0.07-0.08 & 0 & 0 & 0 \\ 

0.08-0.48 & 0 & 0 & 0 \\ 

0.48-1.28 & 41 & 18 & 697 \\ 

1.28-2.78 & 126 & 68 & 605 \\ 

2.78-3.78 & 36 & 12 & 131 \\ 

3.78-5.88 & 17 & 43 & 34 \\ 

5.88-7.63 & 1 & 12 & 49 \\ 

7.63-12.63 & 239 & 1 & 58 \\ 


\end{tabular}

\caption{ The {\dcstat} values for GRB 110721A tell a different story than GRB 100707A (\tableref{tab:grb1dc}), though both GRBs show a significant improvement in the fit when a blackbody is included. Even thought the fast-cooled fits showed extreme improvement with the inclusion of a blackbody, the fits are still poor compared with the slow-cooled model (See \tableref{tab:grb2c}).}
\label{tab:grb2dc}
\end{table}
It was found that even when the
difference in C-stat between Band and Band+blackbody was greater than
the difference between synchrotron and synchrotron+blackbody fits, the
statistical significance in the goodness of fit after the addition of
the blackbody is high if not greater for the synchrotron+blackbody
model for many cases. Computational time limits kept us from checking
if the significance reached 5$\sigma$.  We now focus on the blackbody
component that is found in the synchrotron+blackbody fits. The
blackbody appears to have a separate temporal structure from the
non-thermal component, typically peaking earlier in time and decaying
before the non-thermal emission (\figureref{fig:fluxComp}).

The form of the blackbody used in this work
(\equationref{eq:blackbody}) is simplified and therefore will likely
only approximate the true form of thermal emission from a GRB
photosphere. Since the blackbody is weaker than the non-thermal
(synchrotron) component in the spectrum, the effects of a broadened
and more realistic relativistic blackbody are masked and would only
slightly affect the fit when combined with the synchrotron
model. However, in the case of GRB 100707A, the blackbody is very
bright (\figureref{fig:fluxComp:c}) and subtle changes in actual shape
of the photospheric emission become more apparent. This is reflected
in the C-stat values in \tableref{tab:grb1c}. The Band function
combined with the standard blackbody is a better fit than when using
synchrotron as the non-thermal component. In this case, the Band
function $\alpha$ is still very hard indicating that the Band function
is making up for additional flux that the blackbody is not taking into
account. To test this hypothesis, we fit the synchrotron model along
with an exponentially cutoff power-law to mimic a modified
blackbody. We found that the fits were as good as the Band function
combined with blackbody fits indicating that when the blackbody is
bright compared to the non-thermal emission a more detailed model of
the photospheric emission is needed to fit the thermal part of the
spectrum.


The HIC index for the blackbody component is expected to be 4 provided
N remains a constant in \equationref{eq:sbl}; however, nearly all the
blackbodies had a HIC index of $q\sim2$ (
\figureref{fig:kTcor}). These results confirm those of
\cite{Ryde:2001} and \cite{Ryde:2005} who fit BATSE spectra with a
combination of a blackbody and a power-law to account for the
non-thermal component. \cite{Ryde:2006} describes a toy model for the
blackbody that allows a range of temperature indices related to the
internal structure of the photosphere that may account for these
results. If it is assumed that the $\Gamma \propto t^{\zeta}$, i.e.,
that the flow has a variation in entropy then we can arrive at $L_{ph}
\propto F_{BB} \propto T^{(19\zeta -24)/3\zeta}$. the variation in
$\zeta$ could explain the deviation from \equationref{eq:sbl} observed
in our sample.
\begin{figure}[t]
\centering

    \subfigure{
      \label{fig:kTcor:a}
      \cfig{7}{kTFlux}{4.7}}
    \subfigure{
      \label{fig:kTcor:b}
      \cfig{7}{kTFluence}{4.7}}

\caption{The HIC and HFC correlations for the blackbody are separate
    from those derived from the synchrotron component. This adds more
    evidence for the presence of the component. However, the HIC for
    the blackbody is not q=4 as expected unless $\mathcal{R}$
      varies as is observed.}
\label{fig:kTcor}
\end{figure}

\begin{table}[h!]
\centering
\begin{tabular}{c| c c c c}
GRB & Flux Index & $\chi^2_{red}$ & $\Phi_0$ & $\chi^2_{red}$ \\
\hline \hline
GRB 081224A & 2.3 $\pm$ 0.3 & 1.4 & 121 $\pm$ 13 & 9 \\ 

GRB 090719A & 2.8 $\pm$ 0.4 & 2.3 & 232 $\pm$ 23 & 3 \\ 

GRB 100707A & 2.2 $\pm$ 0.1 & 17.4 & 319 $\pm$ 8 & 28 \\ 

GRB 110721A & 1.3 $\pm$ 0.2 & 1.8 & 43 $\pm$ 3 & 5 \\ 

GRB 110920A & 2.0 $\pm$ 0.1 & 0.9 & 1147 $\pm$ 21.7 & 4 \\ 



\end{tabular}
\caption{For the subset of bursts that have a strong blackbody component we compute the flux and fluence correlation for the blackbody.}
\label{tab:bbCor}
\end{table}

Another interesting quantity that can be obtained from the blackbody
is the HFC. All GRBs in our blackbody subset had blackbodies
consistent with the HFC (\figureref{fig:kTcor}). The decay
constants were all of similar value. \cite{Crider:1999} noted that
similar values of $\Phi_0$ for non-thermal components arise as a
consequence of a narrow parent distribution. A deeper investigation of
a larger sample is required to assess if the same is true for the
blackbody components.


The temporal evolution of kT for the blackbody of each burst appears
to follow a broken power-law (\figureref{fig:kTEvo}). The evolution
is fit with the function derived in \cite{Ryde:2004} where we fixed
the curvature parameter, $\delta$, to 0.15. The coarse time binning
derived from Bayesian blocks does not allow for the decay indices to
be constrained for all the bursts but a small subset are close to
$-$2/3, as expected (see \tableref{tab:bbEvo}). The temporal decay
of the blackbody is different than the power-law decay of $E_*$,
indicating a different emission component.
\begin{figure}[h!]

  \centering

  \subfigure[]{
    \label{fig:ktEvo:a}
    \cfig{7}{081224887-evo}{3}}\subfigure[]{
    \label{fig:ktEvo:b}
    \cfig{7}{090719063-evo}{3}}
\subfigure[]{
    \label{fig:ktEvo:c}
    \cfig{7}{100707032-evo}{3}}\subfigure[]{
    \label{fig:ktEvo:d}
    \cfig{7}{110721200-evo}{3}}
\subfigure[]{
    \label{fig:ktEvo:e}
    \cfig{7}{110920546-evo}{3}
}

\caption{The time evolution of kT for four of the GRBs in our
  sample. GRB110721A is shown without a fit because the coarse time
  binning used did not allow for constraining the fit
  parameters. However, in \cite{Axelsson:2012}, the evolution is
  shown to follow a broken power-law.}

   \label{fig:kTEvo}

 \end{figure}


\begin{table}[h!]
\centering
\begin{tabular}{l| c c c c}
GRB & $F_{bb}/F_{syn}$ & $1^{st}$ Decay Index & $2^{nd}$ Decay Index & $\chi^2_{red}$ \\
\hline \hline 

GRB 081224A & 0.3 & -0.6 $\pm$ 0.07 & -20 $\pm$ 243 & 0.5 \\ 

GRB 090719A & 0.4 & -0.1 $\pm$ 0.05 & -2.0 $\pm$ 0.7 & 11.3 \\ 

GRB 100707A & 0.5 & -0.4 $\pm$ 822749 & -0.8 $\pm$ 0.03 & 22.7 \\ 

%GRB110721A & 0.015 & 819.314 $\pm$ 5120492.859 & -1.432 $\pm$ 0.257 & 0.91 \\ 

GRB 110920A & 0.8 & -0.3 $\pm$ 0.03 & -0.9 $\pm$ 0.04 & 2.0\\ 


\end{tabular}
\caption{The evolution of the blackbody follows a broken power-law. However, the coarse time bins recovered by the Bayesian blocks algorithm make it difficult to constrain the decay indices.}
\label{tab:bbEvo}
\end{table}

The $\mathcal{R}$ parameter was observed to increase with time for all
the GRBs (\figureref{fig:scR}). There are breaks and
plateau in the trends that do not seem to correlate with
the breaks observed in the evolution of kT or with the flux history of
the blackbody. \cite{Ryde:2009} found that the evolution of
$\mathcal{R}$ can be quite complex but mostly follows an increasing
power-law that seems independent of the flux history even for very
complex GRBs. For those complex GRB lightcurves, it was found that
analyzing different intervals of the overlapping pulses yielded an HIC
index for the blackbody of $q\sim 4$ for each interval. In those
intervals, $\mathcal{R}$ was approximately constant indicating the
emission size of the photosphere was constant. Owing to the small
number of time bins, it is difficult to quantify the evolution of
$\mathcal{R}$ for the single pulse GRBs of this study with the coarse
time binning used, but the fact that the HIC index for the blackbodies
differs from $q\sim 4$ and that $\mathcal{R}$ increases indicates that
the evolution of the photosphere is very complex.


\begin{figure}[h!]

  \centering



\subfigure[]{
    \label{fig:scR:a}
    \cfig{7}{R_0}{3}}\subfigure[]{
    \label{fig:scR:b}
    \cfig{7}{R_1}{3}}
\subfigure[]{
    \label{fig:scR:c}
    \cfig{7}{R_2}{3}}\subfigure[]{
    \label{fig:scR:d}
    \cfig{7}{R_3}{3}}
\subfigure[]{
    \label{fig:scR:e}
    \cfig{7}{R_4}{3}
}





\caption{The evolution of the $\mathcal{R}$ parameter is increases
  with time and shows no relation to the photon flux of the blackbody
  component.}
   \label{fig:scR}

 \end{figure}

\section{Discussion }
\label{sec:discussion}
\subsection{ Importance of Fitting with Physical Models}


By using physical synchrotron emissivities in analysis of GRB data, we
have shown that the {\it Fermi} data are consistent with synchrotron
emission from electrons that have not cooled (i.e, slow-cooling
spectra) and are inconsistent with synchrotron emission from electrons
that are cooling (fast-cooling).  The method leads to some interesting
conclusions for empirical modeling.  There is a positive correlation
between hard $\alpha$ and the inability of model to fit the data, but
the low-energy index of synchrotron seems to be clustered around
$-$0.8 rather than near the asymptotic $-$2/3 found in
\cite{preece:1998}. Not only do the fits with Band $\alpha$ near
$-$0.8 lead to better fits with synchrotron, but simulated synchrotron
spectra are best fit with a Band function having $\alpha\approx -0.8$.


Previously, GRB spectra have been successfully fitted with a thermal +
non-thermal model by using a blackbody function combined with a Band
function. A thermal spectral component has indeed been shown to be
significant in several cases, foremost in GRB 100724B and in GRB
110721A \cite{Guiriec:2011,Axelsson:2012}. The Band function in these
fits is, however, not based on any physical arguments but is merely an
empirical function that has a convenient parameterization. A general
problem that arises in this type of fitting is that Band $\alpha$ and
the strength of the blackbody component give fits with degenerate
parameters. When the data are fit by a Band function alone, even if an
additional component really exists in the data at lower energies, it
may not be identified because the Band $\alpha$ can accommodate the
additional low energy flux by changing its slope.

The slow-cooling synchrotron model we use here is more restrictive
compared to the Band function. In particular, a limit to the low
energy slope and the curvature of the spectrum are predicted by the
model.  We find that the spectrum below the synchrotron $\vFv$ peak is
not always satisfactorily fit using just the synchrotron model. Except
for extreme cases such as fast and marginally fast-cooling, which
affect the width of the peak as much as the low-energy index, we find
that the the low-energy photon spectrum is actually well modeled with
a slope equal to the low-energy slope of the single particle
synchrotron emissivity. This is only possible with very low-radiative
efficiency if the standard GRB acceleration model described in Section
2.1 is considered.


In many of our fits an additional component is suggested by the
residuals, and the simulations show that this additional component is
statistically significant. Additional components can also be favored
in GRB spectra fit with the Band function, but we find that the
significance of the additional component can be greater when using
physical synchrotron emission fits than when using Band fits. Because
the Band function can accommodate the extra emission using a suitable
power-law index $\alpha$, but the synchrotron function is more
restrictive, an additional component may be more significantly
required when using synchrotron emission for a prescribed electron
distribution.



Another point in \figureref{fig:fluxComp} is that when using the
synchrotron model, the temporal evolution of the blackbody flux
exhibits well-defined pulses and a spectral evolution that is clearly
separated from the non-thermal emission. This is in contrast to the
less smooth blackbody flux variations when using the Band function as
the non-thermal process. This fact again reflects that the Band
function is less restrictive than the synchrotron function and thereby
gives rise to further scatter in the derived fluxes in the light
curves. These results suggest that:
\begin{enumerate}[(i)]
 \item the synchrotron function is a good physical model to use;
 \item the thermal component does exist; and
 \item multi-component fitting with the Band function can be misleading.
\end{enumerate}



\subsection{Alleviating Problems with Synchrotron Models}
\label{sec:ascp}
The fact that the non-thermal spectra seem to be consistent with
slow-cooled synchrotron rather than the fast-cooled synchrotron regime
places strong constraints on the emission model of GRBs.  The
low-energy spectral index of 11 bright BATSE GRBs fall between the
cooled and uncooled limits \cite{Cohen:1997}.  \cite{Ghisellini:2000}
showed that it was difficult to reconcile the implied fast-cooling
from a comparison of cooling and dynamical timescales with the many
GRB spectra that require a slow-cooling electron distribution, leading
to spectral problems for the internal shock model. In \appendixref{ch:pap2app}, we
show that a weak-cooled system requires $\lesssim 100$ G fields for
typical bright GRBs detected with {\it Fermi}, rather than 100 kG
fields, with typical electron Lorentz factors $\gamma^\prime \approx
10,000$ rather than $300$.


In our simple strong-field synchrotron model, we can neglect the
effects of Compton cooling, which can significantly alter the value of
the low-energy slope in certain parameter regimes
\cite{Daigne:2011}. This could make some spectra less consistent with
fast-cooling, but requires further study.  Klein-Nishina effects on
Compton cooling were not considered, but in the absence of extra
spectral components, either from SSC, hadronic emissions, or external
Compton processes, our synchrotron study is consistent. The need for a
slow-cooling scenario, or marginally slow-cooling system in order to
have reasonable radiative efficiency, is obtained in external shock
model calculations by choosing the $\epsilon_B$ parameter $\approx
10^{-3}$ -- 10$^{-4}$ \cite{Chiang:1999}.

%Neglecting Klein-Nishina effects, 

The fast-cooling internal-shock scenario cannot be reconciled with our
observations. Additionally, \cite{Iyyani:2013} found that for GRB
110721A, the standard slow-cooling synchrotron scenario from impulsive
energy input such as internal shocks places the non-thermal emission
region below the photosphere. This may be understood if the electrons
are highly radiative, yet without displaying a cooling spectrum.

% CD: I found the above text to be not-so-enlightening

Models with ongoing acceleration via first-order and second-order
Fermi acceleration \cite{Waxman:1995,Dermer:2001}, or magnetic
reconnection and turbulence models, including the ICMART model
\cite{zhang:2011}, have the ability to balance synchrotron cooling
with stochastic heating, or to have multiple acceleration events,
which keep $\gamma_{\rm cool}$ above $\gammaMin$, in which case the
spectrum would resemble a slow-cooled synchrotron spectrum.
%The shell collision model has been questioned \cite{Kumar:2009}.
Magnetized jet or subjet models \cite{Lazar:2009}
can extend the non-thermal emission site far above the photosphere, and 
relativistic MHD turbulence provides an alternative second-order mechanism
\cite{Lyutikov:2013}.
%which eliminates the problem mentioned above. 
In such a scenario, the electrons cool by synchrotron, but are at the
same time subject to ongoing acceleration, contrary to the low implied
value of the cooling frequency. Slow-cooling or fast-heating scenarios
explain the data much better than a fast-cooling internal-shock model,
though the latter is more radiatively efficient.



\subsection{Conclusion}
We have demonstrated that for a set of {\it Fermi} GRBs we can fit a
physical, slow-cooling synchrotron model directly to the data. Most of
the fitted spectra also require a weaker blackbody component with a
temperature that places its peak below the synchrotron $\vFv$
peak. The temporal evolution of both radiative components shows how
GRB jet properties change, and are free of some of the assumptions
required when fitting GRB spectra with empirical functions. In our
model, a disordered magnetic field is assumed, which could be shown to
be invalid from X-ray and $\gamma$-ray polarization observations,
which are yet inconclusive.  Several parameters in our model cannot be
separately constrained by the fits, namely $\gamth$, $\gammaMin$, and
B, so we focus on a highly magnetized scenario where the self-Compton
component can be neglected.

We find that the energy flux varies as the peak photon energy $E_{p}$
of the peak of the $\nu F_\nu$ spectrum according to $E_{p}^q$, with
$1.1 \lesssim q \lesssim 2.4$.  The dependence of $E_{p}$ is found to
follow the exponential-decay behavior with accumulated fluence
$\Phi(t)$ given by (\equationref{eq:hfc}), with decay constant
$\Phi_0\approx 100$ -- $700$ phts cm$^{-2}$. (see \figureref{fig:Epcor}).
For the GRBs where both synchrotron and blackbody components can be
resolved, we find that their parameters follow a separate temporal
behavior.

The temporally evolving spectra were examined in terms of fast-cooling
and slow-cooling electron distributions, considering parameters for a
highly magnetized GRB jet.  The temporal evolution of both synchrotron
and blackbody parameters imply that in the GRBs studied, a photosphere
is formed below a non-thermal emitting region found at a radius
corresponding to the characteristic internal shock scenario.  The
electrons in the non-thermal emitting region must undergo continuous
acceleration to produce an apparently slow-cooling synchrotron
spectrum, which can be provided by magnetic reconnection events or
second-order stochastic gyroresonant acceleration with MHD turulence
downstream of the forward and reverse shocks formed in shell
collisions.  If, on the other hand, the jet fluid is not strongly
magnetized, then it will be radiatively inefficient and have a strong
inverse Compton component.  The use of physical models provides
stronger constraints on jet model parameters, and in future studies we
can relax choices of electron Lorentz factors and magnetic fields by
considering leptonic Compton cascading, and ultra-high energy cosmic
rays.



% In conclusion, even though the blackbody component is significant in

% many cases when the Band function is used, the extra freedom in shape

% given by the Band function obscures the assessment of the thermal

% component.  The direct fitting of physical models thus alleviates the

% problems of the Band function in explaining the physical origin of GRB

% spectra.



\section{Inferred GRB Jet Properties}
The presence of the blackbody in addition the synchrotron component
allows for the calculation of several fundamental properties of the
GRB jet \cite{Peer:2007}. If the emission occurs at $r_s<r$ then the values of
$r_0$, $r_{\rm ph}$, and $\Gamma$ can be infer ed from the spectral fit
results. For each GRB in the sample that includes a blackbody, we calculate the time resolved values of each of these quantities. 

\begin{figure}[t]
  \centering
  \cfig{7}{grbjet.pdf}{5}
  \caption{Conceptual diagram of various GRB radii that can be derived from the spectral fits of synchrotron+blackbody.}
  \label{fig:grbjet2}
\end{figure}


\subsection{Calculating $\Gamma$, $r_0$, and $r_{\rm ph}$ }
The value of $r_0$, the base of the GRB jet, is expected to be within
an order of magnitude of the Swarzchild radius, $r_{sc}$. To calculate
the value of $r_0$, the value of $\mathcal{R}$ is examined in the
region where $r_s<r_{\rm ph}$ and the value of $\Gamma$ is derived in
terms of observed quantities. In this regime,
\begin{equation}
  \label{eq:rph2}
  r_{\rm ph}=\dover{L \sigma_{\rm T}}{8 \pi \Gamma^3 m_p c^3}
\end{equation}
and therefore the comoving temperature is easily shown to be
\begin{equation}
  \label{eq:tempcalc}
  T^{\prime}(r_{\rm ph})=\left( \dover{L}{4 \pi r_0^2 c a}  \right)^{1/4}\Gamma^{-1}\left( \dover{r_{\rm ph}}{r_s}  \right)^{-2/3}.
\end{equation}
To relate \equationref{eq:tempcalc} to the observations, assume that
$L=4\pi d_L^2YF_E$ where $Y\ge 1$ is the ratio of total fireball
energy to the energy emitted in {\gray}s, $d_L$ is the luminosity
distance, and $F_E$ is the observed total energy flux. Combining this
with \equationref{eq:scR} the bulk Lorentz factor can be derived in
terms of the observations:
\begin{equation}
  \label{eq:blf}
  \Gamma = \left[(1.06)(1+z)^2d_L\dover{YF_E \sigma_{\rm T}}{2 m_p c^3 \mathcal{R}}  \right]^{1/4}.
\end{equation}
With $\Gamma$ derived in terms of the observations, the value of $r_0$ can be found,
\begin{equation}
  \label{eq:r0}
  r_0 = \dover{4^{3/2}}{(1.48)^6(1.06)^4}\dover{d_L}{(1+z)^2}\left( \dover{F_{BB}}{Y F_E} \right)^{3/2}\mathcal{R}.
\end{equation}
Finally, using the above derived quantities and \equationref{eq:scR}, the values of $r_{ph}$ is found to be,
\begin{equation}
  \label{eq:rphObs}
  r_{\rm ph}=\mathcal{R}\dover{d_L\Gamma}{1.06 (1+z)^2}
\end{equation}


\subsection{Observed values of $\Gamma$, $r_0$, and $r_{\rm ph}$}
Using \equationref{eq:blf,eq:r0,eq:rphObs}, the values of $\Gamma$,
$r_0$, and $r_{\rm ph}$ are calculated for the sub-sample of GRBs with
blackbodies.  The results are graphed and tabulated
in \appendixref{ch:jetParms}. These values are for an assumed redshift
of z=1 and Y=1,10, and 100. It possible that Y$\gg$1, scaling the
values calculated. However, the order of magnitude of the values
warrants a discussion. In particular, the value of $r_{\rm ph}$ is of
great interest to investigating the nature of the synchrotron
emission. The relation $r_{\rm ph}<r_{\rm nt}$ should hold if the
non-thermal spectrum is optically-thin. The value of r$_{\rm nt}$ can
be approximated by examining the cooling time of the electrons
emitting synchrotron radiation (see \appendixref{ch:rnt}). Using the
measured $\Ep$ the values of r$_{\rm nt}$ are calculated. As an
example, the values of GRB 110920A are plotted here for z=1 and
Y=1. The dotted line represents the last stable orbit of a stellar
mass black hole.
\begin{figure}[t]
  \centering
  \cfig{7}{GRB110920A_RNT_Y_1}{4}
  \caption{The evolution of the various jet radii as a function of
    time. The photospheric radius is shown in red, r$_0$ is indicated in green and the maximum r$_{\rm nt}$ in the blue shaded region. The dotted line indicates the last stable orbit of a stellar
    mass blackhole.}
  \label{fig:rnt110920}
\end{figure}
The results of the entire sample are in \appendixref{ch:jetParms}.
While the values of r$_{\rm ph}$ are typically smaller than r$_{\rm
  nt}$, some values are not. The only way to alleviate this problem is
to assume that the electrons have already cooled and therefore $E_{\rm
  cool} < \Ep $. This reverses the inequality in
\equationref{eq:derv4} and allows for $r_{\rm ph}\le r_{\rm nt}$. If
this is assumed, then the only way to have a slow-cooled synchrotron
spectrum is if the electrons are re-accelerated.




 

%%% Local Variables: 
%%% mode: latex
%%% TeX-master: "../thesis"
%%% End: 
