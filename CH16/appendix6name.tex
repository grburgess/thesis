\chapter{Derivation of $L-\Ep$ Relation}
\label{ch:lep}

Assume that the $\vFv$ peak of the non-thermal spectrum is the $\Ep$ of the synchrotron which yields
\begin{eqnarray}
  \label{eq:ep1}
  \Ep \propto \Gamma B \gammaMin^2\\
L\propto \Gamma^2 B^2 \gammaMin^2.
\end{eqnarray}
If we consider that the adiabatic losses dominate over synchrotron due to the observed slow-cooling spectrum, then the evolution of $\gammaMin$ with radius is
\begin{equation}
  \label{eq:gr}
  \gammaMin \propto R^{-1}.
\end{equation}
The essential assumption to reproduce the observed $L-\Ep$ curve in the data is that the magnetic field is frozen into the flow, i.e.,
\begin{eqnarray}
  \label{eq:fluxfrz}
  BR^2\propto constant\;\Leftrightarrow\;B\propto R^{-2}.
\end{eqnarray}
Substituting \equationref{eq:gr,eq:fluxfrz} into \equationref{eq:ep1}
\begin{eqnarray}
  \label{eq:ep1}
  \Ep \propto \Gamma R^{-4}\\
L\propto \Gamma^2 R{-6}
\end{eqnarray}
and eliminating $R$ between them yields $L \propto \Gamma^{}
\Ep^{3/2}$ or
\begin{equation}
  \label{eq:Lep}
  L \propto \Ep^{3/2}
\end{equation}
as desired.


%%% Local Variables: 
%%% mode: latex
%%% TeX-master: "../thesis"
%%% End: 
