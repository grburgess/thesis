\chapter{A New Correlation Between Thermal and Non-Thermal Emission in Gamma-Ray Bursts} 
\label{ch:cor}
\begin{chapterquote}{Mark Twain}
Such is professional jealousy;\\ a scientist will never show any kindness\\ for a theory which he did not start himself.
\end{chapterquote}

\section{Introduction}

Gamma-ray bursts (GRBs) are believed to be the death of super massive
stars or the coalescence of two compact objects resulting in an
explosive, relativistic outflow. The energy of the outflow is
dissipated into the particles of the plasma which then radiate this
energy in the form of $\gamma$-rays. While numerous theories exist to
explain GRB formation and emission, a key question is how the energy
of the outflow is distributed, i.e., whether or not the energy is in
the magnetic field or as a baryon kinetic energy, and how this energy
distribution evolves with time. This question is particularly hard to
address because of the unknown progenitor source of these events and
the fact that prompt GRB observations consist of time-resolved
$\gamma$-ray spectra that have been historically described with the
empirical Band function
\cite{band:1993,Kaneko:2006,Goldstein:2012}. The standard
interpretation of the prompt gamma-ray emission of GRBs requires a
highly-relativistic, jetted fireball that has become
optically-thin. While the emission from this fireball is expected to
be thermal \cite{Goodman:1986,Paczynski:1986}, observations of the
prompt emission over the past two decades have been found to be highly
non-thermal
\cite{Mazets:1981,Fenimore:1982,matz,Kaneko:2006,Goldstein:2012}. Recent
analysis of new data collected by the {\it Fermi} Gamma-ray Space
Telescope found that the emission spectra contain at least two
components, the original non-thermal component, and sub-dominant
thermal component fit with a blackbody
\cite{Guiriec:2010,Axelsson:2012}. This component has been
speculatively linked to the fireball photosphere of the GRB. The
temporal evolution of the thermal component has been shown to follow
what would be expected for an expanding photosphere and is separate
from the evolution of the non-thermal component.

In this work we fit the non-thermal component with a synchrotron
photon model and the thermal component with a blackbody developed in
\cite{Burgess:2012,Burgess:2013}. We find that the peak energies of
the two components are highly correlated across all the GRBs in our
sample. Such a correlation points to a scaling parameter common among
GRBs which we will show is related to the magnetic and kinetic energy
content of the GRB jet.

\section{Observations}
In this work, seven long and bright {\it Fermi} GRBs were selected
that consist of single pulses in their broad-band $\gamma$-ray time
histories (see \tableref{tab:epktcor}). While this sample is limited
by the number of bright, single-pulsed GRBs in the {\it Fermi} data
set, the fine-time resolved spectroscopy used to analyze these GRBs
allows for a detailed understanding of the evolution of the thermal
and non-thermal components. All the GRBs contained significant,
sub-dominate thermal components in their $\gamma$-ray spectra as shown
in \cite{Burgess:2013,preece:2013}. Single pulse GRBs have been shown
to have simple spectral evolution and provide the cleanest signal for
fitting physical models directly to the detector count data
\cite{Burgess:2013,Burgess:2012,Ryde:2009}.

\begin{figure}[t]
  \centering
  \cfig{9}{spectrum.pdf}{4.5}
  \caption{The evolving $\vFv$ spectrum of GRB 130427A. The blackbody component evolves from yellow to red and the synchrotron component evolves from cyan to blue.}
  \label{fig:newSpec}
\end{figure}

To perform time-resolved spectral analysis, each GRB lightcurve was
divided into time bins using a Bayesian blocks algorithm
\cite{Scargle:2013} and each bin's spectrum was fit with a
synchrotron+blackbody photon model using the GRB analysis software
RMFITver4.3 (see \figureref{fig:newSpec}). The synchrotron model consists of a
shock accelerated electron distribution containing a relativistic
Maxwellian and a high-energy power-law tail that is convolved with the
standard synchrotron kernel
\cite{Burgess:2013,Burgess:2012,rybicki:1979}. The spectra were all
well fit by the synchrotron model with fits that were as good if not
better than those made with the Band function.



\section{A Correlation Between Spectral Components}
A strong, positive correlation exists between the peak energies of the
blackbody and synchrotron components (see \figureref{fig:epktcor}). The correlation
was tested with the Spearman's Rank-Order Correlation test, obtaining
a correlation coefficient of $\rho=0.83$ and a p-value of 4.35$\times
10^{-20}$. Tests were carried out to check for a fitting correlation
between the two component peaks and it was found that only a slight,
negative fitting correlation exists.


\begin{figure}[t]
  \centering
  \cfig{9}{correlation}{6}
  \caption{The correlation between $\Ep$ and kT.}
  \label{fig:epktcor}
\end{figure}

\begin{table}[t]

\centering
\begin{tabular}{l || c | c }
 
  &            PL Index  &    F$_{BB}$/F$_{tot}$\\
\hline\hline\\
  GRB 081224A  &	$1.01   \pm	0.14	 $ &      0.29\\
  GRB 090719A   &	$2.33   \pm	0.27	 $ &      0.27\\
  GRB 100707A  &	$1.77	\pm	0.07	 $ &      0.33\\
  GRB 110721A &	$1.24	\pm	0.11	 $ &      0.01\\
  GRB 110920A                    &	$1.97	\pm	0.11	 $ &      0.39\\
  GRB 130427A & 	$1.02	\pm	0.05	 $ &	  0.22\\
  


\end{tabular}
\caption{Correlation fit values and flux ratios for each GRB.}
\label{tab:epktcor}
\end{table}

Of these GRBs, the redshift is known only for GRB 130427A. Since both
values in Figure 2 are energies, shifting to the rest frame would
shift both values of each GRB equally, moving the GRBs along the
correlation but not changing the slope of the correlation. It is
possible that the correlation is tighter in the rest frame but a
greater sample of single pulse GRBs with redshifts is required to
check this assumption. The correlation (\figureref{fig:epktcor}), fit
with a power-law, has a slope $1.26\pm0.03$. We also fit power laws to
the $E_{\rm p}$, kT pairs of the individual GRBs and find dependencies
ranging from $T^{\sim 1}$ to $T^{\sim 2}$ (see
\tableref{tab:epktcor}). The ratio of the blackbody flux to the total
flux was computed for each burst. No correlation between the
correlation index and flux ratio was found.

\section{Interpretation}

A simple model which can interpret these observations is a dissipative
photosphere model \cite{Giannios+07photspec} with general
dynamics. The jet dynamics are parametrized by the dependence of the
Lorentz factor on the radius as $\Gamma\propto R^\mu$ until the jet
reaches its saturation Lorentz factor, $\eta=L/\dot{M}c^2$, where L is
the luminosity and $\dot{M}$ the mass outflow rate. This will be
approximately the jet's Lorentz factor until it reaches the
deceleration radius. For magnetically dominated jets $\mu\approx 1/3$
\cite{Drenkhahn02}, in the baryonic case $\mu\approx1$. The values in
between correspond to a mix of these components \cite{Veres+12fit},
and can be further modified by e.g. the topology of the magnetic
field.

The photosphere will occur where the optical depth of the jet is
unity. This can happen above or below the saturation radius
\cite{Meszaros+93gasdyn}, and accordingly we have two cases.

Closely above the photosphere, instabilities in the flow or magnetic
field line reconnections can lead to mildly relativistic shocks and
accelerate leptons which in turn emit synchrotron radiation
\cite{Meszaros+11gevmag,McKinney+11switch}. To interpret the observed
correlation we restrict the discussion to the dependence of the peak
on the main physical parameters such as the luminosity ($L$), the
coasting Lorentz factor ($\eta$) or the launching radius ($r_0$).

The peak of the synchrotron component is proportional to the Lorentz
factor close to the photosphere, the magnetic field strength and the
random Lorentz factor of the electrons. The magnetic field strength is
usually taken as bearing some fraction $\epsilon_B\sim {\rm const}$ of
the total kinetic energy. The temperature of the photosphere depends
on the luminosity and the photospheric radius.  We derive the
photospheric radius and express other quantities at this radius and
get:


\begin{equation}
E_p \propto \left\{
\begin{array}{ll}
  L^{\frac{3\mu-1}{4\mu+2}} \eta^{-\frac{3\mu-1}{4\mu+2}}
  r_0^{\frac{-5\mu}{4\mu+2}} 		&	{\rm if~photosphere~in~ acceleration~ phase}\\ 
  L^{-1/2} \eta^{3} %r_0^{0}
   	&	{\rm if~photosphere~in~  coasting~ phase} 
\end{array}
\right.
\label{eq:ebreak}
\end{equation}


\begin{equation}
T \propto \left\{
\begin{array}{ll}
  L^{\frac{14\mu-5}{12(2\mu+1)}} \eta^{\frac{2-2\mu}{6\mu+3}}
  r_0^{-\frac{10\mu-1}{6(2\mu+1)}} 	&	{\rm if~photosphere~ in~acceleration~ phase}\\ 
  L^{-5/12} \eta^{8/3} r_0^{1/6}  	&	{\rm if~photosphere~ in~ coasting~ phase} 
\end{array}
\right.
\label{eq:temp}
\end{equation}


It is hard to assess which quantity drives the above dependencies for
either the accelerating or the coasting phase photosphere, or the fact
that it is a single quantity.  One natural assumption is to consider
the luminosity as the main reason for the change in $E_p$ and $T$ as
the flux changes in these bursts.

\begin{itemize}
\item In the accelerating photosphere case, considering the
  appropriate powers of $L$ we get $E_p \propto
  T^{\frac{6(3\mu-1)}{14\mu-5}}$. The exponent is singular at
  $\mu\approx0.36$, but for values up to $\mu < 0.6$ (these are
  the values of $\mu$ for which the photosphere will occur in the
  acceleration phase) we are able to explain exponents from 2 down to
  1.4.

\item For a coasting photosphere $E_p \propto T^{1.2}$. This is
  observed in some bursts. We get similar results if we vary $\eta$.

\end{itemize}

The analysis of the bursts in the framework of this model can
suggestively identify whether the photosphere is in the acceleration
or coasting phase, which in turn can be translated to the composition
of the jet.  We find that for exponents close to 2 the jet dynamics is
dominated by the magnetic field.  Exponents close to 1 are suggestive
of baryonic jets. Our sample spans the values of this model indicating
that the energy content of these GRBs varies from event to event. This
simple method allows for a direct way to assess fundamental properties
of GRBs that have not been previously available through observations.





%%% Local Variables: 
%%% mode: latex
%%% TeX-master: "../thesis"
%%% End: 
