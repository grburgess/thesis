\chapter{GRB Spectral Analysis}
\label{ch:analysis}
\begin{chapterquote}{Christopher Hitchens}
That which can be asserted without\\ evidence,
 can be dismissed without\\ evidence.
\end{chapterquote}

The analysis of {\it Fermi} GRB spectra via physical models will be
performed using the aforementioned RMFIT. The description of the
analysis will be broken into several sections to investigate different
aspects of GRB physics. In \chapterref{ch:grb090820A}, I will test the
validity of physical model fitting and preliminary implications of the
spectral fits. In \chapterref{ch:pap2}, group properties from a large
sample of GRBs will be analyzed to obtain a better understanding of
the emission mechanisms and assess the structure of the GRB
jet. Finally, \chapterref{ch:130427A,ch:cor} contain an analysis of
spectral properties to build a deeper understanding of the GRB jet
properties and evolution. The process of spectral analysis for all of
these test is similar and will therefore be described in this chapter.




\section{Sample Selection}
The study of physical models in GRB spectra places severe limits on
the sample of GRBs that can currently be studied. They must be bright
so that the time-resolved spectra have enough counts to enable the
fitting engine to constrain the model parameters. In addition, the
physical models that are implemented here are derived for single zone
emission, i.e., if two overlapping pulses are fit with one of these
models, then it is highly unlikely the results will be valid because
the photon spectra may contain emission from two event zones. For this
reason only GRBs with single or non-overlapping pulses are
chosen. There are very few GRBs in the {\it Fermi} dataset that have
either property. Using a list of the brightest {\it Fermi} GRBs, a
single-pulsed subset was selected. These pulses were categorized as
single-pulsed by having a monotonic decrease in their bolometric count
rate after the peak intensity of the pulse.

The GRBs in the this study are:
\begin{itemize}
\item GRB 081110A
\item GRB 081224B
\item GRB 090719A
\item GRB 090809A
\item GRB 090820A
\item GRB 100707A
\item GRB 110721A
\item GRB 110920A
\item GRB 130427A
\end{itemize}
The GRBs will all be analyzed in a similar way with two exceptions;
GRB 090820A will serve as a preliminary diagnostic to test the ability
of fitting physical models to the data and GRB 130427A, due to its
brightness and measured redshift will be analyzed in fine detail to
explore the origins of the observed HIC.

\subsection{Time Binning}
\label{sec:tbin}
The selection of time bins for time-resolved spectral analysis has no
unique solution. Ideally, time bins should be as small as possible to
capture the nuances of the evolution of the spectrum in as much detail
as possible. However, the selection of very short time bins decreases
the number of counts in the signal thereby reducing the ability to
constrain fit parameters. A balance between resolution and signal
strength must be arbitrarily made by the observer. There are three
methods for determining time bins in GRB spectral analysis:
\begin{itemize}
\item constant time bin width
\item constant signal-to-noise ratio (S/N) binning
\item Bayesian blocks.
\end{itemize}
Constant time bin width binning is simply that bins are chosen with
the same $\Delta t$ throughout the burst. This method is systematic
and unambiguous though no guarantee can be made about the S/N being
large enough to accommodate spectral analysis. Additionally, the bins
can arbitrarily split or combine physical changes in the
spectrum. Constant S/N binning combines spectra by summing signal and
background counts starting at the beginning of a bin until a desired
S/N is achieved. This method guarantees that the desired number of
signal counts are present in each bin for spectral analysis. However,
the method can arbitrarily sum together significant changes in the
spectral evolution similar to constant time bin width binning. This
presents a significant problem
\begin{figure}[h]
  \centering
  \subfigure{\cfig{5}{constwidth.pdf}{3.2}}\subfigure{\cfig{5}{snbin.pdf}{3.2}}
\subfigure{\cfig{5}{bblc.pdf}{3.2}}
\caption{Example time binnings demonstrating the differences between
  constant time width (\emph{left}), S/N (\emph{right}), and Bayesian
  blocks (\emph{middle}).}
  \label{fig:binmeth}
\end{figure}
for the study of physical emission models because it is important to
map changes in the lightcurve to physical changes in the GRB.  For
this work, the method of time binning chosen is Bayesian blocks
\cite{Scargle:2013}. The Bayesian blocks algorithm selects time bins
by looking for significant changes in count intensity. The
significance level set is determined by a chosen Bayesian prior
($n_{\rm cp}$) that reflects the observer's knowledge about the number
of pulses in the count rate. It was found that setting $n_{\rm cp}=8$
was suitable for the study of single-pulsed GRBs
\cite{Scargle:2013}. The resulting time bins are guaranteed to have a
uniform Poisson count rate with any subdivision of the bin up to
the selected significance level. A key assumption in relying on
Bayesian blocks is that the changes in the Poisson count rate of the
source reflect physical changes in the GRB jet.



\subsection{Source Selection}
A benefit of Bayesian block binning is that a temporal source region
can easily identified. The local background is basically a constant
rate and can be identified as long time bin before and after the
pulse.
\begin{figure}[h]
  \centering
  \cfig{5}{bblcsource.pdf}{4}
  \caption{An example of background and source identification in a
    Bayesian block binned lightcurve.}
  \label{fig:bbselection}
\end{figure}
With the source region identified, the background bins are selected
and a polynomial in time is fit to the background lightcurve. From
this point, time-resolved spectroscopy can be performed on the source
regions.


\section{Spectral Fitting}
\subsection{Fit Statistic}
Time-resolved spectral analysis of {\it Fermi} data occurs in the
low-count regime. This is because the photon model is evaluated in
each energy bin of the data. The classic \chisq statistic will is not
applicable in this regime and therefore a likelihood statistic is
required. Counts data are Poisson distributed and therefore the choice
of a likelihood with a Poisson estimator is required. Therefore, the
C-stat statistic is a useful choice (see \cite{Arnaud:2011}). The
benefit of the C-stat statistic is that it asymptotes to a \chisq
distribution in the limit of a large number of counts which provides a
suggestive goodness of fit. The drawback of a likelihood statistic is
that model comparison is not valid when the models are not nested. Two
models are said to be nested if they contain the same functional form
and parameters but one has at least one additional term e.g.,
\begin{equation}
\begin{array}{l}
\displaystyle y_1=Ax^{B} \\
\displaystyle y_2=Ax^{B}\exp(C).
\end{array}
\label{eq:nest}
\end{equation}
These two models are considered nested and the significance of the
more complex model, $y_2$, can be assessed via a likelihood ratio test
(LRT). For the purposes of this study, direct model comparison will
not be possible via the LRT because none of the models that will be
tested are nested. Even in the case of a linear addition of a second
spectral component like the blackbody, the LRT is not valid. This is
because when a model of the form
\begin{equation}
\begin{array}{l}
\displaystyle y_1=Ax \\
\displaystyle y_2=Ax+B,\;0\le B
\end{array}
\label{eq:nestlin}
\end{equation}
exists with the additional linear parameter is bounded by zero,
\cite{Protassov:2002} has shown that the LRT does not yield a valid
significance. For these reasons, testing the significance and goodness
of fit of models requires monte carlo simulations to sample the
probability distributions (see \sectionref{sec:sigtest}).

\subsection{Forward Folding}
\label{sec:ff}
Spectral analysis of GBM data is possible through a process called
forward-folding. The detected {\gray} photons are converted into
an electrical signal by the detectors' crystals and associated
electronics. The pulse height of the electrical signal is recorded as
a channel energy and is related to the energy of the incident
\gray . The association of the channel energy to the original
photon energy is not direct. Several radiative process can occur in
the crystal which vary the incident photon-channel
energy association non-linearly. This makes directly determining the
initial energy of a given photon impossible. The non-linear channel to
energy relationship is expressed through a response matrix.
\begin{figure}[h]
  \centering
  \cfig{5}{rspMat.pdf}{4.}
  \caption{An NaI response matrix from
    GBM.}
\end{figure}
This matrix is created from monte carlo simulations of the GBM
detectors exposed to different simulated photon energies. The
conversion of the input energy into a detector channel energy can be
represented as a matrix ($D_{ij}$). The input photon model
($f_i$) and background spectrum (${b_i}$) are related to the
detected count spectrum (${c_i}$) via a linear system.
\begin{equation}
\label{eqn:folding}
c_i\;=\;D_{ij}f_j+b_i
\end{equation}
In general, $D_{ij}$ is highly singular and therefore the equation is
non-invertible to obtain $f_i$. The process of forward folding
involves starting with a proposal spectrum $f^{\prime}_i$ and solving
for $c^{\prime}_i$. This is compared to the actual detected count
rate. The photon model will have adjustable parameters such as a
photon index that can be varied. This process is iterated, varying the
spectral parameters of the model until $c^{\prime}_i$ agrees with
$c_i$ via the Levenberg-Marquardt algorithm until C-stat is
minimized. Once the routine is complete, the $f_i$ for describing the
data is obtained along with statistical constraints on the model
parameters.

\section{Significance Testing}
\label{sec:sigtest}
The historical, single-component view of GRB spectra has been shown to
be inadequate in more recent studies. The presence of the blackbody in
several GRBs has modified this canonical view. Many spectra appear to
be better fit by the addition of the blackbody to the non-thermal
(Band or physical) photon model. However, this addition must be shown
to be a \emph{significant} improvement to the fit. The LRT with C-stat
is not viable for testing significance; therefore, monte carlo
simulations are required to assess the probability distributions. The
specific tests will be discussed when required
in \sectionref{sec:results:bb}, but the general details will be shown
here.

When a time bin is found to contain a blackbody in addition to its
non-thermal component by having a lower C-stat when fit with a
non-thermal+blackbody model, a simulation file is created. This
simulation file contains a set number of simulated spectra. These
spectra consist only of the non-thermal model with the fit parameters
fixed to the values derived from the data. The bins all contain a
different random background with a rate sampled from Poisson
distribution with a mean taken from the actual data. The bins in each
file are fit with both the simple model corresponding to the null
hypothesis ($H_0$) and the non-thermal+blackbody corresponding to the
proposal hypothesis ($H_1$). Each series of fits is saved to a file
that includes the fit parameters and the C-stat from each fit. To
assess the distribution of \dcstat the \dcstat for each simulated fit
is calculated. The significance of the blackbody ($\sigma_{\rm BB}$)
component is then calculated by determining the fraction of the
simulated distribution that lies below the value of \dcstat.
\begin{equation}
  \label{eq:sigtest}
  \sigma_{\rm BB}=\dover{\sum_i p(i)\ge {\rm data}\;\Delta_{\rm cstat} }{\sum_i p(i)< {\rm data}\;\Delta_{\rm cstat}}
\end{equation}
where $p(i)$ is the simulated \dcstat distribution. From this value,
the significance of the blackbody component can be ascertained. The
canonical assignment of $\sigma$ values (corresponding to the standard
deviation of the normal distribution) from the p-value is not always
valid using the method above. The distributions of \dcstat are not
necessarily symmetric. Therefore, the quoting of the p-value is more
appropriate. The monte carlo simulations to assess significance are
extremely CPU intensive when using physical models. The significance
level is often limited by the number of simulations that can be run.

\section{Summary of Fitting Procedure}
For each GRB in the sample a standard analysis procedure is applied
except for the caveats in the analysis of GRB 080920A and GRB
130427A. The procedure for analysis is as follows:
\begin{enumerate}[(i)]
\item a custom Bayesian blocks algorithm is applied to the TTE file of
  brightest NaI and BGO detector and the time bins are mapped to all
  detectors used
\item a background fit is applied to all detectors by selecting the
  Bayesian block plateau before and after the pulse
\item each time bin is fit with the Band, Band+blackbody, synchrotron,
  synchrotron+blackbody, fast-cooled synchrotron, and fast-cooled
  synchrotron+blackbody
\item if the fit is successful, the fit parameters are recorded in an
  FITS file
\item the FITS file is fed into a pipeline that calculates the
  following parameters from the fit parameters:
  \begin{itemize}
  \item the bolometric photon and energy flux of each spectral component
  \item $r_{\rm ph}$, $r_o$, $r_{\rm nt}$, and $\Gamma$
  \item HIC and HFC for each component.
  \end{itemize}

\end{enumerate}
Fits that fail in RMFIT either due to the inability to constrain the
fit parameters or due to instabilities in converging to a minimum are
not recorded. This is the case for most fits using the fast-cooled
synchrotron model.
%%% Local Variables: 
%%% mode: latex
%%% TeX-master: "../thesis"
%%% End: 
%
% Move this... but where?
%



