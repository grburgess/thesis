\chapter{GRB 090820A: A Case Study}
\label{ch:grb090820A}
\begin{chapterquote}{Henry Rollins}
You need a little bit of insanity\\
to do great things
\end{chapterquote}

The following is adapted from \cite{Burgess:2012}. GRB 090820A serves
a case study for the testing of the slow-cooled synchrotron photon
model on a single pulse GRB. This analysis was performed to test
physical modeling on a GBM GRB before a systematic analysis was
performed on a larger sample. The analysis here varies from
standardized analysis that will was performed the larger sample in
several ways:
\begin{itemize}
\item time binning was made by the S/N method to ensure constrained fits
\item only select time bins were analyzed to examine the evolution of
  fit parameters
\item the slow-cooled synchrotron model parameter $\epsilon$ was left free during the fits
\item no inferred properties were calculated from the spectral fits.
\end{itemize}




% \shorttitle{Constraints on the Synchrotron Shock Model}
% \shortauthors{J.~Michael Burgess et al.}




% \title{Constraints on the Synchrotron Shock Model for the Fermi GBM Gamma-Ray Burst 090820A}


% %%Authors            
              
% \author{J. Michael Burgess,\altaffilmark{1}, 
% Robert~D.~Preece\altaffilmark{1},
% Matthew~G. Baring,\altaffilmark{2},
% Michael~S.~Briggs\altaffilmark{1}, 
% Valerie Connaughton,\altaffilmark{1},
% Sylvain Guiriec,\altaffilmark{1} 
% William S.~Paciesas\altaffilmark{1}, 
% Charles A.~Meegan\altaffilmark{3}, 
% P.~N. Bhat\altaffilmark{1}, 
% Elisabetta Bissaldi\altaffilmark{4}, 
% Vandiver Chaplin\altaffilmark{1}, 
% Roland Diehl\altaffilmark{4}, 
% Gerald~J.~Fishman\altaffilmark{5}, 
% Gerard Fitzpatrick\altaffilmark{6}, 
% Suzanne Foley\altaffilmark{6},
% Melissa Gibby\altaffilmark{7}, 
% Misty Giles\altaffilmark{7},
% Adam Goldstein\altaffilmark{1}, 
% Jochen Greiner\altaffilmark{4}, 
% David Gruber\altaffilmark{4}, 
% Alexander J.~van der Horst\altaffilmark{3}, 
% Andreas von Kienlin\altaffilmark{4}, 
% Marc Kippen\altaffilmark{8}, 
% Chryssa Kouveliotou\altaffilmark{5}, 
% Sheila McBreen\altaffilmark{6}, 
% Arne Rau\altaffilmark{4}, 
% Dave Tierney\altaffilmark{6}, and 
% Colleen~Wilson-Hodge\altaffilmark{5}}              
              
% %%Affiliations
% \altaffiltext{1}{University of Alabama in Huntsville, 320 Sparkman Drive, Huntsville, AL 35899, USA}
% \altaffiltext{2}{Department of Physics and Astronomy, MS 108,
%       Rice University, Houston, TX 77251, U.S.A.  {\it Email: baring@rice.edu}}
% \altaffiltext{3}{Universities Space Research Association, 320 Sparkman Drive, Huntsville, AL 35899, USA}
% \altaffiltext{4}{Max-Planck-Institut f$\rm \ddot{u}$r extraterrestrische Physik (Giessenbachstrasse 1, 85748 Garching, Germany)}
% \altaffiltext{5}{Space Science Office, VP62, NASA/Marshall Space Flight Center, Huntsville, AL 35812, USA}
% \altaffiltext{6}{School of Physics, University College Dublin, Belfield, Stillorgan Road, Dublin 4, Ireland}
% \altaffiltext{7}{Jacobs Technology}
% \altaffiltext{8}{Los Alamos National Laboratory, PO Box 1663, Los Alamos, NM 87545, USA}              
% %\altaffiltext{1}{Physics Department, The University of Alabama in Huntsville,
% %              Huntsville, AL 35809, U.S.A.  {\it Email: james.burgess@nasa.gov, rdp}}


% \email{james.m.burgess@nasa.gov}



%%Observations
\section{Observations}
 \label{sec:observe}
\begin{centering}

Authors

J. Michael Burgess, Robert~D.~Preece, Matthew~G. Baring,
Michael~S.~Briggs, Valerie Connaughton, Sylvain Guiriec, William
S.~Paciesas, Charles A.~Meegan, P.~N. Bhat, Elisabetta Bissaldi,
Vandiver Chaplin, Roland Diehl, Gerald~J.~Fishman, Gerard Fitzpatrick,
Suzanne Foley, Melissa Gibby, Misty Giles, Adam Goldstein, Jochen
Greiner, David Gruber, Alexander J.~van der Horst, Andreas von
Kienlin, Marc Kippen, Chryssa Kouveliotou, Sheila McBreen, Arne Rau,
Dave Tierney, Colleen~Wilson-Hodge

\end{centering}
\hspace{2 cm}


 On 20 August 2009, at T$_\mathrm{0}$=00:38:16.19 UT, GBM triggered on
 the very bright GRB~090820A~\cite{grb090820A}. This GRB also
 triggered Coronas Photon-RT-2~\cite{GCN-CORONAS}. The burst location
 was initially not in the FOV of the LAT onboard Fermi but was bright
 enough to result in an ARR. However, Earth avoidance constraints
 prevented such a maneuver until 3100 sec after the burst trigger and
 the burst was not detected at higher energies by the LAT.  The most
 precise position for the direction of the burst comes from the GBM
 trigger data which localizes the burst to a patch of sky centered on
 RA = 87.7 degree and Dec = 27.0 degree (J2000) with a 4 degree error,
 statistical and systematic. The current best model for systematic
 errors is 2.8 degrees with 70\% weight and 8.4 degrees with 30\%
 weight \cite{briggsann}. We verified that our analysis does not
 change significantly using instrument response functions for assumed
 source locations throughout this region of uncertainty.
 \figureref{fig:figure1} shows the light curve of GRB~090820A as seen
 by GBM,
\begin{figure}[h]
\cfig{6}{figure1}{5}
\caption{ Light curve of GRB~{\it
    090820}A as observed by GBM. The two panels show the count rate in
  the two NAI detectors (top) and BGO (bottom). The dashed lines
  indicate the time intervals (a, b, c, d) used for the time-resolved
  analysis (see Figure 3 and Table 1). It is clear that the burst
  consists of two main peaks and that this burst is very bright in the
  BGO detectors.}
\label{fig:figure1}
\end{figure}
from 8 to 200 keV in the NaI detectors (top) and from 200 keV to 40
MeV in the BGO detector (bottom). GBM triggered on a weak precursor
which we do not include in the analysis. The main light curve begins
at T$_{0}$ + 28.1s. The main structure of the light curve consists of
a fast rising pulse with an exponential decay lasting until T$_{0}$+60
s. A second, less intense, peak beginning at T$_{0}$+30 s is
superimposed on the main peak.


With such a high intensity and simple structure, this GRB allows for
detailed time-resolved spectroscopy. Because this burst is intense,
calibration issues make the Iodine K-edge (33 keV) prominent in the
count spectra owing to small statistical uncertainties, and we remove
energy channels contributing to this feature from our spectral
fits. In addition, an effective area correction is applied between
each of the NaI detectors and the BGO 0 during the fit process. This
correction of $\approx$~23\% is used to account for possible
imperfections in the response models of the two detector types.
We simultaneously fit the spectral data of the NaI detectors with a
source angle less than 60 degrees (NaI 1 and 5) and the data from the
brightest BGO detector (BGO~0) using the analysis package RMFIT.


We perform a fit to the integrated spectrum and find that it is best
represented by synchrotron emission from thermal and power-law
distributed electrons with an additional blackbody component
characterized by a kT~$\approx$~42 keV (C-Stat/DOF = 558/353). The
\teq{\nu F_{\nu}} spectrum is displayed in \figureref{fig:figure2}
and the best-fit values in \tableref{tab:table1}. We also performed
a fit using the Band function (C-Stat/DOF = 593/355).
\begin{figure}[h]
\cfig{6}{figure2}{5}
\caption{The integrated
  spectrum of GRB~090820A. We are able to resolve three
  components, thermal synchrotron, power-law synchrotron, and a
  blackbody. Energy channels near the NaI K-edge are omitted. The
  deviations in the fit residuals are the due to systematics in the
  detector response resulting from the high count rate and spectral
  hardness of this burst. However, deviations are never greater than
  4$\sigma$ and do not significantly impact the values of the best fit
  parameters. The multiple curves near the peak of the spectrum are an
  artifact of the effective-area correction applied to each detector
  and not related to the different fitted models.}
\label{fig:figure2}
\end{figure}



\begin{figure}
\cfig{6}{figure3}{4}
\label{fig:figure3}
\caption{The electron distribution corresponding to the integrated spectrum. The non-physical jump in the amplitude between the Maxwellian and the power-law distribution (parametrized by $\epsilon$) at $\gammaMin$ is clearly seen.}
\end{figure}



We find in concordance with BB04 that emission from power-law
synchrotron dwarfs the emission from thermal synchrotron by at least 3
orders of magnitude. The value of $\gammaMin$ is fixed to 3, the
choice adopted by BB04: it is a value that accommodates distributions
typically determined by shock acceleration simulations.  When fitting
the power-law synchrotron component we have to fix the value of the
power-law index to its best fit value to remove a correlation between
the amplitude and the index; this does not change the fit statistic
but does mean that the amplitudes obtained are valid only for that
index. The inferred electron distribution from this fit is shown in
\figureref{fig:figure3}. We note that the inability to simultaneously
constrain the power-law index and amplitude of the synchrotron
function may be solved in future studies by including joint fits with
LAT data, whenever available.
% \end{deluxetable}



\begin{table}
\scriptsize
\centering
\begin{tabular}{c | c c c c c c c c}
Time interval&$n_{0}$ {$(\gamma s^{-1}cm^{-2}keV^{-1})$} & $\epsilon$ & $\mathcal{E}_c$ (keV) & $\delta$ & $\gammaMin$ & $A_{BB}$ $(\gamma s^{-1}cm^{-2}keV^{-1})$ & $kT$ (keV) \\
\hline \hline
Time integrated&$0.3437_{-0.065}^{+0.204}$ & $871_{-234}^{+254}$ & $10.39_{-0.245}^{+0.254}$ & $4.9$ & $3.0$ & $2.08_{-0.208}^{+0.367}\times10^{-5}$ & $42.27_{-1.35}^{+1.49}$ \\

a&$ 2.378_{-0.176}^{+0.189}$ & $-$ &   $8.351_{-0.93}^{+1.08}$ & $-$ & $-$ & $-$ & $-$ \\

b&$ 859_{-89.1}^{+94.0}$ & $-$ & $14.24_{-0.776}^{+0.848} $ & $ 4.4 $ & $3.0$ & $1.774_{-0.356}^{+0.410}\times10^{-4}$ & $35.32_{-1.77}^{+1.99}$ \\

c& $1.901_{-0.093}^{+0.094}\times10^{4}$ & $-$ & $15.22_{-0.399}^{+0.411}$ & $ 5.9$ & $3.0$ & $1.818_{-0.344}^{+0.400}\times10^{-4}$ & $ 38.7_{-1.92}^{+2.13}$ \\

d&$ 2.196_{-0.466}^{+0.720}$ & $-$ & $4.035_{-0.715}^{+0.689}$ & $-$ & $-$ & $8.383_{-3.18}^{+4.89}\times10^{-5}$ & $ 28.40_{-3.59}^{+3.73}$ \\
%\hline

\end{tabular}

\caption{The fit parameters for the
  time-integrated (first row) and time-resolved spectra. The fit
  parameters for the blackbody component are its amplitude
  ($A_{BB}$) and energy ($kT$). The fit parameters for the
  non-thermal components are described in
  \sectionref{sec:physmod:rmfit}. The break energy
  $\mathcal{E}_b\equiv \mathcal{E}_c(\gamma \to \eta\gamth )$
  corresponds to employing the substitution $\gamma \to \eta\gamth
  $ in \equationref{eq:elec_dist}.  Note that the ratio of the
  amplitudes is not equal to the ratio of the fluxes.}

\label{tab:table1}
\end{table}




%TRS
For the time resolved analysis we fit four bins labeled {\textbf a},
{\textbf b}, {\textbf c} and {\textbf d} as shown in
\figureref{fig:figure1} with the various synchrotron models. The
corresponding electron distributions inferred from these fits are
displayed in \figureref{fig:figure5}. We also fit the Band function to
each spectrum to show that in nearly all cases the physical models can
fit the data as well as the Band function. We chose the time binning
by finding a balance between high signal-to-noise and evolution of the
spectral shape so that we can identify the time evolution of each
component throughout the burst. Where possible, we fit all three
components simultaneously. Due to the similarity in the spectral
shapes of the low energy portions of the thermal synchrotron and
power-law synchrotron components it is not always possible to
constrain all of the fit parameters especially when one component is
much stronger than the other. Therefore, when one component is
dominant we include only that component in the fit. The ability to fit
both components in the time integrated fit is most likely due to the
fact that both components are significant over the interval.
%Figure4
\begin{figure}[h]
\centering
\cfig{6}{figure4}{6}

\caption{The time-resolved
  spectra for GRB~{\it 090820}A. The spectra represent bin \textbf{a}
  with thermal synchrotron only (top left panel), bin \textbf{b} with
  power-law synchrotron + blackbody (top right panel), bin \textbf{c}
  again with power-law synchrotron + blackbody (bottom left panel),
  and finally bin \textbf{d} with thermal synchrotron + blackbody
  (bottom right panel). As with Fig. \figureref{fig:figure2}, the
  multiple curves are associated with the effective area correction.}
\label{fig:figure4}
\end{figure}


%%Figure5
\begin{figure}
\centering
\cfig{6}{figure5}{4}
\label{fig:figure5}
\caption{The electron distributions for the time-resolved spectra. The choice of $\eta$ with a power-law only distribution is arbitrary due to the fact that $\mathcal{E}_c$ and $\eta$ both scale $\Ep$.}
\end{figure}

From bins {\textbf b} to {\textbf c} the spectrum is best described by
synchrotron emission from power-law distributed electrons in addition
to a blackbody (\tableref{tab:table1} and
\figureref{fig:figure4}). The thermal synchrotron component is too
weak to meaningfully include it in the fit. We find that the intensity
of the power-law synchrotron increases significantly from bin {\textbf
  b} to {\textbf c} while the blackbody component remains nearly
constant in intensity.  The spectral index of the electrons in these
intervals varies from -4.4 to -5.9.  Such values are consistent with
those expected from diffusive acceleration theory, for the specific
case of superluminal shocks \cite{Baring:2011}, i.e. those where the
mean magnetic field angle to the shock normal is significant. This
geometrical requirement establishes efficient convection of particles
downstream of relativistic shocks, thereby steepening their
acceleration distribution.  The blackbody component decreases in
intensity at this point but the temperature remains constant within
errors.  In bins {\textbf a} and {\textbf d}, with weaker emission,
several models are essentially statistically tied.  It is possible
that PLS+BB persists throughout the entire GRB. Alternatively, the GRB
could even begin in bin {\textbf a} with thermal synchrotron emission
and transition to the PLS+BB emission. If this were true we would be
seeing emission from electrons that have not yet been accelerated into
a power-law distribution by the shock. The C-stat values for all of
the models fit in each bin are displayed in Table \tableref{tab:table2}.

\begin{table}
\centering
\begin{tabular}{c|c c c c c}
% & Band & Thermal Synchrotron & Thermal Synchrotron + blackbody & Power Law Synchrotron & Power Law Synchrotron + blackbody \\
Time Interval & Band & TS & TS + BB & PLS & PLS + BB \\
\hline \hline
a & 464/355 & 466/357 & 464/355 & 467/357 & 465/355 \\

b & 432/355 & 742/357 & 445/355 & 555/357 & 434/355 \\

c & 450/355 & 1088/357 & 488/355 & 558/357 & 434/355 \\

d & 404/355 & 421/357 & 403/355 & 406/357 & 405/355 

\end{tabular}
\caption{The c-stat per degree of freedom for each time model in the selected time intervals.}
\label{tab:table2}
\end{table}

While it is not possible to constrain all parameters in all the bins,
it should be stressed that this is due to natural correlations in the
synchrotron functions. These difficulties do not arise when using the
Band function because it has a simpler parametrization.


%%Conclusion
\section{Conclusion of Study}
%

Here, it has been shown that thermal and non-thermal synchrotron
photon models, with an additional blackbody, are well consistent with
the emission spectra of GRB 090820A in various time intervals. These
are physical models that afford the ability to constrain parameters
that are physically meaningful, for example key descriptors of the
electron distribution that is motivated by shock acceleration
theory. By implementing these models into a forward-folding spectral
analysis software we have been able to directly constrain many of the
physical model parameters and their respective errors; a first in the
field of GRB spectroscopy. This constitutes substantial progress over
the use of the empirical Band function to fit prompt GRB spectra,
which has been a nearly universal practice to date.  The results
presented here enable more rigorous statements about the validity of
GRB emission models, moving the study of prompt burst emission into a
new era.

Our modeling has focused on the standard synchrotron shock model with
the addition of a blackbody component. The spectral fitting reveals a
complex temporal evolution of the separate components. While spectral
evolution is a well-known feature of GRBs, this type of fitting can
enable \textit{direct} physical interpretation of the evolution. These
fits provide evidence that the line of death issue
\cite{preece:1998,Preece:2002} can be overcome naturally with a
combination of synchrotron and blackbody emission: the prominence of a
blackbody component with its flat Rayleigh-Jeans portion would derive
a comparably-fitted Band function with a flat low-energy index. This
was also suggested by \cite{Guiriec:2011} where the authors used
simultaneous fits of the Band function and a blackbody. Note that it
is possible that other physical models may, in fact, produce superior
fits to the data for GRB 090820A and other bursts. Strongly-cooled
synchrotron emission, inverse Compton and jitter radiation are popular
candidates, and our work here motivates the future development of
RMFIT software modules for these processes.

A principal finding of the analysis in this paper is that the
power-law synchrotron component is orders of magnitude more intense
than the thermal synchrotron component during the peak of the burst,
the latter contributing at most a few percent of the flux. This
confirms the finding of BB04 for BATSE/EGRET bursts GRB 910503, GRB
910601 and GRB 910814, which was a theoretically-based perspective
that did not fold models through the detector response matrices. They
had noted that full plasma and Monte Carlo diffusion simulations of
shock acceleration clearly predict a power-law tail in the particle
distribution that smoothly extends from the dominant thermal
population (e.g. see also \cite{Baring:2011}, and references therein).
This tail is several orders of magnitude smaller than what is found
when fitting synchrotron emission to burst spectra. It is not clear
how such non-thermally-dominated distributions can arise near shocks,
providing a conundrum for the standard synchrotron shock
model. Limited smoothing of the sharp peak of the non-thermal electron
component will not alter this conclusion.


This result is also in accord with \cite{Guiriec:2010}, in their
analysis of GRB 100724B, who fitted its GBM spectra with a
combination of the Band model and a blackbody. They too found that an
unrealistically high efficiency for the acceleration mechanism or 
a source size smaller than the innermost stable orbit of
a black hole was required to invoke the standard fireball model for
explaining the origin of the $\gamma$-ray emission. Therefore, it was
surmised therein that the outflow from the jet was at least partially
magnetized.

To conclude, the success of this analysis in isolating the relative
contributions of a handful of distinct spectral components indicates
that it is imperative for the field of GRB spectroscopy to move away
from the use of the empirical fitting functions: many physical models
can asymptotically approximate the Band spectral indices, rendering it
difficult to discern between them particularly near the $\nu F_{\nu}$
peak. Instead, direct comparisons of the fitted physical models are
possible, and are required to truly discriminate between the various
emission processes. The fitting of physical synchrotron shock
model/blackbody spectra here offers a clear advance beyond empirical
fits, and provides the impetus for further development and deployment
of physical modeling of prompt burst emission spectra.

\section{Post-Analysis of GRB 090820A and Constraining $\epsilon$}
\label{sec:epsdisc}
The analysis of \cite{Burgess:2012} provided evidence that direct use
of physical models is possible. Several areas of this analysis must be
refined and improved upon. One of the main and troubling findings of
this work was that the non-thermal population of electrons was more
prominent than the thermal population when the two were fit
together. This was found through the value of $\epsilon$ in the
spectral fits. It should be noted as it was in \cite{Burgess:2012}
that the value was highly unconstrained. This happens because the
low-energy slope of the thermal and non-thermal synchrotron photon
spectra have an asymptotic index of -2/3. When treated as separate
components the fitting engine has a difficult time converging on the
relative contribution of each component but has to also account for
the high-energy contribution of the non-thermal synchrotron. Therefore
it quickly converges on a value of $\epsilon$ that favors a relatively
large contribution from power-law electrons.

After this analysis, ways to force the fitting engine to converge of
physical electron distributions were investigated. It has been shown
that the contribution from electrons in the power-law to the
post-shock accelerated distribution should be minuscule ($<10$\%)
compared to the thermal electrons and that the distribution should be
continuous \cite{Spitkovsky:2008}. One way to achieve this is to place
numerical constraints on the value of $\epsilon$ so that the
distribution is always continuous. Due to the shape of the combined
distribution, this constraint will also guarantee that the relative
contribution of power-law electrons is less than that of the thermal
distribution.  By setting equating the thermal and power-law
distributions at and solving for $\epsilon$ one arrives at
\begin{equation}
  \label{eq:epsnorm}
  \epsilon = \dover{\gammaMin}{\gamth}^2 \exp\left(-\dover{\gammaMin}{\gamth}\right).
\end{equation}
\begin{figure}[h!]
  \centering
  \cfig{6}{changeG}{5}
  \caption{The amended electron distribution with a fixed normalization such that the distribution is continuous regardless of the values of $\delta$ and $\gammaMin$.}
  \label{fig:fixeps}
\end{figure}
For all further analysis, this value of $\epsilon$ will be used when
fitting GRB spectra with the slow-cooled synchrotron model.










%Table 1


% This LaTeX table template is generated by emacs 23.2.1






%\end{table}




%Table2






%%% Local Variables: 
%%% mode: latex
%%% TeX-master: "../thesis"
%%% End: 
