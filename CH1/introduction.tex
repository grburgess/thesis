\chapter{Introduction}
\label{ch:intro}

\begin{chapterquote}{Aldous Huxley}
  Science has explained nothing; \\the more we know the more fantastic
  the world becomes and the profounder the surrounding darkness.
\end{chapterquote}

\section{A Brief History of Gamma-Ray Bursts}

\subsection{Discovery and Observation}
The story of Gamma-ray Bursts begins with their accidental discovery
by the Vela satellites in the 1960's. The Vela fleet was designed to
detect nuclear tests by the Soviets via the detection of $\gamma$-rays
emitted during nuclear explosions. While no tests were detected,
unknown signals of non-terrestrial origin were unexpectedly detected
\cite{Klebesadel:1973}. These signals were bright flashes of
$\gamma$-rays that occurred about once every two days. The name coined
for these objects was ``gamma-ray bursts (GRBs)``. Several instruments
made observations of GRBs through the next several decades including
the Solar Maximum Mission (SMM) and The Konus experiment
\cite{1986AdSpR...6..191K,1984BAAS...16..447N,1982ans..conf..229R,1981Ap&SS..80...85M,1979KosIs..17..812M}. The
main finding was that the broadband spectra of GRBs was highly
non-thermal \cite{Fenimore:1982,matz,Mazets:1981}. However, very
little progress was made in discovering the origins of these
events. It was not known if the progenitors were local or
extra-galactic. It wasn't until the 1993 launch of the Burst and
Transient Source Experiment (BATSE) onboard the Compton Gamma-ray
Observatory (CGRO) whose primary mission was to study GRBs that a deep
understanding of the objects was obtained \cite{Fishman:1995}. There
were two main discoveries achieved by the analysis of the BATSE data:
\begin{itemize}
\item GRBs are non-homogeneously distributed in brightness and
\item isotropically distributed on the sky leading to an
  extra-galactic origin as the most plausible explanation
  \cite{Fenimore:1993}.
%\item Their $\gamma$-ray prompt emission is of a highly non-thermal
%  nature \cite{preece:1998,Kaneko:2006}.
\end{itemize}

Another important observation is that GRBs fell into two classes based
on the duration of their emission: long and short. The bimodal
distribution of their emission shows a clustering around emission
lasting $\sim5\times 10^{-1}$ s and $\sim$20 s. Therefore, GRBs are
placed into the long class if the emission last longer than 2 s and
the into short class otherwise \cite{ck:1993} (see
\figureref{fig:t90}). Interestingly, a correlation between the
duration class and the relative hardness (the ratio of high and low
energy counts in the signal) of the detected $\gamma$-rays was found
as well. Such a correlation points to a possibly different physical
origin between these two classes and has spurred much research into
the progenitors of GRBs.

\begin{figure}[h]
  \centering
  \cfig{1}{t90.pdf}{4.3}
  \caption{The duration distribution of GRBs detected by {\it
      Fermi}. Long GRBs are displayed in purple and short GRBs are in
    green.}
  \label{fig:t90}
\end{figure}

\subsection{Origin}
The extra-galactic origin of GRBS means implies that these are the
brightest objects in the sky when they occur. Once a measured redshift
was first detected for GRB 970508 \cite{Paradijs:1997,Costa:1997}, it
was established that these objects emit nearly $10^{51}$ to $10^{53}$
erg s$^{-1}$ during their prompt emission. This amount of energy
release over such a short duration (10$^{-3}$-10$^3$ s) implies
equivalent to that of a solar rest mass, $M_{\rm sun}c^2\sim {\rm few}
\times 10^{54}$ erg emitted isotropically, an amount that is much
larger than the typical energy release of a supernova. This can be
explained if the release is highly beamed in the form of a jet
\cite{Castro:1999,Fruchter:1999,Kulkarni:1999}. If this is true, the
energy release of a GRB is comparable to that of a supernova. However,
the problem of uncovering the sources or progenitors of GRBs remained.

The two most commonly invoked progenitors of extra-galactic GRBs are
either the gravitational collapse of super-massive stars from the
early Universe or the merger of two compact objects such as two
neutron stars (NS-NS) or a neutron star and a black hole (NS-BH)
\cite{Woosley:1993,Paczynski:1998,Paczynski:1986,Eichler:1989}. Both
scenarios would lead to the formation of a black hole and a massive
energy release that would heat the surrounding material to high
temperatures leading, to an expanding fireball in the form of a
jet. This jet would be the source of the observed GRB emission.

\section{Open Problems in Gamma-Ray Bursts Studies}
While GRBs are the most powerful, luminous, and energetic cosmological
events in the Universe next to the Big Bang, they are also almost the
most poorly understood of all $\gamma$-ray astrophysical
phenomena. The mysterious nature of GRBs is largely due to their brief
emission and varied temporal and spectral properties across the
population of observed events. GRBs are very bright, but each unique
event is brief and allows for only one chance to observe the
time-resolved properties of the explosion. This is very different from
other high-energy sources such as $\gamma$-ray pulsars or blazers
whose emission is comparatively long-lasting, if not constant. Even
though there have been thousands of detected GRBs, the detail with
which an individual event can be analyzed is limited by the number of
photons detected by an observing satellite. Therefore, it has been
difficult to ascertain the emission progenitors and emission
mechanisms behind the observations.  As a group, GRBs have many
similar properties, however, very few GRBs share a similar
lightcurve. Such a variety of lightcurves makes it difficult, though
not impossible, to group GRB observations when dealing with the
time-resolved properties. Additionally, the cosmological distances
involved mean that it is hard to optically identify these events to
pinpoint and make an association with the stellar parent of any given
event.

These difficulties in observation have left several open questions in
the study of GRBs, some of which are now beginning to be answered (for
an in depth review see \cite{zhang:2011}).
\begin{enumerate}[(i)]
\item What are stellar progenitor(s) of GRBs?
\item What is the structure and associated evolution of the GRB jet?
\item What particle acceleration mechanisms occur in GRB environs?
\item What is the magnetic field structure (if any) in the GRB jet?
\item What high-energy radiative processes are responsible for
  converting the jet kinetic and/or magnetic energy into the observed radiation?
\end{enumerate}
The importance of addressing these questions lies in the extreme
environments that produce GRBs. They serve as laboratories for testing
theories that cannot be done on Earth.


\section{Summary of Research}
In this work, the central problem of understanding the physical
mechanisms that generate GRB high-energy emission will be addressed in
part. To date, no single emission model has been able to explain the
shape of the detected spectrum of GRBs. Numerous models exist for both
the dynamic structure of the generated jet and the associated emission
of $\gamma$-rays from each stage (see \sectionref{sec:fbm}). The main
issues that will be addressed in this work are
\begin{enumerate}[(i)]
\item What radiative emission mechanisms can account for the observed
  time-resolved spectra of GRBs?

\item What particle acceleration process(es) can generate the necessary
  distributions of electrons available to radiate?

\item Can the radiative mechanisms that account for the observed
  emission be reconciled with current jet dynamics models?


\end{enumerate}

These issues will be addressed through detailed spectral analysis of
detected GRBs with physical emission models and the subsequent use of
the spectral parameters to derive physical quantities that provide
insight into both the radiative mechanisms and jet dynamics related to
GRBs. The data used in this work comes primarily from the Gamma-Ray
Burst Monitor (GBM) \cite{Meegan:2009} and Large Area Telescope (LAT)
\cite{Atwood:2009} onboard the {\it Fermi} space telescope. A key
advancement of this project over previous work is the use of physical
spectral models to fit the spectra of GRBs (see \chapterref{ch:phys}) which
by itself has helped to resolve some key problems in GRB spectroscopy.





%%% Local Variables: 
%%% mode: latex
%%% TeX-master: "../thesis"
%%% End: 
