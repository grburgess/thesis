%   Thesis.tex - Uah ph.D. Dissertation Main File
%
%   This is the main file for my thesis. It loads all content via
%   \include statments

%%% Making the DVI and PDF output the correct size %%%
%   Important - the default pagesize for DVI and PDF output in most LaTeX
%   distributions is a4 (210x297cm, or about 8.27�11.69 in), which won't work
%   for the UAH format.  So, you have to change the DVI and PDF output in your
%   LaTeX distribution.  In most distros, you need to change the following files:

%   (1.)  C:\texmf\dvipdfm\config\dvipdfmx.cfg
%   (2.)  C:\texmf\dvipdfm\config\config
%   (3.)  C:\texmf\pdftex\config\pdftex.cfg

%   For numbers (1.) and (2.), you will change where the files say "p a4" to
%   "p letter" (there should be some explanation in the files themselves, too).
%   For number (3.), you will change two lines of that file to read:

%   page_height 11 true in
%   page_width 8.5 true in

%   There are "local" versions of all the above files in this complete rar
%   package which 'should' override the a4 defaults - but these do not work in
%   some distributions, so editing the above files is usually necessary.
%   For more info, search google for "pdftex.cfg" and "dvipdfmx.cfg"
%%% (End DVI and PDF changes) %%%

%   Define the document class. The proper class is "book." The normal options
%   are "12pt" and "onesided". When printing out copies for purposes other than
%   submission to the Graduate School, the option "twoside" can be used to
%   format the output in a manner which looks good when printed duplex.

\documentclass[12pt,oneside,letterpaper]{book}

%   Load the packages necessary for the thesis. The only required packages are
%   "uahdis", which loads the UAH Dissertation Style, and "chngpage" which is
%   used by "uahdis". The other packages are optional, and should be loaded only
%   if you have a need for the functionality they provide. Documentation on each is
%   available at CTAN.org. The basic purpose of each are described below:
%   units---provides a convenient mechanism for writing quantities with units.
%   bm---provides a convenient mechanism for producing bold symbols in math mode
%   hhline---improved lines and borders in tables
%   rotating---allows landscape oriented pages in a portrait oriented document
%   verbatim---prints text files verbatim. useful for computer code
%   amsfonts---provides additional typefaces
%   amssymb---provides additional mathematical symbols
%   xspace---properly adjusts space after \newcommands which expand to text
%   booktabs---creates traditional scientific tables - no vertical lines, very
%   few horizontal lines, with varying thickness (not meant to be used with hline)

%   Excellent descriptions of all LaTeX packages can be found here:
%   http://www.tug.org/tex-archive/help/Catalogue/index.html

\usepackage[]{uahdis,chngpage,units,bm,hhline,rotating,verbatim,amsfonts,amssymb,xspace}
\usepackage[]{indentfirst,layouts, booktabs, enumerate, gensymb, subfigure}
\usepackage[section]{placeins}
\usepackage{url}
%\usepackage{mathabx}
%   Load the tex file which contains all local \newcommands and \newenvironments

%%%
%%% Environments
%%%

\newsavebox{\speaker}
\newenvironment{chapterquote}[1]
    {\begin{flushright}\begin{minipage}{3 in}\sbox{\speaker}{#1}\itshape\singlespace}
    {\begin{flushright}---\usebox{\speaker}\end{flushright}\end{minipage}\end{flushright}}

% Note: chapterquote works (and looks) best when the chapter begins with a \section{} ...
% If you hadn't planned on beginning the chapter with a \section, try \section{Overview} :-)
% otherwise, some \vspace{} might be necessary, but that won't be consistent throughout the thesis

%%%
%%%  Functions---Commands that take parameters
%%%
\newcommand{\cfig}[3]{\centering\includegraphics[keepaspectratio=true,width=#3in]{./CH#1/EPSFDocs/#2}}
\newcommand{\acro}[1]{\textsc{#1}}
\newcommand{\sci}[2]{\ensuremath{#1 \!  \times \!  10^{#2}}}
\newcommand{\vect}[1]{\boldsymbol{\mathbf{#1}}}
\newcommand{\unitvec}[1]{\vect{\hat{#1}}}
\newcommand{\threebythree}[9]{\renewcommand{\arraystretch}{0.75}\begin{vmatrix}#1&#2&#3\\#4&#5&#6\\#7&#8&#9\end{vmatrix}\renewcommand{\arraystretch}{1.0}}
\newcommand{\mysci}[2]{\ensuremath{#1 \!  \times \!  10^{#2}}}
\newcommand{\myprop}[5]{g(#1,#2;#3,#4;#5)}
\newcommand{\myint}[3]{\int_{#1}^{#2}#3}
\newcommand{\myintinfinf}[1]{\myint{-\infty}{\infty}{#1}}


%%%
%%%  New Math Operators
%%%

\DeclareMathOperator{\polylog}{Li}


%%%
%%%  Symbols---Commands that are shorthand
%%%
\newcommand{\rf}{\acro{rf}\xspace}
\newcommand{\dc}{\acro{dc}\xspace}
\newcommand{\ac}{\acro{ac}\xspace}
\newcommand{\ccd}{\acro{ccd}\xspace}
\newcommand{\mach}{\acro{mach2}\xspace}
\newcommand{\mgmhd}{\acro{mgmhd}\xspace}
\newcommand{\cea}{\acro{cea}\xspace}
\newcommand{\half}{\nicefrac{1}{2}\,}
\newcommand{\xx}{\ensuremath{\unitvec{x}}\xspace}
\newcommand{\yy}{\ensuremath{\unitvec{y}}\xspace}
\newcommand{\zz}{\ensuremath{\unitvec{z}}\xspace}
\newcommand{\ttl}{\acro{ttl}\xspace}
\newcommand{\pmt}{\acro{pmt}\xspace}
\newcommand{\eg}{\textit{e.g.}\xspace}
\newcommand{\ie}{\textit{i.e.}\xspace}
\newcommand{\qmn}{\ensuremath{q^{mn}}\xspace}
\newcommand{\isp}{\ensuremath{I_{sp}}\xspace}



%%%
%%%  Other stuff I've found - Evaluate for usefulness
%%%

%\newcommand{\quan}[2][]{\mbox{$#1\,\mathrm{#2}$}}
%\newcommand{\vv}[1]{\ensuremath{\boldsymbol{#1}}}
%\newcommand{\tempc}[1]{\quan[#1]{^{\circ}C}}
%\DeclareMathOperator{\sinc}{sinc}
%\newcommand{\comment}[1]{\marginpar{\Large \hfill \ddag}\textsf{#1}}
%\newcommand{\comment}[1]{}


%   The next line requires the leading "%". It is only useful if you are using
%   WinEdt as your text editor. It allows WinEdt to collect bibliographic entries into
%   a pop-up table that you can summon when you are \cite-ing a source.

%GATHER{Bibliography.bib}




\def\dover#1#2{\hbox{${{\displaystyle#1 \vphantom{(} }\over{
\displaystyle #2 \vphantom{(} }}$}}
%
{\catcode`\@=11                                                  
\gdef\SchlangeUnter#1#2{\lower2pt\vbox{\baselineskip 0pt\lineskip0pt    
\ialign{$\m@th#1\hfil##\hfil$\crcr#2\crcr\sim\crcr}}}}           
\def\gtrsim{\mathrel{\mathpalette\SchlangeUnter>}}               
\def\lesssim{\mathrel{\mathpalette\SchlangeUnter<}}    
\def\lambar{\lambda\llap {--}}
\def\fsc{\alpha_{\hbox{\sevenrm f}}} 
\def\sigt{\sigma_{\hbox{\sixrm T}}}
\def\erg{\varepsilon}                               
\def\lambdamax{\lambda_{\hbox{\sevenrm max}}} 
\def\tesc{t_{\hbox{\sevenrm esc}}} 
\def\ergNT{\erg_{\hbox{\sixrm NT}}}
\def\calFNT{{\mathcal{F}}_{\hbox{\fiverm NT}}}
\def\ndotsyn{{\dot n}_{\hbox{\fiverm S}}}
\def\ndotssc{{\dot n}_{\hbox{\fiverm SSC}}}
\def\Ndotssc{{\dot N}_{\hbox{\fiverm SSC}}}
\def\es{\erg_{\gamma}}
\def\gammat{\gamma_{\hbox{\fiverm T}}}   
\def\gammin{\gamma_{\hbox{\sevenrm min}}}
\def\gamPL{\gamma_{\rm PL}}
\def\gamth{\gamma_{\rm th}}
\def\gammaMin{\gamma_{\rm min}}
\def\yPL{y_{\hbox{\sixrm PL}}}
\def\vFv{\nu F_{\nu}}
\def\Ep{E_{\rm p}}
\def\Fv{F_{\nu}}
\def\gray{$\gamma$-ray}
\def\chisq{$\chi^{2}\;$}
\def\dcstat{$\Delta_{\rm cstat}\;$}
\def\teq#1{$\, #1\,$}

\def\ve{\varepsilon}
\def\gbm{{\it GBM }}
\def\lat{{\it LAT }}
\def\fermi{{\it Fermi }}
\newcommand{\Mesz}{{M\'esz\'aros}}
\def\mathnew{\mathsurround=0pt}
\def\simov#1#2{\lower .5pt\vbox{\baselineskip0pt \lineskip-.5pt
      \ialign{$\mathnew#1\hfil##\hfil$\crcr#2\crcr\sim\crcr}}}
\def\simg{\mathrel{\mathpalette\simov >}}
\def\siml{\mathrel{\mathpalette\simov <}}
\def\beq{\begin{equation}}
\def\enq{\end{equation}}
\def\bea{\begin{eqnarray}}
\def\ena{\end{eqnarray}}
\def\bitm{\bibitem}
\def\msun{M_\odot}
\def\L54{L_{54}}
\def\E55{E_{55}}
\def\et3{\eta_3}
\def\th1{\theta_{-1}}
\def\r07{r_{0,7}}
\def\x05{x_{0.5}}
\def\et600{\eta_{600}}
\def\et3{\eta_3}
\def\rph{r_{ph}}
\def\vareps{\varepsilon}
\def\fflunit{\hbox{~erg cm}^{-2}~\hbox{s}^{-1}}
\def\eps{\epsilon}
\def\ve{\varepsilon}
\def\Fl{\mathcal{F}}
\def\muh{\hat{\mu}}
\def\cm{\hbox{~cm}}
\def\kpc{\hbox{~kpc}}
\def\Mpc{\hbox{~Mpc}}
\def\s{\hbox{~s}}
\def\gev{\hbox{~GeV}}
\def\Jy{\hbox{~Jy}}
\def\mJy{\hbox{~mJy}}
\def\TeV{\hbox{~TeV}}
\def\GeV{\hbox{~GeV}}
\def\MeV{\hbox{~MeV}}
\def\kev{\hbox{~keV}}
\def\keV{\hbox{~keV}}
\def\eV{\hbox{~eV}}
\def\G{\hbox{~G}}
\def\erg{\hbox{~erg}}
\def\s{{\hbox{~s}}
\def\cm2{\hbox{~cm}^2}}
\def\para{\parallel}


\def\eps{\epsilon}
\def\e{\epsilon}
\def\ep{\epsilon^\prime}
\def\veps{\varepsilon}

\newcommand{\g}{\gamma}
\newcommand{\gp}{{\gamma^\prime}}
\newcommand{\gpp}{{\gamma_p^\prime}}
\newcommand{\dD}{{\delta_{\rm D}}}
\newcommand{\tp}{t^\prime}

%   Define my name
\author{J. Michael Burgess}

%   Define the title of my thesis (user upper and lower case)
\title{Discerning the Physical Properties of Gamma-Ray Bursts via Time Resolved Analysis with Physical Spectral Models}

%   Define the year
\date{2013}

%   Define my department
\uahdepartment{Physics}

%   Define my advisor (no title - i.e., no "Dr.")
\uahadvisor{Robert D. Preece}

%   Define my committee members (no titles)
\uahmema{Matthew G. Baring}

\uahmemb{Valerie Connaughton}

\uahmemc{Massimilio Bonamente}

\uahmemd{Michael S. Briggs}

\uahmeme{Gary P. Zank}

% Note: If you have more than 5 total committee member,
% you must manually edit the uahdis.sty file (Approval Page)

%MWT

%   Define my Department Chair
\uahdeptchair{Richard Lieu}

%   Define my College (do not write "College of")
\uahcollege{Science}

%   Define my College Dean
%   the tex for Dean Aunon is:  Jorge I. Au\~n\a'on
\uahcolldean{Jack Fix}

%   Define my Degree (i.e., Master of Science, Doctor of Philosophy)
\uahdegree{Doctor of Philosophy}

%   Define my Program name
\uahprogram{Physics}

%   Shortened name for degree
%   type "master's" or "doctoral" (without quotes)
\uahdegreeshort{doctoral}

%   Define my Document Type
\uahdoctype{dissertation}

%   Define my Graduate Dean
\uahgraddean{David Berkowitz}


%   Let's get started
\begin{document}

%   To view a layout of the margins of this document,
%   uncomment the below line (requires "layouts" package):
%\layout
%\tocdiagram\tocdesign

%   Here comes the stuff that goes before the main content
\frontmatter

\pagestyle{plain}

\maketitle

\copyrightpage

%MWT - UAH Approval Form
\approvalpage

%\makeabstract % Now issued in ./FRONT/abstract.tex

%   Include my abstract
\chapter*{Abstract}
% the \makeabstract command creates the top portion of the abstract
% page ... must be issued before the abstract content
\makeabstract

%%%%%%%%%%% Your Abstract Text Goes after Here %%%%%%%%%%%%%%%%%%%%%%%
Gamma-ray bursts (GRBs) are the most energetic events in the Universe
but the processes that generate their observed $\gamma$-ray emission
remain unknown. Much of what is known about these processes comes from
fits of the empirical Band function to the photon spectra of
GRBs. However, very little information about the emission mechanisms
can be derived from these empirical fits because extrapolation of
fitted Band parameters to physical photon models is often degenerate
due to the similar shapes of these models.  In this work, physical
models of high-energy radiation mechanisms are numerically implemented
into a data fitting framework in order to test these models on
Gamma-ray Burst (GRB) data from the {\it Fermi} Gamma-ray Space
Telescope. The resulting fit parameters are used to explore the
structure, mechanisms, and evolution of GRB jets to gain a better
understanding of how these relatively unexplained events
occur. Evaluations of plausible models are made from the inferred
properties of the jets enabling a full physical view of the evolution
of GRBs from the event horizon of the parent black hole to the very
edge of the jet.

%%%%%%%%%%%%% Your Abstract Text should be before Here %%%%%%%%%%%%%%

% the abstractsig command creates the signature spaces after the
% abstract, and therefore, must be issued after the abstract.
\abstractsig


% Abstract signatures:
%\abstractsig % command now issued in abstract.tex

%   Include my acknowledgements
\chapter*{Acknowledgments}

This process has taken more from those around me than it ever could
have taken from me. From the moment my parents sent me to bed early
without letting me see the scanning electron microscope image of
molecules, I knew that I wanted to be a scientist. It was there
dedication to keeping me on the right path that lead to me to achieve
this. Thank you Mom for letting me be me and letting me play music and
trusting that I would still return to science (twice!!). Thank you Dad
for showing me what it means to be dedicated and not give up. Both of
you have given up a lot to see me to and through this process and it
is just as much your achievement as it is mine. My entire family has
done nothing but support me through this time in my life and for that
I am lucky and thankful

All the help I have had from the Gamma-ray Burst monitor team is
immeasurable. From the day I started I felt like I was part of a
family. thank you Michael Briggs from keeping me grounded and Valerie
Connaughton for being a great mentor and friend. Bill Paciesas and
Chip Meegan, thanks for your guidance. You guys are legends in this
field and it has been amazing to continue work that you paved the way
for so many years ago.

To my advisor, Rob Preece, thank you for taking a chance on a
graduated student in a desperate situation. Your guidence and
patience with my constant, daily diversion into some strange theory allowed
me to find a rewarding research topic. Though it will never be
communicated in beautiful English, I think we have accomplished some
great things and it is thanks to your help that they were able to be
written down. I will sincerely miss our chats about science and the
world.

I would be foolish to leave out George Clinton. Had I not stayed up
for five days straight with the pure sounds of extraterrestrial p-funk
keeping me going I would not have finished this work on time. It is
true that funk not only moves, it removes, dig?

Lastly, my son Isaac. You have sacrificed the most in all of this. You
came into the world helping me do quantum mechanics problems in the
middle of the night and spent the first parts of your life begging me
to close the books and play. I hope you will see those years as
something to learn from. Sometimes we have to sacrifice precious time
to do what is best for ourselves and those that depend on us. I look
forward to the day when you choose your path and work hard to become
what ever it is you believe in. I love you little buddy and thank you
for bearing with me all these years.



%%% Local Variables: 
%%% mode: latex
%%% TeX-master: "../thesis"
%%% End: 



%   Make a Table of Contents
\tableofcontents

%   Make a List of Figures
\listoffigures

%   Make a List of Tables
\listoftables

%   Make a List of Symbols (Comment out if unwanted)
% List of Symbols
\listofsymbols

\symboldefinition{$B$}{Magnitude of the magnetic field}

\symboldefinition{$b_i$}{Vector representation of the photon background model}

\symboldefinition{$c$}{Speed of light}

\symboldefinition{$c_i$}{Vector representation of the count spectrum}

\symboldefinition{$D_{ij}$}{Matrix representation of the instrument response}

\symboldefinition{$d_L$}{Cosmological luminosity distance}

\symboldefinition{$E_{\star}$}{Parameterized synchrotron $\vFv$ peak}

\symboldefinition{${\rm E}_0$}{Initial energy inout of a GRB}

\symboldefinition{$E_C$}{Characteristic energy of synchrotron emission}

\symboldefinition{$E_{\rm cool}$}{The peak cooling energy of the fast-cooled synchrotron spectrum}

\symboldefinition{$E_{\rm int}$}{Internal energy available in a post shock jet shell}

\symboldefinition{$E_{\rm p}$}{The $\nu F_{\nu}$ peak energy}

\symboldefinition{$E_{\rm p, 0}$}{Initial $E_{\rm p}$ of the hardness-intensity correlation}


\symboldefinition{$\mathcal{E}_s$}{Energy of a seed electron for Compton scattering}

\symboldefinition{$e^{\pm}$}{Electron positron pairs}

\symboldefinition{$F_0$}{Initial energy flux of the hardness-intensity correlation}

\symboldefinition{$F_{BB}$}{The energy flux of the blackbody component}

\symboldefinition{$F_E$}{Integral energy flux}

\symboldefinition{$F_{syn}$}{The energy flux of the synchrotron component}

\symboldefinition{$F_{\nu}$}{Differential energy flux}

\symboldefinition{$f_i$}{Vector representation of a photon model}

\symboldefinition{$G$}{Gravitational constant}

\symboldefinition{$H_0$}{The null hypothesis}

\symboldefinition{$H_1$}{The proposal hypothesis}

\symboldefinition{$\hbar$}{The reduced Planck constant}

\symboldefinition{$k$}{The Boltzmann constant}

\symboldefinition{$L$}{GRB luminosity}

\symboldefinition{$L_{E}$}{Eddington luminosity}

\symboldefinition{$L_{\rm ph}$}{Luminosity of the photosphere}

\symboldefinition{$L_{\rm syn}$}{Luminosity of the synchrotron component}

\symboldefinition{$\dot{M}$}{GRB mass loss rate}

\symboldefinition{$M_0$}{Initial mass of the GRB progenitor}

\symboldefinition{$M_{\rm ext}$}{Mass of the external interstellar medium}

\symboldefinition{$M_{sun}$}{Solar mass}

\symboldefinition{$m_{\rm f}$}{Mass of a fast moving jet shell}

\symboldefinition{$m_p$}{Mass of the proton}

\symboldefinition{$m_{\rm s}$}{Mass of a slow moving jet shell}

\symboldefinition{$N$}{Arbitrary normalizstion constant}

\symboldefinition{$n_{cp}$}{Bayesian prior for Bayesian block time bins}

\symboldefinition{$n_e$}{GRB jet electron density}

\symboldefinition{$n_p$}{GRB jet baryon density}

\symboldefinition{$n_{\rm ph}$}{Soft seed photon distribution for Compton scattering}

\symboldefinition{$p(i)$}{Distribution of \dcstat}

\symboldefinition{$Q_e$}{The injected electron distribution}

\symboldefinition{$q$}{Power-law index of the hardness-intensity correlation}

\symboldefinition{$q_e$}{Charge of the electron}

\symboldefinition{$\mathcal{R}$}{Photosphere geometry parameter}

\symboldefinition{$R$}{Linear jet radius}

\symboldefinition{$r$}{Variable jet radius}

\symboldefinition{${\rm r}_0$}{Initial radius at the base of the GRB jet}

\symboldefinition{$r_{\rm dec}$}{Decelleration radius of the GRB jet}

\symboldefinition{$r_e$}{Classical electron radius}

\symboldefinition{$r_g$}{Elctron gyration radius}

\symboldefinition{${\rm r}_{nt}$}{Radius where non-thermal radiation is generated of the GRB jet}

\symboldefinition{${\rm r}_{ph}$}{Photospheric radius of the GRB jet}

\symboldefinition{${\rm r}_s$}{Saturation radius of the GRB jet}

\symboldefinition{$T^{\prime}$}{Comoving temperature}

\symboldefinition{T$_{max}$}{Time of peak flux in GRB lightcurve}

\symboldefinition{$t$}{Time}

\symboldefinition{$t_{esc}$}{The escape time of particles from an emission region}

\symboldefinition{$t_{\rm dyn}$}{Dynamical time of the GRB jet}

\symboldefinition{$t_v$}{GRB variability time}

\symboldefinition{$u$}{Shock flow velocity}

\symboldefinition{$V^{\prime}$}{Comoving volume}

\symboldefinition{$Y$}{Ration of {\gray} energy to total fireball energy}



%%%%%%%%%%%% GREEKS

\symboldefinition{$\alpha$}{The low-energy power-law index of the Band function}

\symboldefinition{$\beta$}{The high-energy power-law index of the Band function}

\symboldefinition{$\beta_v$}{Dimensionless relativistic velocity scaled to the speed of light}

\symboldefinition{\chisq}{The statistic used for large numbers of counts}

\symboldefinition{\dcstat}{The difference between two models' fit C-stat}

\symboldefinition{$\chi^2_{red}$}{Reduced \chisq ({\chisq}/degrees of freedom)}

\symboldefinition{$\Delta_t$}{The difference of two time intervals}

\symboldefinition{$\delta$}{Spectral index of accelerated electrons in a power-law distribution}

\symboldefinition{$\epsilon$}{Normalization of the electron power-law distribution}

\symboldefinition{$\epsilon_B$}{Fractional energy magnetic energy content of the jet}

\symboldefinition{$\epsilon_{\gamma}$}{$\gamma$-ray energy}

\symboldefinition{$\epsilon_e$}{Fractional energy electronic energy content of the jet}

\symboldefinition{$\eta$}{Dimensionless entropy of the GRB fireball}

\symboldefinition{$\Phi$}{Time running photon fluence}

\symboldefinition{$\Phi_0$}{Time running fluence decay constant}

\symboldefinition{$\Gamma$}{Bulk lorentz factor of the GRB fireball}

\symboldefinition{$\Gamma_{\rm f}$}{Lorentz factor of a fast moving jet shell}

\symboldefinition{$\Gamma_{\rm m}$}{Lorentz factor of a post shock jet shell}

\symboldefinition{$\Gamma_{\rm s}$}{Lorentz factor of a slow moving jet shell}

\symboldefinition{$\dot{\gamma}$}{The cooling rate of electrons via a radiation process}

\symboldefinition{$\gamma^{\prime}$}{Comoving random particle lorentz factor}

\symboldefinition{$\gamma_0$}{Initial random Lorentz factor of particle in the GRB jet}

\symboldefinition{$\gamma_{\rm a}$}{Adiabatic index}

\symboldefinition{$\gamma_{\rm cool}$}{The cooling electron Lorentz factor}

\symboldefinition{$\gamma_{\rm min}$}{Minimum electron lorentz factor for electron power-law distribution}

\symboldefinition{$\gamma_{\rm th}$}{Thermal electron lorentz factor}

\symboldefinition{$\mu$}{Radial exponent of the evolution of $\Gamma$}

\symboldefinition{$\vFv$}{Differential energy power spectrum}

\symboldefinition{$\sigma_{\rm BB}$}{Statistical significance of the blackbody component}

\symboldefinition{$\sigma_{sb}$}{Stefan-Boltzmann constant}

\symboldefinition{$\sigma_{\rm T}$}{Thompson electron cross-section}



\symboldefinition{$\rho_{\rm ext}$}{Density of the external interstellar medium}

\symboldefinition{$\Theta$}{Heaviside step function}

\symboldefinition{$\tau$}{Optical depth}

\symboldefinition{$\tau_{\epsilon}$}{Synchrotron self-absorption opacity}

\symboldefinition{$\zeta$}{Time exponent for the evolution of $\Gamma$}



















% you need the following \clearpage command at the end of the
% list of symbols:
\clearpage

%%% Local Variables: 
%%% mode: plain-tex
%%% TeX-master: "../thesis"
%%% End: 

%   Make a "Chapter" header in TOC, per UAH style
%   issue command after last frontmatter TOC entry
%   cannot come directly before first chapter \include
\addchapheadtotoc


%   Set my dedication
\dedication{To Isaac for being cooler than cool.}

%   Make Epigraph Page
%\epigraphpage{The Mountain Goats}{Astronomy? Impossible to understand and madne%ss to investigate.}
\epigraphpage{The Mountain Goats}{Something here will eventually have to explode}

\clearpage

\pagestyle{myheadings} \markright{}

%   Here comes the main content
\mainmatter

%   Include my chapters
%   LaTeX will look for your chapter files in the appropriate folders,
%   as addressed below.  LaTeX will look for a .tex file with the
%   same name as the name you give it below - i.e., for the CH1 folder,
%   LaTeX will look for introduction.tex (which is where you will type
%   all your chapter 1 stuff).  You can name your chapters/files whatever
%   you want - just make sure the names below match the names in the
%   folders.  Also, you can add or subtract chapters as you like -
%   just make sure that the address and filenames below match your
%   file structure.

\chapter{Introduction}
\label{ch:intro}

\begin{chapterquote}{Aldous Huxley}
  Science has explained nothing; \\the more we know the more fantastic
  the world becomes and the profounder the surrounding darkness.
\end{chapterquote}

\section{A Brief History of Gamma-Ray Bursts}

\subsection{Discovery and Observation}
The story of Gamma-ray Bursts begins with their accidental discovery
by the Vela satellites in the 1960's. The Vela fleet was designed to
detect nuclear tests by the Soviets via the detection of $\gamma$-rays
emitted during nuclear explosions. While no tests were detected,
unknown signals of non-terrestrial origin were unexpectedly detected
\cite{Klebesadel:1973}. These signals were bright flashes of
$\gamma$-rays that occurred about once every two days. The name coined
for these objects was ``gamma-ray bursts (GRBs)``. Several instruments
made observations of GRBs through the next several decades including
the Solar Maximum Mission (SMM) and The Konus experiment
\cite{1986AdSpR...6..191K,1984BAAS...16..447N,1982ans..conf..229R,1981Ap&SS..80...85M,1979KosIs..17..812M}. The
main finding was that the broadband spectra of GRBs was highly
non-thermal \cite{Fenimore:1982,matz,Mazets:1981}. However, very
little progress was made in discovering the origins of these
events. It was not known if the progenitors were local or
extra-galactic. It wasn't until the 1993 launch of the Burst and
Transient Source Experiment (BATSE) onboard the Compton Gamma-ray
Observatory (CGRO) whose primary mission was to study GRBs that a deep
understanding of the objects was obtained \cite{Fishman:1995}. There
were two main discoveries achieved by the analysis of the BATSE data:
\begin{itemize}
\item GRBs are non-homogeneously distributed in brightness and
\item isotropically distributed on the sky leading to an
  extra-galactic origin as the most plausible explanation
  \cite{Fenimore:1993}.
%\item Their $\gamma$-ray prompt emission is of a highly non-thermal
%  nature \cite{preece:1998,Kaneko:2006}.
\end{itemize}

Another important observation is that GRBs fell into two classes based
on the duration of their emission: long and short. The bimodal
distribution of their emission shows a clustering around emission
lasting $\sim5\times 10^{-1}$ s and $\sim$20 s. Therefore, GRBs are
placed into the long class if the emission last longer than 2 s and
the into short class otherwise \cite{ck:1993} (see
\figureref{fig:t90}). Interestingly, a correlation between the
duration class and the relative hardness (the ratio of high and low
energy counts in the signal) of the detected $\gamma$-rays was found
as well. Such a correlation points to a possibly different physical
origin between these two classes and has spurred much research into
the progenitors of GRBs.

\begin{figure}[h]
  \centering
  \cfig{1}{t90.pdf}{4.3}
  \caption{The duration distribution of GRBs detected by {\it
      Fermi}. Long GRBs are displayed in purple and short GRBs are in
    green.}
  \label{fig:t90}
\end{figure}

\subsection{Origin}
The extra-galactic origin of GRBS means implies that these are the
brightest objects in the sky when they occur. Once a measured redshift
was first detected for GRB 970508 \cite{Paradijs:1997,Costa:1997}, it
was established that these objects emit nearly $10^{51}$ to $10^{53}$
erg s$^{-1}$ during their prompt emission. This amount of energy
release over such a short duration (10$^{-3}$-10$^3$ s) implies
equivalent to that of a solar rest mass, $M_{\rm sun}c^2\sim {\rm few}
\times 10^{54}$ erg emitted isotropically, an amount that is much
larger than the typical energy release of a supernova. This can be
explained if the release is highly beamed in the form of a jet
\cite{Castro:1999,Fruchter:1999,Kulkarni:1999}. If this is true, the
energy release of a GRB is comparable to that of a supernova. However,
the problem of uncovering the sources or progenitors of GRBs remained.

The two most commonly invoked progenitors of extra-galactic GRBs are
either the gravitational collapse of super-massive stars from the
early Universe or the merger of two compact objects such as two
neutron stars (NS-NS) or a neutron star and a black hole (NS-BH)
\cite{Woosley:1993,Paczynski:1998,Paczynski:1986,Eichler:1989}. Both
scenarios would lead to the formation of a black hole and a massive
energy release that would heat the surrounding material to high
temperatures leading, to an expanding fireball in the form of a
jet. This jet would be the source of the observed GRB emission.

\section{Open Problems in Gamma-Ray Bursts Studies}
While GRBs are the most powerful, luminous, and energetic cosmological
events in the Universe next to the Big Bang, they are also almost the
most poorly understood of all $\gamma$-ray astrophysical
phenomena. The mysterious nature of GRBs is largely due to their brief
emission and varied temporal and spectral properties across the
population of observed events. GRBs are very bright, but each unique
event is brief and allows for only one chance to observe the
time-resolved properties of the explosion. This is very different from
other high-energy sources such as $\gamma$-ray pulsars or blazers
whose emission is comparatively long-lasting, if not constant. Even
though there have been thousands of detected GRBs, the detail with
which an individual event can be analyzed is limited by the number of
photons detected by an observing satellite. Therefore, it has been
difficult to ascertain the emission progenitors and emission
mechanisms behind the observations.  As a group, GRBs have many
similar properties, however, very few GRBs share a similar
lightcurve. Such a variety of lightcurves makes it difficult, though
not impossible, to group GRB observations when dealing with the
time-resolved properties. Additionally, the cosmological distances
involved mean that it is hard to optically identify these events to
pinpoint and make an association with the stellar parent of any given
event.

These difficulties in observation have left several open questions in
the study of GRBs, some of which are now beginning to be answered (for
an in depth review see \cite{zhang:2011}).
\begin{enumerate}[(i)]
\item What are stellar progenitor(s) of GRBs?
\item What is the structure and associated evolution of the GRB jet?
\item What particle acceleration mechanisms occur in GRB environs?
\item What is the magnetic field structure (if any) in the GRB jet?
\item What high-energy radiative processes are responsible for
  converting the jet kinetic and/or magnetic energy into the observed radiation?
\end{enumerate}
The importance of addressing these questions lies in the extreme
environments that produce GRBs. They serve as laboratories for testing
theories that cannot be done on Earth.


\section{Summary of Research}
In this work, the central problem of understanding the physical
mechanisms that generate GRB high-energy emission will be addressed in
part. To date, no single emission model has been able to explain the
shape of the detected spectrum of GRBs. Numerous models exist for both
the dynamic structure of the generated jet and the associated emission
of $\gamma$-rays from each stage (see \sectionref{sec:fbm}). The main
issues that will be addressed in this work are
\begin{enumerate}[(i)]
\item What radiative emission mechanisms can account for the observed
  time-resolved spectra of GRBs?

\item What particle acceleration process(es) can generate the necessary
  distributions of electrons available to radiate?

\item Can the radiative mechanisms that account for the observed
  emission be reconciled with current jet dynamics models?


\end{enumerate}

These issues will be addressed through detailed spectral analysis of
detected GRBs with physical emission models and the subsequent use of
the spectral parameters to derive physical quantities that provide
insight into both the radiative mechanisms and jet dynamics related to
GRBs. The data used in this work comes primarily from the Gamma-Ray
Burst Monitor (GBM) \cite{Meegan:2009} and Large Area Telescope (LAT)
\cite{Atwood:2009} onboard the {\it Fermi} space telescope. A key
advancement of this project over previous work is the use of physical
spectral models to fit the spectra of GRBs (see \chapterref{ch:phys}) which
by itself has helped to resolve some key problems in GRB spectroscopy.





%%% Local Variables: 
%%% mode: latex
%%% TeX-master: "../thesis"
%%% End: 

\chapter{Theories of Gamma-Ray Bursts}
\label{ch:theoryGRB}
\begin{chapterquote}{Good Luck}
The stars were exploding,\\
One by one as they flicker and they fall
\end{chapterquote}

\section{The Fireball Model}
\label{sec:fbm}
Gamma-ray Bursts are believed to originate from the release of a
massive amount of energy, $E_0$ in a small region of space, $ r_0$
leading to a relativistically expanding fireball
\cite{Cavallo:1978,Goodman:1986,Paczynski:1986}. The current model, as
detailed below, includes the formation of the GRB jet and its evolution, as
well as the charged particle acceleration processes that occurs and the
radiation generated by those particles.

\subsection{Jet Dynamics}
\begin{figure}[h]
  \centering
  \cfig{2}{gammaEvo.pdf}{4}
  \caption{The expansion phases and radii of an expanding GRB jet.}
\end{figure}


\subsubsection{Initial Explosion Energetics}
Assume that an explosion of a stellar mass object deposits an energy
$E_0 \approx 10^{50}-10^{52}$ ergs into a volume $\frac{4}{3}\pi
r_0^2$ \cite{Kobayashi:1999}. Taking the progenitor of the GRB to be
the collapse of either a super-massive star or the coalescence of two
compact objects, $r_0$ can be assumed to be the within an order of
magnitude of the Schwarzschild radius of a black hole. This explosion
occurs over a timescale of a few seconds. The Eddington luminosity,
\begin{equation}
  \label{eq:eddington}
  L_E\;\equiv\;4\pi G M m_p c/\sigma_T\;=\; 1.25\;\times\;10^{38}(M/M_{sun})\;erg\;s^{-1} %%% Fix this
\end{equation}
is the maximum luminosity at which radiation pressure and self-gravity
form a hydrostatic equilibrium \cite{rybicki:1979}. For the energies and
timescales above, the Eddington luminosity is greatly exceeded and radiation
pressure forces an expansion of the fireball.

The velocity of the expansion is governed by the specific entropy of
the fireball:
\begin{equation}
  \eta\;\equiv\;\frac{E_0}{M_0 c^2},
\end{equation}
where $M_0$ is the initial baryonic mass of the jet. This specific
entropy of the jet will be the maximum Lorentz expansion can obtain
(see \sectionref{sec:eoj}). If the fireball contains a high fraction
of baryonic mass, i.e., $\eta\approx\;1$, then the expansion will be
subrelativistic corresponding to a Sedov-Taylor expansion
\cite{Sedov:1946,Taylor:1950}. Conversely, a low baryon load will
cause a relativistic expansion. Observational evidence for
relativistic expansion exists due to the presence of high-energy
{\gray}s. For high photon densities, the photon-photon collisional
cross-section is large for high-energy photons. It is however, a
function of photon propagation angle. If the fireball expands
relativistically, the photons will be beamed which reduces the
photon-photon annihilation cross-section due to the small collisional
angle of the nearly parallel photons.. Therefore, the measurement of
high-energy {\gray}s provides strong evidence for relativistic
expansion and the baryon load is assumed to be small, i.e., $\eta\gg
1$.

An explosion which imparts an energy $E_0$ into a mass
$M_0\ll\frac{E_0}{c^2}$ within a radius $r_0$ results and a
relativistically expanding fireball is formed
\cite{Blandford:1976}. This kinetic energy input energizes the
particles in the outflow to have initial random Lorentz factors
$\gamma_0=\eta$. There are three relevant phases of the jet that will
be focused on; corresponding to specific radii of the outflow. These
include the acceleration phase, which occurs before the saturation
radius, $r_s$, the thermalization phase, corresponding with the
photospheric radius, $r_{ph}$, and the optically-thin coasting phase,
where non-thermal radiation is likely to occur at $r_{nt}$.

\subsubsection{Acceleration of the Jet}
\label{sec:eoj}
Due to the relativistic velocities and initial high densities
involved, the fireball initially expands adiabatically. Adiabatic
expansion occurs when a volume of fluid expands without heat exchange
with its environment, typically when the expansion happens rapidly
and/or when the fluid is thermally insulated from the surrounding
environment. The deposit of energy into the fluid is rapid and the
early stages of the jet are extremely dense, sufficiently insulating
the plasma. From thermodynamics, we know that a relativistically
expanding gas dominated by radiation has an adiabatic index of
$\gamma_{\rm a}=4/3$. It is convenient to work in the rest or comoving
frame of the jet where the volume expands isotropically and then
Lorentz boost into the observer frame at the end. I will use a prime
to indicate quantities in the comoving frame ($x^{\prime}$) from this
point forward. The comoving volume, $V^{\prime}$, and comoving
temperature, $T^{\prime}$ are related via $T^{\prime} \propto
V^{\prime 1-\gamma_{\rm a}}$, from simple thermodynamics. In the
comoving frame, $V^{\prime} \propto r^3$ \cite{Meszaros:1993} which
leads to $T^{\prime} \propto r^{-1}$. The particles in the outflow
have random Lorentz factors $\gamma^{\prime} \propto T^{\prime}$
giving their radial evolution as $\gamma^{\prime} \propto r^{-1}$ as
the jet expands. Therefore, we see that after the initial input of
energy, $E_0$, into the particles, the particles will cool during the
expansion and that energy accelerates the jet.

The bulk Lorentz factor, $\Gamma$, of the outflow must increase to
balance the cooling of the particles. To conserve energy, the kinetic
energy per particle and the bulk Lorentz factor must balance each
other, i.e.,  $\gamma\Gamma = {\rm constant}$ which instantly yields the
evolution of the fireball's bulk expansion velocity, $\Gamma \propto
r$. The acceleration of the fireball continues until $\Gamma = \eta =
E_0/{M_0 c^2}$. This is because there is no more energy is available
for expansion. The
outflow is said to saturate at this point and the radius is referred
to as the saturation radius, $r_{\rm s}$, after which $\Gamma$ coasts
at a constant value \cite{Paczynski:1986,Goodman:1986,Shemi:1990}.

\subsubsection{The Photosphere}
The energies and volumes involved in the fireball scenario imply a
high particle and photon density initially. These high densities imply
that the outflow is optically thick for some part of its evolution
\cite{Paczynski:1986,Goodman:1986,Shemi:1990}. Initially, the plasma
contains a large number of $e^{\pm}$ pairs that are in thermodynamic
equilibrium with the photons. These drop out of equilibrium very deep
within the jet. These pairs could lead to a photosphere but only if
$\Gamma\gg1000$. When these pairs "freeze-out" of the plasma, there
number is much less than pairs coupled to baryons in the flow
\cite{Goodman:1986}.  The baryons in the fireball carry $e^{\pm}$
pairs with them which serve as scattering centers for the
photons. Additionally, pairs can be produced by photon collisions
which can contribute to the overall opacity. If we consider the
outflow to be a wind of varying density as a function of radius then
the opacity properties can be assessed using relativistic fluid
equations \cite{Paczynski:1990}. The flow has a dimensionless entropy
$\eta=L/\dot{M}c^2$ and baryon density
\begin{equation}
  \label{eq:pdensity}
  n^{\prime}_p\;=\;\dot{M}/4 \pi r^2 m_p c \Gamma 
\end{equation}
which can be rewritten
\begin{equation}
  \label{eq:pdensity2}
   n^{\prime}_p\;=\;L/ 4 \pi r^2 m_p c^3 \eta \Gamma.
\end{equation}
The majority of the electrons in the flow at this point are those
which are are coupled to the baryons, i.e.,
$n^{\prime}_p=n^{\prime}_e$. To calculate the optical depth, the
electron density multiplied by the Thomson electron-photon scattering
cross-section, $\sigma_{\rm T}$, must be integrated along the line of
sight. The optical depth as a function of radius is
\cite{Paczynski:1990}:
\begin{equation}
  \tau(r)\;=\;\int_r^{\infty}n^{\prime}_e \sigma_{\rm T}\left(\frac{1-\beta_v}{1+\beta_v}\right)^{1/2}dr^{\prime}\;\simeq\;n^{\prime}_e \sigma_{\rm T}(r/2 \Gamma).
\end{equation}
Here, $\beta_v=v/c$ is the dimensionless velocity of the outflow. The
fireball becomes optically thin with $\tau\approx1$ (but see
\cite{peer:2008} for cases where the photosphere is not a constant
surface) at the photospheric radius, $r_{\rm ph}$. Solving for $r_{\rm
  ph}$ by setting $\tau=1$ yields
\begin{equation}
  \label{eq:rph}
 r_{\rm ph}\;=\;\frac{ L \sigma_{\rm T}}{8 \pi m_p c^3 \eta \Gamma^2}.
\end{equation}

The photosphere can occur above or below the saturation radius
depending on the value of $\eta$. High values of $\eta$ force the
photosphere below $r_s$. This work will focus on values of $\eta$ such
that $r_s<r_{ph}$. The temperature evolution in this region is simply
\cite{Meszaros:1993}
\begin{equation}
  \label{eq:ktevo}
  T^{\prime}\propto \left(\dover{r}{r_s}\right)^{-2/3}.
\end{equation}
When the jet becomes optically thin at $r_{ph}$, the photon emission
should be in the form of a blackbody.


\subsection{Non-Thermal Emission Region}
\label{sec:nter}
The vast majority of GRB spectra are observed to be non-thermal
\cite{Goldstein:2012}. Details of these observations will be discussed
in \sectionref{sec:spectrum}. The source and process of this
non-thermal emission is yet to be fully understood. To understand this
problem, not only do the radiative processes that produce the
$\gamma$-ray emission have to be identified, but also the mechanisms
that accelerated the emitting particles to non-thermal energies. For
this work, it will be assumed that the non-thermal emission occurs in
an optically thin regime, i.e. when $r>r_{ph}$. There are two
macrophysical proceess that serve to extract kinetic energy of the
outflow and convert it into radiation at these radii: internal shocks,
and external shocks. Magnetic reconnection is another process that can
extract energy from the flow, but it extracts magnetic energy and will
be discussed in \sectionref{sec:MR}.


\subsubsection{Internal  Shocks}
\label{sec:nter:is}
The observed non-thermal shape of the broadband emission of GRBs
implies a non-thermal distribution of the emitting electrons. The
commonly-invoked method for generating non-thermal electron
distributions is the Fermi process by which electrons are accelerated
to high non-thermal energies. The details of the particle acceleration
are discussed below in \sectionref{sec:pa::fp}. However, it is important to
discuss the properties of the jet that can lead to particle
acceleration. If it is assumed that random variations in $\Gamma$ form
in the early stages of the jet then these variations can lead to
internal shocks \cite{Rees:1994,Daigne:1998}. These variations form
from random changes in the mass loss rate, $\dot{M}$, and are of order
unity, i.e., $\Delta \dot{M}/\dot{M}\sim 1$ . The variations will be spatially separated by $c t_v$, where
$t_v$ is the variability timescale of the central engine. These stratified portions of the
wind or \emph{shells} will catch up with one another at a radius
$r_{nt} \sim c t_v \Gamma^2$. This can occur above or below $r_{ph}$
depending on the internal properties of the jet. For $r_{nt}<r_{ph}$
energy will be dissipated into the thermalizing electrons. This
scenario will be further detailed in \sectionref{sec:subpht}.

To understand the properties of internal shocks occurring above the
photosphere, consider two shells denoted by their masses and bulk
Lorentz factors $m_{\rm s}$, $\Gamma_{\rm s}$ and $m_{\rm f}$,
$\Gamma_{\rm f}$ where $\Gamma_{\rm f} > \Gamma_{\rm s}$. When the
faster shell catches up with the slower shell and inelastic collision
occurs creating a merged shell with a resulting Lorentz factor

\begin{equation}
\label{eq:intshock}
\Gamma_{\rm m} \sim \sqrt{\frac{m_{\rm f} \Gamma_{\rm f} + m_{\rm s} \Gamma_{\rm s}}{{m_{\rm f} /\Gamma_{\rm f}}+m_{\rm s}/ \Gamma_{\rm s}}}
\end{equation}
by way of conservation of kinetic energy and momentum
\cite{Daigne:1998}. The internal energy available for the electrons to
radiate or be accelerated to high-energies is given by the difference
of the kinetic energy before and after the collision:
\begin{equation}
  \label{eq:intEne}
  E_{\rm int}=m_{\rm f} c^2 (\Gamma_{\rm f}-\Gamma_{\rm m})+m_{\rm s}c^2(\Gamma_{\rm s}-\Gamma_{\rm m}).
\end{equation}
%%%% MORE HERE!!

While internal shocks present a viable option for extracting energy
from the outflow for radiation, several drawbacks exist that have
yet to be resolved. Foremost, internal shocks are extremely
inefficient at converting bulk energy into radiation. The efficiency
of the process is given by
\begin{equation}
  \label{eq:eff}
  {\rm eff} = 1 - \dover{(m_{\rm f} + m_{\rm s})}{\sqrt{m_{\rm f}^2+m_{\rm s}^2 +m_{\rm f}m_{\rm s}\left( \dover{\Gamma_{\rm f}}{\Gamma_{\rm s}} +  \dover{\Gamma_{\rm s}}{\Gamma_{\rm f}}   \right)}}
\end{equation}
which is only on the order of 5-20\% for a typical collision speed
\cite{Daigne:1998,Guetta:2001}. This range of efficiencies can be
modified by changing the value of $r_{nt}$ or more directly, the
initial variation in $\Gamma$ \cite{Spada:2000,Beloborodov:2003}.  The
extremely high luminosities observed in GRBs require an immense amount
of radiation to be generated and the low efficiency of the internal
shock process places severe limits on the radiation processes that
must occur.

\subsection{External Shocks}


An external shock occurs when the outflow material in the jet collides
with the external interstellar medium (ISM) or the layers of material
previously blown off by the progenitor star with density $\rho_{\rm
  ext}$. The external matter is swept up by the jet producing a blast
wave \cite{Rees:1992}. A shock is formed at the velocity discontinuity
and propagates forward with a Lorentz factor $2^{1/2} \Gamma$
\cite{Blandford:1976}.  The shock becomes important when the amount of
energy in swept up material is roughly equal to the energy of the
outflow:
\begin{equation}
  \label{eq:extShockE}
  E_0 \sim \dover{4 \pi}{3} \rho_{ext} \Gamma^2 r_{dec}^3,
\end{equation}
at a deceleration radius $r_{dec}$. The bulk velocity of the jet will
decrease to half its original value at this point and have swept up a
mass approximately $M_{ext}\sim M_0/\Gamma$ \cite{Rees:1992}. The
deceleration causes a reverse shock to form and propagate back into
the jet which dissipates energy to the particles in the outflow. This energy
is free to be radiated in the form of a GRB via synchrotron or
inverse-Compton processes.

External shocks have been shown to be not viable for generating the
prompt non-thermal radiation in GRBs due to several factors. Namely,
external shocks cannot recreate the short-timescale variability
observed in GRB lightcurves \cite{Sari:1997}. They are briefly
reviewed here for completeness.

% It has been suggested that a part of the prompt emission could be due
% to emission from an external shock \cite{}. The emission is in the
% form of the so-called early afterglow. The afterglow is x-ray emission
% that occurs after the prompt phase and can last for several days \cite{}. The
% source of afterglow emission is almost certainly an external shock \cite{}.



\section{Particle Acceleration Processes}
As noted in \sectionref{sec:nter}, the observed non-thermal shape of
GRB spectra require the acceleration of the radiating particles to
non-thermal energies. The two main physical processes that can
accelerate these particle in GRB environs are the Fermi process and
magnetic reconnection.  While the Fermi process has been studied and
simulated extensively, the framework of magnetic reconnection in the
context of GRBs is still poorly understood and its application is
typically only to alleviate problems arising from the Fermi
acceleration. Here the two processes are reviewed.
  


\subsection{The Fermi Process}
\label{sec:pa::fp}
First posed by Enrico Fermi \cite{Fermi:1949}, the Fermi process's use
in high-energy astrophysics is ubiquitous. The mechanism is the
central theoretical underpinning of cosmic-ray generation and was
fully developed in the 1970's
\cite{Axford:1981,Krymskii:1977,Blandford:1978,Bell:1978b,Bell:1978a}. The
theory was extended to the ultra-relativistic regime in the 1980's
\cite{Kirk:1989,Kirk:1987} and has been theorized as a viable
mechanism to accelerate charges in GRB outflows. Two forms of the
process exist, first- and second-order acceleration, referring to the
order of the particle velocity ($u$ and $u^2$) to which the energy
gain is proportional.

\subsubsection{First-Order Acceleration}
First order Fermi acceleration is a process by which fast charges in a
shock wave cross the shock boundary. A velocity discontinuity exists
between a fast and slow region of material, referred to as the upstream
and downstream regions respectively (see \figureref{fig:fermiaccel}).
\begin{figure}[h]
  \centering
  \cfig{2}{fermiaccel.pdf}{4}
  \caption{An illustration of the
    first-order Fermi process. Charges are reflected between the up
    and downstream regions, gaining energy proportional to $u$. }
\label{fig:fermiaccel}
\end{figure}
The relative velocity of the two regions is $u=\beta_v c$.  Assume
that the charges in the up and downstream fluid are initially in
thermal distribution. Suprathermal charges (charges with a high
velocity compared with the bulk of the distribution i.e., those in the
exponential tail of the distribution) can escape ahead of the shock
front, where they see converging scattering centers in the form of
other charges with a velocity relative to the escaped charge. These
escaped charges will be reflected back across the shock boundary by
the scattering centers where they will once again see converging
scattering centers in the new region. This cycle can occur many times,
each resulting in a systematic energy gain of $u/c$. Therefore,
first-order Fermi acceleration can result in particles gaining a
substantial amount of energy compared to their original thermal
distribution \cite{longair}. Relativistic, first-order acceleration
works generally in the same manner and results in charges being
energized into a high-energy power-law. It has been shown that the
index of the electron energy power-law is $\delta \approx 2.2 -2.3$
\cite{Ostrowski:2002,Waxman:1997,Bednarz:1998,Achterberg:2001}.

A key problem in first-order acceleration is how to have enough
suprathermal particles serving as a pool for acceleration to
sufficiently populate the observed non-thermal distribution. This is
known as the injection problem. In addition, the question of whether
shocks can generate in situ the conditions necessary for acceleration
to occur. Monte carlo simulations and analytic simplifications of both
relativistic and non-relativistic shock acceleration have relied on
placing shock conditions and magnetic fields in by hand to validate
the process
\cite{Baring:2012,Baring:2011,Baring:1995,Ellison:1990,Ellison:2004}.
The success of these efforts begs the question of whether the shock
conditions can be self generated.  Recent numerical simulation work
has attempted to address these problems and confirms that first-order
Fermi acceleration is possible in realistic, relativistic shocks
\cite{Spitkovsky:2008}.
\begin{figure}[h]
  \centering
  \cfig{2}{slowE.pdf}{4}
  \caption{Post-shock electron distribution consisting of a
    relativistic Maxwellian and a high-energy power-law tail.}
\end{figure}
The general form of the post-shock particle distribution is that of
relativistic Maxwellian with a high-energy power-law tail
\cite{Baring:2012,Baring:2011,Baring:1995,Ellison:1990,Ellison:2004,Spitkovsky:2008}:
\begin{equation}
  n_e(\gamma )\; =\; n_{0} \biggl\lbrack\;
  \Bigl( \dover{\gamma}{\gamth} \Bigr)^2\,
  e^{-\gamma/\gamth } + \epsilon \,
  \Bigl( \dover{\gamma}{\gamth} \Bigr)^{-\delta}\,
  \Theta \Bigl( \dover{\gamma}{\gammaMin} \Bigr)\, \biggr\rbrack\, .
  \label{eq:elec_dist}
\end{equation}
This distribution is crucial to this work because it serves as the
foundation for the physical modeling of GRB spectra that will be
applied to data.
\subsubsection{Second-Order Acceleration}
Second-order Fermi acceleration is a process by which charges are
reflected off magnetic turbulence, which can be conceptualized as
magnetic clouds moving in random directions with a mean velocity
$u=\lvert\vec{u}\rvert=\beta_v c$. The overall effect of these random
scatterings is a stochastic gain in energy proportional to $u^2$. The
energy gain is much smaller than the first-order process, but is still
relevant to GRB particle acceleration as an attempt to solve the
so-called fast-cooling problem (see \sectionref{sec:spec:lod,sec:ascp}). The process is
also much slower than first-order acceleration.

\subsection{Magnetic Reconnection}
\label{sec:MR}
While extensive analytic and numerical studies of the Fermi process
have had some success in attempting to explain the non-thermal spectra
of GRBs, particle acceleration via magnetic reconnection remains a
plausible theoretical avenue as well
\cite{zhang:2011,Meszaros:1997b,Meszaros:1994,Thompson:1994,Usov:1994}.
The progenitor of a GRB can have a strong initial magnetic field,
e.g., as in the collision of two highly-magnetized neutron stars. If
it is assumed that the GRB jet ejects a highly-magnetized or
Poynting-flux-dominated (PDF) outflow, then the observed $\gamma$-ray
emission could result from the release of this magnetic energy when it
is transferred to charged particles in the jet and then radiated away.

The main process for releasing stored magnetic energy is via magnetic
reconnection. The details of magnetic reconnection are not well
understood; however, the basic mechanism occurs when two magnetic
regions of opposite orientation approach each other and then reconnect
via the so-called Sweet-Parker process
\cite{Parker:1957,Sweet:1958}. This classical scenario occurs over a
timescale that is far too slow to explain the rapid variability seen
in GRB events. Recently, a new mechanism was proposed, and verified by
numerical simulations, in which reconnection can occur over a rapid
timescale in the presence of magnetic turbulence
\cite{Kowal:2009,Lazarian:1999}. The release of this magnetic energy
serves to accelerate particles non-thermally as well as to
introduce turbulence into the surrounding magnetic field.  This in
turn can serve to accelerate particles via the second-order Fermi
process.


\section{Emission Mechanisms}
\label{sec:emissionMech}
While it is critical to understand the processes that govern the
dynamics of the jet and energize the electrons, ultimately the
radiative mechanisms that produce the observed radiation are key to
understanding the observed $\gamma$-ray flux. A review of the relevant
non-thermal radiative processes follows. These include synchrotron and
inverse-Compton radiation.

\subsection{Synchrotron}
\label{sec:synctheory}

Electrons in the presence of a magnetic field ($\vec{B}$) are accelerated by the Lorentz force:
\begin{equation}
\label{eq:FL}
  \vec{F_L}=\dover{d}{dt}(\gamma_e m \vec{v})=q\left(\vec{E} +\dover{1}{c} \vec{v}\times \vec{B} \right)
\end{equation}
where $\gamma_e$ is the Lorentz factor of the electron
\cite{rybicki:1979,jackson:1998}. For the plasmas considered in this
work, the electric field ($\vec{E}$) is typically shorted out and
therefore $\vec{E}=0$. The force of the magnetic field on the particle
causes a gyration.  Assume that the electrons lose little energy per
gyration implying $\gamma_e\neq \gamma_e(t)$. With these assumptions
\equationref{eq:FL} can be rewritten
 \begin{equation}
   \label{eq:FL2}
   \dover{d \vec{v}}{dt}=\omega_B\left( \vec{v} \times \dover{\vec{B}}{B} \right)
 \end{equation}
where
\begin{equation}
  \label{eq:cycfreq}
  \omega_B = \frac{qB}{\gamma_e m_e c}.
\end{equation}
Accelerated charges emit radiation with power, P, calculated via the Lamour formula:
\begin{equation}
  \label{eq:Psync}
  P=\dover{2 q^2}{3 c^3}\gamma_e^2\omega_B^2v^2=\dover{4}{3}\sigma_{\rm T}c\beta_v^2\gamma_e^2\dover{B^2}{8 \pi}.
\end{equation}
This radiation
is called synchrotron radiation. 

In GRB studies, synchrotron radiation is theorized to be emitted by
the distribution of electrons, $n_e$, that have been accelerated in
the jet by one of the processes described above. To calculate the
synchrotron energy flux, $F_{\nu}$ (erg s$^{-1}$ cm$^{-2}$), emitted
by a distribution of electrons ($n_e(\gamma)$), the single particle
emissivity of synchrotron radiation \cite{rybicki:1979},

\begin{equation}
  \mathcal{F}\left(w\right) \; =\; w \int_w^{\infty } K_{5/3}(x) \, dx,
  \label{eq:synch_func}
\end{equation}
expressed here in dimensionless form must be convolved with $n_e(\gamma)$ over all energies, i.e., 
\begin{equation}
  F_{\nu}(\mathcal{E})\; \propto\; \int_1^{\infty} n_e(\gamma ) \, 
  \mathcal{F} \left( \dover{\mathcal{E}}{E_c}\right) \, d\gamma\quad .
  \label{eq:synch_flux}
\end{equation}
The quantity $E_c$ is called the characteristic energy of emission and
is defined
\begin{equation}
  \label{eq:wc}
  E_c(\gamma) = \frac{3 \gamma^2q \hbar B \sin\alpha}{2 m_e c}.
\end{equation}
With \equationref{eq:Psync,eq:synch_flux,eq:wc} the full synchrotron
spectrum can be derived for any distribution of electrons. Of
particular interest  due to its prevalence in nature and the fact
that Fermi acceleration generates a power-law
(see \sectionref{sec:pa::fp}), is the synchrotron spectrum from a power-law
distribution of electrons. Let
\begin{equation}
  \label{eq:plelec}
  n_e^{pl}=n_0(\delta -1)\gamma_{\rm min}^{\delta-1}\gamma^{\delta}, \gamma_{\rm min} \leq \gamma
\end{equation}
where $\delta$ is the electron spectral index of the power-law. The $\Fv$ synchrotron spectrum of from this distribution can be described asymptotically described by
\begin{equation}
  \label{eq:plsync}
  F_{\nu}(\mathcal{E})\propto\left\{
     \begin{array}{lr}
       \mathcal{E}^{1/3} & : \mathcal{E}\leq E_{\rm p}\\
       \mathcal{E}^{-\frac{1-\delta}{2}} & : \mathcal{E}>E_{\rm p}
     \end{array}.
   \right.
\end{equation}
with a peak energy calculated from \equationref{eq:wc},
\begin{equation}
  \label{eq:synchPeak}
  E_{\rm p} = \frac{3 \gamma_{\rm min}^2 q \hbar B \sin{\alpha}}{2 m_e c}.
\end{equation}
The 1/3 index below the peak of the spectrum holds for the majority of
physical electron distributions. However, electrons will lose a
significant amount of their energy via synchrotron radiation. This
cooling can alter the initial $n_e(\gamma)$ and therefore we now
consider the synchrotron spectrum from electrons that have been
significantly cooled.

The cooling of the electron distribution is inevitable when there is
no source of heating because synchrotron is a highly efficient
radiator. Adapting from the derivation of \cite{Burgess:2013}
(hereafter B13), assume that electrons distributed as in
\equationref{eq:plelec} are injected via and acceleration process into
a region where they are allowed to cool by the emission of synchrotron
radiation. The cooling of the electrons is governed by the continuity
equation \cite{Blumenthal:1970}
\begin{equation}
  \label{eq:komp}
  \frac{\partial n_e(\gamma,t)}{\partial t}+\frac{\partial}{\partial \gamma}[\dot{\gamma}n_e(\gamma,t)]+\frac{n_e(\gamma,t)}{t_{esc}}\;=\;Q_e(\gamma)
\end{equation}
where $n_e/t_{esc}$ represents the loss of particles from the emission
region from which we can define the maximal cooling scale
$\gamma/\dot{\gamma}\;\sim\;t_{esc}$ and $Q_e$ is the injection term. This corresponds to a Lorentz
factor, $\gamma_{\rm cool}$, below which cooling shuts off. For very high
electron Lorentz factors, $\gamma/\dot{\gamma}\;\ll\;t_{esc}$ and
therefore the loss term can be neglected, i.e., the dynamical
timescale is much longer than the radiative timescale. With this
assumption, we can simplify \equationref{eq:komp} to become
time independent. The resulting electron distribution can then easily
be solved
\begin{equation}
  \label{eq:simpKomp}
  n_e(\gamma,t)\;\approx\;\frac{1}{\dot{\gamma}}\int_{\gamma}^{\infty}Q_e(\gamma')d\gamma'.
\end{equation}
Substituting in the synchrotron cooling rate,

\begin{equation}
  \label{synchCool}
  \dot{\gamma}\;=\;-\frac{\pi}{3}\frac{r_0c}{r_g^2}\gamma^2,
\end{equation}
where $r_g\;=\;m_ec^2/(eB)$ and $r_e\;=\;e^2/(m_ec^3)$, \equationref{eq:simpKomp} yields the synchrotron-cooled broken
power-law distribution of electrons (see \figureref{fig:fastE})

\begin{equation}
  \label{eq:necool}
  n_e^{cool}\;\propto\;\frac{q_e\gammaMin}{\gamma^2}\;\min\left\{\left(\frac{\gamma}{\gammaMin}\right)^{-(\delta-1)},1\right\}\;,\;\gamma_{cool}\leq\gamma.
\end{equation}

Convolving this distribution with the single-particle synchrotron
emissivity yields an energy flux spectrum
\begin{equation}
  \label{eq:coolSynch}
   F_{\nu}^{\rm cool}(\mathcal{E})\propto\left\{
     \begin{array}{lr}
       \mathcal{E}^{1/3} & : \mathcal{E}\leq E_{\rm p}\\
       \mathcal{E}^{-1/2} & : E_{cool}<\mathcal{E}\leq E_{\rm p}\\
       \mathcal{E}^{-\delta/2} & : \mathcal{E}>E_{\rm p}{\rm p}
     \end{array}.
   \right.
\end{equation}
Here, $E_{\rm cool}$ is cooling energy of the energy flux spectrum corresponding to $\gamma_{\rm cool}$, i.e., 
\begin{equation}
  \label{eq:Ec}
  E_{\rm cool}= \frac{3 \gamma_{\rm cool}^2 q \hbar B \sin{\alpha}}{2 m_e c}.
\end{equation}

\begin{figure}[t]

  \centering
  \cfig{2}{fastE.pdf}{4}
  \caption{The resulting
    electron distribution resulting from the synchrotron cooling of a
    shock injected power-law electron distribution. The distance
    between $\gamma_{\rm cool}$ and $\gamma_{\rm min}$ is dependent on
    the amount of time the electrons have been allowed to cool.}
\label{fig:fastE}
\end{figure}

\begin{figure}[ht]
  \centering
  \cfig{2}{sync.pdf}{4}
  \caption{The energy flux spectrum of synchrotron  ({\emph red}) and fast-cooling synchrotron ({\emph blue}).}
\end{figure}

For this distribution of electrons (\equationref{eq:necool}), the
low-energy spectral index of the $\gamma$-ray emission is 1/2, which
is steeper than that of the un-cooled electrons. This difference in
emitted spectral indices will be crucial for interpreting the
observations of GRB spectra in order to determine models of emission.





\subsection{Inverse Compton}

The process of Compton scattering occurs when a photon inelastically
scatters with an relativistic electron and changes energy and
direction to conserve momentum \cite{rybicki:1979}. The inverse of
this process occurs when the electron has a high kinetic energy
compared to the scattering photon and some of this energy is transfer
ed to the photon. This process is called inverse Compton
scattering. This mechanism is of importance to GRBs because it is
possible that low-energy seed photons of energy $\mathcal{E}_s$ present in
the outflow from thermal emission are scattered to $\gamma$-ray
energies by electrons concurrently existing in the outflow. The seed
photons could even come from the synchrotron radiation generated by
the electrons themselves. This is called synchrotron self-Compton
emission.

Considering only inverse Compton in optically thin environs, the $\Fv$
spectrum can be calculated by convolving the single-particle
scattering kernel with the given electron distribution
\cite{rybicki:1979,Baring:2004}:
\begin{equation}
  \label{eq:ic}
  F_{\nu}^{\rm IC}(\mathcal{E}) = \dover{3 \mathcal{E}}{4}\sigma_{T}\int_0^{\infty}d\mathcal{E}_s n_{\rm ph}(\mathcal{E}_s)\int_{\gamma_{\rm min}}^{\infty}\gamma n_e(\gamma)\dover{f(z)}{4 \gamma^2 \mathcal{E}_s}
\end{equation}
where $n_{\rm ph}$ is the ambient photon field, and the function
\begin{equation}
  \label{eq:fz}
  f(z)=(2z+\ln z+z+1-2z^2)\Theta(z),\;\;z=\dover{\mathcal{E}}{4 \gamma^2 \mathcal{E}_s}
\end{equation}
is the angle averaged scattering kernel and the Heaviside step
function $\Theta(z)=1$ for $0\le z \le 1$ and $\Theta(z)=0$
otherwise. The high-energy portion of the spectrum is related to the
index of the electron power-law the same as with synchrotron
emission. However, the low-energy portion of the spectrum can be much
harder than with synchrotron, depending on the seed photon
distribution. For a mono-energetic photon source scattering on a
power-law electron source, the asymptotic energy flux spectrum can be
written
\begin{equation}
  \label{eq:icflux}
   F_{\nu}(\mathcal{E})\propto\left\{
     \begin{array}{lr}
       \mathcal{E} & : \mathcal{E}\ll \mathcal{E}_s \gamma_{min}^2\\
       \mathcal{E}^{-\frac{1-\delta}{2} }& : \mathcal{E}\gg \mathcal{E}_s \gamma_{min}^2
     \end{array}.
   \right.
\end{equation}
Inverse Compton emission of GRBs is attractive because no magnetic
field is required to be present in the jet, unlike synchrotron
emission. However, if the source of GRB non-thermal emission in the 10
keV - 10 MeV range is from inverse-Compton emission, then there should
be no observable synchrotron emission at optical wavelengths. This
contradicts many observations \cite{}.

If a magnetic field is present in the outflow, the electrons will
radiate synchrotron photons that the electrons can ``upscatter'' to
higher energies. The so-called synchrotron self-Compton emission has
been considered as viable source of GRB prompt emission
\cite{Freedman:2001,Spada:2000}. The difference between self-Compton
emission and mono-energetic inverse-Compton emission is the low-energy
spectral index which is 1/3, similar to synchrotron emission. However,
self-Compton and primary synchrotron emission differ in that the self-Compton spectrum has a much broader spectral curvature.


\subsection{Sub-Photospheric Dissipation}
\label{sec:subpht}
The possibility for shocks to form below the photosphere arises from
the fact that the range of outflow parameters allows for
$r_{nt}<r_{ph}$ \cite{Peer:2005}. To illustrate, assume that energy
dissipation occurs at $r>r_s \equiv \eta r_0$. The minimum radius that internal shocks can form is
\begin{equation}
\label{eq:ri}
r_i\approx 2\Gamma r_s.
\end{equation}
Assuming $L=10^{52}$ erg, and $r_0 = 6\times10^{6}$ cm corresponding
to the last stable orbit of a stellar mass black hole. Using
\equationref{eq:rph,eq:ri}, \figureref{fig:subph} shows the allowed
values of $r_{\rm ph}$ and $r_i$.
\begin{figure}[t]
  \centering
  \cfig{2}{subph}{4}
  
  
  \caption{The allowed radii of internal shocks as a function of
    $\Gamma$ are shown in the purple shaded region with the values of
    $r_{\rm ph}$ indicated by the red line. It is evident that shocks
    may form below the photosphere.}
\label{fig:subph}
\end{figure}
This implies that internal shocks can form and dissipate energy into
the electrons before the jet becomes optically thin. In this scenario,
the electrons are heated by the dissipated energy and a balance
between Compton and inverse-Compton scattering occurs. If a low-energy
photon component such as a blackbody exists in this region of the
outflow, it will be upscattered to higher energies by the electrons
that have been heated by the dissipation. This component could explain
the observed non-thermal emission in GRBs. However, this model has yet
to be able to explain the lightcurves and variability present in the
data.




%%% Local Variables: 
%%% mode: latex
%%% TeX-master: "../thesis"
%%% End: 
\chapter{Gamma-Ray Burst Observations and Modeling}

\label{ch:obsGRB}
\begin{chapterquote}{Edwin Hubble}
  Equipped with his five senses,\\ man explores the universe around him\\
  and calls the adventure Science.
\end{chapterquote}

\section{The {\it Fermi} Space Telescope}
In this work, GRB data from the {\it Fermi} Space Telescope were
analyzed both temporally and spectrally. {\it Fermi} consists of two
instruments, the Gamma-Ray Burst Monitor (GBM) and the Large Area
Telescope (LAT). Together, they cover an energy range from 10 keV to
300 GeV. This extensive bandpass allows for observing the entire
$\gamma$-ray spectrum both above and below the $\vFv$ peak. The data
are readily available via the Fermi Science Support Center (FSSC)
\cite{fssc}. A description of both instruments and their utilized data
types follows.
\subsection{GBM}
The GBM consists of twelve Sodium Iodide (NaI) and two Bismuth
Germinate (BGO) detectors that cover an effective energy range of 10
keV to 40 MeV. It continuously observes the non-Earth occulted sky
except when entering the region of high charged particle activity know
as the South Atlantic Anomaly (SAA). The instrument's main goal is to
trigger when an increase in $\gamma$-ray counts is detected signaling
a GRB event. The design and placement of the detectors around the
spacecraft enables a coarse localization ($> 1^{\degree}$) of GRBs. In
the event of an extremely bright detection, GBM can trigger an
Automated Repoint Request (ARR) that serves to point the LAT into an
orientation that allows for pointed, spatially-resolved, high-energy
emission observations.

Detection of {\gray}s is possible via their conversion of into optical
wavelengths via scintillation \cite{knoll}. The {\gray} interacts with
a scintillation crystal atom primarily by the photoelectric
effect, freeing an electron. The electron moves through the crystal
and loses energy as it excites the surrounding ions that dexcite by
emitting optical photons. These optical photons are measured by a
photomultiplier tube at the base of the crystal that converts their
signal into an electrical pulse proportional to the incident \gray's
energy.

The GBM is primarily a time-domain spectrometer. Three publicly
available data types from the GBM exist. These are CTIME, CSPEC, and
TTE. CTIME is a high time-resolution (0.256 s) binned data with low
energy resolution (8 channels). CSPEC is temporally-binned high energy
resolution (128 channels) with a lower time resolution (4.096 s) than
CTIME. TTE data is event data with time-tags and a channel energy for
each individual detected count. CSPEC and CTIME are both continuously
made during the spacecraft's orbit. Until recently, TTE was only
available during GRB triggers; however, as of 2013 a flight software
upgrade allows for the continuous generation of TTE data.




\subsection{LAT}
The LAT is a pair-conversion tracker that is primarily designed to
image $\gamma$-rays with energies $>10$ MeV. A pair-conversion tracker
operates by converting an incident $\gamma$-ray into an
electron-position pair ($e^{\pm}$) via a collision with a high-Z
material. The $e^{\pm}$ then moves in the same direction through the
tracker where its trajectory is tracked with Silicon strips until it
is finally absorbed in a calorimeter at the base. By reconstructing
the path of the $e^{\pm}$ and measuring the energy deposited in the
calorimeter, the direction and energy of the incident $\gamma$-ray can
be determined.

The effective energy range of the LAT is 100 MeV - 300 GeV using the
standard event type data. However, a new data type named the LAT Low
Energy (LLE) data extends the energy range down to $\sim$30 MeV
\cite{Vero:2010}. This data is made possible by collecting all counts in the
detector that pass basic cuts instead of rejecting counts that have
poor spatial and energy information. These poorly-measured events are
typically of lower energy. The side effect of accepting these events
is a high background; therefore, the LLE data is only viable for
analyzing temporally transient events such as GRBs or pulsars.



\section{Spectrum}
\label{sec:spectrum}
This focus of this work is on the prompt emission of GRBs. It is
necessary to draw a distinction between the prompt emission and the
so-called afterglow which occurs after the primary high-energy
emission. Discussion of the afterglow is beyond the scope of this
project. To date, over 3000 GRBs have been observed by both BATSE and {\it
  Fermi}. Several catalogs detail their spectral, temporal, and
intensity properties
\cite{Goldstein:2012,Kaneko:2006,Nava:2011}. These catalogs identify
several key features of the GRB population including their subdivision
into two classes based on duration \cite{ck:1993}, the grouping of
their respective $\vFv$ peak energies \cite{Schaefer:2003}, and their
distribution of spectral indices. Key to all these findings are the
 parameters found via spectral analysis. The distribution of
photons that make up the observed spectra of GRBs is one of the main
clues to understanding the physical mechanisms behind GRB emission.
\subsection{The Band Function}
\label{sec:bandfunc}
Because of the inversion problem (see \sectionref{sec:ff}), it is impossible to directly assess
the shape of the photon spectrum. The shape of the typical
GRB spectrum is curved and asymptotically approaches a power-law at
high and low energies. A comprehensive study of empirical photon
models fit to GRBs in the BATSE data found that a specialized function
was able to fit the majority of time-integrated and time-resolved
spectra \cite{band:1993}. The Band function is a smoothly broken
power-law connected exponentially at the $\vFv$ peak.
\begin{equation}
  \label{eq:band}
   \Fv(\mathcal{E})\;=\;F_0
  \begin{cases}
    {\left(\frac{\mathcal{E}}{\rm 100\;keV}\right)^{\alpha} {\rm exp}\left(-\frac{(2+\alpha)\mathcal{E}}{E_{\rm p}}\right)} & {\mathcal{E}\le (\alpha-\beta)\frac{E_{\rm p}}{ (\alpha+2)} }\\
    {\left(\frac{\mathcal{E}}{\rm 100\;keV}\right)^{\beta}{\rm
        exp}{\left(\beta-\alpha\right)}\left[\frac{(\alpha-\beta)E_{\rm
            p}}{{\rm 100\;keV} (2+\alpha)}\right]^{\alpha-\beta}} &
    {\mathcal{E}>(\alpha-\beta)\frac{E_{\rm p}}{ (\alpha+2)} }
  \end{cases}
\end{equation}
\begin{figure}[t]
  \centering
  \cfig{3}{bandExp}{5}
  \caption{The $\vFv$ spectrum of the Band function illustrating its
    three important shape parameters: $\alpha$, $\beta$, and $\Ep$
    (see \equationref{eq:band}).}
  \label{fig:bandSpecEx}
\end{figure}




\begin{figure}[t]
  \centering
  \cfig{3}{band}{5}
  \caption{Demonstrating the variety of Band function shapes
    corresponding to differing values of $F_0$, $\alpha$, $\beta$, and
    $\Ep$ (see \equationref{eq:band}).}
  \label{fig:bandSpec}
\end{figure}
Nearly all GRBs detected by {\it Fermi} and BATSE can be fit with the
Band function. Though the function is empirical, attempts have been
made to relate its fit parameters to physical quantities. The most
important association was made by \cite{preece:1998} with the
so-called synchrotron 'line-of-death'. As noted
in \sectionref{sec:synctheory}, the low-energy slope (Band's $\alpha$
index) of a synchrotron photon spectrum is -2/3. It found that nearly
1/3 of GRB spectra had $\alpha$'s greater than -2/3 indicating that
they were inconsistent with synchrotron emission
(see \sectionref{sec:spec:lod} for a detailed discussion).

\begin{figure}[h]
  \centering 
\subfigure{\cfig{3}{epDist.pdf}{2.}}\subfigure{\cfig{3}{alpDist.pdf}{2.}}\subfigure{\cfig{3}{betaDist.pdf}{2.}}
\caption{Band function fit parameters from the first two
  years of GBM data.}
  \label{fig:catparam}
\end{figure}
Similar associations with other emission mechanisms have been
made. Much research has focused on finding an emission mechanism that
can correctly account for the Band $\alpha$ distribution
\cite{preece:1998,Beloborodov:2010,Daigne:2011,piran:2013}. To date,
no one model can account for the entire parameter space. Similarly,
E$_{\rm p}$ can be associated with radiative mechanisms through their
respective physical $\vFv$ peaks e.g. see
\equationref{eq:synchPeak}. The physical relation to $\beta$ is
typically made to the power-law index of the emitting electron
distribution.
\subsection{The Blackbody Component}
With few exceptions \cite{gonzalez:2003,abdo+GRB090902B}, the spectra
of GRBs were found to consist of only one broadband component in the
BATSE era and the early {\it Fermi} era. The overall non-thermal shape
of GRB spectra fit with the Band function left little hope for finding
the thermal signature of a blackbody predicted by the basic fireball
model. However, after the launch of {\it Fermi}, several GRBs were
satisfactorily fit with a two component model consisting of the Band
function and a blackbody
\cite{guireic:2010,Axelsson:2012,guiriec:2013}. The component was
shown to be statistically significant. When fit along with a
blackbody, the Band component of the spectra appeared to be more
consistent with synchrotron than before with Band's $\alpha$ moving
closer to the expected value of the low-energy index.

\begin{figure}[h]
  \centering
  \cfig{3}{vFv.pdf}{4}
  \caption{The BATSE
    era blackbody, which was fit to the entire spectrum is the same as
    the GBM era blackbody due to the difference in bandpass of the
    instruments.}
  \label{fig:batseBB}
\end{figure}


Prior to the discovery of the blackbody component existing in
combination with the Band function, several works studied the
existence of blackbodies in BATSE GRB spectra
\cite{Ryde:2009,Ryde:2006,Ryde:2005,Ryde:2004}. These studies found
that the entire GRB spectrum consisted of a blackbody or a blackbody
and a power-law extending into high energies. The evolution of this
blackbody component was studied extensively. It was found that the
temperature of the blackbody decayed as a broken power-law in time
with the index after the power-law break was typically $\sim-2/3$. A
post analysis of this component reveals that it is the same blackbody
found in {\it Fermi} GRBs. This is because BATSE had a limited
high-energy response above 2 MeV (see
\figureref{fig:batseBB}). Therefore, BATSE could only detect the
low-energy blackbody plus the low-energy power-law of the Band
function.

\subsection{The Synchrotron Line-of-Death and the Fast Cooling Problem}
\label{sec:spec:lod}
The distribution of Band $\alpha$ has been studied extensively in an
attempt to understand the underlying emission mechanisms. As shown
in \sectionref{sec:emissionMech}, the low-energy index of a
$\gamma$-ray spectrum is unique for many models. Synchrotron emission
from internal shocks is the simplest and most widely invoked model for
explaining GRB observations, however, the 'line-of-death' problem
presents a challenge to the theory.

\begin{figure}[h]
  \centering
  \cfig{3}{lod.pdf}{4}
  \caption{The $\alpha$ distribution
    of GBM detected GRBs illustrating the 'line-of-death' problem. The
    fast-cooling line (\emph{blue}) and slow-cooling line (\emph{red})
    are superimposed on the distribution.}
  \label{fig:lod}
\end{figure}
GRBs fit with the Band function possessing $\alpha$'s greater than
-2/3 are presumably inconsistent with synchrotron emission. Even more
problematic is that nearly \emph{all} GRBs have $\alpha$'s greater
than -3/2, the fast-cooling synchrotron index. Due to the time scales
involved, the electrons in GRBs must be in the fast-cooling regime. If
the cooling timescale argument is ignored then the problem of
efficiency forces the requirement of fast cooling for the internal
shock model. The low dissipation efficiency of internal shocks
(5-20\%) is the upper limit of the radiative efficiency, i.e., if the
radiative efficiency is 100\%, then only 5-20\% of bulk kinetic energy
of the jet can be radiated in the form of {\gray}s. Slow-cooling
electrons are much less efficient than those that are being quickly
cooled by synchrotron; therefore, the internal shock model requires
fast-cooling to be radiatively efficient. This is the so-called
fast-cooling problem.



\subsection{Spectral Evolution}
\label{sec:spec:evo}
The evolution of spectral parameters in GRBs is useful in identifying
the evolution of the jet structure and/or the evolution of the
magnetic field during the outflow. The most well studied parameter
evolution is the that of $\Ep$
\cite{Medvedev:2006,Liang:1997,Liang:1996,golenetskii:1983,Ryde:2001}. While
there is no universal evolution of $\Ep$, many GRBs exhibit the
so-called hard-to-soft evolution, i.e., the monotonic evolution of
$\Ep$ from an initially high value to a lower value. The physical
explanation of this evolution is not fully understood and will be
addressed in \sectionref{sec:results:hec,sec:fluxfrx}. Accompanying hard-to-soft evolution is the
correlation of $\Ep$ with $\Fv$ in many bursts. This correlation is
most prominent in single-pulsed GRBs. This hardness-intensity
correlation (HIC) is also not fully understood but should be related
to the radial change in the GRB jet parameters \cite{preece:2013}.

\section{Lightcurves}

\begin{figure}[h]
  \centering
  \cfig{3}{lc.pdf}{6}
  \caption{A sample of GBM lightcurves
    demonstrating the diversity of pulse shape and complexity
    \cite{valerieSite}. For this study, the concentration will be on
    single pulsed GRBs such as GRB 110605183 (above).}
  \label{fig:gbmlc}
\end{figure}

The varied nature of GRB lightcurves is impossible to describe
categorically (see \figureref{fig:gbmlc}). There are; however, a
subclass of lightcurves that have been studied intensely
\cite{Ruiz:2000,Fenimore:1996,Dermer:2004,Norris:1999}. These are
those with a fast rise and exponential decay (FREDs) in time. GRBs with well
separated FRED pulses have very similar properties in their associated
spectral evolution. The FRED shape has not been fully explained by
theory. The simplest explanation comes from special relativity. If a
spherical source emitting photons isotropically expands
relativistically, then it can be shown that the observer from pulse
resembles that of a FRED \cite{Rees:1966}. Studies of FRED pulses have
shown that this simple model cannot account for all FRED shapes
\cite{Kocevski:2003}. The relation of the rise and decay times of
pulses should give an indication of the emission model of GRB pulses
\cite{Kocevski:2003}; however, no unique solution for the association
of pulse shape has been established.

For this work, the focus will be on GRBs with single, separable pulses
that have FRED-like shapes. These GRBs have been shown to have simple
and clean spectral evolution \cite{Ryde:2009} which is imperative for the study
of physical emission models.

\section{Simulations of GRB Emission}
\label{sec:sims}
In order to understand the observations and their relation to theory,
several theoretical models have been numerically simulated and
compared to observed spectra and lightcurves
\cite{Daigne:1998,Daigne:2009,Asano:2009,Pana:1998,Chiang:1999,Cannizzo:2004,Peer:2005,Kobayashi:1997}. The
simulations typically attempt to assess either the dynamic evolution
of the jet or the emission of the electrons that have been accelerated
and their evolution. Some simulations attempt to address both
properties. The internal shock model has received the most attention
via simulation due to its relative simplicity compared to other
models.

\subsection{Simulations of the Internal Shock Model}
To simulate the internal shock model, the dynamics derived
in \sectionref{sec:nter:is} must be numerically calculated for a large
number of emitted shells. The first attempts to do so were simplistic
and attempted to simulate the shape and variability of the typical GRB
lightcurve only \cite{Kobayashi:1997}. These studies showed that the
internal shock model could reproduce the observed lightcurve shapes
but gave no information about the spectra. The next set of simulations
enhanced these results by including synchrotron emission and then
later a full radiative code that included all radiative processes
relevant to GRB emission \cite{Daigne:1998,Daigne:2009}.

These simulations consist of a large number of shells of varying mass
and $\Gamma$. They are emitted one after another with faster shells
coming after the slower shells in time. This forces a collision where the
dissipated energy is calculated from \equationref{eq:intEne}. This
dissipated energy is then used to numerically calculate an accelerated
electron distribution that cools via a radiative code that
numerically calculates:
\begin{equation}
  \label{eq:coolingcode}
  \dover{\partial n_e}{\partial t^{\prime}}= - \dover{\partial}{\partial \gamma}\left[\dover{\mathrm{d}\gamma}{\mathrm{d} t^{\prime}}\bigg|_{{\rm sync} + {\rm ic}}n_e(\gamma,t^{\prime})\right]
\end{equation}
As the electrons cool, the radiation spectrum is calculated from the
different radiative emissivities. The radiative codes used in these
simulations are simple to increase the speed of computation due to
the large number of shell collisions involved. Once the radiation
spectrum is calculated in the GRB jet rest frame, the emission is
transformed into the observer frame. The total emission from the jet
is then summed together forming a lightcurve. These lightcurves and
their associated spectra have been shown to reproduce many of the
observed features of GRBs including hard-to-soft evolution, Band-like
shape, and lightcurve shape. This success has helped to answer many
questions about viable mechanisms for producing GRBs. However, these
simulations rely on many processes that are put in by hand and not
self-consistent. These include the particle acceleration and radiative
timescales. While these simplifications aid in decreasing CPU time,
they neglect key questions which must be self-consistently validated
in order to fully assess the viability of the internal shock model.

\subsection{Full Radiative Codes and Sub-Photospheric Dissipation}
A fully self-consistent radiative code can calculate the evolution of
accelerated electrons as they cool from the various emission
mechanisms relevant to GRBs. The most complete radiative code in the
literature is that of \cite{Peer:2005}. In these simulations of GRB
emission, an injected electron distribution is tracked as it cools in
a very detailed manner. Different radiative processes occur at
different timescales making the calculations numerically challenging
and CPU intensive. Therefore, this code neglects the GRB jet dynamics
implemented in \cite{Daigne:2009}. Still, these simulations show that
the evolution of the electron distribution is highly complex and the
simplifications made in \cite{Daigne:1998,Daigne:2009} neglect
important aspects of the radiative process. Ideally, a fully detailed
radiative code should be coupled with a full jet dynamics code to
fully understand the internal shock model. The simulation collects the
associated radiation from the electrons and computes the observed
spectra and its evolution. These spectra can then be compared to data.

One key benefit of this full radiative code is that it can simulate
the evolution of electrons when the particle densities imply a high optical
depth. This allows for testing of the sub-photospheric dissipation
model. It has been shown that the spectra produced by this model can
have Band-like shapes. However, due to the limitations discussed above
no lightcurve has been produced to show that the model is fully
consistent with observations.

\subsection{Simulations of PFD and Magnetic Reconnection in GRBs}
Due to the lack of theoretical understanding of magnetic reconnection
and PFD models, very little simulation work has been done in the field
of GRBs to test their viability as an emission mechanism. No lightcurves or spectra have been numerically simulated and
therefore, no evaluation of the validity of these models can be
made. The association of these models to data has been made purely on
considerations of timescales and energy requirements that fall into
the expected GRB regime but detailed simulations must be carried out
to validate these models fully.




% \section{Tables---Kinda like Figures in Reverse}

% In the UAH format, table require ``titles'' (where figures require
% ``captions'').  So, all that really means is that you put the
% \verb|\caption{Table Title}| \textsl{before} you actually begin the
% table (but, of course, you must be in the table environment).  For
% example, this \TeX:
% \begin{verbatim}
% \begin{table}
% \begin{center}
% \caption{This is a Table Made with the Booktabs package}
% \label{tbl:table}
% \begin{tabular}{@{}lccc@{}}
% \toprule \rule[-1pt]{0pt}{14pt}Title 1-Left&Centered&Centered Again&
% Centered\\
% \midrule \rule[-1pt]{0pt}{14pt}Item 1&These are all&separated
% by&Ampersands
% \verb|&|\\
% \rule[-1pt]{0pt}{14pt}Math Works too&$E=mc^2$&$F=ma$&see?\\
% \bottomrule
% \end{tabular}
% \end{center}
% \end{table}
% \end{verbatim}

% \noindent produced the following table, \tableref{tbl:table}.
% \begin{table}
% \begin{center}
% \caption{This is a Table Made with the Booktabs package}
% \label{tbl:table}
% \begin{tabular}{@{}lccc@{}}
% \toprule
% \rule[-1pt]{0pt}{14pt}Title 1-Left&Centered&Centered Again&
% Centered\\
% \midrule
% \rule[-1pt]{0pt}{14pt}Item 1&These are all&separated by&Ampersands
% \verb|&|\\
% \rule[-1pt]{0pt}{14pt}Math Works too&$E=mc^2$&$F=ma$&see?\\
% \bottomrule
% \end{tabular}
% \end{center}
% \end{table}

% These files are set up such that \LaTeX\ will put figures and tables
% (which are collectively called ``floats'') at the top of the first
% available page (as near to where you inserted the figure or table
% environment as possible). You can change that (\ie, force \LaTeX\ to
% put the floats in different locations).  But, I prefer it simple.

%%% Local Variables: 
%%% mode: latex
%%% TeX-master: "../thesis"
%%% End: 
\chapter{Physical Model Fitting of GRB Spectra}
\label{ch:phys}
\begin{chapterquote}{Edwin Hubble}
  Equipped with his five senses,\\ man explores the universe around him\\
  and calls the adventure Science.
\end{chapterquote}
\section{Problems with the Band Function}
While the Band function has been successful at categorizing the
majority of detected GRB spectra, several limitations and problems
arise from the use of empirical models when fitting data. The Band
function has suggested a viable, comprehensive physical origin since its
canonical use in the fitting of GRBs. The first problem comes from the
fact that GRB spectra are fit with the forward-folding method. This
method assumes a model a priori and then convolves this model with the
detector response matrix to produce a count spectrum that is fit to the observed count
data. Therefore, the initial assumption of a photon model limits the
allowed shape parameters that can be tested because the photon model
is essentially imprinted on the data from the start. The Band function
approximates many non-thermal photon models, but it is not exact. The
curvature of the Band function around the $\vFv$ peak of the spectrum
is fixed by its spectral shape and differs from the curvature of
actual models such as synchrotron. The association of the Band
function parameters with physical models has typically been via the
low-energy $\alpha$ index. However, this association neglects the
curvature of the physical model. A comparison of Band shapes to
different physical models is shown in \figureref{fig:bandCompPhys}.
\begin{figure}[h]
  \centering
  \subfigure{\cfig{4}{bandMods.pdf}{2.95}}
  \subfigure{\cfig{4}{mods.pdf}{2.95}}
  \caption{A comparison a physical photon spectra (\emph{left}) and
    their associated Band function approximations (\emph{right})
    demonstrates the problems of using the Band function parameters to
    infer a physical emission model from observed spectra.}
  \label{fig:bandCompPhys}
\end{figure}
The varying curvature of these models
can result in the false association of Band $\alpha$ to a physical
origin depending on where the Band $\vFv$ peak falls with respect to
the curvature of the physical model. 

The danger of relating Band to physical models becomes very apparent
when comparing fast and slow-cooling synchrotron models. Slow-cooling
predicts $\alpha=-2/3$ while fast-cooling predicts $\alpha=-3/2$. If a
spectrum is fit with the Band function and an $\alpha\sim -3/2$ is
measured, the conclusion that the spectrum results from fast-cooling
synchrotron could be made in error. This is because the $\vFv$
curvature of the Band function is much narrower than that of the
fast-cooling synchrotron even when they possess the same low-energy
index. These issues imply that a new method for assessing the physical
origin of GRB spectra is required. The approach in this work is to fit
numerical physical models directly to the GRB count data to make a
direct association of the data to a model.

\section{Historical Fitting of Physical Models to Data}
The first attempt to fit GRB data with physical models was made by
\cite{Baring:2004} (hereafter, BB04). In this work numerical
evaluations of physical emissivities convolved with an electron
distribution resulted in photon models that were fit to GRB photon
spectra that had been deconvolved with the Band function (see
\figureref{fig:bb04}). While enlightening, this study suffered from
the limitation that the photon data that was fit had been deconvolved
with an empirical function first. This means that the data already
had the shape of the Band function imprinted on it. Therefore,
incorrect conclusions were drawn about the validity of these physical
photon models. Since the Band function's curvature was imprinted on
the data, the broader synchrotron curvature was unable to fit the data
without converging on non-physical values for the electron
distribution. The solution to this problem is to implement physical
models directly into the forward-folding scheme and is the main focus
of this work.

\begin{figure}[h]
  \centering
  \cfig{4}{bb04.pdf}{5.5}
  \caption{Examples of physical model fitting to deconvolved BATSE $\vFv$
    spectra from \cite{Baring:2004}.}
  \label{fig:bb04}
\end{figure}

\section{Numerical Physical Fitting in RMFIT}
For all spectral fitting in this work, the publicly-available analysis
package, {\tt RMFIT} \cite{rmfit} was used. {\tt RMFIT} is a
combination of IDL and FORTRAN code that is able to read GBM and LAT
data and fit photon models to the counts data. The graphical interface
is written in IDL and the fitting engine, {\tt MFIT}, is in
FORTRAN. {\tt MFIT} utilizes a Levenberg-Marquardt minimization
routine \cite{numrecipes} to minimize a chosen fitting statistic
(e.g. {\chisq}, log-likelihood) to optimize the spectral parameters of
a given photon model to the data. It is optimized to work with GBM
counts data and was chosen over the fitting package {\tt XSPEC}
\cite{xspec} because of its close association with the GBM data
types. {\tt RMFIT} contains several photon fit models but lacks the
physical models required for this study. Therefore, custom FORTRAN
modules had to be designed to enable the fitting of these models to
data.

\subsection{Physical Models in MFIT}
\label{sec:physmod:rmfit}
To enable the fitting of physical models in {\tt RMFIT}, the source
code of {\tt MFIT} was extended to contain the following models:
\begin{itemize}
\item slow-cooling synchrotron
\item fast-cooling synchrotron
\item inverse-compton from a mono-energetic seed source
\item synchrotron self-compton.
\end{itemize}
For all models, numeric integration was required. To add this
functionality into {\tt MFIT} the numeric integration routines of the
GNU Scientific Library (GSL) \cite{gsl} were added to the {\tt MFIT}
source code. These routines were written in C, and therefore, a
FORTRAN/C interface code was designed that allowed for the passing of
variables between GSL and {\tt MFIT}.

For each photon model, the emitting electron distribution must be numerically
evaluated. For all models except the slow and fast-cooled synchrotron model a
power-law was chosen for the electron distribution. For the
slow-cooled synchrotron model, the distribution of
\equationref{eq:elec_dist} was used, consisting of a relativistic
Maxwellian with a high-energy power-law tail. The fast-cooled electron distribution (\equationref{eq:necool}) was used for fast-cooled synchrotron. The electron
distribution is then numerically convolved with single particle
emissivity of the selected photon model. The functional form of the
photon models are then added to the list of the {\tt MFIT} photon
models. When {\tt MFIT} calls the models for parameter optimization, a
numerical evaluation of the model is made at each energy bin of the
counts data. The parameters of the photon model are compared with the
data and the process is iterated until the fitting statistic converges
to a minimum. This process is very CPU intensive compared to fitting
empirical models.

The fit parameters of each photon model are listed here:
\begin{itemize}
\item slow-cooled synchrotron
  \begin{itemize}
  \item total normalization (A)
  \item power-law normalization ($\epsilon$)
  \item thermal electron Lorentz factor ($\gamth$)
  \item power-law electron Lorentz factor ($\gammaMin$)
  \item electron spectral index ($\delta$)
    \item peak energy ($\Ep$)
  \end{itemize}
\item fast-cooled synchrotron
  \begin{itemize}
  \item total normalization (A)
    \item power-law electron Lorentz factor ($\gammaMin$)
  \item electron spectral index ($\delta$)
    \item peak energy ($\Ep$)
  \end{itemize}
\item mono-energetic inverse-Compton
  \begin{itemize}
  \item total normalization (A)
    \item power-law electron Lorentz factor ($\gammaMin$)
  \item electron spectral index ($\delta$)
    \item peak energy ($\Ep$)
  \end{itemize}
\item synchrotron self-Compton
  \begin{itemize}
  \item total normalization (A)
    \item power-law electron Lorentz factor ($\gammaMin$)
  \item electron spectral index ($\delta$)
    \item peak energy ($\Ep$)
  \end{itemize}

\end{itemize}
These physical models are very rigid and difficult to fit to the data
compared to the Band function. The fitting engine will often fail
completely when fitting models that are very different from the
data. This is the case for the inverse Compton and self-Compton
models. For this reason, they will be left out of the discussion and
we will focus on the synchrotron based models. 

An important difficulty arises when trying to fit these models to the
data. The number of free parameters greatly exceeds the number of
parameters that can be simultaneously constrained. To alleviate these
problems, the models have to be formulated such that the degeneracies
in the shape parameters do not exist. This reformulation is detailed
for the synchrotron model in \appendixref{ch:degen}.


%%% Local Variables: 
%%% mode: latex
%%% TeX-master: "../thesis"
%%% End: 
\chapter{GRB Spectral Analysis}
\label{ch:analysis}
\begin{chapterquote}{Christopher Hitchens}
That which can be asserted without\\ evidence,
 can be dismissed without\\ evidence.
\end{chapterquote}

The analysis of {\it Fermi} GRB spectra via physical models will be
performed using the aforementioned RMFIT. The description of the
analysis will be broken into several sections to investigate different
aspects of GRB physics. In \chapterref{ch:grb090820A}, I will test the
validity of physical model fitting and preliminary implications of the
spectral fits. In \chapterref{ch:pap2}, group properties from a large
sample of GRBs will be analyzed to obtain a better understanding of
the emission mechanisms and assess the structure of the GRB
jet. Finally, \chapterref{ch:130427A,ch:cor} contain an analysis of
spectral properties to build a deeper understanding of the GRB jet
properties and evolution. The process of spectral analysis for all of
these test is similar and will therefore be described in this chapter.




\section{Sample Selection}
The study of physical models in GRB spectra places severe limits on
the sample of GRBs that can currently be studied. They must be bright
so that the time-resolved spectra have enough counts to enable the
fitting engine to constrain the model parameters. In addition, the
physical models that are implemented here are derived for single zone
emission, i.e., if two overlapping pulses are fit with one of these
models, then it is highly unlikely the results will be valid because
the photon spectra may contain emission from two event zones. For this
reason only GRBs with single or non-overlapping pulses are
chosen. There are very few GRBs in the {\it Fermi} dataset that have
either property. Using a list of the brightest {\it Fermi} GRBs, a
single-pulsed subset was selected. These pulses were categorized as
single-pulsed by having a monotonic decrease in their bolometric count
rate after the peak intensity of the pulse.

The GRBs in the this study are:
\begin{itemize}
\item GRB 081110A
\item GRB 081224B
\item GRB 090719A
\item GRB 090809A
\item GRB 090820A
\item GRB 100707A
\item GRB 110721A
\item GRB 110920A
\item GRB 130427A
\end{itemize}
The GRBs will all be analyzed in a similar way with two exceptions;
GRB 090820A will serve as a preliminary diagnostic to test the ability
of fitting physical models to the data and GRB 130427A, due to its
brightness and measured redshift will be analyzed in fine detail to
explore the origins of the observed HIC.

\subsection{Time Binning}
\label{sec:tbin}
The selection of time bins for time-resolved spectral analysis has no
unique solution. Ideally, time bins should be as small as possible to
capture the nuances of the evolution of the spectrum in as much detail
as possible. However, the selection of very short time bins decreases
the number of counts in the signal thereby reducing the ability to
constrain fit parameters. A balance between resolution and signal
strength must be arbitrarily made by the observer. There are three
methods for determining time bins in GRB spectral analysis:
\begin{itemize}
\item constant time bin width
\item constant signal-to-noise ratio (S/N) binning
\item Bayesian blocks.
\end{itemize}
Constant time bin width binning is simply that bins are chosen with
the same $\Delta t$ throughout the burst. This method is systematic
and unambiguous though no guarantee can be made about the S/N being
large enough to accommodate spectral analysis. Additionally, the bins
can arbitrarily split or combine physical changes in the
spectrum. Constant S/N binning combines spectra by summing signal and
background counts starting at the beginning of a bin until a desired
S/N is achieved. This method guarantees that the desired number of
signal counts are present in each bin for spectral analysis. However,
the method can arbitrarily sum together significant changes in the
spectral evolution similar to constant time bin width binning. This
presents a significant problem
\begin{figure}[h]
  \centering
  \subfigure{\cfig{5}{constwidth.pdf}{3.2}}\subfigure{\cfig{5}{snbin.pdf}{3.2}}
\subfigure{\cfig{5}{bblc.pdf}{3.2}}
\caption{Example time binnings demonstrating the differences between
  constant time width (\emph{left}), S/N (\emph{right}), and Bayesian
  blocks (\emph{middle}).}
  \label{fig:binmeth}
\end{figure}
for the study of physical emission models because it is important to
map changes in the lightcurve to physical changes in the GRB.  For
this work, the method of time binning chosen is Bayesian blocks
\cite{Scargle:2013}. The Bayesian blocks algorithm selects time bins
by looking for significant changes in count intensity. The
significance level set is determined by a chosen Bayesian prior
($n_{\rm cp}$) that reflects the observer's knowledge about the number
of pulses in the count rate. It was found that setting $n_{\rm cp}=8$
was suitable for the study of single-pulsed GRBs
\cite{Scargle:2013}. The resulting time bins are guaranteed to have a
uniform Poisson count rate with any subdivision of the bin up to
the selected significance level. A key assumption in relying on
Bayesian blocks is that the changes in the Poisson count rate of the
source reflect physical changes in the GRB jet.



\subsection{Source Selection}
A benefit of Bayesian block binning is that a temporal source region
can easily identified. The local background is basically a constant
rate and can be identified as long time bin before and after the
pulse.
\begin{figure}[h]
  \centering
  \cfig{5}{bblcsource.pdf}{4}
  \caption{An example of background and source identification in a
    Bayesian block binned lightcurve.}
  \label{fig:bbselection}
\end{figure}
With the source region identified, the background bins are selected
and a polynomial in time is fit to the background lightcurve. From
this point, time-resolved spectroscopy can be performed on the source
regions.


\section{Spectral Fitting}
\subsection{Fit Statistic}
Time-resolved spectral analysis of {\it Fermi} data occurs in the
low-count regime. This is because the photon model is evaluated in
each energy bin of the data. The classic \chisq statistic will is not
applicable in this regime and therefore a likelihood statistic is
required. Counts data are Poisson distributed and therefore the choice
of a likelihood with a Poisson estimator is required. Therefore, the
C-stat statistic is a useful choice (see \cite{Arnaud:2011}). The
benefit of the C-stat statistic is that it asymptotes to a \chisq
distribution in the limit of a large number of counts which provides a
suggestive goodness of fit. The drawback of a likelihood statistic is
that model comparison is not valid when the models are not nested. Two
models are said to be nested if they contain the same functional form
and parameters but one has at least one additional term e.g.,
\begin{equation}
\begin{array}{l}
\displaystyle y_1=Ax^{B} \\
\displaystyle y_2=Ax^{B}\exp(C).
\end{array}
\label{eq:nest}
\end{equation}
These two models are considered nested and the significance of the
more complex model, $y_2$, can be assessed via a likelihood ratio test
(LRT). For the purposes of this study, direct model comparison will
not be possible via the LRT because none of the models that will be
tested are nested. Even in the case of a linear addition of a second
spectral component like the blackbody, the LRT is not valid. This is
because when a model of the form
\begin{equation}
\begin{array}{l}
\displaystyle y_1=Ax \\
\displaystyle y_2=Ax+B,\;0\le B
\end{array}
\label{eq:nestlin}
\end{equation}
exists with the additional linear parameter is bounded by zero,
\cite{Protassov:2002} has shown that the LRT does not yield a valid
significance. For these reasons, testing the significance and goodness
of fit of models requires monte carlo simulations to sample the
probability distributions (see \sectionref{sec:sigtest}).

\subsection{Forward Folding}
\label{sec:ff}
Spectral analysis of GBM data is possible through a process called
forward-folding. The detected {\gray} photons are converted into
an electrical signal by the detectors' crystals and associated
electronics. The pulse height of the electrical signal is recorded as
a channel energy and is related to the energy of the incident
\gray . The association of the channel energy to the original
photon energy is not direct. Several radiative process can occur in
the crystal which vary the incident photon-channel
energy association non-linearly. This makes directly determining the
initial energy of a given photon impossible. The non-linear channel to
energy relationship is expressed through a response matrix.
\begin{figure}[h]
  \centering
  \cfig{5}{rspMat.pdf}{4.}
  \caption{An NaI response matrix from
    GBM.}
\end{figure}
This matrix is created from monte carlo simulations of the GBM
detectors exposed to different simulated photon energies. The
conversion of the input energy into a detector channel energy can be
represented as a matrix ($D_{ij}$). The input photon model
($f_i$) and background spectrum (${b_i}$) are related to the
detected count spectrum (${c_i}$) via a linear system.
\begin{equation}
\label{eqn:folding}
c_i\;=\;D_{ij}f_j+b_i
\end{equation}
In general, $D_{ij}$ is highly singular and therefore the equation is
non-invertible to obtain $f_i$. The process of forward folding
involves starting with a proposal spectrum $f^{\prime}_i$ and solving
for $c^{\prime}_i$. This is compared to the actual detected count
rate. The photon model will have adjustable parameters such as a
photon index that can be varied. This process is iterated, varying the
spectral parameters of the model until $c^{\prime}_i$ agrees with
$c_i$ via the Levenberg-Marquardt algorithm until C-stat is
minimized. Once the routine is complete, the $f_i$ for describing the
data is obtained along with statistical constraints on the model
parameters.

\section{Significance Testing}
\label{sec:sigtest}
The historical, single-component view of GRB spectra has been shown to
be inadequate in more recent studies. The presence of the blackbody in
several GRBs has modified this canonical view. Many spectra appear to
be better fit by the addition of the blackbody to the non-thermal
(Band or physical) photon model. However, this addition must be shown
to be a \emph{significant} improvement to the fit. The LRT with C-stat
is not viable for testing significance; therefore, monte carlo
simulations are required to assess the probability distributions. The
specific tests will be discussed when required
in \sectionref{sec:results:bb}, but the general details will be shown
here.

When a time bin is found to contain a blackbody in addition to its
non-thermal component by having a lower C-stat when fit with a
non-thermal+blackbody model, a simulation file is created. This
simulation file contains a set number of simulated spectra. These
spectra consist only of the non-thermal model with the fit parameters
fixed to the values derived from the data. The bins all contain a
different random background with a rate sampled from Poisson
distribution with a mean taken from the actual data. The bins in each
file are fit with both the simple model corresponding to the null
hypothesis ($H_0$) and the non-thermal+blackbody corresponding to the
proposal hypothesis ($H_1$). Each series of fits is saved to a file
that includes the fit parameters and the C-stat from each fit. To
assess the distribution of \dcstat the \dcstat for each simulated fit
is calculated. The significance of the blackbody ($\sigma_{\rm BB}$)
component is then calculated by determining the fraction of the
simulated distribution that lies below the value of \dcstat.
\begin{equation}
  \label{eq:sigtest}
  \sigma_{\rm BB}=\dover{\sum_i p(i)\ge {\rm data}\;\Delta_{\rm cstat} }{\sum_i p(i)< {\rm data}\;\Delta_{\rm cstat}}
\end{equation}
where $p(i)$ is the simulated \dcstat distribution. From this value,
the significance of the blackbody component can be ascertained. The
canonical assignment of $\sigma$ values (corresponding to the standard
deviation of the normal distribution) from the p-value is not always
valid using the method above. The distributions of \dcstat are not
necessarily symmetric. Therefore, the quoting of the p-value is more
appropriate. The monte carlo simulations to assess significance are
extremely CPU intensive when using physical models. The significance
level is often limited by the number of simulations that can be run.

\section{Summary of Fitting Procedure}
For each GRB in the sample a standard analysis procedure is applied
except for the caveats in the analysis of GRB 080920A and GRB
130427A. The procedure for analysis is as follows:
\begin{enumerate}[(i)]
\item a custom Bayesian blocks algorithm is applied to the TTE file of
  brightest NaI and BGO detector and the time bins are mapped to all
  detectors used
\item a background fit is applied to all detectors by selecting the
  Bayesian block plateau before and after the pulse
\item each time bin is fit with the Band, Band+blackbody, synchrotron,
  synchrotron+blackbody, fast-cooled synchrotron, and fast-cooled
  synchrotron+blackbody
\item if the fit is successful, the fit parameters are recorded in an
  FITS file
\item the FITS file is fed into a pipeline that calculates the
  following parameters from the fit parameters:
  \begin{itemize}
  \item the bolometric photon and energy flux of each spectral component
  \item $r_{\rm ph}$, $r_o$, $r_{\rm nt}$, and $\Gamma$
  \item HIC and HFC for each component.
  \end{itemize}

\end{enumerate}
Fits that fail in RMFIT either due to the inability to constrain the
fit parameters or due to instabilities in converging to a minimum are
not recorded. This is the case for most fits using the fast-cooled
synchrotron model.
%%% Local Variables: 
%%% mode: latex
%%% TeX-master: "../thesis"
%%% End: 
%
% Move this... but where?
%




\chapter{GRB 090820A: A Case Study}
\label{ch:grb090820A}
\begin{chapterquote}{Henry Rollins}
You need a little bit of insanity\\
to do great things
\end{chapterquote}

The following is adapted from \cite{Burgess:2012}. GRB 090820A serves
a case study for the testing of the slow-cooled synchrotron photon
model on a single pulse GRB. This analysis was performed to test
physical modeling on a GBM GRB before a systematic analysis was
performed on a larger sample. The analysis here varies from
standardized analysis that will was performed the larger sample in
several ways:
\begin{itemize}
\item time binning was made by the S/N method to ensure constrained fits
\item only select time bins were analyzed to examine the evolution of
  fit parameters
\item the slow-cooled synchrotron model parameter $\epsilon$ was left free during the fits
\item no inferred properties were calculated from the spectral fits.
\end{itemize}




% \shorttitle{Constraints on the Synchrotron Shock Model}
% \shortauthors{J.~Michael Burgess et al.}




% \title{Constraints on the Synchrotron Shock Model for the Fermi GBM Gamma-Ray Burst 090820A}


% %%Authors            
              
% \author{J. Michael Burgess,\altaffilmark{1}, 
% Robert~D.~Preece\altaffilmark{1},
% Matthew~G. Baring,\altaffilmark{2},
% Michael~S.~Briggs\altaffilmark{1}, 
% Valerie Connaughton,\altaffilmark{1},
% Sylvain Guiriec,\altaffilmark{1} 
% William S.~Paciesas\altaffilmark{1}, 
% Charles A.~Meegan\altaffilmark{3}, 
% P.~N. Bhat\altaffilmark{1}, 
% Elisabetta Bissaldi\altaffilmark{4}, 
% Vandiver Chaplin\altaffilmark{1}, 
% Roland Diehl\altaffilmark{4}, 
% Gerald~J.~Fishman\altaffilmark{5}, 
% Gerard Fitzpatrick\altaffilmark{6}, 
% Suzanne Foley\altaffilmark{6},
% Melissa Gibby\altaffilmark{7}, 
% Misty Giles\altaffilmark{7},
% Adam Goldstein\altaffilmark{1}, 
% Jochen Greiner\altaffilmark{4}, 
% David Gruber\altaffilmark{4}, 
% Alexander J.~van der Horst\altaffilmark{3}, 
% Andreas von Kienlin\altaffilmark{4}, 
% Marc Kippen\altaffilmark{8}, 
% Chryssa Kouveliotou\altaffilmark{5}, 
% Sheila McBreen\altaffilmark{6}, 
% Arne Rau\altaffilmark{4}, 
% Dave Tierney\altaffilmark{6}, and 
% Colleen~Wilson-Hodge\altaffilmark{5}}              
              
% %%Affiliations
% \altaffiltext{1}{University of Alabama in Huntsville, 320 Sparkman Drive, Huntsville, AL 35899, USA}
% \altaffiltext{2}{Department of Physics and Astronomy, MS 108,
%       Rice University, Houston, TX 77251, U.S.A.  {\it Email: baring@rice.edu}}
% \altaffiltext{3}{Universities Space Research Association, 320 Sparkman Drive, Huntsville, AL 35899, USA}
% \altaffiltext{4}{Max-Planck-Institut f$\rm \ddot{u}$r extraterrestrische Physik (Giessenbachstrasse 1, 85748 Garching, Germany)}
% \altaffiltext{5}{Space Science Office, VP62, NASA/Marshall Space Flight Center, Huntsville, AL 35812, USA}
% \altaffiltext{6}{School of Physics, University College Dublin, Belfield, Stillorgan Road, Dublin 4, Ireland}
% \altaffiltext{7}{Jacobs Technology}
% \altaffiltext{8}{Los Alamos National Laboratory, PO Box 1663, Los Alamos, NM 87545, USA}              
% %\altaffiltext{1}{Physics Department, The University of Alabama in Huntsville,
% %              Huntsville, AL 35809, U.S.A.  {\it Email: james.burgess@nasa.gov, rdp}}


% \email{james.m.burgess@nasa.gov}



%%Observations
\section{Observations}
 \label{sec:observe}
\begin{centering}

Authors

J. Michael Burgess, Robert~D.~Preece, Matthew~G. Baring,
Michael~S.~Briggs, Valerie Connaughton, Sylvain Guiriec, William
S.~Paciesas, Charles A.~Meegan, P.~N. Bhat, Elisabetta Bissaldi,
Vandiver Chaplin, Roland Diehl, Gerald~J.~Fishman, Gerard Fitzpatrick,
Suzanne Foley, Melissa Gibby, Misty Giles, Adam Goldstein, Jochen
Greiner, David Gruber, Alexander J.~van der Horst, Andreas von
Kienlin, Marc Kippen, Chryssa Kouveliotou, Sheila McBreen, Arne Rau,
Dave Tierney, Colleen~Wilson-Hodge

\end{centering}
\hspace{2 cm}


 On 20 August 2009, at T$_\mathrm{0}$=00:38:16.19 UT, GBM triggered on
 the very bright GRB~090820A~\cite{grb090820A}. This GRB also
 triggered Coronas Photon-RT-2~\cite{GCN-CORONAS}. The burst location
 was initially not in the FOV of the LAT onboard Fermi but was bright
 enough to result in an ARR. However, Earth avoidance constraints
 prevented such a maneuver until 3100 sec after the burst trigger and
 the burst was not detected at higher energies by the LAT.  The most
 precise position for the direction of the burst comes from the GBM
 trigger data which localizes the burst to a patch of sky centered on
 RA = 87.7 degree and Dec = 27.0 degree (J2000) with a 4 degree error,
 statistical and systematic. The current best model for systematic
 errors is 2.8 degrees with 70\% weight and 8.4 degrees with 30\%
 weight \cite{briggsann}. We verified that our analysis does not
 change significantly using instrument response functions for assumed
 source locations throughout this region of uncertainty.
 \figureref{fig:figure1} shows the light curve of GRB~090820A as seen
 by GBM,
\begin{figure}[h]
\cfig{6}{figure1}{5}
\caption{ Light curve of GRB~{\it
    090820}A as observed by GBM. The two panels show the count rate in
  the two NAI detectors (top) and BGO (bottom). The dashed lines
  indicate the time intervals (a, b, c, d) used for the time-resolved
  analysis (see Figure 3 and Table 1). It is clear that the burst
  consists of two main peaks and that this burst is very bright in the
  BGO detectors.}
\label{fig:figure1}
\end{figure}
from 8 to 200 keV in the NaI detectors (top) and from 200 keV to 40
MeV in the BGO detector (bottom). GBM triggered on a weak precursor
which we do not include in the analysis. The main light curve begins
at T$_{0}$ + 28.1s. The main structure of the light curve consists of
a fast rising pulse with an exponential decay lasting until T$_{0}$+60
s. A second, less intense, peak beginning at T$_{0}$+30 s is
superimposed on the main peak.


With such a high intensity and simple structure, this GRB allows for
detailed time-resolved spectroscopy. Because this burst is intense,
calibration issues make the Iodine K-edge (33 keV) prominent in the
count spectra owing to small statistical uncertainties, and we remove
energy channels contributing to this feature from our spectral
fits. In addition, an effective area correction is applied between
each of the NaI detectors and the BGO 0 during the fit process. This
correction of $\approx$~23\% is used to account for possible
imperfections in the response models of the two detector types.
We simultaneously fit the spectral data of the NaI detectors with a
source angle less than 60 degrees (NaI 1 and 5) and the data from the
brightest BGO detector (BGO~0) using the analysis package RMFIT.


We perform a fit to the integrated spectrum and find that it is best
represented by synchrotron emission from thermal and power-law
distributed electrons with an additional blackbody component
characterized by a kT~$\approx$~42 keV (C-Stat/DOF = 558/353). The
\teq{\nu F_{\nu}} spectrum is displayed in \figureref{fig:figure2}
and the best-fit values in \tableref{tab:table1}. We also performed
a fit using the Band function (C-Stat/DOF = 593/355).
\begin{figure}[h]
\cfig{6}{figure2}{5}
\caption{The integrated
  spectrum of GRB~090820A. We are able to resolve three
  components, thermal synchrotron, power-law synchrotron, and a
  blackbody. Energy channels near the NaI K-edge are omitted. The
  deviations in the fit residuals are the due to systematics in the
  detector response resulting from the high count rate and spectral
  hardness of this burst. However, deviations are never greater than
  4$\sigma$ and do not significantly impact the values of the best fit
  parameters. The multiple curves near the peak of the spectrum are an
  artifact of the effective-area correction applied to each detector
  and not related to the different fitted models.}
\label{fig:figure2}
\end{figure}



\begin{figure}
\cfig{6}{figure3}{4}
\label{fig:figure3}
\caption{The electron distribution corresponding to the integrated spectrum. The non-physical jump in the amplitude between the Maxwellian and the power-law distribution (parametrized by $\epsilon$) at $\gammaMin$ is clearly seen.}
\end{figure}



We find in concordance with BB04 that emission from power-law
synchrotron dwarfs the emission from thermal synchrotron by at least 3
orders of magnitude. The value of $\gammaMin$ is fixed to 3, the
choice adopted by BB04: it is a value that accommodates distributions
typically determined by shock acceleration simulations.  When fitting
the power-law synchrotron component we have to fix the value of the
power-law index to its best fit value to remove a correlation between
the amplitude and the index; this does not change the fit statistic
but does mean that the amplitudes obtained are valid only for that
index. The inferred electron distribution from this fit is shown in
\figureref{fig:figure3}. We note that the inability to simultaneously
constrain the power-law index and amplitude of the synchrotron
function may be solved in future studies by including joint fits with
LAT data, whenever available.
% \end{deluxetable}



\begin{table}
\scriptsize
\centering
\begin{tabular}{c | c c c c c c c c}
Time interval&$n_{0}$ {$(\gamma s^{-1}cm^{-2}keV^{-1})$} & $\epsilon$ & $\mathcal{E}_c$ (keV) & $\delta$ & $\gammaMin$ & $A_{BB}$ $(\gamma s^{-1}cm^{-2}keV^{-1})$ & $kT$ (keV) \\
\hline \hline
Time integrated&$0.3437_{-0.065}^{+0.204}$ & $871_{-234}^{+254}$ & $10.39_{-0.245}^{+0.254}$ & $4.9$ & $3.0$ & $2.08_{-0.208}^{+0.367}\times10^{-5}$ & $42.27_{-1.35}^{+1.49}$ \\

a&$ 2.378_{-0.176}^{+0.189}$ & $-$ &   $8.351_{-0.93}^{+1.08}$ & $-$ & $-$ & $-$ & $-$ \\

b&$ 859_{-89.1}^{+94.0}$ & $-$ & $14.24_{-0.776}^{+0.848} $ & $ 4.4 $ & $3.0$ & $1.774_{-0.356}^{+0.410}\times10^{-4}$ & $35.32_{-1.77}^{+1.99}$ \\

c& $1.901_{-0.093}^{+0.094}\times10^{4}$ & $-$ & $15.22_{-0.399}^{+0.411}$ & $ 5.9$ & $3.0$ & $1.818_{-0.344}^{+0.400}\times10^{-4}$ & $ 38.7_{-1.92}^{+2.13}$ \\

d&$ 2.196_{-0.466}^{+0.720}$ & $-$ & $4.035_{-0.715}^{+0.689}$ & $-$ & $-$ & $8.383_{-3.18}^{+4.89}\times10^{-5}$ & $ 28.40_{-3.59}^{+3.73}$ \\
%\hline

\end{tabular}

\caption{The fit parameters for the
  time-integrated (first row) and time-resolved spectra. The fit
  parameters for the blackbody component are its amplitude
  ($A_{BB}$) and energy ($kT$). The fit parameters for the
  non-thermal components are described in
  \sectionref{sec:physmod:rmfit}. The break energy
  $\mathcal{E}_b\equiv \mathcal{E}_c(\gamma \to \eta\gamth )$
  corresponds to employing the substitution $\gamma \to \eta\gamth
  $ in \equationref{eq:elec_dist}.  Note that the ratio of the
  amplitudes is not equal to the ratio of the fluxes.}

\label{tab:table1}
\end{table}




%TRS
For the time resolved analysis we fit four bins labeled {\textbf a},
{\textbf b}, {\textbf c} and {\textbf d} as shown in
\figureref{fig:figure1} with the various synchrotron models. The
corresponding electron distributions inferred from these fits are
displayed in \figureref{fig:figure5}. We also fit the Band function to
each spectrum to show that in nearly all cases the physical models can
fit the data as well as the Band function. We chose the time binning
by finding a balance between high signal-to-noise and evolution of the
spectral shape so that we can identify the time evolution of each
component throughout the burst. Where possible, we fit all three
components simultaneously. Due to the similarity in the spectral
shapes of the low energy portions of the thermal synchrotron and
power-law synchrotron components it is not always possible to
constrain all of the fit parameters especially when one component is
much stronger than the other. Therefore, when one component is
dominant we include only that component in the fit. The ability to fit
both components in the time integrated fit is most likely due to the
fact that both components are significant over the interval.
%Figure4
\begin{figure}[h]
\centering
\cfig{6}{figure4}{6}

\caption{The time-resolved
  spectra for GRB~{\it 090820}A. The spectra represent bin \textbf{a}
  with thermal synchrotron only (top left panel), bin \textbf{b} with
  power-law synchrotron + blackbody (top right panel), bin \textbf{c}
  again with power-law synchrotron + blackbody (bottom left panel),
  and finally bin \textbf{d} with thermal synchrotron + blackbody
  (bottom right panel). As with Fig. \figureref{fig:figure2}, the
  multiple curves are associated with the effective area correction.}
\label{fig:figure4}
\end{figure}


%%Figure5
\begin{figure}
\centering
\cfig{6}{figure5}{4}
\label{fig:figure5}
\caption{The electron distributions for the time-resolved spectra. The choice of $\eta$ with a power-law only distribution is arbitrary due to the fact that $\mathcal{E}_c$ and $\eta$ both scale $\Ep$.}
\end{figure}

From bins {\textbf b} to {\textbf c} the spectrum is best described by
synchrotron emission from power-law distributed electrons in addition
to a blackbody (\tableref{tab:table1} and
\figureref{fig:figure4}). The thermal synchrotron component is too
weak to meaningfully include it in the fit. We find that the intensity
of the power-law synchrotron increases significantly from bin {\textbf
  b} to {\textbf c} while the blackbody component remains nearly
constant in intensity.  The spectral index of the electrons in these
intervals varies from -4.4 to -5.9.  Such values are consistent with
those expected from diffusive acceleration theory, for the specific
case of superluminal shocks \cite{Baring:2011}, i.e. those where the
mean magnetic field angle to the shock normal is significant. This
geometrical requirement establishes efficient convection of particles
downstream of relativistic shocks, thereby steepening their
acceleration distribution.  The blackbody component decreases in
intensity at this point but the temperature remains constant within
errors.  In bins {\textbf a} and {\textbf d}, with weaker emission,
several models are essentially statistically tied.  It is possible
that PLS+BB persists throughout the entire GRB. Alternatively, the GRB
could even begin in bin {\textbf a} with thermal synchrotron emission
and transition to the PLS+BB emission. If this were true we would be
seeing emission from electrons that have not yet been accelerated into
a power-law distribution by the shock. The C-stat values for all of
the models fit in each bin are displayed in Table \tableref{tab:table2}.

\begin{table}
\centering
\begin{tabular}{c|c c c c c}
% & Band & Thermal Synchrotron & Thermal Synchrotron + blackbody & Power Law Synchrotron & Power Law Synchrotron + blackbody \\
Time Interval & Band & TS & TS + BB & PLS & PLS + BB \\
\hline \hline
a & 464/355 & 466/357 & 464/355 & 467/357 & 465/355 \\

b & 432/355 & 742/357 & 445/355 & 555/357 & 434/355 \\

c & 450/355 & 1088/357 & 488/355 & 558/357 & 434/355 \\

d & 404/355 & 421/357 & 403/355 & 406/357 & 405/355 

\end{tabular}
\caption{The c-stat per degree of freedom for each time model in the selected time intervals.}
\label{tab:table2}
\end{table}

While it is not possible to constrain all parameters in all the bins,
it should be stressed that this is due to natural correlations in the
synchrotron functions. These difficulties do not arise when using the
Band function because it has a simpler parametrization.


%%Conclusion
\section{Conclusion of Study}
%

Here, it has been shown that thermal and non-thermal synchrotron
photon models, with an additional blackbody, are well consistent with
the emission spectra of GRB 090820A in various time intervals. These
are physical models that afford the ability to constrain parameters
that are physically meaningful, for example key descriptors of the
electron distribution that is motivated by shock acceleration
theory. By implementing these models into a forward-folding spectral
analysis software we have been able to directly constrain many of the
physical model parameters and their respective errors; a first in the
field of GRB spectroscopy. This constitutes substantial progress over
the use of the empirical Band function to fit prompt GRB spectra,
which has been a nearly universal practice to date.  The results
presented here enable more rigorous statements about the validity of
GRB emission models, moving the study of prompt burst emission into a
new era.

Our modeling has focused on the standard synchrotron shock model with
the addition of a blackbody component. The spectral fitting reveals a
complex temporal evolution of the separate components. While spectral
evolution is a well-known feature of GRBs, this type of fitting can
enable \textit{direct} physical interpretation of the evolution. These
fits provide evidence that the line of death issue
\cite{preece:1998,Preece:2002} can be overcome naturally with a
combination of synchrotron and blackbody emission: the prominence of a
blackbody component with its flat Rayleigh-Jeans portion would derive
a comparably-fitted Band function with a flat low-energy index. This
was also suggested by \cite{Guiriec:2011} where the authors used
simultaneous fits of the Band function and a blackbody. Note that it
is possible that other physical models may, in fact, produce superior
fits to the data for GRB 090820A and other bursts. Strongly-cooled
synchrotron emission, inverse Compton and jitter radiation are popular
candidates, and our work here motivates the future development of
RMFIT software modules for these processes.

A principal finding of the analysis in this paper is that the
power-law synchrotron component is orders of magnitude more intense
than the thermal synchrotron component during the peak of the burst,
the latter contributing at most a few percent of the flux. This
confirms the finding of BB04 for BATSE/EGRET bursts GRB 910503, GRB
910601 and GRB 910814, which was a theoretically-based perspective
that did not fold models through the detector response matrices. They
had noted that full plasma and Monte Carlo diffusion simulations of
shock acceleration clearly predict a power-law tail in the particle
distribution that smoothly extends from the dominant thermal
population (e.g. see also \cite{Baring:2011}, and references therein).
This tail is several orders of magnitude smaller than what is found
when fitting synchrotron emission to burst spectra. It is not clear
how such non-thermally-dominated distributions can arise near shocks,
providing a conundrum for the standard synchrotron shock
model. Limited smoothing of the sharp peak of the non-thermal electron
component will not alter this conclusion.


This result is also in accord with \cite{Guiriec:2010}, in their
analysis of GRB 100724B, who fitted its GBM spectra with a
combination of the Band model and a blackbody. They too found that an
unrealistically high efficiency for the acceleration mechanism or 
a source size smaller than the innermost stable orbit of
a black hole was required to invoke the standard fireball model for
explaining the origin of the $\gamma$-ray emission. Therefore, it was
surmised therein that the outflow from the jet was at least partially
magnetized.

To conclude, the success of this analysis in isolating the relative
contributions of a handful of distinct spectral components indicates
that it is imperative for the field of GRB spectroscopy to move away
from the use of the empirical fitting functions: many physical models
can asymptotically approximate the Band spectral indices, rendering it
difficult to discern between them particularly near the $\nu F_{\nu}$
peak. Instead, direct comparisons of the fitted physical models are
possible, and are required to truly discriminate between the various
emission processes. The fitting of physical synchrotron shock
model/blackbody spectra here offers a clear advance beyond empirical
fits, and provides the impetus for further development and deployment
of physical modeling of prompt burst emission spectra.

\section{Post-Analysis of GRB 090820A and Constraining $\epsilon$}
\label{sec:epsdisc}
The analysis of \cite{Burgess:2012} provided evidence that direct use
of physical models is possible. Several areas of this analysis must be
refined and improved upon. One of the main and troubling findings of
this work was that the non-thermal population of electrons was more
prominent than the thermal population when the two were fit
together. This was found through the value of $\epsilon$ in the
spectral fits. It should be noted as it was in \cite{Burgess:2012}
that the value was highly unconstrained. This happens because the
low-energy slope of the thermal and non-thermal synchrotron photon
spectra have an asymptotic index of -2/3. When treated as separate
components the fitting engine has a difficult time converging on the
relative contribution of each component but has to also account for
the high-energy contribution of the non-thermal synchrotron. Therefore
it quickly converges on a value of $\epsilon$ that favors a relatively
large contribution from power-law electrons.

After this analysis, ways to force the fitting engine to converge of
physical electron distributions were investigated. It has been shown
that the contribution from electrons in the power-law to the
post-shock accelerated distribution should be minuscule ($<10$\%)
compared to the thermal electrons and that the distribution should be
continuous \cite{Spitkovsky:2008}. One way to achieve this is to place
numerical constraints on the value of $\epsilon$ so that the
distribution is always continuous. Due to the shape of the combined
distribution, this constraint will also guarantee that the relative
contribution of power-law electrons is less than that of the thermal
distribution.  By setting equating the thermal and power-law
distributions at and solving for $\epsilon$ one arrives at
\begin{equation}
  \label{eq:epsnorm}
  \epsilon = \dover{\gammaMin}{\gamth}^2 \exp\left(-\dover{\gammaMin}{\gamth}\right).
\end{equation}
\begin{figure}[h!]
  \centering
  \cfig{6}{changeG}{5}
  \caption{The amended electron distribution with a fixed normalization such that the distribution is continuous regardless of the values of $\delta$ and $\gammaMin$.}
  \label{fig:fixeps}
\end{figure}
For all further analysis, this value of $\epsilon$ will be used when
fitting GRB spectra with the slow-cooled synchrotron model.










%Table 1


% This LaTeX table template is generated by emacs 23.2.1






%\end{table}




%Table2






%%% Local Variables: 
%%% mode: latex
%%% TeX-master: "../thesis"
%%% End: 

\chapter{Spectral Analysis of {\it Fermi} GRBs with Fast and Slow-Cooled Synchrotron Photon Models}
\label{ch:pap2}
\begin{chapterquote}{Dr. Dog}
It's like that old black hole,\\
no matter how you try,\\
you set out each day\\
 never to arrive
\end{chapterquote}




With the validity of using physical models to directly fit GRB
spectral data established, the analysis can be expanded to a larger
sample to allow for a categorical analysis from which physical
implications can be derived. An important question not addressed in
\cite{Burgess:2012} (hereafter B12) is the fast-cooling problem. The spectrum of GRB
090820A was successfully fit with a slow-cooling model. However, GRBs
are expected to be in the fast cooling regime. To address this a
fast-cooling model must be tested along with the slow-cooling
model. The analysis of \cite{Burgess:2013} focuses on these questions
as well as looking at the physical implications of the spectral fit
parameters. The following is an adaptation of \cite{Burgess:2013}.

\begin{centering}

Authors

J.~M.~Burgess,
R.~D.~Preece,
V.~Connaughton,
M.~S.~Briggs,
A.~Goldstein,
P.~N. Bhat,
J.~Greiner,
D.~Gruber,
A.~Kienlin,
C.~Kouveliotou,
S.~McGlynn,
C.~A.~Meegan,
W.~S.~Paciesas,
A.~Rau,
S.~Xiong

M.~Axelsson,
M.~G.~Baring,
C.~D.~Dermer,
S.~Iyyani,
D.~Kocevski,
N.~Omodei,
F.~Ryde,
G.~Vianello

\end{centering}

\hspace{2 cm}

\section{Model Spectral Components}
\label{sec:model}



In the fireball model of GRBs, the majority of the flux is
theoretically expected to be in the form of thermal emission coming
from the photosphere of the jet. However, nearly all of the low-energy
indices implied by GRB spectral analysis with Band-function spectral
inputs have $\alpha<+1$, i.e. too soft to be thermal -- see
  for example \cite{Goldstein:2012} for the BATSE database.
This points to a non-thermal emission process for most
GRBs. Multi-spectral component analysis of {\it Fermi} GRBs has shown
that while the majority of the emission is non-thermal, a small
fraction of the energy radiated apparently originates from a blackbody
component \cite{Guiriec:2010,Axelsson:2012}. In B12, a blackbody
component was also identified when the non-thermal emission was fit
with a synchrotron photon model. This combination of blackbody and
non-thermal emission was predicted by \cite{Meszaros:2000}.
In this paper, before proceeding to the details of the
  analysis, we first review the synchrotron model (see B12), and also
several observable relations of the blackbody component. These
components are then implemented into a fitting program which directly
convolves the physical models with the GBM detector response to
compare with observations.



In principle, there are six spectral parameters that can be
constrained by the fits: $n_0$, $E_{*}$, $\delta$, $\epsilon$,
$\gamth$, and $\gammaMin$; however, we fix $\gamth$, $\gammaMin$, and
$\epsilon$ due to fitting correlations as explained below. The
parameter $E_{*}$ scales the energy of the fit and is linearly related
to the Band function's E$_{\rm p}$. Numerical simulations of particle
acceleration at relativistic shocks have shown \cite{Baring:2004}
that the non-thermal population is generated directly from the thermal
one. To match these circumstances, we set the ratio of $\gamth$ and
$\gammaMin$ to be $\sim$3, following \cite{Baring:2004}.  The
parameters $E_*$, $\gamth$, and $\gammaMin$ all directly scale the
peak energy of the spectrum but do not alter its shape and thus cannot
be independently determined. For this reason we chose values of
$\gamth=300$ and $\gammaMin=900$ for all fits and left $E_*$ free to
be constrained from the fit.

As shown in the \appendixref{ch:pap2app}, such parameter values are on the outer edge
for what is allowed energetically.  For these parameters, the flow is
strongly Poynting flux/magnetic-field energy dominated in order that
electrons with $\gamma\sim 300$ -- 900 can produce radiation in the
MeV regime.  Magnetized jet models are advantageous for energy
dissipation through magnetic reconnection, to produce short timescale
variability, and to accelerate ultra-high energy cosmic ray (see
\appendixref{ch:pap2app}). In fact, a wide range of parameter values with much larger
electron Lorentz factors and smaller magnetic fields are possible. In
weak magnetic-field models, a strong self-Compton component and
$\gamma\gamma$ opacity effects can make a cascade that modifies the
standard emission spectrum of GRBs. By considering a strongly
magnetically dominated model, these issues can be neglected.

For the chosen parameters, the system is always in the strongly cooled
regime (see \appendixref{ch:pap2app}). Nevertheless, we adopt the expression,
\equationref{eq:elec_dist}, to approximate an electron spectrum in the
slow-cooling regime.  The parameter $\epsilon$, corresponding to the
relative amplitude between the thermal and non-thermal portions of the
electron distribution, was not easily constrained in the fitting
process used in B12 and produced small non-physical discontinuities in
the electron distribution, as pointed out in
\cite{Beloborodov:2012}. Therefore, here we numerically fix this
parameter to the small value of $(\gammaMin/\gamth)^2\times
\exp(-\gammaMin/\gamth)$, so that there is no discernible
discontinuity between the thermal and non-thermal parts of the
distribution. The thermal component helps smooth out the spectral
structure at $E_{\ast}$, but does not alter the asymptotic index of
$\alpha = 2/3$ realized for synchrotron emission from populations with
lower bounds to their particle energies. After these simplifications,
three shape parameters remain free: $E_{*}$, $\delta$ and $n_0$, which
corresponds to the amplitude. Compared with the Band function's four
fit parameters this model is simpler yet tied to actual physical
processes.


Even though we examine a model with $\gamma_{th} = 300$ and
$\gamma_{min} = 900$, the spectral fitting is insensitive to the exact
value of the product $\Gamma B\gamma^2$ provided that the constraints
discussed in \appendixref{ch:pap2app} are satisfied. Even in a strong cooling regime
defined by these low assumed values of $\gamma_{th}$ and $\gamma_{min}
$, second-order processes in GRBs \cite{Waxman:1995,Dermer:2001},
which can become more important than first-order processes in
relativistic shocks \cite{2009herb.book.....D}, allow us to consider a
model that is effectively slowly cooled. Simulations typically have
more parameters than our current model
\cite{Peer:2004,Asano:2009,Daigne:2009}, and constraining those models
via spectral templates using data from {\it Fermi} may be difficult.



\subsection{Blackbody Component}
\label{sec:model:bb}
The pure fireball scenario for GRB emission predicts that most of the
flux is from thermal emission
\cite{Goodman:1986,Paczynski:1986}. This is because as the jet
becomes optically thin at some photospheric radius, $r_{ph}$, it
releases radiation that has undergone many scatterings with the
optically thick electrons below the photosphere. We model this
emission as a blackbody
\begin{equation}
  \label{eq:blackbody}
  F(\mathcal{E})\;=\;A \mathcal{E}^3 \frac{1}{ e^{\frac{\mathcal{E}}{kT}}-1}
\end{equation}
where A is the normalization and kT scales the energy of the
function. This is simplified thermal emission from the photosphere
that does not take into account the effects of relativistic broadening
that can produce a multi-color blackbody emerging from the photosphere
\cite{Beloborodov:2010,Ryde:2010,Peer:2011}. \cite{Ryde:2006} showed
that if it is assumed that the thermal component is emanating from the
photospheric radius of the jet, several properties about the blackbody
component are derivable. The cooling behavior is well predicted for a
thermal component. The temperature of the blackbody should decay as
$T\propto r_{ph}^{-2/3}$. If $\Gamma$ is assumed to remain constant
during the coasting phase of the jet then it can be shown that the
temperature should decay as $T \propto t^{-2/3}$ in time.  It has been
found observationally that the evolution of kT often follows a broken
power-law trend with the index below the break averaging to $\sim
-$2/3 \cite{Ryde:2009}. Finally, a true blackbody has a well defined
relation between energy flux and temperature:
\begin{equation}
   \label{eq:sbl}
  F_{BB}=N\sigma_{sb} T^4
\end{equation}
where $N$ is a normalization related to the transverse size of the
emitting surface, $r_{ph}/\Gamma$
\cite{Ryde:2005,Ryde:2009,Iyyani:2013}, and $\sigma_{sb}$ is the
Stephan-Boltzmann constant.


The photospheric radius and the transverse size of the photospheric
emitting region are also of great importance to understanding the geometry
and energetics of GRBs. In \cite{Ryde:2009}, a parameter
$\mathcal{R}$ 
\begin{equation}
  \label{eq:scR}
  \mathcal{R}(t)\;\equiv\;\left(\frac{F_{BB}(t)}{\sigma_{sb}T(t)^4} \right)^{1/2} \propto r_{ph}.
\end{equation}
is used to track the outflow dynamics of the burst (see \figureref{fig:grbjet2} for a conceptual view of $\mathcal{R}$). The connection
between $\mathcal{R}$ and N from \equationref{eq:sbl} is established
by noting that $N = \mathcal{R}^2$. Thereby, only if N is constant
would we expect to recover the relation established in
\equationref{eq:sbl}.  Several BATSE GRBs were found to have a
power-law increase of $\mathcal{R}$ with time. However, the connection
between $\mathcal{R}$, T, and $F_{BB}$ was difficult to establish
because the error on the data points was large. Understanding these
connections is essential to unmasking the structure and temporal
evolution of GRB jets.

%% Observations
\section{Time Resolved Analysis}

\label{sec:observe}

\subsection{Summary of Technique}
The GRBs in our sample were selected based on two criteria: large peak
flux and single-peaked, non-overlapping temporal structure. The GRBs
were binned temporally in an objective way described in
\sectionref{sec:tbin} and spectral fits were performed on each time
bin using four different photon models (Band, Band+blackbody,
synchrotron, synchrotron+blackbody). When fitting synchrotron we
compared the fits of slow-cooling and fast-cooling synchrotron. In
many cases the fits from fast-cooling synchrotron completely
failed. From these spectral fits a photon flux lightcurve was
generated for each component and fitted with a pulse model to
determine the decay phase of the pulse. We describe each step in the
following subsections.

\subsection{Sample Selection}
To fully constrain the parameters of the fitted models, we selected
GRBs with a requirement that the peak flux be greater than 5 photons
s$^{-1}$ cm$^{-2}$ between 10 keV and 40 MeV. It is important for our
GRBs to have a simple, single-peaked lightcurve structure to avoid the
overlapping of different emission episodes. This facilitates the
identification of distinct evolutionary trends in the physical
parameters for the emission region. While we cannot be sure that a
weaker emission episode does not lie beneath the main peak, the bursts
we selected have no significant additional peak during the rise or
decay phase of the pulse. These two cuts left us a sample of eight
GRBs: GRB 081110A \cite{GRB081110A}, GRB 081224A \cite{GRB081224A},
GRB 090719A \cite{GRB090719A}, GRB 090809B, GRB 100707A \cite{
  GRB100707A}, GRB 110407A, GRB 110721A, \cite{GRB110721A} and, GRB
110920A (\figureref{fig:lc}). GRB 081224A and GRB 110721A were both
analyzed including the new LAT Low-Energy (LLE) data that provides a
high effective area above 30 MeV for the analysis of short-lived
phenomena, thanks to a loosened set of cuts with respect to LAT
standard classes \cite{Vero:2010, Ack:2012a}. This data selection
bypasses the typical photon classification \cite{Ackerman:2012} tree
and includes events that would normally be excluded but can be
selected temporally when the signal to background rate is high, such
as with GRBs. GRB 081224A had very little data above 30 MeV but the
LLE data helped to constrain the spectral fits. From this sample, five
GRBs (GRB 081224A, GRB 090719A, GRB 100707A, GRB 110721A, and GRB
110920A) had blackbody components that were bright enough to analyze
(\tableref{tab:grbs}).


\begin{figure}

 \centering



  \subfigure[]{

    \label{fig:lc:a}
    \cfig{7}{lc3}{2.5}}\subfigure[]{
    \label{fig:lc:b}
    \cfig{7}{lc4}{2.5}}
  \subfigure[]{
    \label{fig:lc:c}
    \cfig{7}{lc5}{2.5}}\subfigure[]{
    \label{fig:lc:d}
    \cfig{7}{lc6}{2.5}}
\subfigure[]{
    \label{fig:lc:e}
    \cfig{7}{lc7}{2.5}}\subfigure[]{
    \label{fig:lc:f}
    \cfig{7}{lc8}{2.5}}
\subfigure[]{
    \label{fig:lc:g}
    \cfig{7}{lc9}{2.5}}\subfigure[]{
    \label{fig:lc:h}
    \cfig{7}{lc10}{2.5}}

  \caption{The energy flux lightcurves of the synchrotron component
    for the entire sample (\emph{black} curve). The integration range
    is from 10 keV - 40 MeV for all GRBs except GRB 081224A and GRB
    110721A which are from 10 keV - 300 MeV. Superimposed is the
    slow-cooled synchrotron $\Ep$ (\emph{red} curve) demonstrating the
    hard to soft evolution of the bursts. }

  \label{fig:lc}

\end{figure}


\begin{table}

\centering
\scriptsize
\begin{tabular}{c | c c c c}



GRB & Peak Flux (p/s/cm$^2$) & Duration Analyzed (s) & Blackbody Component & LLE data\\ 

\hline \hline

GRB 081110A & 20.88  &  4.61 &   \text{\sffamily X}         & \text{\sffamily X}\\ 

GRB 081224A & 17.11  & 18.36  & $\checkmark$ & $\checkmark$\\

GRB 090719A & 26.52   & 30.09  & $\checkmark$ & \text{\sffamily X}\\

GRB 090809B & 18.36  & 14.64  & \text{\sffamily X} & \text{\sffamily X}\\

GRB 100707A & 18.77  & 22.39  & $\checkmark$ & \text{\sffamily X}\\

GRB 110407A & 15.6  & 20.48  & \text{\sffamily X} & \text{\sffamily X}\\

GRB 110721A & 29.82   & 12.7  & $\checkmark$ & $\checkmark$\\

GRB 110920A & 8.08   & 238.29 & $\checkmark$ & \text{\sffamily X} \\



\end{tabular}

\caption{The GRBS in our sample. The peak fluxes were taken from the brightest bin of each GRB with a duration determined by Bayesian blocks.}

\label{tab:grbs}

\end{table}



\subsection{Spectral Analysis}
\label{sec:specan}
For spectral fits we used the RMFIT
ver4.1\footnote{http://fermi.gsfc.nasa.gov/ssc/data/analysis/user/}
software package developed by the GBM team. Fitting the synchrotron
photon model requires a custom module developed and used in B12. Each
time bin was fit with one of the four spectral models mentioned
above. We fit the physical models to compare the validity of each one
against the other and the Band function to try and understand how the
Band function parameters correlate with the best fit physical
model. If the addition of blackbody component did not make a
significant improvement of at least 10 units of C-stat
\cite{Arnaud:2011} for any time bins of a particular GRB, then we did
not include the blackbody component in the analyzed fits for that
burst. However,
near the end of the prompt emission in some GRBs, the blackbody
component becomes weak but has spectral evolution consistent with more
significant time bins in the burst. The spectral parameters of the
blackbody in those bins were included even though they contributed
large error bars to some quantities. We checked with simulations that
this cut was sufficient to identify a significant addition of a
blackbody to the fit model.



\section{Results}
\label{sec:results}
\subsection{Test of Slow-Cooling Synchrotron}
\label{sec:results:scs}

In nearly all cases the synchrotron or synchrotron+blackbody model
produced a fit with a comparable or better C-stat than the Band
function. The GRBs exhibited hard to soft spectral evolution (
\figureref{fig:specEvo}) for both components. From these fits we can
derive several interesting properties of the bursts. The results of
fitting the non-thermal part of each time bin in our sample with
slow-cooled synchrotron indicate that this model can indeed fit the
data well. The spectral parameters are summarized
in \appendixref{ch:fitparams}.
\begin{figure}[h]

  \centering

 \cfig{7}{spectrum}{5}

 \caption{The spectral evolution of GRB 081224A is an example of the
   typical evolution observed for the entire sample.  The synchrotron
   (from \emph{light blue} to \emph{dark blue}) and blackbody (from
   \emph{yellow} to \emph{red}) both evolve from hard to soft peak
   energies with time. For this GRB, the high-energy power-law
   corresponding to the electron spectral index does not evolve
   significantly over the duration of the burst.}

  \label{fig:specEvo}

\end{figure}
The C-stat fit statistic per degree of freedom was at or
near 1 for most time bins. The spectral fit residuals cluster around
zero, with no deviations at low-energy that might indicate the
presence of an additional power-law component
(\figureref{fig:counts}). The residuals are below 4$\sigma$ for the
entire energy range. 
\begin{figure}

 \centering

 \subfigure[]{
   \label{fig:counts:a}
  \cfig{7}{counts}{3.2}
}\subfigure[]{
   \label{fig:counts:b}
   \cfig{7}{counts-1001}{3.}
}





\caption{A time bin of GRB 110721A (left panel) and GRB 100707A (right
  panel) demonstrating typical count spectra from the sample. Two
  extreme cases are shown: a subdominant and dominant blackbody
  component. The response has been convolved with synchrotron
  (\equationref{eq:synch_flux}) and a blackbody to produce counts. The
  residuals from the fits indicate that the model is fitting the data
  well.}

\label{fig:counts}

\end{figure}
As an example, the fit C-stats for GRB 100707A
and GRB 110721A are shown in \tableref{tab:grb1c,tab:grb2c}
respectively. They show that the slow-cooled synchrotron model fits
the data as well as the Band function when a blackbody is included in
both cases. The fit C-stats for fast-cooled synchrotron are shown for
GRB 110721A to compare all three non-thermal models. These results
imply that slow-cooled synchrotron is a viable model for GRB prompt
emission. We cannot claim it provides a better fit to the data than
other untested models and we will investigate and compare other
physical models in future work.

\begin{table}[h!]

\centering

\begin{tabular}{c | c c c c}



Time Bin & Band C-Stat & Band+BB C-Stat & Synchrotron C-stat & Synchrotron+BB C-stat \\ 

\hline \hline

-0.2-0.2 & 453 & 450 & 485 & 455 \\ 



0.2-0.8 & 374 & 362 & 695 & 364 \\ 



0.8-2.4 & 427 & 405 & 2546 & 487 \\ 



2.4-3.0 & 408 & 400 & 903 & 413 \\ 



3.0-4.3 & 431 & 415 & 1214 & 430 \\ 



4.3-5.7 & 397 & 363 & 791 & 399 \\ 



5.7-7.2 & 411 & 390 & 598 & 422 \\ 



7.2-12.8 & 488 & 414 & 829 & 447 \\ 



12.8-22.2 & 558 & 412 & 594 & 423 \\ 





\end{tabular}



\caption{The time resolved C-Stat values for GRB100707A show that while the Band function and synchrotron models combined with a blackbody function both fit the data well, the non-thermal functions fit the data very differently when not combined with a blackbody. Specifically, where the blackbody is the brightest (\figureref{fig:fluxComp:c,fig:fluxComp:d} intervals 2, 3, and 4) the Band function alone fits the data acceptably while the synchrotron model alone fits the data poorly. This shows that the flexibility of the Band function can mask the need for the blackbody component. The Band+blackbody fits actually fit the data better when the blackbody is very bright in this case. This is most likely due to the blackbody function (\equationref{eq:blackbody}) used is simplified and the actual emission may be broadened due to beaming effects that are only important to the fit when the blackbody is bright and the synchrotron fit is used. The Band function makes up for these effects by having a harder $\alpha$. We tested using an exponentially cutoff power-law combined with the synchrotron model and the fits were as good as those with the Band function. We will examine the use of a more realistic photosphere model in future work.}

\label{tab:grb1c}

\end{table}











\begin{table}[h!]

\centering
\scriptsize
\begin{tabular}{c | c c c c c c}



Time Bin & Band C-Stat & Band+BB C-Stat & Synchrotron C-stat & Synchrotron+BB C-stat & Fast C-stat & Fast+BB C-stat \\ 

\hline \hline

-0.07-0.08 & 640 & 640 & 673 & 673 & 709 & 709 \\ 

0.08-0.48 & 690 & 690 & 704 & 704 & 1088 & 1088 \\ 

0.48-1.28 & 709 & 668 & 688 & 670 & 1654 & 957 \\ 

1.28-2.78 & 887 & 761 & 838 & 770 & 1646 & 1041 \\ 

2.78-3.78 & 678 & 642 & 655 & 643 & 797 & 666 \\ 

3.78-5.88 & 648 & 631 & 677 & 634 & 694 & 660 \\ 

5.88-7.63 & 729 & 728 & 733 & 721 & 773 & 724 \\ 

7.63-12.63 & 932 & 693 & 693 & 692 & 756 & 698 \\ 
\end{tabular}



\caption{The C-stat values for GRB 110721A. The significance of the addition of the blackbody is not as large as with GRB 100707A (\tableref{tab:grb1c}) due to the weakness of the blackbody component. The fits for fast-cooled synchrotron are included to demonstrate the poor quality fits that are obtained both with fast-cooling synchrotron and fast-cooling synchrotron with a blackbody.}

\label{tab:grb2c}

\end{table}


An important parameter constrained in these fits is the electron
index, $\delta$, of the accelerated power-law. The canonical value for
diffusive acceleration at ultra-relativistic, parallel shocks is
$\delta$=2.2 \cite{Kirk:1987,Kirk:2000}. The distribution of
constrained $\delta$'s (\figureref{fig:index}) is broad and centered
around $\delta=5$ (i.e., $\beta = 3$).
\begin{figure}[h!]

 \centering

  \cfig{7}{index}{4}

  \caption{The distribution of electron indices from the slow-cooling
    synchrotron fits. Only indices that were constrained are
    plotted. The distribution is broad but centered at $\delta=$5
    which is much steeper than expected from simple relativistic shock
    acceleration.}

  \label{fig:index}

\end{figure}
This steep index could provide
clues for the structure and magnetic turbulence spectrum of the
shocks. \cite{Baring:2006}, \cite{Ellison:2004} and \cite{Baring:2012}
show that shock speed, obliquity, and turbulence all have a strong
effect on the electron spectral index of the accelerated
electrons. Steeper indices correspond to increasing shock obliquity in
superluminal shocks. Fit models which are built from the electron
distribution such as the one used in this work enable a direct
diagnostic of the GRB shock structure.






\subsection{Test of Fast-Cooling Synchrotron}
\label{sec:fast}
In order to see if any spectra were consistent with the fast-cooling
synchrotron spectrum, we implemented a fast-cooled synchrotron model
where the electrons were distributed according to the broken power-law
in \equationref{eq:necool}. These apply to the ``undisturbed plasma''
outside the shock acceleration/injection zone. As with the
slow-cooling fits, $\gammaMin$ was held fixed to 900. Several spectra
were tested and all resulted in very poor fits regardless of whether
the low-energy index found with the Band function was much harder than
$-$3/2 (\figureref{fig:fastS} and \tableref{tab:fast}). This is due to
the broad spectral curvature of the fast-cooled spectrum around the
$\nu F_{\nu}$ peak. 
\begin{figure}

 \centering

  \cfig{7}{fast}{5}

  \caption{The fast-cooled synchrotron fits are poor for nearly all of our
    sample because none of the spectra have a low-energy
    index as steep as $-$3/2. Therefore the fast-cooled synchrotron
    spectrum is too broad around the $\vFv$ peak as shown in
    this example spectrum.}

  \label{fig:fastS}

\end{figure}


\begin{table}[h!]
\scriptsize
\centering

\begin{tabular}{c | c | c c | c}



Time Bin & Band $\alpha$ & Slow-cooled Synchrotron C-stat & Fast-cooled Synchrotron C-stat & $\Delta_{\rm C-stat}$ \\ 

\hline \hline



-5.38-2.82 & -0.9 & 523 & 599 & 76 \\ 



2.82-3.84 & -0.7 & 507 & 604 & 97 \\ 



3.84-4.86 & -0.8 & 506 & 596 & 90 \\ 



4.86-6.91 & -1.0 & 534 & 626 & 92 \\ 



6.91-9.98 & -1.1 & 591 & 639 & 48 \\ 



9.98-15.1 & -1.5 & 494 & 494 & 0 \\ 



\end{tabular}

\caption{For each time bin of GRB 110407A we examine the C-stat value of synchrotron and fast-cooled synchrotron. This GRB did not have a blackbody in its spectrum. While the Band function and slow-cooling synchrotron fit the spectrum well, fast-cooling synchrotron does not fit the spectrum unless Band $\alpha=-1.5$. In this case, the fast-cooled synchrotron peak energy was very unconstrained due to the curvature of the data being narrower than the photon model's curvature.}

\label{tab:fast}

\end{table}

The broken power-law nature of the electron
distribution is smeared out by the synchrotron kernel and cannot fit
the typical curvature of the GBM data. In fact, the fast-cooling
synchrotron spectrum has a spectral index of $-$2/3 below the
$\gamma_c$ which we have fixed at 1. The fitting algorithm increased
$E_*$ to high values to align the $-$2/3 index with the data, which
resulted in poor fits (\figureref{fig:fastComp}). Even when a
blackbody is present in the bursts, fast-cooled synchrotron is not a
good fit to the non-thermal part of the spectrum
(\tableref{tab:grb2c}). The lack of GRBs with low-energy indices as
steep as $-$3/2, additionally disfavors fast-cooled synchrotron as the
non-thermal emission component in GRB spectra.
\begin{figure}[h!]

 \centering

  \cfig{7}{fastComp}{5}

  \caption{An example time bin of GRB 110407A comparing the fitted
    $\vFv$ spectra of the Band Function (\emph{green}),
    slow-cooled synchrotron (\emph{red}), and fast-cooled synchrotron
    (\emph{blue}). While the Band function and slow-cooled synchrotron
    fits resemble each other, the fast-cooled synchrotron fit is only
    able to fit the low-energy part of the spectrum. Because
    fast-cooled synchrotron has an index of $-$2/3 below the cooling
    frequency, the fitting engine pushes the value of E$_*$ very high
    to fit the low-energy part of the spectrum resulting in a $-$3/2
    index near the $\vFv$ peak. The high-energy power-law of
    the fast-cooling synchrotron spectrum is pushed out of the data
    energy window.}

  \label{fig:fastComp}

\end{figure}

\begin{table}[h!]
\centering
\begin{tabular}{c | c c c}

Time Bin & Band-BB {\dcstat} & Synchrotron-BB {\dcstat} & Fast-BB {\dcstat} \\ 
\hline \hline

-0.07-0.08 & 0 & 0 & 0 \\ 

0.08-0.48 & 0 & 0 & 0 \\ 

0.48-1.28 & 41 & 18 & 697 \\ 

1.28-2.78 & 126 & 68 & 605 \\ 

2.78-3.78 & 36 & 12 & 131 \\ 

3.78-5.88 & 17 & 43 & 34 \\ 

5.88-7.63 & 1 & 12 & 49 \\ 

7.63-12.63 & 239 & 1 & 58 \\ 


\end{tabular}

\caption{ The {\dcstat} values for GRB 110721A tell a different story than GRB 100707A (\tableref{tab:grb1dc}), though both GRBs show a significant improvement in the fit when a blackbody is included. Even thought the fast-cooled fits showed extreme improvement with the inclusion of a blackbody, the fits are still poor compared with the slow-cooled model (See \tableref{tab:grb2c}).}
\label{tab:grb2dc}
\end{table}


\subsection{Synchrotron vs. Band}
\label{sec:results:bvs}
The Band function has been used in the literature as a proxy for
distinguishing among non-thermal emission mechanisms. The predicted
non-thermal emission of GRBs is typically characterized as a
smoothly-broken power-law with the high-energy spectral index related
to the index of accelerated electrons and the low-energy index related
to the radiative emission process. Therefore, fitting a Band function
to the emission spectrum of a GRB {\em should} serve as a diagnostic
of the radiative process responsible for the
emission. \cite{preece:1998} examined the BATSE GRB catalog and
looked at the distribution low-energy indices from Band function
fits. They found that the distribution peaked at $\alpha\approx -1$
and that 1/3 of the fitted spectra had low-energy indices too hard for
synchrotron radiation. The assumption is that the Band function's
shape approximates synchrotron but has an added degree of freedom in
the low-energy index. However, the Band function has a broader range
of curvatures around the $\vFv$ peak allowing it the possibility to
deviate from the shape of synchrotron above and around the $\vFv$
peak. The synchrotron $\vFv$ peak is $\propto\gammaMin^2 B\propto
E_*$, leading to the relation between Band and synchrotron models
E$_{\rm p}\propto E_*$. This relationship is easily recovered from our
sample (\figureref{fig:EpEc}). 
\begin{figure}[h!]

  \centering

  \cfig{7}{EpEc}{4}

  \caption{Derived values of the parameter E$_{\rm p}$ (obtained using
    the Band function to fit GRB spectra), versus $E_*$ (obtained using
    an optically-thin non-thermal synchrotron to fit GRB spectra).}

  \label{fig:EpEc}

\end{figure}
Direct comparison of the quality of
the fits using Band and the synchrotron model is not the goal of this
study. Both Band and the synchrotron model fit the data well with
their respective fit residuals not deviating more than 4$\sigma$ and
centered around zero (\figureref{fig:counts}). It is important to
stress that the questions being asked are does the synchrotron model
fit the data?  and, what temporal evolution do the synchrotron
parameters undergo?

For all GRBs in our sample that include both a blackbody and
non-thermal component we compare the photon flux (photons s$^{-1}$
cm$^{-2}$) lightcurves (integrated from 10 keV - 40 MeV) derived from
synchrotron fits with those derived from Band fits (
\figureref{fig:fluxComp}). It is seen that while both methods recover
the same total flux, the flux from the individual components is much
better constrained when using the synchrotron model for the
non-thermal component. This is due to the pliability of the Band
function below E$_{\rm p}$ that is not afforded to the synchrotron
model.
\begin{figure}

  \centering

   \subfigure{
    \label{fig:fluxComp:a}
    \cfig{7}{10fc}{2.3}}\subfigure{
    \label{fig:fluxComp:b}
    \cfig{7}{11fc}{2.3}}
  \subfigure{
    \label{fig:fluxComp:c}
    \cfig{7}{0fc}{2.3}}\subfigure{
    \label{fig:fluxComp:d}
    \cfig{7}{1fc}{2.3}}
\subfigure{
    \label{fig:fluxComp:e}
    \cfig{7}{5fc}{2.3}}\subfigure{
    \label{fig:fluxComp:f}
    \cfig{7}{6fc}{2.3}}

  \caption{A subset of flux lightcurves illustrating both the temporal
    structure of the different components and the advantages of using a
    physical model to deconvolve the detector response. The left
    column contains the lightcurves using synchrotron (\emph{blue
      thick line}) and blackbody (\emph{red thin line}) while the
    right column contains the lightcurves made from using the Band
    function (\emph{green thick line}) and blackbody (\emph{red thin
      line}). The total flux lightcurve (\emph{black dotted line}) of
    both approaches are the same. The components have a very simple
    and constrained evolution when using synchrotron as the
    non-thermal component. This is potentially indicative that
    synchrotron is the actual emission mechanism and the response is
    being properly deconvolved. In contrast, the lightcurves where the
    Band function is used have large errors and the blackbody does not
    have a consistent evolution.}

  \label{fig:fluxComp}

\end{figure}



The C-stat fit values for the synchrotron model loosely correlate with
the value of Band $\alpha$ found by fitting the same interval with the
Band function. When Band $\alpha$ was much harder than zero, the
synchrotron fit was poor and typically required adding blackbody to
fit the data. The flexibility of the Band function with its low-energy
power law creates the possibility that the index alpha of that power
law will not accurately measure the true slope if E$_{\rm p}$ is too
close to the low-energy boundary of GBM data. Simulated spectra using
the Band function were created with a grid in both Band $\alpha$ and
E$_{\rm p}$ to ensure that low values of E$_{\rm p}$ do not affect the
reconstruction of Band $\alpha$ in our fits. It was found that Band
$\alpha$ could be accurately measured when E$_{\rm p}$ was as low as
$\sim$20 keV. While the asymptotic value of synchrotron is $-$2/3,
fitting the photon model with an empirical function like Band with a
slightly different curvature could result in measured low-energy
indices that are different. To measure this effect, simulated
synchrotron spectra with different $E_*$ were fit with the Band
function. The Band $\alpha$ showed a slight dependence on the
synchrotron peak; moving to softer values for lower $E_*$. The
distribution of fitted Band $\alpha$ values from these simulations
centered around $-$0.81 $\pm$ 0.1, a slightly softer value than $-$2/3
which may explain the clustering of Band $\alpha$ at $-$0.82 in the
GBM spectral catalog \cite{Goldstein:2012} if a majority of the
non-thermal spectra are the result of synchrotron emission.



\subsection{High-Energy Correlations}
\label{sec:results:hec}
There is a well-known spectral evolution in GRB pulses of $E_{\rm
  peak}$ evolving from hard to soft (see
\figureref{fig:lc,fig:specEvo}). This leads to two time-resolved
correlations between hardness (measured as E$_{\rm p}$) and flux
\cite{golenetskii:1983,Liang:1996,Ghisellini:2010}. \cite{Liang:1996}
(hereafter LK96) showed that the hardness intensity correlation (HIC)
which relates the instantaneous energy flux $F_{E}$ to spectral
hardness and can be defined as
\begin{equation}
  F_E\;=\;F_0\left(\frac{E_{\rm p}}{E_{{\rm p},0}} \right)^q \;,
\end{equation}
where $F_0$ and E$_{\rm p,0}$ are the initial values at the start of the
pulse decay phase and $q$ is the HIC index. \cite{Ryde:2001} found
that 57\% of a sample of 82 BATSE GRBs were consistent with this
relation. The second relation is the hardness-fluence correlation
(HFC) which relates hardness to the time-running fluence of the
GRB. Time-running fluence, $\Phi(t)$, is defined as the cumulative,
time-integrated flux of each time bin in a GRB. The HFC is expressed
as
\begin{equation}
  \label{eq:hfc}
  E_{\rm p}\;=\;E_{\rm p,0}e^{-\Phi(t)/\Phi_0}\;,
\end{equation}
where $\Phi_0$ is the decay constant. LK96 noted that this equation is
similar to the form of a confined radiating plasma. This should not be
the case for optically-thin synchrotron. Upon differentiating
\equationref{eq:hfc} it becomes apparent that the change in hardness
is nearly equal to the energy density:
\begin{equation}
  \label{eq:plasma}
  -\frac{dE_{\rm p}}{dt}\;=\;-\frac{F_{\nu}E_{\rm p}}{\Phi_0}\;\approx\;-\frac{F_E}{\Phi_0}.
\end{equation}
The HFC could be the result of a confined plasma with a fixed number
of particles cooling via $\gamma$-radiation as proposed by LK96. Since
these relations are only applicable during the decay phase of a pulse
the value of $T_{max}$ (the time of the peak flux) from the pulse fit
of each GRB is used as the initial point for $F_0$ and $E_{\rm p,0}$.

The use of a hardness indicator is somewhat ambiguous. Historically,
the ratio of counts in low and high-energy channels was used as a
hardness measure. This has an advantage of being model-independent but
suffers from the lack of information associated with the instrument
response. High-energy photons can scatter in the detector and not
deposit their full energy thereby artificially lowering the hardness
ratio. LK96 used the Band function E$_{\rm p}$ to compute hardness,
which as a deconvolved quantity is less instrument-dependent but
introduces a model dependence. We take this approach for both the Band
and synchrotron model fluxes. For synchrotron we use the $E_*$
parameter as our hardness indicator. This is justified by the
relationship between $E_*$ and E$_{\rm p}$ (see
\sectionref{sec:results:bvs} and \figureref{fig:EpEc}).


We compute the HIC and HFC for the synchrotron fits for each GRB in
our sample (\figureref{fig:Epcor} and \tableref{tab:cor}). All the
GRBs seemed to follow the HIC to some extent. We find that the HIC
index for $E_*$ ranges between $\approx$1-2. When using the Band
function E$_{\rm p}$ as a hardness indicator, it is expected that
E$_{\rm p} \propto {\rm L}^{1/2} \propto F_{E}^{1/2}$, which follows
from synchrotron theory, supposing that only the $\Gamma$ factor
changes while the internal properties remain (however unlikely) the
same \cite{Ghisellini:2010}. Decay behavior due to light travel-time
effects of a briefly illuminated relativistic spherical shell varies
according to $E_{p}\propto L^{1/3}$, that is, $q = 3$
\cite{Kumar:2000,Dermer:2004,Genet:2009}, whereas GRB observations
here show $L\propto F \propto E_{p}^{1.1}$ -- $E_{p}^{2.3}$
(\tableref{tab:cor}).  Evolution of internal parameters that would
explain the observed correlations is an open question. The synchrotron
fits seem to obey the HFC fairly well. Owing to the large errors in
the Band flux, fits with synchrotron are more consistent with the HFC
and HIC than those with Band. The deviations of the data from the
expected synchrotron HIC may be due to the fact that there are
overlapping pulses under the main emission that alter the decay
profile. In addition, the use of Bayesian blocks to select time bins
ignores spectral evolution. If bins with very different E$_{\rm p}$
are combined then it could affect the the HIC and HFC data.

\begin{figure}[h!]

  \begin{center}
    \subfigure{
      \label{fig:Epcor:a}
      \cfig{7}{flux}{4.2}
    }    \subfigure{
      \label{fig:Epcor:b}
      \cfig{7}{fluence}{4.2}
    }
  \end{center}  

  \caption{The non-thermal emission of all of the bursts in the sample
    loosely follow $F_E$-$E_p$ and $E_p$-fluence relations. See 
    \tableref{tab:cor} for the numerical results. }

  \label{fig:Epcor}

\end{figure}


\begin{table}[h!]
\centering
\begin{tabular}{c | c c c c}
GRB & Flux Index $q$ & $\chi^2_{red}$ & $\Phi_0$ & $\chi^2_{red}$ \\ 
\hline \hline
GRB 081110A & 2.32$\pm$0.4 & 0.6 & 97$\pm$23 & 0.4 \\ 

GRB 081224A & 1.74$\pm$0.1 & 1.5 & 253$\pm$23 & 0.3 \\ 

GRB 090719A & 1.14$\pm$0.07 & 0.98 & 245$\pm$17 & 1.2\\

GRB 09080B & 1.58$\pm$0.05 & 8.0 & 188$\pm$9 & 1.0 \\ 

GRB 100707A & 1.04$\pm$0.02 & 1.2 & 444$\pm$24 & 7.3 \\ 

GRB 110407A & 1.72$\pm$0.20 & 0.5 & 214$\pm$32 & 4.2 \\ 

GRB 110721A & 1.08$\pm$0.03 & 14.4 & 269$\pm$13 & 15.4 \\ 

GRB 110920A & 1.37$\pm$0.06 & 0.5 & 669$\pm$33 & 1.2 \\ 



\end{tabular}
\caption{Sample correlations for both flux and fluence for the synchrotron component.}
\label{tab:cor}
\end{table}

%%



\subsection{Blackbody component}
\label{sec:results:bb}
For most of the spectra in our sample, the blackbody's $\vFv$ peak is
below the $\vFv$ peak of the non-thermal component. There is sometimes
a much larger change in C-stat between fits with synchrotron and
synchrotron+blackbody than those of Band and Band+blackbody owing to
the fact that the Band function has more freedom in the shape below
E$_{\rm p}$ (\tableref{tab:grb1dc,tab:grb2dc}). Simulations of both
Band and slow-cooling synchrotron were used to find the significance
of adding a blackbody to the spectrum. As an example, the time bin
covering 0.8 s to 2.4 s is examined here. The \dcstat between the fit
with the Band function and the fit with Band and the blackbody is 22
while \dcstat using the synchrotron model is 2059. 10000 simulations
of each model were created as described in \sectionref{sec:sigtest}
and fit with $H_0$ and $H_1$. For the Band function a p-value of
$4\times 10^{-4}$ was obtained while the p-value of the Synchrotron
fits was not obtainable due to CPU time constraints but is clearly
smaller than that of the Band function. Therefore, the addition of the
blackbody is significant.

\begin{figure}[h!]
  \centering
  \subfigure{\cfig{7}{banddcstat}{3.2}}\subfigure{\cfig{7}{bandfracFig}{3.2}}
  \subfigure{\cfig{7}{syncdcstat}{3.2}}\subfigure{\cfig{7}{syncfracFig}{3.2}}
  \caption{The \dcstat distribution and cumulative distribution from
    the Band+blackbody (\emph{top}) and synchrotron+blackbody
    (\emph{bottom}) simulations. Two p-value levels are shown in the
    cumulative plots indicating the classical 1$\sigma$ and 2$\sigma$
    significance levels. The blue line indicates the fraction of the
    distribution below the fitted \dcstat value and the red line
    indicates the fraction above it. It can be seen that the \dcstat
    value from the synchrotron fit is far out in the tail of the
    distribution.}
  \label{fig:sigtest}
\end{figure}


\begin{table}[h]
\centering
\begin{tabular}{c | c c}

Time Bin & Band-BB {\dcstat} & Synchrotron-BB {\dcstat} \\ 
\hline \hline

-0.2-0.2 & 3 & 30 \\ 

0.2-0.8 & 12 & 331 \\ 

0.8-2.4 & 22 & 2059 \\ 

2.4-3.0 & 8 & 490 \\ 

3.0-4.3 & 16 & 784 \\ 

4.3-5.7 & 34 & 392 \\ 

5.7-7.2 & 21 & 176 \\ 

7.2-12.8 & 74 & 382 \\ 

12.8-22.2 & 146 & 171 \\ 

\end{tabular}

\caption{The {\dcstat} between the Band function and synchrotron model fits with and without the inclusion of a blackbody for GRB 100707A. The blackbody has a significantly larger impact on the fit when included with the synchrotron model.}
\label{tab:grb1dc}
\end{table}



\begin{table}[h]
\centering
\begin{tabular}{c | c c c}

Time Bin & Band-BB {\dcstat} & Synchrotron-BB {\dcstat} & Fast-BB {\dcstat} \\ 
\hline \hline

-0.07-0.08 & 0 & 0 & 0 \\ 

0.08-0.48 & 0 & 0 & 0 \\ 

0.48-1.28 & 41 & 18 & 697 \\ 

1.28-2.78 & 126 & 68 & 605 \\ 

2.78-3.78 & 36 & 12 & 131 \\ 

3.78-5.88 & 17 & 43 & 34 \\ 

5.88-7.63 & 1 & 12 & 49 \\ 

7.63-12.63 & 239 & 1 & 58 \\ 


\end{tabular}

\caption{ The {\dcstat} values for GRB 110721A tell a different story than GRB 100707A (\tableref{tab:grb1dc}), though both GRBs show a significant improvement in the fit when a blackbody is included. Even thought the fast-cooled fits showed extreme improvement with the inclusion of a blackbody, the fits are still poor compared with the slow-cooled model (See \tableref{tab:grb2c}).}
\label{tab:grb2dc}
\end{table}
It was found that even when the
difference in C-stat between Band and Band+blackbody was greater than
the difference between synchrotron and synchrotron+blackbody fits, the
statistical significance in the goodness of fit after the addition of
the blackbody is high if not greater for the synchrotron+blackbody
model for many cases. Computational time limits kept us from checking
if the significance reached 5$\sigma$.  We now focus on the blackbody
component that is found in the synchrotron+blackbody fits. The
blackbody appears to have a separate temporal structure from the
non-thermal component, typically peaking earlier in time and decaying
before the non-thermal emission (\figureref{fig:fluxComp}).

The form of the blackbody used in this work
(\equationref{eq:blackbody}) is simplified and therefore will likely
only approximate the true form of thermal emission from a GRB
photosphere. Since the blackbody is weaker than the non-thermal
(synchrotron) component in the spectrum, the effects of a broadened
and more realistic relativistic blackbody are masked and would only
slightly affect the fit when combined with the synchrotron
model. However, in the case of GRB 100707A, the blackbody is very
bright (\figureref{fig:fluxComp:c}) and subtle changes in actual shape
of the photospheric emission become more apparent. This is reflected
in the C-stat values in \tableref{tab:grb1c}. The Band function
combined with the standard blackbody is a better fit than when using
synchrotron as the non-thermal component. In this case, the Band
function $\alpha$ is still very hard indicating that the Band function
is making up for additional flux that the blackbody is not taking into
account. To test this hypothesis, we fit the synchrotron model along
with an exponentially cutoff power-law to mimic a modified
blackbody. We found that the fits were as good as the Band function
combined with blackbody fits indicating that when the blackbody is
bright compared to the non-thermal emission a more detailed model of
the photospheric emission is needed to fit the thermal part of the
spectrum.


The HIC index for the blackbody component is expected to be 4 provided
N remains a constant in \equationref{eq:sbl}; however, nearly all the
blackbodies had a HIC index of $q\sim2$ (
\figureref{fig:kTcor}). These results confirm those of
\cite{Ryde:2001} and \cite{Ryde:2005} who fit BATSE spectra with a
combination of a blackbody and a power-law to account for the
non-thermal component. \cite{Ryde:2006} describes a toy model for the
blackbody that allows a range of temperature indices related to the
internal structure of the photosphere that may account for these
results. If it is assumed that the $\Gamma \propto t^{\zeta}$, i.e.,
that the flow has a variation in entropy then we can arrive at $L_{ph}
\propto F_{BB} \propto T^{(19\zeta -24)/3\zeta}$. the variation in
$\zeta$ could explain the deviation from \equationref{eq:sbl} observed
in our sample.
\begin{figure}[t]
\centering

    \subfigure{
      \label{fig:kTcor:a}
      \cfig{7}{kTFlux}{4.7}}
    \subfigure{
      \label{fig:kTcor:b}
      \cfig{7}{kTFluence}{4.7}}

\caption{The HIC and HFC correlations for the blackbody are separate
    from those derived from the synchrotron component. This adds more
    evidence for the presence of the component. However, the HIC for
    the blackbody is not q=4 as expected unless $\mathcal{R}$
      varies as is observed.}
\label{fig:kTcor}
\end{figure}

\begin{table}[h!]
\centering
\begin{tabular}{c| c c c c}
GRB & Flux Index & $\chi^2_{red}$ & $\Phi_0$ & $\chi^2_{red}$ \\
\hline \hline
GRB 081224A & 2.3 $\pm$ 0.3 & 1.4 & 121 $\pm$ 13 & 9 \\ 

GRB 090719A & 2.8 $\pm$ 0.4 & 2.3 & 232 $\pm$ 23 & 3 \\ 

GRB 100707A & 2.2 $\pm$ 0.1 & 17.4 & 319 $\pm$ 8 & 28 \\ 

GRB 110721A & 1.3 $\pm$ 0.2 & 1.8 & 43 $\pm$ 3 & 5 \\ 

GRB 110920A & 2.0 $\pm$ 0.1 & 0.9 & 1147 $\pm$ 21.7 & 4 \\ 



\end{tabular}
\caption{For the subset of bursts that have a strong blackbody component we compute the flux and fluence correlation for the blackbody.}
\label{tab:bbCor}
\end{table}

Another interesting quantity that can be obtained from the blackbody
is the HFC. All GRBs in our blackbody subset had blackbodies
consistent with the HFC (\figureref{fig:kTcor}). The decay
constants were all of similar value. \cite{Crider:1999} noted that
similar values of $\Phi_0$ for non-thermal components arise as a
consequence of a narrow parent distribution. A deeper investigation of
a larger sample is required to assess if the same is true for the
blackbody components.


The temporal evolution of kT for the blackbody of each burst appears
to follow a broken power-law (\figureref{fig:kTEvo}). The evolution
is fit with the function derived in \cite{Ryde:2004} where we fixed
the curvature parameter, $\delta$, to 0.15. The coarse time binning
derived from Bayesian blocks does not allow for the decay indices to
be constrained for all the bursts but a small subset are close to
$-$2/3, as expected (see \tableref{tab:bbEvo}). The temporal decay
of the blackbody is different than the power-law decay of $E_*$,
indicating a different emission component.
\begin{figure}[h!]

  \centering

  \subfigure[]{
    \label{fig:ktEvo:a}
    \cfig{7}{081224887-evo}{3}}\subfigure[]{
    \label{fig:ktEvo:b}
    \cfig{7}{090719063-evo}{3}}
\subfigure[]{
    \label{fig:ktEvo:c}
    \cfig{7}{100707032-evo}{3}}\subfigure[]{
    \label{fig:ktEvo:d}
    \cfig{7}{110721200-evo}{3}}
\subfigure[]{
    \label{fig:ktEvo:e}
    \cfig{7}{110920546-evo}{3}
}

\caption{The time evolution of kT for four of the GRBs in our
  sample. GRB110721A is shown without a fit because the coarse time
  binning used did not allow for constraining the fit
  parameters. However, in \cite{Axelsson:2012}, the evolution is
  shown to follow a broken power-law.}

   \label{fig:kTEvo}

 \end{figure}


\begin{table}[h!]
\centering
\begin{tabular}{l| c c c c}
GRB & $F_{bb}/F_{syn}$ & $1^{st}$ Decay Index & $2^{nd}$ Decay Index & $\chi^2_{red}$ \\
\hline \hline 

GRB 081224A & 0.3 & -0.6 $\pm$ 0.07 & -20 $\pm$ 243 & 0.5 \\ 

GRB 090719A & 0.4 & -0.1 $\pm$ 0.05 & -2.0 $\pm$ 0.7 & 11.3 \\ 

GRB 100707A & 0.5 & -0.4 $\pm$ 822749 & -0.8 $\pm$ 0.03 & 22.7 \\ 

%GRB110721A & 0.015 & 819.314 $\pm$ 5120492.859 & -1.432 $\pm$ 0.257 & 0.91 \\ 

GRB 110920A & 0.8 & -0.3 $\pm$ 0.03 & -0.9 $\pm$ 0.04 & 2.0\\ 


\end{tabular}
\caption{The evolution of the blackbody follows a broken power-law. However, the coarse time bins recovered by the Bayesian blocks algorithm make it difficult to constrain the decay indices.}
\label{tab:bbEvo}
\end{table}

The $\mathcal{R}$ parameter was observed to increase with time for all
the GRBs (\figureref{fig:scR}). There are breaks and
plateau in the trends that do not seem to correlate with
the breaks observed in the evolution of kT or with the flux history of
the blackbody. \cite{Ryde:2009} found that the evolution of
$\mathcal{R}$ can be quite complex but mostly follows an increasing
power-law that seems independent of the flux history even for very
complex GRBs. For those complex GRB lightcurves, it was found that
analyzing different intervals of the overlapping pulses yielded an HIC
index for the blackbody of $q\sim 4$ for each interval. In those
intervals, $\mathcal{R}$ was approximately constant indicating the
emission size of the photosphere was constant. Owing to the small
number of time bins, it is difficult to quantify the evolution of
$\mathcal{R}$ for the single pulse GRBs of this study with the coarse
time binning used, but the fact that the HIC index for the blackbodies
differs from $q\sim 4$ and that $\mathcal{R}$ increases indicates that
the evolution of the photosphere is very complex.


\begin{figure}[h!]

  \centering



\subfigure[]{
    \label{fig:scR:a}
    \cfig{7}{R_0}{3}}\subfigure[]{
    \label{fig:scR:b}
    \cfig{7}{R_1}{3}}
\subfigure[]{
    \label{fig:scR:c}
    \cfig{7}{R_2}{3}}\subfigure[]{
    \label{fig:scR:d}
    \cfig{7}{R_3}{3}}
\subfigure[]{
    \label{fig:scR:e}
    \cfig{7}{R_4}{3}
}





\caption{The evolution of the $\mathcal{R}$ parameter is increases
  with time and shows no relation to the photon flux of the blackbody
  component.}
   \label{fig:scR}

 \end{figure}

\section{Discussion }
\label{sec:discussion}
\subsection{ Importance of Fitting with Physical Models}


By using physical synchrotron emissivities in analysis of GRB data, we
have shown that the {\it Fermi} data are consistent with synchrotron
emission from electrons that have not cooled (i.e, slow-cooling
spectra) and are inconsistent with synchrotron emission from electrons
that are cooling (fast-cooling).  The method leads to some interesting
conclusions for empirical modeling.  There is a positive correlation
between hard $\alpha$ and the inability of model to fit the data, but
the low-energy index of synchrotron seems to be clustered around
$-$0.8 rather than near the asymptotic $-$2/3 found in
\cite{preece:1998}. Not only do the fits with Band $\alpha$ near
$-$0.8 lead to better fits with synchrotron, but simulated synchrotron
spectra are best fit with a Band function having $\alpha\approx -0.8$.


Previously, GRB spectra have been successfully fitted with a thermal +
non-thermal model by using a blackbody function combined with a Band
function. A thermal spectral component has indeed been shown to be
significant in several cases, foremost in GRB 100724B and in GRB
110721A \cite{Guiriec:2011,Axelsson:2012}. The Band function in these
fits is, however, not based on any physical arguments but is merely an
empirical function that has a convenient parameterization. A general
problem that arises in this type of fitting is that Band $\alpha$ and
the strength of the blackbody component give fits with degenerate
parameters. When the data are fit by a Band function alone, even if an
additional component really exists in the data at lower energies, it
may not be identified because the Band $\alpha$ can accommodate the
additional low energy flux by changing its slope.

The slow-cooling synchrotron model we use here is more restrictive
compared to the Band function. In particular, a limit to the low
energy slope and the curvature of the spectrum are predicted by the
model.  We find that the spectrum below the synchrotron $\vFv$ peak is
not always satisfactorily fit using just the synchrotron model. Except
for extreme cases such as fast and marginally fast-cooling, which
affect the width of the peak as much as the low-energy index, we find
that the the low-energy photon spectrum is actually well modeled with
a slope equal to the low-energy slope of the single particle
synchrotron emissivity. This is only possible with very low-radiative
efficiency if the standard GRB acceleration model described in Section
2.1 is considered.


In many of our fits an additional component is suggested by the
residuals, and the simulations show that this additional component is
statistically significant. Additional components can also be favored
in GRB spectra fit with the Band function, but we find that the
significance of the additional component can be greater when using
physical synchrotron emission fits than when using Band fits. Because
the Band function can accommodate the extra emission using a suitable
power-law index $\alpha$, but the synchrotron function is more
restrictive, an additional component may be more significantly
required when using synchrotron emission for a prescribed electron
distribution.



Another point in \figureref{fig:fluxComp} is that when using the
synchrotron model, the temporal evolution of the blackbody flux
exhibits well-defined pulses and a spectral evolution that is clearly
separated from the non-thermal emission. This is in contrast to the
less smooth blackbody flux variations when using the Band function as
the non-thermal process. This fact again reflects that the Band
function is less restrictive than the synchrotron function and thereby
gives rise to further scatter in the derived fluxes in the light
curves. These results suggest that:
\begin{enumerate}[(i)]
 \item the synchrotron function is a good physical model to use;
 \item the thermal component does exist; and
 \item multi-component fitting with the Band function can be misleading.
\end{enumerate}



\subsection{Alleviating Problems with Synchrotron Models}
\label{sec:ascp}
The fact that the non-thermal spectra seem to be consistent with
slow-cooled synchrotron rather than the fast-cooled synchrotron regime
places strong constraints on the emission model of GRBs.  The
low-energy spectral index of 11 bright BATSE GRBs fall between the
cooled and uncooled limits \cite{Cohen:1997}.  \cite{Ghisellini:2000}
showed that it was difficult to reconcile the implied fast-cooling
from a comparison of cooling and dynamical timescales with the many
GRB spectra that require a slow-cooling electron distribution, leading
to spectral problems for the internal shock model. In \appendixref{ch:pap2app}, we
show that a weak-cooled system requires $\lesssim 100$ G fields for
typical bright GRBs detected with {\it Fermi}, rather than 100 kG
fields, with typical electron Lorentz factors $\gamma^\prime \approx
10,000$ rather than $300$.


In our simple strong-field synchrotron model, we can neglect the
effects of Compton cooling, which can significantly alter the value of
the low-energy slope in certain parameter regimes
\cite{Daigne:2011}. This could make some spectra less consistent with
fast-cooling, but requires further study.  Klein-Nishina effects on
Compton cooling were not considered, but in the absence of extra
spectral components, either from SSC, hadronic emissions, or external
Compton processes, our synchrotron study is consistent. The need for a
slow-cooling scenario, or marginally slow-cooling system in order to
have reasonable radiative efficiency, is obtained in external shock
model calculations by choosing the $\epsilon_B$ parameter $\approx
10^{-3}$ -- 10$^{-4}$ \cite{Chiang:1999}.

%Neglecting Klein-Nishina effects, 

The fast-cooling internal-shock scenario cannot be reconciled with our
observations. Additionally, \cite{Iyyani:2013} found that for GRB
110721A, the standard slow-cooling synchrotron scenario from impulsive
energy input such as internal shocks places the non-thermal emission
region below the photosphere. This may be understood if the electrons
are highly radiative, yet without displaying a cooling spectrum.

% CD: I found the above text to be not-so-enlightening

Models with ongoing acceleration via first-order and second-order
Fermi acceleration \cite{Waxman:1995,Dermer:2001}, or magnetic
reconnection and turbulence models, including the ICMART model
\cite{zhang:2011}, have the ability to balance synchrotron cooling
with stochastic heating, or to have multiple acceleration events,
which keep $\gamma_{\rm cool}$ above $\gammaMin$, in which case the
spectrum would resemble a slow-cooled synchrotron spectrum.
%The shell collision model has been questioned \cite{Kumar:2009}.
Magnetized jet or subjet models \cite{Lazar:2009}
can extend the non-thermal emission site far above the photosphere, and 
relativistic MHD turbulence provides an alternative second-order mechanism
\cite{Lyutikov:2013}.
%which eliminates the problem mentioned above. 
In such a scenario, the electrons cool by synchrotron, but are at the
same time subject to ongoing acceleration, contrary to the low implied
value of the cooling frequency. Slow-cooling or fast-heating scenarios
explain the data much better than a fast-cooling internal-shock model,
though the latter is more radiatively efficient.



\subsection{Conclusion}
We have demonstrated that for a set of {\it Fermi} GRBs we can fit a
physical, slow-cooling synchrotron model directly to the data. Most of
the fitted spectra also require a weaker blackbody component with a
temperature that places its peak below the synchrotron $\vFv$
peak. The temporal evolution of both radiative components shows how
GRB jet properties change, and are free of some of the assumptions
required when fitting GRB spectra with empirical functions. In our
model, a disordered magnetic field is assumed, which could be shown to
be invalid from X-ray and $\gamma$-ray polarization observations,
which are yet inconclusive.  Several parameters in our model cannot be
separately constrained by the fits, namely $\gamth$, $\gammaMin$, and
B, so we focus on a highly magnetized scenario where the self-Compton
component can be neglected.

We find that the energy flux varies as the peak photon energy $E_{p}$
of the peak of the $\nu F_\nu$ spectrum according to $E_{p}^q$, with
$1.1 \lesssim q \lesssim 2.4$.  The dependence of $E_{p}$ is found to
follow the exponential-decay behavior with accumulated fluence
$\Phi(t)$ given by (\equationref{eq:hfc}), with decay constant
$\Phi_0\approx 100$ -- $700$ phts cm$^{-2}$. (see \figureref{fig:Epcor}).
For the GRBs where both synchrotron and blackbody components can be
resolved, we find that their parameters follow a separate temporal
behavior.

The temporally evolving spectra were examined in terms of fast-cooling
and slow-cooling electron distributions, considering parameters for a
highly magnetized GRB jet.  The temporal evolution of both synchrotron
and blackbody parameters imply that in the GRBs studied, a photosphere
is formed below a non-thermal emitting region found at a radius
corresponding to the characteristic internal shock scenario.  The
electrons in the non-thermal emitting region must undergo continuous
acceleration to produce an apparently slow-cooling synchrotron
spectrum, which can be provided by magnetic reconnection events or
second-order stochastic gyroresonant acceleration with MHD turulence
downstream of the forward and reverse shocks formed in shell
collisions.  If, on the other hand, the jet fluid is not strongly
magnetized, then it will be radiatively inefficient and have a strong
inverse Compton component.  The use of physical models provides
stronger constraints on jet model parameters, and in future studies we
can relax choices of electron Lorentz factors and magnetic fields by
considering leptonic Compton cascading, and ultra-high energy cosmic
rays.



% In conclusion, even though the blackbody component is significant in

% many cases when the Band function is used, the extra freedom in shape

% given by the Band function obscures the assessment of the thermal

% component.  The direct fitting of physical models thus alleviates the

% problems of the Band function in explaining the physical origin of GRB

% spectra.



\section{Inferred GRB Jet Properties}
The presence of the blackbody in addition the synchrotron component
allows for the calculation of several fundamental properties of the
GRB jet \cite{Peer:2007}. If the emission occurs at $r_s<r$ then the values of
$r_0$, $r_{\rm ph}$, and $\Gamma$ can be infer ed from the spectral fit
results. For each GRB in the sample that includes a blackbody, we calculate the time resolved values of each of these quantities. 

\begin{figure}[t]
  \centering
  \cfig{7}{grbjet.pdf}{5}
  \caption{Conceptual diagram of various GRB radii that can be derived from the spectral fits of synchrotron+blackbody.}
  \label{fig:grbjet2}
\end{figure}


\subsection{Calculating $\Gamma$, $r_0$, and $r_{\rm ph}$ }
The value of $r_0$, the base of the GRB jet, is expected to be within
an order of magnitude of the Swarzchild radius, $r_{sc}$. To calculate
the value of $r_0$, the value of $\mathcal{R}$ is examined in the
region where $r_s<r_{\rm ph}$ and the value of $\Gamma$ is derived in
terms of observed quantities. In this regime,
\begin{equation}
  \label{eq:rph2}
  r_{\rm ph}=\dover{L \sigma_{\rm T}}{8 \pi \Gamma^3 m_p c^3}
\end{equation}
and therefore the comoving temperature is easily shown to be
\begin{equation}
  \label{eq:tempcalc}
  T^{\prime}(r_{\rm ph})=\left( \dover{L}{4 \pi r_0^2 c a}  \right)^{1/4}\Gamma^{-1}\left( \dover{r_{\rm ph}}{r_s}  \right)^{-2/3}.
\end{equation}
To relate \equationref{eq:tempcalc} to the observations, assume that
$L=4\pi d_L^2YF_E$ where $Y\ge 1$ is the ratio of total fireball
energy to the energy emitted in {\gray}s, $d_L$ is the luminosity
distance, and $F_E$ is the observed total energy flux. Combining this
with \equationref{eq:scR} the bulk Lorentz factor can be derived in
terms of the observations:
\begin{equation}
  \label{eq:blf}
  \Gamma = \left[(1.06)(1+z)^2d_L\dover{YF_E \sigma_{\rm T}}{2 m_p c^3 \mathcal{R}}  \right]^{1/4}.
\end{equation}
With $\Gamma$ derived in terms of the observations, the value of $r_0$ can be found,
\begin{equation}
  \label{eq:r0}
  r_0 = \dover{4^{3/2}}{(1.48)^6(1.06)^4}\dover{d_L}{(1+z)^2}\left( \dover{F_{BB}}{Y F_E} \right)^{3/2}\mathcal{R}.
\end{equation}
Finally, using the above derived quantities and \equationref{eq:scR}, the values of $r_{ph}$ is found to be,
\begin{equation}
  \label{eq:rphObs}
  r_{\rm ph}=\mathcal{R}\dover{d_L\Gamma}{1.06 (1+z)^2}
\end{equation}


\subsection{Observed values of $\Gamma$, $r_0$, and $r_{\rm ph}$}
Using \equationref{eq:blf,eq:r0,eq:rphObs}, the values of $\Gamma$,
$r_0$, and $r_{\rm ph}$ are calculated for the sub-sample of GRBs with
blackbodies.  The results are graphed and tabulated
in \appendixref{ch:jetParms}. These values are for an assumed redshift
of z=1 and Y=1,10, and 100. It possible that Y$\gg$1, scaling the
values calculated. However, the order of magnitude of the values
warrants a discussion. In particular, the value of $r_{\rm ph}$ is of
great interest to investigating the nature of the synchrotron
emission. The relation $r_{\rm ph}<r_{\rm nt}$ should hold if the
non-thermal spectrum is optically-thin. The value of r$_{\rm nt}$ can
be approximated by examining the cooling time of the electrons
emitting synchrotron radiation (see \appendixref{ch:rnt}). Using the
measured $\Ep$ the values of r$_{\rm nt}$ are calculated. As an
example, the values of GRB 110920A are plotted here for z=1 and
Y=1. The dotted line represents the last stable orbit of a stellar
mass black hole.
\begin{figure}[t]
  \centering
  \cfig{7}{GRB110920A_RNT_Y_1}{4}
  \caption{The evolution of the various jet radii as a function of
    time. The photospheric radius is shown in red, r$_0$ is indicated in green and the maximum r$_{\rm nt}$ in the blue shaded region. The dotted line indicates the last stable orbit of a stellar
    mass blackhole.}
  \label{fig:rnt110920}
\end{figure}
The results of the entire sample are in \appendixref{ch:jetParms}.
While the values of r$_{\rm ph}$ are typically smaller than r$_{\rm
  nt}$, some values are not. The only way to alleviate this problem is
to assume that the electrons have already cooled and therefore $E_{\rm
  cool} < \Ep $. This reverses the inequality in
\equationref{eq:derv4} and allows for $r_{\rm ph}\le r_{\rm nt}$. If
this is assumed, then the only way to have a slow-cooled synchrotron
spectrum is if the electrons are re-accelerated.




 

%%% Local Variables: 
%%% mode: latex
%%% TeX-master: "../thesis"
%%% End: 


\chapter{Origin of the Hardness-Intensity Correlation via GRB 130427A}
\label{ch:130427A}
\begin{chapterquote}{Beastie Boys}
{Now here we go dropping science,\\ dropping it all over}
\end{chapterquote}


On April 27, 2013 GBM triggered on the brightest GRB ever
detected. The GRB was so intense that the data bus was overloaded
during the brightest portion of the burst. In addition to being
detected by {\it Fermi}, this burst was seen by several satellites
enabling the measurement of a redshift of z=0.34. This measurement
allows for the calculation of rest frame properties such as L and the
intrinsic $\Ep$.

For this analysis, only the first pulse will be studied due to its
non-overlapping, single pulse nature that falls in line with the
previous studies with physical modeling. The intensity of this pulse
allows for extremely fine time-resolved spectroscopy. In
\cite{preece:2013}, the first pulse was analyzed with both the
Band+blackbody and synchrotron+blackbody models. However, when using
the Band function, there was no significant detection of the
blackbody. This finding helps illustrate the importance of using
physical models for GRB spectral analysis.

\section{Spectral Analysis}
The intensity of this GRB allowed for the binning of the lightcurve at
high-resolution while maintaining enough counts to perform spectral
analysis. The time binning method chosen was constant time width bins
of $\Delta t=0.05$s. Each bin was fit with the Band+blackbody and
synchrotron+blackbody models as is described in \sectionref{sec:specan} (see
\figureref{fig:specEvo130427A}). It was found that when using the Band
function as the non-thermal component that the blackbody component was
not significant.
\begin{figure}[t]
  \centering
  \subfigure{\cfig{8}{Bandspectrum}{5.4}}
\subfigure{\cfig{8}{Syncspectrum}{5.4}}
\caption{Spectral evolution of GRB 130427A using the Band function
  (\emph{top}) and the synchrotron+blackbody model
  (\emph{bottom}). The time evolution for the non-thermal spectra is
  from cyan to blue and from yellow to red for the blackbody.}
  \label{fig:specEvo130427A}
\end{figure}
This difference in the two analysis models leads to a different
evolution of $\Ep$ in the two models. For the Band only model, $\Ep$
decays as a single power-law in time, $\Ep \propto t^{-0.97}$. The
$\Ep$ of the synchrotron model evolves as a broken power-law with the
break occurring before the peak of the emission. The index before and
after the break are $\Ep \propto t^{-0.37\pm0.24}$ and $\Ep \propto
t^{-1.17\pm0.04}$ respectively (see \figureref{fig:epevo130427A}).
\begin{figure}[t]
  \centering
  \cfig{8}{lightcurve}{5}
  \caption{The photon flux lightcurve of the first pulse of GRB
    130427A. The $\Ep$ evolution of the Band function (\emph{blue})
    and the synchrotron model (\emph{red}) is superimposed to
    demonstrate the hard-to-soft evolution.}
  \label{fig:lightcurve130427A}
\end{figure}
\begin{figure}[t]
  \centering
  \cfig{8}{multiRp}{5}
  \caption{The log-log evolution of the $\Ep$ evolution of the Band
    function (\emph{blue}) and the synchrotron model (\emph{red}).}
  \label{fig:epevo130427A}
\end{figure}
\section{The  $L-\Ep$ Plane and Magnetic Flux-Freezing}
\label{sec:fluxfrx}
The brightness of this burst allows for the computation of the HIC in
the rest-frame and therefore the $L-\Ep$ relation is shown in
\figureref{fig:feep}. The values before the peak flux are plotted in
red and those in the decay phase are in black. The break in the
relation between the rise phase and decay phase is very prominent. The
index of the relation if $L\propto \Ep^{1.4\pm 0.06}$. This clear
measurement warrants a deeper investigation into the origin of the
relation. Attempts to explain the relation (see \sectionref{sec:results:hec}) have been unsuccessful.  
\begin{figure}[t]
  \centering
  \cfig{8}{corRestSyncBBRest}{6.5}
  \caption{The $L-\Ep$ plane of GRB 130427A. The decay phase is
    plotted in black and the rise phase in red. The fit to the decay
    is plotted in green with the 1$\sigma$ error contour. The flux
    freezing model described in \appendixref{ch:lep} is plotted in
    blue.}
  \label{fig:feep}
\end{figure}
However, it can be shown that a natural explanation for the relation
can be derived if one considers magnetic flux-freezing. In this
scenario, the flux of the magnetic field is conserved with radius,
i.e., $BR^2\propto constant$. With this assumption it can be shown
(see \appendixref{ch:lep}) that $\Ep\propto L^{3/2}$, consistent with
the data of GRB 130427A.

The GRBs in \chapterref{ch:pap2} have HIC indices that range from
$\sim 1- 2.3$. Including the errors, most of these are consistent with
the flux-freezing value of $q=-3/2$ within 2$\sigma$. Therefore, it is
plausible that a similar mechanism is responsible for the observed
correlation across the sample. 

%%% Local Variables: 
%%% mode: latex
%%% TeX-master: "../thesis"
%%% End: 

\chapter{A New Correlation Between Thermal and Non-Thermal Emission in Gamma-Ray Bursts} 
\label{ch:cor}
\begin{chapterquote}{Mark Twain}
Such is professional jealousy;\\ a scientist will never show any kindness\\ for a theory which he did not start himself.
\end{chapterquote}

\section{Introduction}

Gamma-ray bursts (GRBs) are believed to be the death of super massive
stars or the coalescence of two compact objects resulting in an
explosive, relativistic outflow. The energy of the outflow is
dissipated into the particles of the plasma which then radiate this
energy in the form of $\gamma$-rays. While numerous theories exist to
explain GRB formation and emission, a key question is how the energy
of the outflow is distributed, i.e., whether or not the energy is in
the magnetic field or as a baryon kinetic energy, and how this energy
distribution evolves with time. This question is particularly hard to
address because of the unknown progenitor source of these events and
the fact that prompt GRB observations consist of time-resolved
$\gamma$-ray spectra that have been historically described with the
empirical Band function
\cite{band:1993,Kaneko:2006,Goldstein:2012}. The standard
interpretation of the prompt gamma-ray emission of GRBs requires a
highly-relativistic, jetted fireball that has become
optically-thin. While the emission from this fireball is expected to
be thermal \cite{Goodman:1986,Paczynski:1986}, observations of the
prompt emission over the past two decades have been found to be highly
non-thermal
\cite{Mazets:1981,Fenimore:1982,matz,Kaneko:2006,Goldstein:2012}. Recent
analysis of new data collected by the {\it Fermi} Gamma-ray Space
Telescope found that the emission spectra contain at least two
components, the original non-thermal component, and sub-dominant
thermal component fit with a blackbody
\cite{Guiriec:2010,Axelsson:2012}. This component has been
speculatively linked to the fireball photosphere of the GRB. The
temporal evolution of the thermal component has been shown to follow
what would be expected for an expanding photosphere and is separate
from the evolution of the non-thermal component.

In this work we fit the non-thermal component with a synchrotron
photon model and the thermal component with a blackbody developed in
\cite{Burgess:2012,Burgess:2013}. We find that the peak energies of
the two components are highly correlated across all the GRBs in our
sample. Such a correlation points to a scaling parameter common among
GRBs which we will show is related to the magnetic and kinetic energy
content of the GRB jet.

\section{Observations}
In this work, seven long and bright {\it Fermi} GRBs were selected
that consist of single pulses in their broad-band $\gamma$-ray time
histories (see \tableref{tab:epktcor}). While this sample is limited
by the number of bright, single-pulsed GRBs in the {\it Fermi} data
set, the fine-time resolved spectroscopy used to analyze these GRBs
allows for a detailed understanding of the evolution of the thermal
and non-thermal components. All the GRBs contained significant,
sub-dominate thermal components in their $\gamma$-ray spectra as shown
in \cite{Burgess:2013,preece:2013}. Single pulse GRBs have been shown
to have simple spectral evolution and provide the cleanest signal for
fitting physical models directly to the detector count data
\cite{Burgess:2013,Burgess:2012,Ryde:2009}.

\begin{figure}[t]
  \centering
  \cfig{9}{spectrum.pdf}{4.5}
  \caption{The evolving $\vFv$ spectrum of GRB 130427A. The blackbody component evolves from yellow to red and the synchrotron component evolves from cyan to blue.}
  \label{fig:newSpec}
\end{figure}

To perform time-resolved spectral analysis, each GRB lightcurve was
divided into time bins using a Bayesian blocks algorithm
\cite{Scargle:2013} and each bin's spectrum was fit with a
synchrotron+blackbody photon model using the GRB analysis software
RMFITver4.3 (see \figureref{fig:newSpec}). The synchrotron model consists of a
shock accelerated electron distribution containing a relativistic
Maxwellian and a high-energy power-law tail that is convolved with the
standard synchrotron kernel
\cite{Burgess:2013,Burgess:2012,rybicki:1979}. The spectra were all
well fit by the synchrotron model with fits that were as good if not
better than those made with the Band function.



\section{A Correlation Between Spectral Components}
A strong, positive correlation exists between the peak energies of the
blackbody and synchrotron components (see \figureref{fig:epktcor}). The correlation
was tested with the Spearman's Rank-Order Correlation test, obtaining
a correlation coefficient of $\rho=0.83$ and a p-value of 4.35$\times
10^{-20}$. Tests were carried out to check for a fitting correlation
between the two component peaks and it was found that only a slight,
negative fitting correlation exists.


\begin{figure}[t]
  \centering
  \cfig{9}{correlation}{6}
  \caption{The correlation between $\Ep$ and kT.}
  \label{fig:epktcor}
\end{figure}

\begin{table}[t]

\centering
\begin{tabular}{l || c | c }
 
  &            PL Index  &    F$_{BB}$/F$_{tot}$\\
\hline\hline\\
  GRB 081224A  &	$1.01   \pm	0.14	 $ &      0.29\\
  GRB 090719A   &	$2.33   \pm	0.27	 $ &      0.27\\
  GRB 100707A  &	$1.77	\pm	0.07	 $ &      0.33\\
  GRB 110721A &	$1.24	\pm	0.11	 $ &      0.01\\
  GRB 110920A                    &	$1.97	\pm	0.11	 $ &      0.39\\
  GRB 130427A & 	$1.02	\pm	0.05	 $ &	  0.22\\
  


\end{tabular}
\caption{Correlation fit values and flux ratios for each GRB.}
\label{tab:epktcor}
\end{table}

Of these GRBs, the redshift is known only for GRB 130427A. Since both
values in Figure 2 are energies, shifting to the rest frame would
shift both values of each GRB equally, moving the GRBs along the
correlation but not changing the slope of the correlation. It is
possible that the correlation is tighter in the rest frame but a
greater sample of single pulse GRBs with redshifts is required to
check this assumption. The correlation (\figureref{fig:epktcor}), fit
with a power-law, has a slope $1.26\pm0.03$. We also fit power laws to
the $E_{\rm p}$, kT pairs of the individual GRBs and find dependencies
ranging from $T^{\sim 1}$ to $T^{\sim 2}$ (see
\tableref{tab:epktcor}). The ratio of the blackbody flux to the total
flux was computed for each burst. No correlation between the
correlation index and flux ratio was found.

\section{Interpretation}

A simple model which can interpret these observations is a dissipative
photosphere model \cite{Giannios+07photspec} with general
dynamics. The jet dynamics are parametrized by the dependence of the
Lorentz factor on the radius as $\Gamma\propto R^\mu$ until the jet
reaches its saturation Lorentz factor, $\eta=L/\dot{M}c^2$, where L is
the luminosity and $\dot{M}$ the mass outflow rate. This will be
approximately the jet's Lorentz factor until it reaches the
deceleration radius. For magnetically dominated jets $\mu\approx 1/3$
\cite{Drenkhahn02}, in the baryonic case $\mu\approx1$. The values in
between correspond to a mix of these components \cite{Veres+12fit},
and can be further modified by e.g. the topology of the magnetic
field.

The photosphere will occur where the optical depth of the jet is
unity. This can happen above or below the saturation radius
\cite{Meszaros+93gasdyn}, and accordingly we have two cases.

Closely above the photosphere, instabilities in the flow or magnetic
field line reconnections can lead to mildly relativistic shocks and
accelerate leptons which in turn emit synchrotron radiation
\cite{Meszaros+11gevmag,McKinney+11switch}. To interpret the observed
correlation we restrict the discussion to the dependence of the peak
on the main physical parameters such as the luminosity ($L$), the
coasting Lorentz factor ($\eta$) or the launching radius ($r_0$).

The peak of the synchrotron component is proportional to the Lorentz
factor close to the photosphere, the magnetic field strength and the
random Lorentz factor of the electrons. The magnetic field strength is
usually taken as bearing some fraction $\epsilon_B\sim {\rm const}$ of
the total kinetic energy. The temperature of the photosphere depends
on the luminosity and the photospheric radius.  We derive the
photospheric radius and express other quantities at this radius and
get:


\begin{equation}
E_p \propto \left\{
\begin{array}{ll}
  L^{\frac{3\mu-1}{4\mu+2}} \eta^{-\frac{3\mu-1}{4\mu+2}}
  r_0^{\frac{-5\mu}{4\mu+2}} 		&	{\rm if~photosphere~in~ acceleration~ phase}\\ 
  L^{-1/2} \eta^{3} %r_0^{0}
   	&	{\rm if~photosphere~in~  coasting~ phase} 
\end{array}
\right.
\label{eq:ebreak}
\end{equation}


\begin{equation}
T \propto \left\{
\begin{array}{ll}
  L^{\frac{14\mu-5}{12(2\mu+1)}} \eta^{\frac{2-2\mu}{6\mu+3}}
  r_0^{-\frac{10\mu-1}{6(2\mu+1)}} 	&	{\rm if~photosphere~ in~acceleration~ phase}\\ 
  L^{-5/12} \eta^{8/3} r_0^{1/6}  	&	{\rm if~photosphere~ in~ coasting~ phase} 
\end{array}
\right.
\label{eq:temp}
\end{equation}


It is hard to assess which quantity drives the above dependencies for
either the accelerating or the coasting phase photosphere, or the fact
that it is a single quantity.  One natural assumption is to consider
the luminosity as the main reason for the change in $E_p$ and $T$ as
the flux changes in these bursts.

\begin{itemize}
\item In the accelerating photosphere case, considering the
  appropriate powers of $L$ we get $E_p \propto
  T^{\frac{6(3\mu-1)}{14\mu-5}}$. The exponent is singular at
  $\mu\approx0.36$, but for values up to $\mu < 0.6$ (these are
  the values of $\mu$ for which the photosphere will occur in the
  acceleration phase) we are able to explain exponents from 2 down to
  1.4.

\item For a coasting photosphere $E_p \propto T^{1.2}$. This is
  observed in some bursts. We get similar results if we vary $\eta$.

\end{itemize}

The analysis of the bursts in the framework of this model can
suggestively identify whether the photosphere is in the acceleration
or coasting phase, which in turn can be translated to the composition
of the jet.  We find that for exponents close to 2 the jet dynamics is
dominated by the magnetic field.  Exponents close to 1 are suggestive
of baryonic jets. Our sample spans the values of this model indicating
that the energy content of these GRBs varies from event to event. This
simple method allows for a direct way to assess fundamental properties
of GRBs that have not been previously available through observations.





%%% Local Variables: 
%%% mode: latex
%%% TeX-master: "../thesis"
%%% End: 

\chapter{Discussion} 

\begin{chapterquote}{Bertrand Russell}
  We know very little, and yet it is astonishing that we know so much,
  and still more astonishing that so little knowledge can give us so
  much power.
\end{chapterquote}

\section{Physical Modeling}
It has been shown that using physical photon models to directly fit
GRB data is feasible and provides insights into the physical
mechanisms occurring in GRB outflow jets. Not only do they provide a
direct way to uncover the emission mechanisms that are responsible for
the observed GRB flux, additionally they provide better constraints on
the spectra and flux from the individual components. This leads to
smoother observed pulses in the flux evolution of the components. All
GRBs in the sample used here have their non-thermal emission best fit
with a slow-cooled synchrotron model. This is a significant step
forward in the ability to constrain the spectral models of GRBs
alleviating the degeneracies present in the use empirical photon
models.

\section{Interpretation}
There are three main interpretations that can be made from the work
here concerning the emission mechanisms and structure of the GRB
jet. These are that, at least for single pulse GRBs, the non-thermal
emission is that from electrons that have not fully cooled and piled
up at low energies. The magnetic field flux appears to be frozen into
the outflow, conserving its magnitude as a function of
radius. Additionally, GRBs range from having their internal energy
magnetically dominated to kinetically dominated. These inferences are
drawn from the spectral fits with physical models and the evolution of
the fluxes. Knowing the actual emission mechanism removes the
degeneracy of not knowing the physical form of $\Ep$ which then
enables the determination of actual physical parameters from the
collection of parameters derived from the fits.

With these considerations, the interpretation of the GRB emissions
that can be arrived at is a mixed magnetic and kinetic fireball
model. A fraction of the initial fireball energy is tied up in the
magnetic field of the jet which may serve to accelerate electrons to
high-energy via magnetic reconnection. The important function of this
magnetic energy is that the turbulence in the magnetic field serves to
re-energize the electrons as they radiate synchrotron emission. Some
of the energy must be in kinetic form due to the presence of a
photospheric component in the spectra. Additionally, the individual
indices of the kT-$\Ep$ relationship indicate that the GRBs have a
mixed amount of both energy content.

A model that predicts these features is the ICMART model mentioned
in \chapterref{ch:pap2}. The model relies heavily on the fact that at
least some of the initial explosion energy is carried in the magnetic
field of the jet. The amount of energy in the magnetic field is
quantified by a parameter
\begin{equation}
  \label{eq:sigM}
  \sigma_{\rm M}=\dover{F_P}{F_b}=\dover{B^2}{4 \pi \Gamma \rho c^2}=\dover{B^{\prime2}}{4 \pi \rho^{\prime}c^2}
\end{equation}
which is the ratio of Poynting flux ($F_P$) to baryon flux
($F_b$). The effect of $\sigma_{\rm}$ on the overall view of GRB
emission can be profound. It reduces the required brightness of the
photospheric component since the kinetic energy used to fuel that
component is placed into the magnetic field. The value of $\sigma_{\rm
  M}$ can change as magnetic energy is released during an ICMART
event. If we let $\sigma_{\rm M}^{\rm int}$ be the value before the
event and $\sigma_{\rm M}^{\rm end}$ be the value after the event then
the \equationref{eq:eff} for the efficiency is modified to the form:
\begin{equation}
  \label{eq:si}
  {\rm eff}=\dover{1}{1+\sigma_{\rm M}^{\rm end}}-\dover{\Gamma_{\rm m}(m_{\rm f}+m_{\rm s})}{(\Gamma_{\rm f}m_{\rm f}+\Gamma_{\rm s}m_{\rm s})(1+\sigma_{\rm M}^{\rm int})}.
\end{equation}
If $\sigma_{\rm M}^{\rm int}\ll 1$ then the efficiency can reach 90\%
or more. As mentioned in \chapterref{ch:pap2}, the resulting emission
from an ICMART event should be a two-component model with the
non-thermal emission in the form of synchrotron. Therefore, this model
is supported by the observations in this work.


\section{Conclusion}
In this dissertation I have attempted to show that the physical
modeling of GRB spectra is a significant and viable method for the
study of GRB prompt emission spectra. The physical parameters derived
from spectral fits using the slow-cooled synchrotron model have
allowed for an examination of GRB jet properties including the spatial
and temporal evolution of the outflow, the topology and magnitude of
the jet magnetic field, and the specific emission mechanisms by which
the electrons in the jet radiate. These insights are in many respects
deeper than those that have been derived from analysis performed with
the Band function because the physical shape of the spectrum
(slow-cooled synchrotron) is assumed a priori. The ambiguity of
interpreting Band function parameters to assess the emission
mechanisms of electrons in the outflow severely limits the extension of
spectral analysis to physical parameters because the phase space of
possibilities is large and degenerate with an empirical fitting
function. Not only are these ambiguities eliminated, but I have shown
that the use of physical photon models produces better constraints
when fitting multi-component models \figureref{fig:fluxComp}.

In particular, the 'line-of-death' problem can be resolved when using
physical models for two reasons. First, the physical interpretation of
the Band function $\alpha$ is most likely inaccurate as shown
in \sectionref{sec:results:bvs}. Only the fitting of a physical photon
model can correctly determine the physical origin of a photon
spectrum. Second, the confirmation of a photospheric blackbody
component below the non-thermal $\Ep$ softens the low-energy spectrum
and allows for the synchrotron photon model to be fit to the data. The
second 'line-of-death' problem corresponding to the fast-cooling
synchrotron is not solved but I have shown that the this model cannot
fit the data. This forces the conclusion that the electrons in the
outflow are being re-energized through some process that occurs on a
time scale comparable to $t_{\rm dyn}$. Such scenarios are a
slow-heating by magnetic turbulence via a second-order Fermi
process. While these scenarios have been theoretically discussed, the
observation made here require them for my results to be physically
viable.

The inference of internal GRB properties (e.g. $\Gamma$, r$_{\rm ph}$,
r$_0$) from the observations combined with the physical implications
of the $L-\Ep$ and $\Ep-kT$ correlations provide tools that will
ultimately lead to a deeper understanding of the evolution of the GRB
jet and its magnetic field. A larger sample of GRBs is required to
fully take advantage of these new methods. In addition, the study of
multi-episodic GRBs, i.e., those with complex lightcurves, to assess
whether or not they obey the correlations is required. Though more
complex physical models may be needed to fully explain their spectra.




%%% Local Variables: 
%%% mode: latex
%%% TeX-master: "../thesis"
%%% End: 




%\chapter{Conclusions}
\label{ch:thisChapterLabel}

\section{What I Learned}

I still need to be written.

\section{No Figures}

Remember---the UAH format does not allow for figures in the
``Conclusions'' chapter.


%   This set of LaTeX files uses the Appendix Package to handle
%   the appendices - it creates the "Appendices" page and puts
%   "Appendix A:" in the TOC.  So, strictly speaking, the
%   \appendix command is not necessary (because the appendix
%   package uses an "appendices environment").  But, for
%   the typedref package to operate correctly, you must include the
%   \appendix command - so, it's included after the beginning of
%   the appendices environment.  If you only have one appendix,
%   then you need to remove the "page" option from the uahdis.sty
%   file ('page' option occurs when loading appendix package).
%   You can add or subtract appendices - they are treated just like
%   chapters - the only difference is that they occur within the
%   appendices environment, which tells LaTeX to give them letters
%   instead of numbers.

\begin{appendices}
\appendix

%   Include my appendices
\chapter{Degeneracies in Physical Models}
\label{ch:degen}
The number of free parameters in the physical models implemented in
{\tt RMFIT} exceeds the number of parameters than can be constrained
by the fitting engine. This is due to two factors: spectral resolution
and degeneracies in the fit parameters. The spectral resolution of the
instrument is a fixed quantity and can not be altered; however, the
degeneracies in the fit parameters can be dealt with be reformulating
their numerical expressions. Here, degeneracies in the slow-cooled
synchrotron model will be dealt with.


The synchrotron spectrum has three characteristics that determine its
shape including the high-energy electron index, $\vFv$ peak position,
and the overall amplitude. However, their are six free parameters in
our formulation of the model: $n_0$, $\gammaMin$, $\gamth$, $B$,
$\delta$, $\epsilon$, and $\Gamma$. The main degeneracy exists in the determination of the of $\Ep$:
\begin{equation}
  \label{eq:epdeg}
  \Ep \propto \Gamma B \gammaMin^2.
\end{equation}
It is not possible to simply fit the value of $\Ep$ as is done with
the Band function because of the integration over the electron
distribution. The numerical integration is done in two steps for each
part of the electron distribution:
\begin{eqnarray}
  \label{eq:n}
  \Fv^{\rm thermal}=n_0\int_1^{\gammaMin} d\gamma\; n_e^{\rm thermal}(\gamma)\mathcal{F}\left(\dover{\mathcal{E}}{E_c(\gamma)} \right)\\
\Fv^{\rm power-law}=n_0\int_{\gammaMin}^{\infty} d\gamma\; n_e^{\rm power-law}(\gamma)\mathcal{F}\left(\dover{\mathcal{E}}{E_c(\gamma)} \right)\\
\Fv = \Fv^{\rm thermal} + \Fv^{\rm power-law}.
\end{eqnarray}
Therefore, the value of $\gammaMin$ must be specified even though
there is no way to determine its actual value. For this reason, $\Ep$
is broken into two parts:
\begin{equation}
  \label{eq:estar}
  \Ep = E_*\gammaMin^2
\end{equation}
and the value of $E_*$ is used to scale the energy of the $\vFv$ peak
during fitting. It is not possible to leave both $E_*$ and $\gammaMin$
free during the fit because both parameters scale $\Ep$ but do not
alter the shape of the spectrum independently (see \figureref{fig:scalep}). 
\begin{figure}[t]
  \centering
  \subfigure{
    \cfig{11}{gMin.pdf}{3.2}\cfig{11}{eCrit.pdf}{3.2}}

  
\caption{The slow-cooled synchrotron spectrum as a function of
  $\gammaMin$ (\emph{left}) and varying E$_*$ (\emph{right}). While
  the amplitude and $\vFv$ peak of the spectrum are altered, the
  overall shape remains the same.}
  \label{fig:scalep}
\end{figure}
In this work, $\gammaMin$ was chosen to be fixed and the value of
$E_*$ left free. The values of $E_*$ and $\gammaMin$ are not
independently physical for this reason, and only the value of $\Ep$
can be used to make inferences about physical parameters. Estimations
of $\gammaMin$ can be made from very general considerations about
shock acceleration theory. The total amount of energy dissipated by shocks into the electrons is
\begin{equation}
  \label{eq:gammamin}
  \int_{\gammaMin}^{\infty}d\gamma\; n_e(\gamma)(\gamma-1) = \xi \Gamma \epsilon_e \dover{m_p}{m_e}
\end{equation}
where $\xi$ is an unknown parameter characterizing the efficiency of
mildly-relativistic shocks. Solving this for $\gammaMin$ yields
\begin{equation}
  \label{eq:gminth}
  \gammaMin \approx \dover{\delta - 2}{\delta -1}\epsilon_c \dover{m_p}{m_e}.
\end{equation}
The maximum value of $\gammaMin\approx (m_p/m_e)\approx
1,800$. Therefore, we set $\gammaMin=900$ in the fits.

The value of $\gamth$ has to be fixed in the fits as well. Simulations
of relativistic shocks have shown that the ratio
$\gammaMin/\gamth\approx 3$ and therefore $\gamth=300$ is the value
chosen for the fits. In \sectionref{sec:epsdisc} the choosing of the highly
unconstrained value of $\epsilon$ is discussed. With these values set,
a tractable parametrization of the slow-cooled synchrotron model is
available.




%%% Local Variables: 
%%% mode: latex
%%% TeX-master: "../thesis"
%%% End: 

\chapter{GRB Fit Parameters}
\label{ch:fitparams}


\section{Synchrotron Fits}
\begin{table}[h]
\centering
\scriptsize
\label{tab:}
\begin{tabular}{c| c c c c}
Time [s] & $\Fv$ [erg s$^{-1}$ cm$^{-2}$] & Energy Fluence [erg cm$^{-2}$] & $\Ep$ [keV] & $\delta$ \\
\hline \hline\\ 

-0.85-0.25 & 4.89E-06$\pm$1.45E-06 & 5.38E-06$\pm$1.60E-06 & 1139.40$\pm$543.49 & 3.00$\pm$0.78 \\ 

0.25-0.56 & 1.23E-05$\pm$2.34E-06 & 3.80E-06$\pm$7.25E-07 & 866.30$\pm$189.58 & 4.46$\pm$1.20 \\ 

0.56-1.20 & 4.32E-06$\pm$4.84E-07 & 2.76E-06$\pm$3.09E-07 & 792.51$\pm$763.74 & 98.23$\pm$0.00 \\ 

1.20-1.52 & 1.98E-06$\pm$2.87E-07 & 6.32E-07$\pm$9.19E-08 & 488.98$\pm$95.64 & 7.00$\pm$0.00 \\ 

1.52-3.76 & 6.88E-07$\pm$2.23E-07 & 1.54E-06$\pm$4.99E-07 & 186.01$\pm$75.59 & 5.13$\pm$3.11 \\ 

1.52-3.76 & 6.88E-07$\pm$2.23E-07 & 1.54E-06$\pm$4.99E-07 & 186.13$\pm$75.68 & 5.13$\pm$3.12 \\ 

\end{tabular}
\caption{The slow-cooled synchrotron model fit parameters of GRB 081110A.}
\end{table}


\begin{table}[h]
\centering
\scriptsize
\label{tab:}
\begin{tabular}{c| c c c c}
Time [s] & $\Fv$ [erg s$^{-1}$ cm$^{-2}$] & Energy Fluence [erg cm$^{-2}$] & $\Ep$ [keV] & $\delta$ \\
\hline \hline\\ 

-0.16-0.28 & 5.47E-06$\pm$1.67E-06 & 2.41E-06$\pm$7.36E-07 & 2293.82$\pm$870.32 & 6.60$\pm$1.53 \\ 

0.28-0.90 & 5.63E-06$\pm$1.61E-06 & 3.49E-06$\pm$1.00E-06 & 1322.09$\pm$482.10 & 5.57$\pm$0.88 \\ 

0.90-1.91 & 5.95E-06$\pm$1.04E-06 & 6.01E-06$\pm$1.05E-06 & 1190.67$\pm$261.24 & 6.73$\pm$1.29 \\ 

1.91-4.13 & 5.69E-06$\pm$4.29E-07 & 1.26E-05$\pm$9.53E-07 & 592.18$\pm$43.86 & 6.00$\pm$0.00 \\ 

4.13-6.46 & 2.57E-06$\pm$2.06E-07 & 5.98E-06$\pm$4.80E-07 & 318.77$\pm$27.50 & 6.00$\pm$0.00 \\ 

6.46-7.46 & 1.65E-06$\pm$2.23E-07 & 1.65E-06$\pm$2.23E-07 & 261.01$\pm$39.77 & 6.00$\pm$0.00 \\ 

7.46-10.77 & 7.40E-07$\pm$9.41E-08 & 2.45E-06$\pm$3.11E-07 & 159.20$\pm$22.47 & 6.00$\pm$0.00 \\ 

10.77-12.50 & 4.50E-07$\pm$7.74E-08 & 7.79E-07$\pm$1.34E-07 & 152.79$\pm$33.06 & 7.00$\pm$0.00 \\ 

12.50-18.20 & 2.35E-07$\pm$2.04E-08 & 1.34E-06$\pm$1.16E-07 & 141.32$\pm$29.89 & 7.00$\pm$0.00 \\ 

\end{tabular}
\caption{The slow-cooled synchrotron model fit parameters of GRB 081224A.}
\end{table}

\begin{table}[h]
\centering
\scriptsize
\label{tab:}
\begin{tabular}{c| c c c}
Time [s] & $\Fv$ [erg s$^{-1}$ cm$^{-2}$] & Energy Fluence [erg cm$^{-2}$] & $kT$ [keV] \\
\hline \hline\\ 

-0.16-0.28 & 1.14E-06$\pm$5.45E-07 & 5.02E-07$\pm$2.40E-07 & 163.54$\pm$46.48 \\ 

0.28-0.90 & 4.10E-06$\pm$6.15E-07 & 2.54E-06$\pm$3.81E-07 & 146.15$\pm$10.80 \\ 

0.90-1.91 & 3.67E-06$\pm$3.59E-07 & 3.70E-06$\pm$3.63E-07 & 106.04$\pm$5.45 \\ 

1.91-4.13 & 1.89E-06$\pm$1.53E-07 & 4.19E-06$\pm$3.41E-07 & 69.04$\pm$2.98 \\ 

4.13-6.46 & 3.82E-07$\pm$7.99E-08 & 8.91E-07$\pm$1.86E-07 & 42.62$\pm$4.95 \\ 

6.46-7.46 & 9.30E-08$\pm$8.48E-08 & 9.30E-08$\pm$8.48E-08 & 35.21$\pm$18.96 \\ 

7.46-10.77 & 8.92E-08$\pm$4.51E-08 & 2.95E-07$\pm$1.49E-07 & 31.74$\pm$8.32 \\ 

10.77-12.50 & 1.00E-08$\pm$3.54E-08 & 1.74E-08$\pm$6.12E-08 & 21.71$\pm$44.30 \\ 

12.50-18.20 & 1.01E-08$\pm$7.96E-09 & 5.73E-08$\pm$4.54E-08 & 5.25$\pm$1.87 \\ 

\end{tabular}
\caption{The blackbody model fit parameters of GRB 081224A.}
\end{table}

\begin{table}[h]
\centering
\scriptsize
\label{tab:}
\begin{tabular}{c| c c c c}
Time [s] & $\Fv$ [erg s$^{-1}$ cm$^{-2}$] & Energy Fluence [erg cm$^{-2}$] & $\Ep$ [keV] & $\delta$ \\
\hline \hline\\ 

-0.10-0.73 & 3.19E-06$\pm$1.78E-15 & 2.65E-06$\pm$1.47E-15 & 1287.65$\pm$1581.22 & 11.08$\pm$64.65 \\ 

0.73-3.80 & 3.21E-06$\pm$3.31E-16 & 9.86E-06$\pm$1.02E-15 & 515.56$\pm$37.96 & 7.00$\pm$0.00 \\ 

3.80-4.44 & 6.74E-06$\pm$1.16E-15 & 4.32E-06$\pm$7.44E-16 & 663.12$\pm$72.89 & 7.00$\pm$0.00 \\ 

4.44-5.64 & 7.75E-06$\pm$7.06E-16 & 9.30E-06$\pm$8.47E-16 & 560.09$\pm$34.74 & 7.00$\pm$0.00 \\ 

5.64-6.78 & 5.33E-06$\pm$4.94E-16 & 6.08E-06$\pm$5.64E-16 & 354.14$\pm$22.82 & 7.00$\pm$0.00 \\ 

6.78-7.46 & 2.96E-06$\pm$5.34E-16 & 2.01E-06$\pm$3.63E-16 & 217.55$\pm$26.32 & 7.00$\pm$0.00 \\ 

7.46-8.07 & 2.08E-06$\pm$2.85E-16 & 1.27E-06$\pm$1.74E-16 & 204.54$\pm$24.39 & 7.00$\pm$0.00 \\ 

8.07-10.02 & 1.03E-06$\pm$1.92E-16 & 2.01E-06$\pm$3.74E-16 & 89.57$\pm$11.85 & 7.00$\pm$0.00 \\ 

10.02-12.48 & 8.28E-07$\pm$2.05E-16 & 2.04E-06$\pm$5.05E-16 & 82.20$\pm$43.26 & 4.09$\pm$1.01 \\ 

12.48-13.88 & 8.53E-07$\pm$6.92E-16 & 1.19E-06$\pm$9.69E-16 & 45.97$\pm$43.57 & 3.09$\pm$0.70 \\ 

13.88-16.20 & 2.18E-07$\pm$8.02E-17 & 5.06E-07$\pm$1.86E-16 & 83.12$\pm$58.29 & 5.35$\pm$3.37 \\ 

16.20-29.99 & 3.71E-08$\pm$3.75E-17 & 5.12E-07$\pm$5.18E-16 & 22.52$\pm$40.81 & 4.19$\pm$2.21 \\ 

\end{tabular}
\caption{The slow-cooled synchrotron model fit parameters of GRB 090719A.}
\end{table}

\begin{table}[h]
\centering
\scriptsize
\label{tab:}
\begin{tabular}{c| c c c}
Time [s] & $\Fv$ [erg s$^{-1}$ cm$^{-2}$] & Energy Fluence [erg cm$^{-2}$] & $kT$ [keV] \\
\hline \hline\\ 

-0.10-0.73 & 2.31E-06$\pm$6.71E-16 & 1.92E-06$\pm$5.57E-16 & 104.29$\pm$8.27 \\ 

0.73-3.80 & 1.64E-06$\pm$1.12E-16 & 5.03E-06$\pm$3.44E-16 & 44.39$\pm$1.10 \\ 

3.80-4.44 & 1.95E-06$\pm$3.92E-16 & 1.25E-06$\pm$2.51E-16 & 62.28$\pm$4.49 \\ 

4.44-5.64 & 2.61E-06$\pm$2.49E-16 & 3.14E-06$\pm$2.98E-16 & 50.30$\pm$1.74 \\ 

5.64-6.78 & 1.72E-06$\pm$2.10E-16 & 1.96E-06$\pm$2.39E-16 & 41.09$\pm$1.81 \\ 

6.78-7.46 & 1.38E-06$\pm$2.89E-16 & 9.41E-07$\pm$1.97E-16 & 39.76$\pm$2.63 \\ 

7.46-8.07 & 4.27E-07$\pm$1.42E-16 & 2.61E-07$\pm$8.68E-17 & 23.65$\pm$2.75 \\ 

8.07-10.02 & 2.75E-07$\pm$1.34E-16 & 5.35E-07$\pm$2.60E-16 & 28.52$\pm$3.27 \\ 

10.02-12.48 & 1.09E-07$\pm$1.40E-16 & 2.68E-07$\pm$3.44E-16 & 18.50$\pm$3.14 \\ 

12.48-13.88 & 3.74E-08$\pm$9.18E-17 & 5.24E-08$\pm$1.29E-16 & 11.97$\pm$5.90 \\ 

13.88-16.20 & 2.03E-09$\pm$4.28E-17 & 4.72E-09$\pm$9.93E-17 & 8.37$\pm$53.40 \\ 

16.20-29.99 & 1.41E-08$\pm$2.11E-17 & 1.95E-07$\pm$2.90E-16 & 13.06$\pm$5.73 \\ 

\end{tabular}
\caption{The blackbody model fit parameters of GRB 090719A.}
\end{table}

\begin{table}[h]
\centering
\scriptsize
\label{tab:}
\begin{tabular}{c| c c c c}
Time [s] & $\Fv$ [erg s$^{-1}$ cm$^{-2}$] & Energy Fluence [erg cm$^{-2}$] & $\Ep$ [keV] & $\delta$ \\
\hline \hline\\ 

-1.89-1.42 & 1.28E-06$\pm$4.83E-07 & 4.22E-06$\pm$1.60E-06 & 1199.98$\pm$490.27 & 5.76$\pm$0.00 \\ 

1.42-2.11 & 8.42E-06$\pm$1.26E-06 & 5.81E-06$\pm$8.66E-07 & 865.49$\pm$195.33 & 3.56$\pm$0.00 \\ 

2.11-4.51 & 8.29E-06$\pm$3.82E-07 & 1.99E-05$\pm$9.16E-07 & 362.86$\pm$25.01 & 3.50$\pm$0.00 \\ 

4.51-5.60 & 2.49E-06$\pm$1.14E-07 & 2.72E-06$\pm$1.24E-07 & 262.54$\pm$24.67 & 10.00$\pm$0.00 \\ 

5.60-8.01 & 1.73E-06$\pm$1.57E-07 & 4.17E-06$\pm$3.78E-07 & 97.48$\pm$10.73 & 3.98$\pm$0.00 \\ 

8.01-9.95 & 6.74E-07$\pm$2.82E-08 & 1.31E-06$\pm$5.46E-08 & 64.72$\pm$4.60 & 4.89$\pm$0.00 \\ 

9.95-12.75 & 2.71E-07$\pm$1.82E-08 & 7.58E-07$\pm$5.10E-08 & 50.65$\pm$6.11 & 4.76$\pm$0.00 \\ 

\end{tabular}
\caption{The slow-cooled synchrotron model fit parameters of GRB 090809B.}
\end{table}

\begin{table}[h]
\centering
\scriptsize
\label{tab:}
\begin{tabular}{c| c c c}
Time [s] & $\Fv$ [erg s$^{-1}$ cm$^{-2}$] & Energy Fluence [erg cm$^{-2}$] & $kT$ [keV] \\
\hline \hline\\ 

-1.89-1.42 & 1.18E-07$\pm$5.85E-08 & 3.90E-07$\pm$1.93E-07 & 53.58$\pm$15.24 \\ 

1.42-2.11 & 4.93E-07$\pm$1.03E-07 & 3.40E-07$\pm$7.12E-08 & 36.38$\pm$4.10 \\ 

2.11-4.51 & 5.10E-07$\pm$5.33E-08 & 1.22E-06$\pm$1.28E-07 & 31.12$\pm$1.95 \\ 

4.51-5.60 & 7.79E-08$\pm$5.54E-08 & 8.50E-08$\pm$6.04E-08 & 14.75$\pm$3.71 \\ 

5.60-8.01 & 1.44E-07$\pm$5.54E-08 & 3.47E-07$\pm$1.34E-07 & 27.97$\pm$5.16 \\ 

8.01-9.95 & 0.00E+00$\pm$0.00E+00 & 0.00E+00$\pm$0.00E+00 & 0.00$\pm$0.00 \\ 

9.95-12.75 & 0.00E+00$\pm$0.00E+00 & 0.00E+00$\pm$0.00E+00 & 0.00$\pm$0.00 \\ 

\end{tabular}
\caption{The blackbody model fit parameters of GRB 090809B.}
\end{table}

\begin{table}[h]
\centering
\scriptsize
\label{tab:}
\begin{tabular}{c| c c c c}
Time [s] & $\Fv$ [erg s$^{-1}$ cm$^{-2}$] & Energy Fluence [erg cm$^{-2}$] & $\Ep$ [keV] & $\delta$ \\
\hline \hline\\ 

-0.20-0.20 & 8.81E-06$\pm$5.57E-15 & 3.52E-06$\pm$2.23E-15 & 4340.09$\pm$1703.19 & 5.00$\pm$0.00 \\ 

0.20-0.80 & 1.70E-05$\pm$4.63E-15 & 1.02E-05$\pm$2.78E-15 & 3760.07$\pm$689.59 & 5.00$\pm$0.00 \\ 

0.80-2.40 & 1.93E-05$\pm$2.60E-15 & 3.09E-05$\pm$4.16E-15 & 2088.29$\pm$192.14 & 5.00$\pm$0.00 \\ 

2.40-3.00 & 1.34E-05$\pm$2.88E-15 & 8.01E-06$\pm$1.73E-15 & 1267.70$\pm$202.18 & 5.00$\pm$0.00 \\ 

3.00-4.30 & 6.46E-06$\pm$9.18E-16 & 8.40E-06$\pm$1.19E-15 & 557.09$\pm$62.26 & 5.00$\pm$0.00 \\ 

4.30-5.70 & 4.08E-06$\pm$4.84E-16 & 5.71E-06$\pm$6.78E-16 & 314.86$\pm$32.08 & 5.00$\pm$0.00 \\ 

5.70-7.20 & 2.28E-06$\pm$3.84E-16 & 3.42E-06$\pm$5.77E-16 & 213.46$\pm$26.74 & 5.00$\pm$0.00 \\ 

7.20-12.80 & 1.55E-06$\pm$1.49E-16 & 8.65E-06$\pm$8.35E-16 & 161.44$\pm$11.94 & 5.00$\pm$0.00 \\ 

12.80-22.20 & 6.38E-07$\pm$3.43E-17 & 5.99E-06$\pm$3.22E-16 & 74.27$\pm$6.32 & 5.00$\pm$0.00 \\ 

\end{tabular}
\caption{The slow-cooled synchrotron model fit parameters of GRB 100707A.}
\end{table}

\begin{table}[h]
\centering
\scriptsize
\label{tab:}
\begin{tabular}{c| c c c}
Time [s] & $\Fv$ [erg s$^{-1}$ cm$^{-2}$] & Energy Fluence [erg cm$^{-2}$] & $kT$ [keV] \\
\hline \hline\\ 

-0.20-0.20 & 5.18E-06$\pm$2.08E-15 & 2.07E-06$\pm$8.33E-16 & 267.41$\pm$48.88 \\ 

0.20-0.80 & 9.69E-06$\pm$1.02E-15 & 5.82E-06$\pm$6.10E-16 & 129.61$\pm$4.89 \\ 

0.80-2.40 & 1.06E-05$\pm$4.78E-16 & 1.70E-05$\pm$7.65E-16 & 83.72$\pm$1.36 \\ 

2.40-3.00 & 5.23E-06$\pm$4.65E-16 & 3.14E-06$\pm$2.79E-16 & 54.38$\pm$1.72 \\ 

3.00-4.30 & 3.07E-06$\pm$2.06E-16 & 3.99E-06$\pm$2.68E-16 & 37.83$\pm$0.94 \\ 

4.30-5.70 & 1.48E-06$\pm$1.40E-16 & 2.08E-06$\pm$1.97E-16 & 28.28$\pm$0.96 \\ 

5.70-7.20 & 8.44E-07$\pm$1.32E-16 & 1.27E-06$\pm$1.97E-16 & 30.28$\pm$1.75 \\ 

7.20-12.80 & 4.72E-07$\pm$5.48E-17 & 2.64E-06$\pm$3.07E-16 & 25.54$\pm$1.07 \\ 

12.80-22.20 & 5.90E-08$\pm$2.59E-17 & 5.55E-07$\pm$2.43E-16 & 9.52$\pm$0.86 \\ 

\end{tabular}
\caption{The blackbody model fit parameters of GRB 100707A.}
\end{table}

\begin{table}[h]
\centering
\scriptsize
\label{tab:}
\begin{tabular}{c| c c c c}
Time [s] & $\Fv$ [erg s$^{-1}$ cm$^{-2}$] & Energy Fluence [erg cm$^{-2}$] & $\Ep$ [keV] & $\delta$ \\
\hline \hline\\ 

-5.38-2.82 & 1.43E-06$\pm$9.38E-08 & 1.17E-05$\pm$7.68E-07 & 841.37$\pm$68.38 & 11.12$\pm$0.00 \\ 

2.82-3.84 & 7.49E-06$\pm$9.07E-07 & 7.67E-06$\pm$9.29E-07 & 481.49$\pm$64.23 & 3.76$\pm$0.00 \\ 

3.84-4.86 & 5.12E-06$\pm$1.09E-06 & 5.24E-06$\pm$1.12E-06 & 320.26$\pm$71.19 & 3.81$\pm$0.00 \\ 

4.86-6.91 & 2.81E-06$\pm$6.52E-07 & 5.76E-06$\pm$1.33E-06 & 325.97$\pm$77.43 & 4.91$\pm$0.00 \\ 

6.91-9.98 & 1.59E-06$\pm$3.58E-07 & 4.90E-06$\pm$1.10E-06 & 145.92$\pm$38.28 & 3.72$\pm$0.00 \\ 

9.98-15.10 & 7.32E-08$\pm$1.56E-08 & 3.75E-07$\pm$8.00E-08 & 37.63$\pm$12.65 & 19.00$\pm$0.00 \\ 

\end{tabular}
\caption{The slow-cooled synchrotron model fit parameters of GRB 110407A.}
\end{table}

\begin{table}[h]
\centering
\scriptsize
\label{tab:}
\begin{tabular}{c| c c c}
Time [s] & $\Fv$ [erg s$^{-1}$ cm$^{-2}$] & Energy Fluence [erg cm$^{-2}$] & $kT$ [keV] \\
\hline \hline\\ 

-5.38-2.82 & 0.00E+00$\pm$0.00E+00 & 0.00E+00$\pm$0.00E+00 & 0.00$\pm$0.00 \\ 

2.82-3.84 & 5.12E-07$\pm$1.92E-07 & 5.25E-07$\pm$1.97E-07 & 61.73$\pm$12.54 \\ 

3.84-4.86 & 1.27E-06$\pm$4.04E-07 & 1.30E-06$\pm$4.14E-07 & 90.73$\pm$10.36 \\ 

4.86-6.91 & 7.65E-07$\pm$4.00E-07 & 1.57E-06$\pm$8.19E-07 & 140.38$\pm$24.29 \\ 

6.91-9.98 & 2.00E-07$\pm$1.61E-07 & 6.14E-07$\pm$4.95E-07 & 81.40$\pm$19.37 \\ 

9.98-15.10 & 1.17E-07$\pm$1.82E-08 & 5.99E-07$\pm$9.33E-08 & 47.21$\pm$8.77 \\ 

\end{tabular}
\caption{The blackbody model fit parameters of GRB 110407A.}
\end{table}

\begin{table}[h]
\centering
\scriptsize
\label{tab:}
\begin{tabular}{c| c c c c}
Time [s] & $\Fv$ [erg s$^{-1}$ cm$^{-2}$] & Energy Fluence [erg cm$^{-2}$] & $\Ep$ [keV] & $\delta$ \\
\hline \hline\\ 

-0.07-0.08 & 4.66E-05$\pm$3.79E-06 & 7.00E-06$\pm$5.69E-07 & 9039.36$\pm$1429.45 & 5.33$\pm$0.65 \\ 

0.08-0.48 & 4.76E-05$\pm$2.33E-06 & 1.90E-05$\pm$9.34E-07 & 3776.38$\pm$313.91 & 4.78$\pm$0.20 \\ 

0.48-1.28 & 3.10E-05$\pm$1.22E-06 & 2.48E-05$\pm$9.75E-07 & 2068.85$\pm$156.03 & 5.18$\pm$0.19 \\ 

1.28-2.78 & 1.18E-05$\pm$4.00E-07 & 1.76E-05$\pm$6.01E-07 & 665.14$\pm$40.48 & 4.68$\pm$0.17 \\ 

2.78-3.78 & 6.07E-06$\pm$3.16E-07 & 6.07E-06$\pm$3.16E-07 & 263.39$\pm$26.40 & 4.14$\pm$0.20 \\ 

3.78-5.88 & 3.67E-06$\pm$1.78E-07 & 7.71E-06$\pm$3.74E-07 & 287.30$\pm$27.22 & 4.08$\pm$0.16 \\ 

5.88-7.63 & 1.82E-06$\pm$2.34E-07 & 3.19E-06$\pm$4.10E-07 & 249.43$\pm$43.79 & 4.59$\pm$0.77 \\ 

7.63-12.63 & 1.04E-06$\pm$9.58E-08 & 5.20E-06$\pm$4.79E-07 & 211.56$\pm$28.16 & 4.27$\pm$0.39 \\ 

\end{tabular}
\caption{The slow-cooled synchrotron model fit parameters of GRB 110721A.}
\end{table}

\begin{table}[h]
\centering
\scriptsize
\label{tab:}
\begin{tabular}{c| c c c}
Time [s] & $\Fv$ [erg s$^{-1}$ cm$^{-2}$] & Energy Fluence [erg cm$^{-2}$] & $kT$ [keV] \\
\hline \hline\\ 

-0.07-0.08 & 0.00E+00$\pm$0.00E+00 & 0.00E+00$\pm$0.00E+00 & 0.00$\pm$0.00 \\ 

0.08-0.48 & 0.00E+00$\pm$0.00E+00 & 0.00E+00$\pm$0.00E+00 & 0.00$\pm$0.00 \\ 

0.48-1.28 & 2.87E-07$\pm$8.23E-08 & 2.30E-07$\pm$6.59E-08 & 29.36$\pm$3.68 \\ 

1.28-2.78 & 4.67E-07$\pm$5.91E-08 & 7.01E-07$\pm$8.87E-08 & 27.65$\pm$1.63 \\ 

2.78-3.78 & 1.63E-07$\pm$4.71E-08 & 1.63E-07$\pm$4.71E-08 & 12.46$\pm$1.19 \\ 

3.78-5.88 & 1.07E-07$\pm$1.73E-08 & 2.24E-07$\pm$3.63E-08 & 8.15$\pm$0.57 \\ 

5.88-7.63 & 2.05E-08$\pm$9.90E-09 & 3.59E-08$\pm$1.73E-08 & 4.90$\pm$1.30 \\ 

7.63-12.63 & 3.83E-09$\pm$5.43E-09 & 1.91E-08$\pm$2.72E-08 & 4.95$\pm$3.80 \\ 

\end{tabular}
\caption{The blackbody model fit parameters of GRB 110721A.}
\end{table}

\begin{table}[h]
\centering
\scriptsize
\label{tab:}
\begin{tabular}{c| c c c c}
Time [s] & $\Fv$ [erg s$^{-1}$ cm$^{-2}$] & Energy Fluence [erg cm$^{-2}$] & $\Ep$ [keV] & $\delta$ \\
\hline \hline\\ 

-1.60-1.10 & 1.64E-06$\pm$8.70E-07 & 4.43E-06$\pm$2.35E-06 & 2211.89$\pm$1031.15 & 10.00$\pm$0.00 \\ 

1.10-4.90 & 2.10E-06$\pm$5.60E-07 & 7.99E-06$\pm$2.13E-06 & 1130.78$\pm$273.82 & 10.00$\pm$0.00 \\ 

4.90-6.90 & 2.55E-06$\pm$6.47E-07 & 5.10E-06$\pm$1.29E-06 & 934.77$\pm$217.99 & 10.00$\pm$0.00 \\ 

6.90-16.40 & 3.73E-06$\pm$2.69E-07 & 3.54E-05$\pm$2.55E-06 & 1245.76$\pm$86.21 & 15.00$\pm$0.00 \\ 

16.40-20.20 & 2.23E-06$\pm$3.17E-07 & 8.49E-06$\pm$1.20E-06 & 833.72$\pm$112.85 & 15.00$\pm$0.00 \\ 

20.20-28.70 & 2.02E-06$\pm$1.80E-07 & 1.72E-05$\pm$1.53E-06 & 774.00$\pm$67.48 & 15.00$\pm$0.00 \\ 

28.70-37.70 & 8.98E-07$\pm$1.29E-07 & 8.08E-06$\pm$1.16E-06 & 392.49$\pm$55.51 & 15.00$\pm$0.00 \\ 

37.70-47.60 & 7.04E-07$\pm$9.27E-08 & 6.97E-06$\pm$9.17E-07 & 355.53$\pm$47.45 & 15.00$\pm$0.00 \\ 

47.60-58.50 & 4.04E-07$\pm$8.15E-08 & 4.40E-06$\pm$8.89E-07 & 197.91$\pm$41.47 & 15.00$\pm$0.00 \\ 

58.50-83.40 & 2.44E-07$\pm$3.44E-08 & 6.07E-06$\pm$8.56E-07 & 154.46$\pm$23.27 & 15.00$\pm$0.00 \\ 

83.40-105.50 & 1.06E-07$\pm$3.31E-08 & 2.34E-06$\pm$7.32E-07 & 76.64$\pm$27.05 & 15.00$\pm$0.00 \\ 

105.50-122.20 & 9.25E-08$\pm$2.73E-08 & 1.55E-06$\pm$4.56E-07 & 76.36$\pm$25.74 & 15.00$\pm$0.00 \\ 

122.20-161.00 & 5.37E-08$\pm$1.00E-08 & 2.08E-06$\pm$3.89E-07 & 95.44$\pm$22.90 & 15.00$\pm$0.00 \\ 

161.00-182.60 & 4.32E-08$\pm$1.32E-08 & 9.33E-07$\pm$2.85E-07 & 61.59$\pm$23.03 & 15.00$\pm$0.00 \\ 

182.60-236.70 & 1.64E-08$\pm$1.39E-08 & 8.90E-07$\pm$7.50E-07 & 25.95$\pm$26.09 & 15.00$\pm$0.00 \\ 

\end{tabular}
\caption{The slow-cooled synchrotron model fit parameters of GRB 110920A.}
\end{table}

\begin{table}[h]
\centering
\scriptsize
\label{tab:}
\begin{tabular}{c| c c c}
Time [s] & $\Fv$ [erg s$^{-1}$ cm$^{-2}$] & Energy Fluence [erg cm$^{-2}$] & $kT$ [keV] \\
\hline \hline\\ 

-1.60-1.10 & 2.29E-07$\pm$2.18E-07 & 6.17E-07$\pm$5.88E-07 & 144.99$\pm$72.24 \\ 

1.10-4.90 & 1.18E-06$\pm$1.92E-07 & 4.49E-06$\pm$7.31E-07 & 109.47$\pm$8.11 \\ 

4.90-6.90 & 1.94E-06$\pm$2.39E-07 & 3.87E-06$\pm$4.79E-07 & 95.92$\pm$5.53 \\ 

6.90-16.40 & 1.82E-06$\pm$7.21E-08 & 1.73E-05$\pm$6.85E-07 & 77.29$\pm$1.67 \\ 

16.40-20.20 & 1.60E-06$\pm$1.10E-07 & 6.07E-06$\pm$4.17E-07 & 70.25$\pm$2.46 \\ 

20.20-28.70 & 1.08E-06$\pm$5.89E-08 & 9.18E-06$\pm$5.01E-07 & 61.43$\pm$1.77 \\ 

28.70-37.70 & 9.54E-07$\pm$6.74E-08 & 8.58E-06$\pm$6.07E-07 & 57.26$\pm$1.69 \\ 

37.70-47.60 & 6.49E-07$\pm$4.47E-08 & 6.43E-06$\pm$4.42E-07 & 46.93$\pm$1.55 \\ 

47.60-58.50 & 5.28E-07$\pm$5.56E-08 & 5.75E-06$\pm$6.06E-07 & 45.19$\pm$1.53 \\ 

58.50-83.40 & 3.18E-07$\pm$2.20E-08 & 7.92E-06$\pm$5.47E-07 & 32.84$\pm$0.88 \\ 

83.40-105.50 & 2.32E-07$\pm$2.68E-08 & 5.14E-06$\pm$5.92E-07 & 25.93$\pm$0.69 \\ 

105.50-122.20 & 1.35E-07$\pm$2.04E-08 & 2.26E-06$\pm$3.40E-07 & 21.37$\pm$1.13 \\ 

122.20-161.00 & 8.90E-08$\pm$6.23E-09 & 3.45E-06$\pm$2.42E-07 & 16.35$\pm$0.66 \\ 

161.00-182.60 & 4.22E-08$\pm$9.36E-09 & 9.11E-07$\pm$2.02E-07 & 14.97$\pm$1.73 \\ 

182.60-236.70 & 3.23E-08$\pm$1.31E-08 & 1.75E-06$\pm$7.10E-07 & 14.69$\pm$0.84 \\ 

\end{tabular}
\caption{The blackbody model fit parameters of GRB 110920A.}
\end{table}


%%%%%%%%%%%%%%%%%%%%%%%%%%%%%%%%%%%%%%




\section{Band Fits}

\begin{table}[h]
\centering
\scriptsize
\label{tab:}
\begin{tabular}{c| c c c c c}
Time [s] & $\Fv$ [erg s$^{-1}$ cm$^{-2}$] & Energy Fluence [erg cm$^{-2}$] & $\Ep$ [keV] & $\alpha$ & $\beta$ \\
\hline \hline\\ 

-0.85-0.25 & 4.53E-06$\pm$1.41E-06 & 4.98E-06$\pm$1.55E-06 & 1394.43$\pm$0.00 & -0.82$\pm$0.15 & -1.95$\pm$0.51 \\ 

0.25-0.56 & 1.22E-05$\pm$2.47E-06 & 3.77E-06$\pm$7.65E-07 & 774.49$\pm$150.78 & -0.81$\pm$0.07 & -2.32$\pm$0.30 \\ 

0.56-1.20 & 5.65E-06$\pm$1.28E-06 & 3.62E-06$\pm$8.21E-07 & 300.31$\pm$41.97 & -0.53$\pm$0.11 & -2.22$\pm$0.21 \\ 

1.20-1.52 & 1.34E-06$\pm$1.55E-07 & 4.30E-07$\pm$4.96E-08 & 264.47$\pm$42.93 & -0.37$\pm$0.22 & -10.00$\pm$0.00 \\ 

1.52-3.76 & 7.35E-07$\pm$3.56E-07 & 1.65E-06$\pm$7.97E-07 & 221.68$\pm$56.38 & -1.02$\pm$0.13 & -2.62$\pm$1.08 \\ 

1.52-3.76 & 1.78E-07$\pm$9.56E-08 & 3.99E-07$\pm$2.14E-07 & 42.46$\pm$20.02 & 0.59$\pm$2.04 & -2.04$\pm$0.24 \\ 

\end{tabular}
\caption{The Band function fit parameters of GRB 081110A.}
\end{table}

\begin{table}[h]
\scriptsize
\centering
\label{tab:}
\begin{tabular}{c| c c c c c}
Time [s] & $\Fv$ [erg s$^{-1}$ cm$^{-2}$] & Energy Fluence [erg cm$^{-2}$] & $\Ep$ [keV] & $\alpha$ & $\beta$ \\
\hline \hline\\ 

-0.16-0.28 & 4.52E-06$\pm$1.67E-06 & 1.99E-06$\pm$7.33E-07 & 4087.35$\pm$1940.83 & -1.13$\pm$0.17 & -4.03$\pm$1.42 \\ 

0.28-0.90 & 6.15E-06$\pm$2.12E-06 & 3.81E-06$\pm$1.32E-06 & 779.29$\pm$318.97 & -0.49$\pm$0.21 & -2.93$\pm$0.27 \\ 

0.90-1.91 & 6.85E-06$\pm$4.49E-06 & 6.92E-06$\pm$4.53E-06 & 511.73$\pm$266.86 & -0.24$\pm$0.14 & -3.26$\pm$0.55 \\ 

1.91-4.13 & 5.17E-06$\pm$2.00E-06 & 1.15E-05$\pm$4.44E-06 & 299.57$\pm$104.56 & -0.41$\pm$0.07 & -3.07$\pm$0.31 \\ 

4.13-6.46 & 1.29E-06$\pm$3.61E-07 & 3.02E-06$\pm$8.42E-07 & 111.22$\pm$18.00 & -0.33$\pm$0.14 & -2.91$\pm$0.61 \\ 

6.46-7.46 & 1.37E-06$\pm$4.56E-07 & 1.37E-06$\pm$4.56E-07 & 157.12$\pm$64.07 & -0.71$\pm$0.19 & -2.68$\pm$0.37 \\ 

7.46-10.77 & 7.48E-07$\pm$1.40E-07 & 2.48E-06$\pm$4.64E-07 & 169.43$\pm$21.76 & -0.73$\pm$0.18 & -3.81$\pm$3.43 \\ 

10.77-12.50 & 4.24E-07$\pm$3.02E-07 & 7.34E-07$\pm$5.22E-07 & 172.61$\pm$69.34 & -1.02$\pm$0.31 & -6.97$\pm$1214.41 \\ 

12.50-18.20 & 2.39E-07$\pm$1.72E-07 & 1.36E-06$\pm$9.80E-07 & 149.39$\pm$35.47 & -1.22$\pm$0.14 & -6.87$\pm$1432.49 \\ 

\end{tabular}
\caption{The Band function fit parameters of GRB 081224A.}
\end{table}

\begin{table}[h]
\centering
\scriptsize
\label{tab:}
\begin{tabular}{c| c c c}
Time [s] & $\Fv$ [erg s$^{-1}$ cm$^{-2}$] & Energy Fluence [erg cm$^{-2}$] & $kT$ [keV] \\
\hline \hline\\ 

-0.16-0.28 & 1.87E-06$\pm$6.36E-07 & 8.24E-07$\pm$2.80E-07 & 161.32$\pm$25.14 \\ 

0.28-0.90 & 2.81E-06$\pm$1.42E-06 & 1.74E-06$\pm$8.79E-07 & 160.22$\pm$41.03 \\ 

0.90-1.91 & 1.73E-06$\pm$3.91E-06 & 1.75E-06$\pm$3.95E-06 & 157.22$\pm$86.17 \\ 

1.91-4.13 & 1.64E-06$\pm$1.75E-06 & 3.64E-06$\pm$3.88E-06 & 141.03$\pm$26.12 \\ 

4.13-6.46 & 1.22E-06$\pm$2.12E-07 & 2.84E-06$\pm$4.94E-07 & 82.83$\pm$8.63 \\ 

6.46-7.46 & 4.13E-07$\pm$3.07E-07 & 4.13E-07$\pm$3.07E-07 & 118.78$\pm$64.58 \\ 

7.46-10.77 & 2.53E-08$\pm$2.40E-08 & 8.39E-08$\pm$7.96E-08 & 10.05$\pm$4.45 \\ 

10.77-12.50 & 1.66E-08$\pm$4.54E-08 & 2.88E-08$\pm$7.85E-08 & 14.60$\pm$16.80 \\ 

12.50-18.20 & 0.00E+00$\pm$0.00E+00 & 0.00E+00$\pm$0.00E+00 & 0.00$\pm$0.00 \\ 

\end{tabular}
\caption{The blackbody model fit parameters of GRB 081224A.}
\end{table}

\begin{table}[h]
\centering
\scriptsize
\label{tab:}
\begin{tabular}{c| c c c c c}
Time [s] & $\Fv$ [erg s$^{-1}$ cm$^{-2}$] & Energy Fluence [erg cm$^{-2}$] & $\Ep$ [keV] & $\alpha$ & $\beta$ \\
\hline \hline\\ 

-0.10-0.73 & 3.28E-06$\pm$1.33E-06 & 2.72E-06$\pm$1.10E-06 & 984.15$\pm$498.89 & -0.82$\pm$0.24 & -3.46$\pm$4.02 \\ 

0.73-3.80 & 3.71E-06$\pm$1.05E-07 & 1.14E-05$\pm$3.23E-07 & 279.57$\pm$13.90 & -0.19$\pm$0.08 & -7.59$\pm$0.00 \\ 

3.80-4.44 & 7.74E-06$\pm$1.37E-06 & 4.95E-06$\pm$8.75E-07 & 402.21$\pm$73.21 & -0.52$\pm$0.17 & -3.08$\pm$0.74 \\ 

4.44-5.64 & 9.36E-06$\pm$7.48E-07 & 1.12E-05$\pm$8.97E-07 & 320.79$\pm$28.87 & -0.39$\pm$0.09 & -3.11$\pm$0.35 \\ 

5.64-6.78 & 6.08E-06$\pm$3.46E-07 & 6.93E-06$\pm$3.95E-07 & 248.91$\pm$15.00 & -0.43$\pm$0.08 & -4.28$\pm$1.66 \\ 

6.78-7.46 & 3.93E-06$\pm$2.65E-07 & 2.67E-06$\pm$1.80E-07 & 188.69$\pm$10.56 & -0.27$\pm$0.18 & -4.74$\pm$3.26 \\ 

7.46-8.07 & 2.30E-06$\pm$3.82E-07 & 1.40E-06$\pm$2.33E-07 & 124.54$\pm$12.92 & 3.20$\pm$4.80 & -2.59$\pm$0.21 \\ 

8.07-10.02 & 1.25E-06$\pm$9.76E-08 & 2.43E-06$\pm$1.90E-07 & 108.30$\pm$7.34 & -0.74$\pm$0.14 & -4.87$\pm$6.13 \\ 

10.02-12.48 & 8.73E-07$\pm$9.89E-08 & 2.15E-06$\pm$2.43E-07 & 69.14$\pm$49.07 & -0.98$\pm$0.44 & -2.64$\pm$0.76 \\ 

12.48-13.88 & 4.82E-07$\pm$6.97E-07 & 6.75E-07$\pm$9.75E-07 & 216.82$\pm$202.04 & -1.62$\pm$0.21 & -4.43$\pm$185.30 \\ 

13.88-16.20 & 2.03E-07$\pm$1.04E-07 & 4.71E-07$\pm$2.42E-07 & 62.55$\pm$32.41 & -1.36$\pm$0.69 & -3.10$\pm$2.84 \\ 

16.20-29.99 & 3.85E-08$\pm$2.71E-08 & 5.31E-07$\pm$3.74E-07 & 18.02$\pm$71.53 & -1.60$\pm$1.63 & -2.56$\pm$2.51 \\ 

\end{tabular}
\caption{The Band function fit parameters of GRB 090719A.}
\end{table}

\begin{table}[h]
\centering
\scriptsize
\label{tab:}
\begin{tabular}{c| c c c}
Time [s] & $\Fv$ [erg s$^{-1}$ cm$^{-2}$] & Energy Fluence [erg cm$^{-2}$] & $kT$ [keV] \\
\hline \hline\\ 

-0.10-0.73 & 2.16E-06$\pm$5.13E-07 & 1.80E-06$\pm$4.26E-07 & 101.43$\pm$8.43 \\ 

0.73-3.80 & 3.92E-07$\pm$1.18E-07 & 1.20E-06$\pm$3.64E-07 & 28.17$\pm$3.43 \\ 

3.80-4.44 & 4.47E-07$\pm$7.38E-07 & 2.86E-07$\pm$4.72E-07 & 53.70$\pm$17.94 \\ 

4.44-5.64 & 4.44E-07$\pm$3.21E-07 & 5.33E-07$\pm$3.86E-07 & 31.46$\pm$7.21 \\ 

5.64-6.78 & 2.59E-07$\pm$1.44E-07 & 2.95E-07$\pm$1.64E-07 & 20.90$\pm$5.24 \\ 

6.78-7.46 & 1.52E-07$\pm$1.07E-07 & 1.03E-07$\pm$7.28E-08 & 11.79$\pm$3.16 \\ 

7.46-8.07 & 5.18E-07$\pm$2.70E-07 & 3.16E-07$\pm$1.65E-07 & 10.17$\pm$2.35 \\ 

8.07-10.02 & 3.12E-08$\pm$3.92E-08 & 6.09E-08$\pm$7.65E-08 & 8.46$\pm$4.93 \\ 

10.02-12.48 & 3.54E-08$\pm$1.58E-07 & 8.71E-08$\pm$3.89E-07 & 19.18$\pm$51.28 \\ 

12.48-13.88 & 6.71E-08$\pm$3.07E-08 & 9.39E-08$\pm$4.29E-08 & 10.83$\pm$2.49 \\ 

13.88-16.20 & 1.95E-08$\pm$4.51E-08 & 4.52E-08$\pm$1.05E-07 & 9.05$\pm$6.43 \\ 

16.20-29.99 & 1.36E-08$\pm$4.24E-08 & 1.88E-07$\pm$5.85E-07 & 13.08$\pm$7.97 \\ 

\end{tabular}
\caption{The blackbody model fit parameters of GRB 090719A.}
\end{table}

\begin{table}[h]
\centering
\scriptsize
\label{tab:}
\begin{tabular}{c| c c c c c}
Time [s] & $\Fv$ [erg s$^{-1}$ cm$^{-2}$] & Energy Fluence [erg cm$^{-2}$] & $\Ep$ [keV] & $\alpha$ & $\beta$ \\
\hline \hline\\ 

-1.89-1.42 & 1.09E-06$\pm$7.04E-07 & 3.60E-06$\pm$2.33E-06 & 1029.28$\pm$965.19 & -0.82$\pm$0.31 & -3.38$\pm$5.87 \\ 

1.42-2.11 & 6.07E-06$\pm$1.92E-06 & 4.19E-06$\pm$1.33E-06 & 499.84$\pm$163.37 & -0.47$\pm$0.22 & -2.28$\pm$0.37 \\ 

2.11-4.51 & 7.36E-06$\pm$1.02E-06 & 1.77E-05$\pm$2.44E-06 & 400.64$\pm$73.69 & -0.79$\pm$0.09 & -2.25$\pm$0.15 \\ 

4.51-5.60 & 2.35E-06$\pm$8.65E-07 & 2.56E-06$\pm$9.42E-07 & 251.42$\pm$39.72 & -0.99$\pm$0.10 & -10.48$\pm$35148.20 \\ 

5.60-8.01 & 1.83E-06$\pm$3.22E-07 & 4.42E-06$\pm$7.77E-07 & 134.21$\pm$16.98 & -0.84$\pm$0.18 & -2.49$\pm$0.26 \\ 

8.01-9.95 & 6.40E-07$\pm$1.53E-07 & 1.24E-06$\pm$2.97E-07 & 93.93$\pm$14.26 & -1.06$\pm$0.42 & -2.95$\pm$1.06 \\ 

9.95-12.75 & 2.11E-07$\pm$5.76E-08 & 5.92E-07$\pm$1.61E-07 & 68.76$\pm$10.79 & 1.02$\pm$3.72 & -2.88$\pm$0.60 \\ 

\end{tabular}
\caption{The Band function fit parameters of GRB 090809B.}
\end{table}

\begin{table}[h]
\centering
\scriptsize
\label{tab:}
\begin{tabular}{c| c c c}
Time [s] & $\Fv$ [erg s$^{-1}$ cm$^{-2}$] & Energy Fluence [erg cm$^{-2}$] & $kT$ [keV] \\
\hline \hline\\ 

-1.89-1.42 & 1.10E-07$\pm$1.04E-07 & 3.63E-07$\pm$3.44E-07 & 50.21$\pm$16.22 \\ 

1.42-2.11 & 2.70E-07$\pm$1.32E-07 & 1.86E-07$\pm$9.08E-08 & 25.01$\pm$5.80 \\ 

2.11-4.51 & 3.55E-07$\pm$1.13E-07 & 8.53E-07$\pm$2.70E-07 & 25.42$\pm$2.72 \\ 

4.51-5.60 & 1.36E-07$\pm$7.19E-08 & 1.48E-07$\pm$7.84E-08 & 16.27$\pm$4.06 \\ 

5.60-8.01 & 5.29E-09$\pm$2.98E-08 & 1.28E-08$\pm$7.19E-08 & 7.01$\pm$16.71 \\ 

8.01-9.95 & 1.39E-08$\pm$3.22E-08 & 2.70E-08$\pm$6.25E-08 & 3.95$\pm$1.83 \\ 

9.95-12.75 & 4.23E-08$\pm$4.24E-08 & 1.18E-07$\pm$1.19E-07 & 4.46$\pm$1.28 \\ 

\end{tabular}
\caption{The blackbody model fit parameters of GRB 090809B.}
\end{table}

\begin{table}[h]
\centering
\scriptsize
\label{tab:}
\begin{tabular}{c| c c c c c}
Time [s] & $\Fv$ [erg s$^{-1}$ cm$^{-2}$] & Energy Fluence [erg cm$^{-2}$] & $\Ep$ [keV] & $\alpha$ & $\beta$ \\
\hline \hline\\ 

-0.20-0.20 & 5.95E+03$\pm$6.31E-06 & 2.38E+03$\pm$2.52E-06 & 1119.49$\pm$985.94 & -0.05$\pm$0.37 & -2.42$\pm$0.99 \\ 

0.20-0.80 & 1.21E+04$\pm$3.36E-06 & 7.29E+03$\pm$2.01E-06 & 1418.53$\pm$923.10 & -0.42$\pm$0.33 & -2.22$\pm$0.25 \\ 

0.80-2.40 & 1.29E+04$\pm$1.87E-06 & 2.06E+04$\pm$2.99E-06 & 690.11$\pm$109.88 & -0.10$\pm$0.14 & -2.69$\pm$0.17 \\ 

2.40-3.00 & 8.36E+03$\pm$2.71E-06 & 5.02E+03$\pm$1.63E-06 & 428.41$\pm$142.72 & -0.19$\pm$0.29 & -2.50$\pm$0.29 \\ 

3.00-4.30 & 5.23E+03$\pm$1.39E-06 & 6.80E+03$\pm$1.81E-06 & 217.21$\pm$74.34 & -0.11$\pm$0.40 & -2.41$\pm$0.18 \\ 

4.30-5.70 & 2.99E+03$\pm$5.12E-07 & 4.18E+03$\pm$7.16E-07 & 135.99$\pm$6.74 & 3.86$\pm$3.41 & -2.50$\pm$0.11 \\ 

5.70-7.20 & 1.68E+03$\pm$3.14E-07 & 2.53E+03$\pm$4.71E-07 & 130.25$\pm$8.04 & 3.36$\pm$2.88 & -2.66$\pm$0.16 \\ 

7.20-12.80 & 1.13E+03$\pm$1.29E-07 & 6.30E+03$\pm$7.23E-07 & 124.96$\pm$5.77 & 0.34$\pm$0.41 & -2.83$\pm$0.17 \\ 

12.80-22.20 & 3.70E+02$\pm$6.48E-08 & 3.48E+03$\pm$6.09E-07 & 87.26$\pm$7.44 & -0.61$\pm$0.25 & -2.90$\pm$0.33 \\ 

\end{tabular}
\caption{The Band function fit parameters of GRB 100707A.}
\end{table}

\begin{table}[h]
\centering
\scriptsize
\label{tab:}
\begin{tabular}{c| c c c}
Time [s] & $\Fv$ [erg s$^{-1}$ cm$^{-2}$] & Energy Fluence [erg cm$^{-2}$] & $kT$ [keV] \\
\hline \hline\\ 

-0.20-0.20 & 5.75E-06$\pm$1.08E-14 & 2.30E-06$\pm$4.31E-15 & 683.86$\pm$415.26 \\ 

0.20-0.80 & 7.20E-06$\pm$3.23E-15 & 4.32E-06$\pm$1.94E-15 & 120.00$\pm$8.50 \\ 

0.80-2.40 & 5.36E-06$\pm$1.83E-15 & 8.58E-06$\pm$2.94E-15 & 67.80$\pm$3.65 \\ 

2.40-3.00 & 2.67E-06$\pm$2.11E-15 & 1.60E-06$\pm$1.26E-15 & 45.75$\pm$4.83 \\ 

3.00-4.30 & 1.30E-06$\pm$1.74E-15 & 1.69E-06$\pm$2.26E-15 & 34.56$\pm$4.09 \\ 

4.30-5.70 & 8.96E-07$\pm$6.65E-16 & 1.25E-06$\pm$9.30E-16 & 13.11$\pm$2.63 \\ 

5.70-7.20 & 4.75E-07$\pm$3.14E-16 & 7.12E-07$\pm$4.71E-16 & 10.82$\pm$2.01 \\ 

7.20-12.80 & 1.38E-07$\pm$8.73E-17 & 7.71E-07$\pm$4.89E-16 & 8.17$\pm$0.57 \\ 

12.80-22.20 & 6.85E-08$\pm$2.94E-17 & 6.44E-07$\pm$2.76E-16 & 6.62$\pm$0.76 \\ 

\end{tabular}
\caption{The blackbody model fit parameters of GRB 100707A.}
\end{table}

\begin{table}[h]
\centering
\scriptsize
\label{tab:}
\begin{tabular}{c| c c c c c}
Time [s] & $\Fv$ [erg s$^{-1}$ cm$^{-2}$] & Energy Fluence [erg cm$^{-2}$] & $\Ep$ [keV] & $\alpha$ & $\beta$ \\
\hline \hline\\ 

-5.38-2.82 & 8.34E+02$\pm$1.93E+04 & 6.84E+03$\pm$1.58E+05 & 688.68$\pm$101.82 & -0.90$\pm$0.06 & -9.47$\pm$3755.81 \\ 

2.82-3.84 & 4.21E+03$\pm$9.31E+05 & 4.31E+03$\pm$9.54E+05 & 517.25$\pm$122.61 & -0.74$\pm$0.11 & -2.48$\pm$0.42 \\ 

3.84-4.86 & 4.64E+03$\pm$1.15E+06 & 4.75E+03$\pm$1.18E+06 & 749.75$\pm$437.21 & -1.14$\pm$0.14 & -2.04$\pm$0.24 \\ 

4.86-6.91 & 2.23E+03$\pm$1.40E+06 & 4.57E+03$\pm$2.86E+06 & 481.21$\pm$528.67 & -0.96$\pm$0.12 & -2.67$\pm$1.32 \\ 

6.91-9.98 & 9.34E+02$\pm$1.87E+05 & 2.87E+03$\pm$5.73E+05 & 472.70$\pm$189.99 & -1.20$\pm$0.14 & -3.57$\pm$8.23 \\ 

9.98-15.10 & 1.18E+02$\pm$1.72E+05 & 6.05E+02$\pm$8.80E+05 & 215.04$\pm$730.55 & -1.85$\pm$0.32 & -12.63$\pm$28637996.00 \\ 

\end{tabular}
\caption{The Band function fit parameters of GRB 110407A.}
\end{table}

\begin{table}[h]
\centering
\scriptsize
\label{tab:}
\begin{tabular}{c| c c c}
Time [s] & $\Fv$ [erg s$^{-1}$ cm$^{-2}$] & Energy Fluence [erg cm$^{-2}$] & $kT$ [keV] \\
\hline \hline\\ 

-5.38-2.82 & 0.00E+00$\pm$0.00E+00 & 0.00E+00$\pm$0.00E+00 & 0.00$\pm$0.00 \\ 

2.82-3.84 & 6.53E+01$\pm$9.21E+03 & 6.69E+01$\pm$9.43E+03 & 28.96$\pm$17.01 \\ 

3.84-4.86 & 7.66E+02$\pm$3.53E+04 & 7.85E+02$\pm$3.62E+04 & 72.97$\pm$9.06 \\ 

4.86-6.91 & 2.12E+02$\pm$1.25E+06 & 4.34E+02$\pm$2.57E+06 & 149.83$\pm$173.47 \\ 

6.91-9.98 & 5.15E+01$\pm$3.10E+03 & 1.58E+02$\pm$9.52E+03 & 36.83$\pm$11.76 \\ 

9.98-15.10 & 3.27E+01$\pm$4.77E+02 & 1.67E+02$\pm$2.44E+03 & 41.52$\pm$16.65 \\ 

\end{tabular}
\caption{The blackbody model fit parameters of GRB 110407A.}
\end{table}

\begin{table}[h]
\centering
\scriptsize
\label{tab:}
\begin{tabular}{c| c c c c c}
Time [s] & $\Fv$ [erg s$^{-1}$ cm$^{-2}$] & Energy Fluence [erg cm$^{-2}$] & $\Ep$ [keV] & $\alpha$ & $\beta$ \\
\hline \hline\\ 

-0.07-0.08 & 4.49E-05$\pm$3.57E-06 & 6.73E-06$\pm$5.35E-07 & 15672.66$\pm$1659.60 & -1.04$\pm$0.04 & -3.40$\pm$0.39 \\ 

0.08-0.48 & 5.22E-05$\pm$3.01E-06 & 2.09E-05$\pm$1.20E-06 & 6332.89$\pm$696.01 & -0.93$\pm$0.03 & -2.99$\pm$0.13 \\ 

0.48-1.28 & 3.13E-05$\pm$1.32E-06 & 2.50E-05$\pm$1.06E-06 & 2613.78$\pm$206.68 & -0.90$\pm$0.02 & -2.99$\pm$0.09 \\ 

1.28-2.78 & 1.24E-05$\pm$5.87E-07 & 1.87E-05$\pm$8.80E-07 & 1305.33$\pm$146.23 & -1.04$\pm$0.03 & -2.96$\pm$0.11 \\ 

2.78-3.78 & 6.25E-06$\pm$3.97E-07 & 6.25E-06$\pm$3.97E-07 & 680.15$\pm$122.58 & -1.20$\pm$0.04 & -2.71$\pm$0.14 \\ 

3.78-5.88 & 3.68E-06$\pm$1.98E-07 & 7.72E-06$\pm$4.16E-07 & 495.30$\pm$60.64 & -1.13$\pm$0.06 & -2.54$\pm$0.08 \\ 

5.88-7.63 & 1.74E-06$\pm$2.21E-07 & 3.04E-06$\pm$3.87E-07 & 303.00$\pm$46.33 & -0.99$\pm$0.15 & -2.66$\pm$0.30 \\ 

7.63-12.63 & 1.00E-06$\pm$9.49E-08 & 5.00E-06$\pm$4.75E-07 & 304.57$\pm$46.47 & -1.02$\pm$0.09 & -2.59$\pm$0.18 \\ 

\end{tabular}
\caption{The Band function fit parameters of GRB 110721A.}
\end{table}

\begin{table}[h]
\centering
\scriptsize
\label{tab:}
\begin{tabular}{c| c c c}
Time [s] & $\Fv$ [erg s$^{-1}$ cm$^{-2}$] & Energy Fluence [erg cm$^{-2}$] & $kT$ [keV] \\
\hline \hline\\ 

-0.07-0.08 & 0.00E+00$\pm$0.00E+00 & 0.00E+00$\pm$0.00E+00 & 0.00$\pm$0.00 \\ 

0.08-0.48 & 4.80E-07$\pm$2.38E-07 & 1.92E-07$\pm$9.53E-08 & 65.64$\pm$18.56 \\ 

0.48-1.28 & 5.60E-07$\pm$1.02E-07 & 4.48E-07$\pm$8.17E-08 & 38.42$\pm$4.23 \\ 

1.28-2.78 & 8.48E-07$\pm$7.35E-08 & 1.27E-06$\pm$1.10E-07 & 32.23$\pm$1.46 \\ 

2.78-3.78 & 3.89E-07$\pm$6.07E-08 & 3.89E-07$\pm$6.07E-08 & 20.59$\pm$1.64 \\ 

3.78-5.88 & 8.76E-08$\pm$2.19E-08 & 1.84E-07$\pm$4.61E-08 & 10.12$\pm$1.16 \\ 

5.88-7.63 & 1.84E-08$\pm$2.03E-08 & 3.22E-08$\pm$3.56E-08 & 5.49$\pm$1.70 \\ 

7.63-12.63 & 1.09E-08$\pm$1.19E-08 & 5.46E-08$\pm$5.94E-08 & 11.05$\pm$6.10 \\ 

\end{tabular}
\caption{The blackbody model fit parameters of GRB 110721A.}
\end{table}

\begin{table}[h]
\centering
\scriptsize
\label{tab:}
\begin{tabular}{c| c c c c c}
Time [s] & $\Fv$ [erg s$^{-1}$ cm$^{-2}$] & Energy Fluence [erg cm$^{-2}$] & $\Ep$ [keV] & $\alpha$ & $\beta$ \\
\hline \hline\\ 

-1.60-1.10 & 4.39E-07$\pm$1.28E-06 & 1.18E-06$\pm$3.45E-06 & 525.46$\pm$6552.12 & -1.26$\pm$0.74 & -2.09$\pm$4.01 \\ 

1.10-4.90 & 1.81E-06$\pm$9.15E-07 & 6.90E-06$\pm$3.48E-06 & 525.34$\pm$617.41 & -0.88$\pm$0.19 & -2.27$\pm$0.80 \\ 

4.90-6.90 & 2.71E-06$\pm$1.14E-06 & 5.42E-06$\pm$2.27E-06 & 524.78$\pm$712.80 & -1.05$\pm$0.21 & -2.02$\pm$0.51 \\ 

6.90-16.40 & 3.97E-06$\pm$4.63E-07 & 3.77E-05$\pm$4.40E-06 & 524.37$\pm$64.50 & -0.57$\pm$0.09 & -2.67$\pm$0.28 \\ 

16.40-20.20 & 2.20E-06$\pm$2.77E-07 & 8.36E-06$\pm$1.05E-06 & 685.22$\pm$179.73 & -0.89$\pm$0.13 & -7.00$\pm$0.00 \\ 

20.20-28.70 & 2.06E-06$\pm$1.40E-07 & 1.75E-05$\pm$1.19E-06 & 528.18$\pm$73.51 & -0.75$\pm$0.10 & -5.25$\pm$0.00 \\ 

28.70-37.70 & 9.45E-07$\pm$1.18E-07 & 8.50E-06$\pm$1.06E-06 & 338.95$\pm$61.70 & -0.91$\pm$0.13 & -8.12$\pm$0.00 \\ 

37.70-47.60 & 7.05E-07$\pm$8.47E-08 & 6.98E-06$\pm$8.39E-07 & 334.08$\pm$79.33 & -1.01$\pm$0.15 & -7.42$\pm$0.00 \\ 

47.60-58.50 & 4.29E-07$\pm$7.08E-08 & 4.68E-06$\pm$7.72E-07 & 207.33$\pm$54.92 & -1.11$\pm$0.17 & -7.24$\pm$0.00 \\ 

58.50-83.40 & 2.31E-07$\pm$4.00E-08 & 5.76E-06$\pm$9.95E-07 & 112.81$\pm$21.41 & -0.82$\pm$0.23 & -5.00$\pm$0.00 \\ 

83.40-105.50 & 5.76E-08$\pm$1.07E-08 & 1.27E-06$\pm$2.36E-07 & 35.29$\pm$3.72 & 1.38$\pm$0.75 & -5.00$\pm$0.00 \\ 

105.50-122.20 & 3.93E-08$\pm$5.02E-08 & 6.57E-07$\pm$8.38E-07 & 35.26$\pm$8.53 & 2.07$\pm$1.44 & -4.32$\pm$8.77 \\ 

122.20-161.00 & 3.12E-08$\pm$6.65E-09 & 1.21E-06$\pm$2.58E-07 & 30.00$\pm$2.80 & 3.08$\pm$1.43 & -4.32$\pm$0.00 \\ 

161.00-182.60 & 1.21E-08$\pm$1.23E-08 & 2.61E-07$\pm$2.66E-07 & 27.70$\pm$13.43 & 1.19$\pm$2.83 & -4.32$\pm$0.00 \\ 

182.60-236.70 & 1.59E-08$\pm$5.00E-09 & 8.61E-07$\pm$2.70E-07 & 19.43$\pm$3.30 & 1.11$\pm$2.66 & -4.32$\pm$0.00 \\ 

\end{tabular}
\caption{The Band function fit parameters of GRB 110920A.}
\end{table}

\begin{table}[h]
\centering
\scriptsize
\label{tab:}
\begin{tabular}{c| c c c}
Time [s] & $\Fv$ [erg s$^{-1}$ cm$^{-2}$] & Energy Fluence [erg cm$^{-2}$] & $kT$ [keV] \\
\hline \hline\\ 

-1.60-1.10 & 6.75E-07$\pm$8.73E-07 & 1.82E-06$\pm$2.36E-06 & 140.97$\pm$37.54 \\ 

1.10-4.90 & 1.52E-06$\pm$5.92E-07 & 5.76E-06$\pm$2.25E-06 & 118.92$\pm$17.70 \\ 

4.90-6.90 & 2.44E-06$\pm$4.86E-07 & 4.88E-06$\pm$9.72E-07 & 98.32$\pm$8.94 \\ 

6.90-16.40 & 1.43E-06$\pm$1.76E-07 & 1.36E-05$\pm$1.67E-06 & 80.74$\pm$3.72 \\ 

16.40-20.20 & 1.56E-06$\pm$1.76E-07 & 5.93E-06$\pm$6.69E-07 & 68.36$\pm$2.35 \\ 

20.20-28.70 & 8.91E-07$\pm$1.19E-07 & 7.58E-06$\pm$1.01E-06 & 58.53$\pm$2.03 \\ 

28.70-37.70 & 8.96E-07$\pm$8.41E-08 & 8.06E-06$\pm$7.56E-07 & 55.97$\pm$2.09 \\ 

37.70-47.60 & 6.53E-07$\pm$6.35E-08 & 6.47E-06$\pm$6.28E-07 & 45.65$\pm$1.49 \\ 

47.60-58.50 & 5.16E-07$\pm$4.77E-08 & 5.62E-06$\pm$5.19E-07 & 43.25$\pm$2.01 \\ 

58.50-83.40 & 2.50E-07$\pm$3.30E-08 & 6.22E-06$\pm$8.22E-07 & 31.65$\pm$1.96 \\ 

83.40-105.50 & 2.51E-07$\pm$9.18E-09 & 5.54E-06$\pm$2.03E-07 & 26.73$\pm$0.91 \\ 

105.50-122.20 & 1.23E-07$\pm$4.36E-08 & 2.06E-06$\pm$7.27E-07 & 21.50$\pm$1.18 \\ 

122.20-161.00 & 9.32E-08$\pm$5.82E-09 & 3.62E-06$\pm$2.26E-07 & 19.69$\pm$0.91 \\ 

161.00-182.60 & 4.34E-08$\pm$1.11E-08 & 9.38E-07$\pm$2.40E-07 & 15.71$\pm$1.72 \\ 

182.60-236.70 & 3.55E-08$\pm$4.51E-09 & 1.92E-06$\pm$2.44E-07 & 14.77$\pm$1.29 \\ 

\end{tabular}
\caption{The blackbody model fit parameters of GRB 110920A.}
\end{table}


\section{Spectral Evolution Plots}

\begin{figure}[h]
  \centering
  \cfig{12}{081110601.pdf}{4}
  \caption{The spectral evolution of GRB 081110 A. Synchrotron in
    indicated as evolving from cyan to blue.}
  \label{fig:specEvoGRB1}
\end{figure}

\begin{figure}[h]
  \centering
  \cfig{12}{081224887.pdf}{4}
  \caption{The spectral evolution of GRB 081224 A. Synchrotron in
    indicated as evolving from cyan to blue. The blackboy is indicated
    as evolving from yellow to red.}
  \label{fig:specEvoGRB2}
\end{figure}

\begin{figure}[h]
  \centering
  \cfig{12}{090719063.pdf}{4}
  \caption{The spectral evolution of GRB 090719 A. Synchrotron in
    indicated as evolving from cyan to blue. The blackboy is indicated
    as evolving from yellow to red.}
  \label{fig:specEvoGRB3}
\end{figure}

\begin{figure}[h]
  \centering
  \cfig{12}{090809978.pdf}{4}
  \caption{The spectral evolution of GRB 090809 A. Synchrotron in
    indicated as evolving from cyan to blue.}
  \label{fig:specEvoGRB4}
\end{figure}

\begin{figure}[h]
  \centering
  \cfig{12}{100707032.pdf}{4}
  \caption{The spectral evolution of GRB 100707 A. Synchrotron in
    indicated as evolving from cyan to blue. The blackboy is indicated
    as evolving from yellow to red.}
  \label{fig:specEvoGRB5}
\end{figure}

\begin{figure}[h]
  \centering
  \cfig{12}{110407998.pdf}{4}
  \caption{The spectral evolution of GRB 110407 A. Synchrotron in
    indicated as evolving from cyan to blue.}
  \label{fig:specEvoGRB6}
\end{figure}

\begin{figure}[h]
  \centering
  \cfig{12}{110721200.pdf}{4}
  \caption{The spectral evolution of GRB 110721 A. Synchrotron in
    indicated as evolving from cyan to blue. The blackboy is indicated
    as evolving from yellow to red.}
  \label{fig:specEvoGRB7}
\end{figure}

\begin{figure}[h]
  \centering
  \cfig{12}{110920546.pdf}{4}
  \caption{The spectral evolution of GRB 1109020 A. Synchrotron in
    indicated as evolving from cyan to blue. The blackboy is indicated
    as evolving from yellow to red.}
  \label{fig:specEvoGRB8}
\end{figure}


%%% Local Variables: 
%%% mode: latex
%%% TeX-master: "../thesis"
%%% End: 

\chapter{Justification of the Synchrotron Parameters}
\label{ch:pap2app}
\section{Synchrotron-Shell-Model Constraints}
Broad ranges of parameter values are possible in a GRB colliding shell
model. Here we justify the values used to fit the {\it Fermi} GBM and
LAT GRBs, assuming that the bright keV -- MeV emission of the GRBs in
our sample is primarily nonthermal synchrotron radiation emitted by
nonthermal electrons with an isotropic pitch-angle distribution that
radiate in a spherical shell expanding at relativistic speeds, within
which is entrained randomly directed magnetic field on coherence
length scales small in comparison with the shell volume. For
additional considerations about synchrotron models, see
\cite{2013ApJ...769...69B}.


The constraints that we consider are (1) particle and magnetic-field
energetics; (2) a negligible synchrotron self-Compton (SSC) component
so that we can neglect any high-energy $\gamma$-rays that could be
absorbed through $\gamma\gamma$ pair production and make additional
radiation at energies where the data are fit; (3) small synchrotron
self-absorption; and (4) minimum bulk Lorentz factor $\Gamma_{min}$ to
avoid strong $\gamma\gamma$ opacity. We also examine (5) the criterion
for being in the strong cooling regime.  To suppress SSC, we focus on
magnetically dominated models, which are also required in some
theories of GRBs to trigger magnetic reconnection events and produce
the prompt GRB emission through synchrotron emission
\cite{2009JPhCS.189a2018G,zhang:2011}. Magnetically
dominated GRB synchrotron models are also required for efficient
acceleration of ultra-high energy cosmic rays
\cite{2010OAJ.....3..150R}.


Our fiducial parameters are: characteristic electron Lorentz factor
$\gamma^\prime = 10^3 \gamma^\prime_3$; bulk Lorentz factor $\Gamma =
300\Gamma_{300}$, and fluid magnetic field $B^\prime = 10^5
B^\prime_5$ G. Radiation with characteristic $\nu F_\nu$ peak
frequency $\nu_{obs} = m_ec^2 \epsilon /h(1+z)$ is observed during the
prompt phase of the GRB. If non-thermal lepton synchrotron radiation,
then $\epsilon \cong 3\Gamma B^\prime \gamma^{\prime 2}/2B_{cr}$, and
$z$ is the source redshift, so $B_5^\prime \cong \epsilon/\Gamma_{300}
\gamma_3^{\prime 2}$.



\subsection{Energetics}
The electron energy content $\epsilon_e^{(\prime)}$ in the source
(comoving) frame is given by $\epsilon_e^\prime = \epsilon_e/\Gamma =
N_{e0}\gamma^\prime m_ec^2$, where $N_{e0}$ is the number of
electrons, so that
\begin{equation}
\epsilon_e = {\epsilon_{par}\over 1+\zeta}
 = 
{6\pi m_ec L_{syn}\over \sigma_{\rm T} B^{\prime 2} \gamma^\prime \Gamma^2}
=
\,{27 \pi m_ec L_{syn}\over 2  \sigma_{\rm T} B_{cr}^2 \epsilon^2}
\,\Gamma\gamma^{\prime 3}
 \cong 
10^{45} \, {L_{51} \Gamma_3 \gamma_3^{\prime 3}\over \epsilon^{2}}\;\;{\rm erg}\;, 
\label{Epar}
\end{equation}
where the total particle energy is denoted $\epsilon_{par} $, and
$\zeta$ represents the additional energy in hadrons. Here the
synchrotron luminosity $L_{syn}=10^{51}L_{51}$ erg s$^{-1}$ is derived
from the synchrotron electron energy-loss rate formula, using
$L^\prime_{syn} = c \sigma_{\rm T} B^{\prime 2}\gamma^{\prime
  2}N_{e0}/6\pi$.

The magnetic-field energy density $\epsilon_B = \Gamma {\cal
  E}_B^\prime = \Gamma 4\pi r^2 \Delta r^\prime (B^{\prime 2}/8\pi)$.
The shell width $\Delta r^\prime = k r/\Gamma$, with $k$ a factor of
order unity (for details see \cite{2013ApJ...769...69B}), using the
relations $\Delta r^\prime \cong \Gamma c t_{var}$ and $r \cong
\Gamma^2 c t_{var}$, where $t_{var}$ is the measured variability time
scale in the source frame. Thus the isotropic magnetic-field energy
\begin{equation}
\epsilon_{B} = {2k \Gamma^4\over 9}\, {c^3 t_{var}^3 B_{cr}^2 \epsilon^2\over \gamma^{\prime 4}} \cong 
10^{56}  k({\Gamma_{300}\over \gamma_3^{\prime}})^4 t_{var}({\rm s})^3 \epsilon^{2}\;\;{\rm erg}.
\label{eq:EB}
\end{equation}
The absolute magnetic field energy $\epsilon_{B,abs} \cong
(\theta_j^2/2)\epsilon_{B}$ for this system greatly out of
equipartition can be reduced to acceptable values (i.e., ${\cal
  E}_{abs}\ll 10^{54}$ erg) with a sufficiently small jet opening
angle $\theta_j$ between $\approx 0.01$ and $0.1$.


\subsection{SSC Component}
The ratio of the SSC and synchrotron luminosities is related to the
ratio of the synchrotron and magnetic field energy densities through
the relation $L_{SSC}/L_{syn} \lesssim
u^\prime_{syn}/u^\prime_{B^\prime}$, with the inequality arising from
the neglect of Klein-Nishina effects on the SSC emission.  Because
$u^\prime_{syn} \cong L^\prime_{syn}/4\pi r^2 c$, we have
\begin{equation}
{L_{SSC}\over L_{syn}} \approx {2 L_{syn}\over c^3 \Gamma^6 t_{var}^2 B^{\prime 2}} \cong {10^{-5}  L_{52}\over \Gamma_{300}^6 t_{var}^2({\rm s}) B^{\prime 2}_5 } \;,
\label{SSCsynratio}
\end{equation}
and so can be safely neglected here.



\subsection{Synchrotron Self-Absorption}
For a log-parabolic description of the $\gamma^{\prime 2} N^\prime
(\gamma_p)$ electron distribution, the SSA opacity in the
$\delta$-function approximation is given by
\begin{equation}
  \tau_{\epsilon^\prime} = 2\kappa_{\epsilon^\prime} \Delta r^\prime \cong {\pi \over 9}\,{\epsilon_e^\prime \Delta r^\prime\over m_ec^2 I(b) V_b^\prime \gamma_p^{\prime 4}}\, {\lambda_{\rm C} r_e\over \epsilon^\prime} (2+ b\log x)\,x^{-(4+b\log x)}
  \equiv \tau_0 (2+ b\log x)\,x^{-(4+b\log x)}\;
\label{tauep}
\end{equation}
\cite{2009herb.book.....D,2013arXiv1304.6680D}, where
$\kappa_{\epsilon^\prime}$ is the SSA absorption coefficient (units of
inverse length), $x \equiv
\sqrt{\ep/2\varepsilon_B^\prime}/\g_p^\prime$, $\e \cong \Gamma\ep$,
shell volume $V_b^\prime = 4\pi r^2 \Delta r^\prime$, and $I(b) =
\sqrt{\pi \ln 10/b}$ normalizes the electron spectrum depending on the
value of the log-parabola width parameter $b$.  Using \equationref{eq:EB}, we
obtain
\begin{equation}
\tau_{0} \cong {\pi \over 6\e }\,{ {\lambda_{\rm C} r_e L_{syn} \over c^3 \sigma_{\rm T} B^{\prime 2} t_{var}^2 \Gamma^5 \gamma^{\prime 5} I(b)}}
\approx {10^{-16} \over \e^3 }\,{ { L_{51} \over  t_{var}^2({\rm s}) \Gamma_{300}^3 \gamma_3^\prime I(b)}}\;, 
\label{eq:tau0}
\end{equation}
using the relation $\epsilon \cong \Gamma_{300} B^\prime_5
\gamma^{\prime 2}_3$ characterizing the condition that $x \approx 1$.
Thus SSA is utterly negligible at $x \gtrsim 0.1$, where the question
of SSA opacity is most important, noting from \equationref{eq:tau0} that the
opacity can grow as fast as $x^{-4}$ at $x\approx 0.1$, when
$b\lesssim 1$.

\subsection{$\gamma$-$\gamma$ Opacity}
The minimum bulk Lorentz factor giving a $\gamma$-ray with energy
$\epsilon_\gamma = 1.96\times 10^{5} E_\gamma$(GeV) unit optical depth
for $\gamma\gamma$ absorption by the target synchrotron photons is
estimated fairly accurately by the expression
\begin{equation}
\Gamma \geq \Gamma_{min} = \left[ { \sigma_{\rm T} \hat\epsilon L(\hat \epsilon ) \epsilon_{\gamma}
\over 16 \pi m_ec^4 t_{var} }\right ]^{1/6}\;,\; \hat \epsilon \cong 2\Gamma^2/\epsilon_\gamma .
\label{tauep}
\end{equation}
Taking $\e L(\e) \cong 10^{51}L_{51}/\ln (100)$ erg s$^{-1}$, i.e., a
flat $\nu F_\nu$ spectrum over 2 decades in frequency, then the
minimum bulk Lorentz factor $\Gamma_{min} \approx 300 \,[L_{51}
E_\gamma$(100~GeV)$/t_{var}({\rm s})]^{1/6}$.  For GRB synchrotron
radiation emitted in the $0.1 \lesssim \epsilon \lesssim 10$ range,
$\gamma$-rays with energies between $\approx (0.01$ --
1)$\Gamma_{300}^2$ TeV are subject to $\gamma\gamma$ opacity.
Provided that the energy radiated at 100 GeV and TeV energies is much
smaller that the total GRB photon energy, opacity effects and
cascading can be neglected.





\subsection{Cooling Regime}

The minimum and cooling frequencies in a colliding shell are derived
in the same way as the case of a blast wave decelerating by sweeping
up external medium material at a shock \cite{sari:1998}, recognizing
that the relative Lorentz factor between two shells is more likely to
be $\Gamma_{rel}\sim 10$, compared to the external shock Lorentz
factor $\Gamma \sim 300$. The system is in the slow cooling regime
when the cooling Lorentz factor
\begin{equation}
\gamma_c^\prime \cong {6\pi m_e c\over \sigma_{\rm T} B^{\prime 2}\Gamma t_{var} }
\gtrsim
\gamma^\prime_{min} \cong \epsilon_e {m_p\over m_e} f(p) \Gamma_{rel}\;,
\label{gammaprimemin}
\end{equation}
where $\gamma^\prime_{min}$ is the minimum electron Lorentz factor,
$p$ is the injection number index of relativistic electrons,
$\epsilon_e$ is the fraction of energy dissipated at the shock that
goes into nonthermal electrons, and the factor $f(p) = (p-2)/(p-1)$
normalizes the number and energy of the energized electrons.  Solving
gives
\begin{equation}
B^{\prime}
\lesssim 
\sqrt{ 6\pi m_e c (m_e/m_p)\over \sigma_{\rm T} \Gamma t_{var}\epsilon_e f(p)\Gamma_{rel} }
\approx {120 {\rm ~G}\over \sqrt{\Gamma_{300} (\epsilon_e/0.1) t_{var}({\rm s}) f(p)\Gamma_{rel}}}\;.
\label{gammaprimemin}
\end{equation}
A system with $\sim 100$ kG fields is always in the fast cooling
regime according to this criterion.


%%% Local Variables: 
%%% mode: latex
%%% TeX-master: "../thesis"
%%% End: 

\chapter{Derived Jet Parameters}
\label{ch:jetParms}

\section{Jet Parameter Values}


\begin{table}[htp]
\scriptsize
\label{tab:}
\begin{tabular}{c c c c c}
Time [s] & $\Gamma$ & $r_0$ [cm] & $r_{\rm ph}$ [cm] & $r_{\rm nt}$ [cm] \\
\hline \hline\\ 

-0.16-0.28 & 573.62$\pm$93.07 & 8.78E+06$\pm$1.01E+07 & 1.10E+11$\pm$5.11E+10 & 6.33E+12$\pm$7.81E+12 \\ 

0.28-0.90 & 509.05$\pm$25.78 & 7.95E+07$\pm$3.00E+07 & 2.31E+11$\pm$2.95E+10 & 9.31E+12$\pm$7.35E+12 \\ 

0.90-1.91 & 438.43$\pm$15.25 & 1.23E+08$\pm$3.09E+07 & 3.58E+11$\pm$3.14E+10 & 4.68E+12$\pm$2.28E+12 \\ 

1.91-4.13 & 362.11$\pm$9.40 & 1.10E+08$\pm$2.13E+07 & 5.00E+11$\pm$3.62E+10 & 6.01E+12$\pm$1.29E+12 \\ 

4.13-6.46 & 274.37$\pm$17.78 & 4.87E+07$\pm$2.36E+07 & 4.48E+11$\pm$8.57E+10 & 3.92E+12$\pm$1.67E+12 \\ 

6.46-7.46 & 260.99$\pm$76.54 & 9.29E+06$\pm$1.97E+07 & 3.08E+11$\pm$2.70E+11 & 4.34E+12$\pm$7.74E+12 \\ 

7.46-10.77 & 206.81$\pm$30.29 & 3.21E+07$\pm$3.67E+07 & 2.94E+11$\pm$1.28E+11 & 2.89E+12$\pm$2.66E+12 \\ 

10.77-12.50 & 193.98$\pm$215.66 & 2.10E+06$\pm$1.71E+07 & 1.98E+11$\pm$6.60E+11 & 2.13E+12$\pm$1.43E+13 \\ 

12.50-18.20 & 81.46$\pm$16.70 & 9.28E+07$\pm$1.62E+08 & 1.42E+12$\pm$8.70E+11 & 1.37E+10$\pm$1.78E+10 \\ 

\end{tabular}
\caption{Inferred jet paramerters for GRB081224A assuming Y=1.}
\end{table}

\begin{table}[htp]
\scriptsize
\label{tab:}
\begin{tabular}{c c c c c}
Time [s] & $\Gamma$ & $r_0$ [cm] & $r_{\rm ph}$ [cm] & $r_{\rm nt}$ [cm] \\
\hline \hline\\ 

-0.10-0.73 & 400.64$\pm$15.88 & 1.17E+08$\pm$1.85E+07 & 2.69E+11$\pm$3.20E+10 & 2.33E+12$\pm$5.75E+12 \\ 

0.73-3.80 & 264.35$\pm$3.26 & 3.92E+08$\pm$1.93E+07 & 8.24E+11$\pm$3.05E+10 & 1.20E+12$\pm$1.98E+11 \\ 

3.80-4.44 & 354.58$\pm$12.77 & 1.17E+08$\pm$1.68E+07 & 6.12E+11$\pm$6.61E+10 & 4.23E+12$\pm$1.30E+12 \\ 

4.44-5.64 & 320.98$\pm$5.56 & 2.48E+08$\pm$1.72E+07 & 9.84E+11$\pm$5.11E+10 & 3.26E+12$\pm$5.28E+11 \\ 

5.64-6.78 & 277.61$\pm$6.10 & 2.87E+08$\pm$2.52E+07 & 1.03E+12$\pm$6.82E+10 & 3.41E+12$\pm$6.29E+11 \\ 

6.78-7.46 & 248.55$\pm$8.22 & 4.11E+08$\pm$5.43E+07 & 8.87E+11$\pm$8.80E+10 & 4.66E+12$\pm$1.46E+12 \\ 

7.46-8.07 & 193.61$\pm$11.24 & 2.52E+08$\pm$5.84E+07 & 1.09E+12$\pm$1.89E+11 & 1.18E+12$\pm$4.97E+11 \\ 

8.07-10.02 & 190.83$\pm$10.93 & 1.90E+08$\pm$4.36E+07 & 5.90E+11$\pm$1.01E+11 & 5.63E+12$\pm$2.44E+12 \\ 

10.02-12.48 & 158.81$\pm$13.46 & 1.17E+08$\pm$3.97E+07 & 7.34E+11$\pm$1.87E+11 & 2.22E+12$\pm$2.59E+12 \\ 

12.48-13.88 & 144.14$\pm$35.52 & 3.56E+07$\pm$3.51E+07 & 9.33E+11$\pm$6.90E+11 & 3.97E+12$\pm$9.54E+12 \\ 

13.88-16.20 & 122.27$\pm$390.16 & 1.75E+06$\pm$2.24E+07 & 3.78E+11$\pm$3.61E+12 & 4.52E+11$\pm$8.68E+12 \\ 

16.20-29.99 & 83.29$\pm$18.27 & 3.09E+08$\pm$2.71E+08 & 2.78E+11$\pm$1.83E+11 & 6.15E+11$\pm$2.37E+12 \\ 

\end{tabular}
\caption{Inferred jet paramerters for GRB090719A assuming Y=1.}
\end{table}

\begin{table}[htp]
\scriptsize
\label{tab:}
\begin{tabular}{c c c c c}
Time [s] & $\Gamma$ & $r_0$ [cm] & $r_{\rm ph}$ [cm] & $r_{\rm nt}$ [cm] \\
\hline \hline\\ 

-0.20-0.20 & 732.31$\pm$66.92 & 2.20E+07$\pm$8.03E+06 & 1.12E+11$\pm$3.06E+10 & 7.65E+12$\pm$7.33E+12 \\ 

0.20-0.80 & 554.26$\pm$10.45 & 1.24E+08$\pm$9.35E+06 & 4.93E+11$\pm$2.79E+10 & 1.92E+12$\pm$7.36E+11 \\ 

0.80-2.40 & 452.97$\pm$3.68 & 3.01E+08$\pm$9.80E+06 & 1.01E+12$\pm$2.46E+10 & 1.85E+12$\pm$3.53E+11 \\ 

2.40-3.00 & 354.13$\pm$5.62 & 3.54E+08$\pm$2.24E+07 & 1.31E+12$\pm$6.25E+10 & 1.15E+12$\pm$3.82E+11 \\ 

3.00-4.30 & 267.18$\pm$3.33 & 6.85E+08$\pm$3.41E+07 & 1.57E+12$\pm$5.86E+10 & 1.10E+12$\pm$2.58E+11 \\ 

4.30-5.70 & 221.07$\pm$3.74 & 6.43E+08$\pm$4.35E+07 & 1.61E+12$\pm$8.20E+10 & 1.10E+12$\pm$2.51E+11 \\ 

5.70-7.20 & 212.52$\pm$6.15 & 4.32E+08$\pm$5.00E+07 & 1.02E+12$\pm$8.87E+10 & 1.89E+12$\pm$5.76E+11 \\ 

7.20-12.80 & 188.15$\pm$3.94 & 3.65E+08$\pm$3.06E+07 & 9.51E+11$\pm$5.98E+10 & 1.59E+12$\pm$3.09E+11 \\ 

12.80-22.20 & 114.22$\pm$5.18 & 2.03E+08$\pm$3.67E+07 & 1.47E+12$\pm$2.00E+11 & 3.76E+11$\pm$1.21E+11 \\ 

\end{tabular}
\caption{Inferred jet paramerters for GRB100707A assuming Y=1.}
\end{table}

\begin{table}[htp]
\scriptsize
\label{tab:}
\begin{tabular}{c c c c c}
Time [s] & $\Gamma$ & $r_0$ [cm] & $r_{\rm ph}$ [cm] & $r_{\rm nt}$ [cm] \\
\hline \hline\\ 

-0.07-0.08 & 426.02$\pm$31.04 & 1.68E+06$\pm$1.05E+06 & 1.27E+12$\pm$2.75E+11 & 1.31E+12$\pm$6.04E+11 \\ 

0.08-0.48 & 307.56$\pm$10.57 & 2.05E+07$\pm$5.80E+06 & 1.32E+12$\pm$1.32E+11 & 1.79E+12$\pm$4.28E+11 \\ 

0.48-1.28 & 198.97$\pm$12.19 & 3.38E+07$\pm$2.07E+07 & 2.48E+12$\pm$4.48E+11 & 8.36E+11$\pm$3.50E+11 \\ 

1.28-2.78 & 149.77$\pm$6.30 & 7.14E+07$\pm$2.57E+07 & 3.53E+12$\pm$4.29E+11 & 1.28E+11$\pm$4.03E+10 \\ 

2.78-3.78 & 119.28$\pm$17.78 & 2.15E+07$\pm$2.40E+07 & 3.41E+12$\pm$1.49E+12 & 4.33E+10$\pm$4.16E+10 \\ 

3.78-5.88 & 128.23$\pm$54.28 & 1.72E+06$\pm$5.55E+06 & 1.55E+12$\pm$1.97E+12 & 9.28E+10$\pm$2.37E+11 \\ 

\end{tabular}
\caption{Inferred jet paramerters for GRB110721A assuming Y=1.}
\end{table}

\begin{table}[htp]
\scriptsize
\label{tab:}
\begin{tabular}{c c c c c}
Time [s] & $\Gamma$ & $r_0$ [cm] & $r_{\rm ph}$ [cm] & $r_{\rm nt}$ [cm] \\
\hline \hline\\ 

-1.60-1.10 & 481.45$\pm$140.46 & 2.98E+06$\pm$6.64E+06 & 5.25E+10$\pm$4.38E+10 & 2.38E+12$\pm$4.72E+12 \\ 

1.10-4.90 & 392.28$\pm$20.30 & 6.01E+07$\pm$2.40E+07 & 1.71E+11$\pm$2.22E+10 & 2.66E+12$\pm$1.53E+12 \\ 

4.90-6.90 & 373.29$\pm$15.28 & 1.31E+08$\pm$4.08E+07 & 2.71E+11$\pm$2.74E+10 & 2.89E+12$\pm$1.53E+12 \\ 

6.90-16.40 & 356.03$\pm$5.46 & 1.30E+08$\pm$1.40E+07 & 3.86E+11$\pm$1.43E+10 & 1.23E+12$\pm$2.04E+11 \\ 

16.40-20.20 & 314.55$\pm$7.71 & 2.11E+08$\pm$3.77E+07 & 3.86E+11$\pm$2.33E+10 & 1.30E+12$\pm$4.01E+11 \\ 

20.20-28.70 & 292.98$\pm$5.62 & 1.73E+08$\pm$2.41E+07 & 3.87E+11$\pm$1.89E+10 & 9.87E+11$\pm$2.06E+11 \\ 

28.70-37.70 & 252.53$\pm$4.96 & 3.37E+08$\pm$5.51E+07 & 3.61E+11$\pm$1.90E+10 & 1.57E+12$\pm$4.82E+11 \\ 

37.70-47.60 & 221.82$\pm$4.68 & 3.73E+08$\pm$6.11E+07 & 3.89E+11$\pm$2.21E+10 & 8.81E+11$\pm$2.60E+11 \\ 

47.60-58.50 & 203.47$\pm$4.69 & 4.65E+08$\pm$1.05E+08 & 3.47E+11$\pm$2.25E+10 & 1.69E+12$\pm$7.47E+11 \\ 

58.50-83.40 & 162.83$\pm$2.81 & 6.82E+08$\pm$1.05E+08 & 4.08E+11$\pm$1.98E+10 & 7.30E+11$\pm$2.33E+11 \\ 

83.40-105.50 & 132.53$\pm$2.73 & 1.25E+09$\pm$3.00E+08 & 4.56E+11$\pm$2.70E+10 & 8.62E+11$\pm$6.18E+11 \\ 

105.50-122.20 & 116.65$\pm$3.94 & 1.13E+09$\pm$3.66E+08 & 4.51E+11$\pm$4.40E+10 & 4.04E+11$\pm$2.84E+11 \\ 

122.20-161.00 & 95.64$\pm$2.29 & 1.69E+09$\pm$2.89E+08 & 5.12E+11$\pm$3.41E+10 & 7.86E+10$\pm$3.94E+10 \\ 

161.00-182.60 & 88.36$\pm$5.82 & 9.75E+08$\pm$4.96E+08 & 3.88E+11$\pm$7.49E+10 & 1.17E+11$\pm$9.92E+10 \\ 

182.60-236.70 & 78.67$\pm$4.66 & 1.38E+09$\pm$1.13E+09 & 3.14E+11$\pm$5.51E+10 & 3.29E+11$\pm$6.72E+11 \\ 

\end{tabular}
\caption{Inferred jet paramerters for GRB110920A assuming Y=1.}
\end{table}

\begin{table}[htp]
\scriptsize
\label{tab:}
\begin{tabular}{c c c c c}
Time [s] & $\Gamma$ & $r_0$ [cm] & $r_{\rm ph}$ [cm] & $r_{\rm nt}$ [cm] \\
\hline \hline\\ 

-0.16-0.28 & 1020.05$\pm$165.51 & 2.78E+05$\pm$3.20E+05 & 1.95E+11$\pm$9.09E+10 & 3.64E+13$\pm$4.49E+13 \\ 

0.28-0.90 & 905.23$\pm$45.84 & 2.51E+06$\pm$9.49E+05 & 4.11E+11$\pm$5.25E+10 & 5.35E+13$\pm$4.23E+13 \\ 

0.90-1.91 & 779.64$\pm$27.12 & 3.89E+06$\pm$9.78E+05 & 6.37E+11$\pm$5.58E+10 & 2.69E+13$\pm$1.31E+13 \\ 

1.91-4.13 & 643.93$\pm$16.71 & 3.47E+06$\pm$6.74E+05 & 8.90E+11$\pm$6.43E+10 & 3.46E+13$\pm$7.43E+12 \\ 

4.13-6.46 & 487.90$\pm$31.61 & 1.54E+06$\pm$7.45E+05 & 7.96E+11$\pm$1.52E+11 & 2.26E+13$\pm$9.60E+12 \\ 

6.46-7.46 & 464.11$\pm$136.11 & 2.94E+05$\pm$6.23E+05 & 5.48E+11$\pm$4.81E+11 & 2.49E+13$\pm$4.45E+13 \\ 

7.46-10.77 & 367.77$\pm$53.86 & 1.01E+06$\pm$1.16E+06 & 5.23E+11$\pm$2.28E+11 & 1.66E+13$\pm$1.53E+13 \\ 

10.77-12.50 & 344.95$\pm$383.50 & 6.65E+04$\pm$5.41E+05 & 3.52E+11$\pm$1.17E+12 & 1.23E+13$\pm$8.20E+13 \\ 

12.50-18.20 & 144.86$\pm$29.70 & 2.93E+06$\pm$5.11E+06 & 2.53E+12$\pm$1.55E+12 & 7.86E+10$\pm$1.02E+11 \\ 

\end{tabular}
\caption{Inferred jet paramerters for GRB081224A assuming Y=10.}
\end{table}

\begin{table}[htp]
\scriptsize
\label{tab:}
\begin{tabular}{c c c c c}
Time [s] & $\Gamma$ & $r_0$ [cm] & $r_{\rm ph}$ [cm] & $r_{\rm nt}$ [cm] \\
\hline \hline\\ 

-0.10-0.73 & 712.45$\pm$28.23 & 3.69E+06$\pm$5.86E+05 & 4.78E+11$\pm$5.68E+10 & 1.34E+13$\pm$3.31E+13 \\ 

0.73-3.80 & 470.08$\pm$5.80 & 1.24E+07$\pm$6.11E+05 & 1.47E+12$\pm$5.43E+10 & 6.90E+12$\pm$1.14E+12 \\ 

3.80-4.44 & 630.55$\pm$22.71 & 3.70E+06$\pm$5.32E+05 & 1.09E+12$\pm$1.18E+11 & 2.43E+13$\pm$7.49E+12 \\ 

4.44-5.64 & 570.80$\pm$9.89 & 7.84E+06$\pm$5.43E+05 & 1.75E+12$\pm$9.09E+10 & 1.87E+13$\pm$3.03E+12 \\ 

5.64-6.78 & 493.68$\pm$10.85 & 9.08E+06$\pm$7.98E+05 & 1.84E+12$\pm$1.21E+11 & 1.96E+13$\pm$3.62E+12 \\ 

6.78-7.46 & 441.98$\pm$14.61 & 1.30E+07$\pm$1.72E+06 & 1.58E+12$\pm$1.56E+11 & 2.68E+13$\pm$8.38E+12 \\ 

7.46-8.07 & 344.29$\pm$19.99 & 7.96E+06$\pm$1.85E+06 & 1.93E+12$\pm$3.36E+11 & 6.77E+12$\pm$2.86E+12 \\ 

8.07-10.02 & 339.35$\pm$19.44 & 6.02E+06$\pm$1.38E+06 & 1.05E+12$\pm$1.80E+11 & 3.24E+13$\pm$1.40E+13 \\ 

10.02-12.48 & 282.40$\pm$23.93 & 3.71E+06$\pm$1.26E+06 & 1.31E+12$\pm$3.32E+11 & 1.28E+13$\pm$1.49E+13 \\ 

12.48-13.88 & 256.33$\pm$63.17 & 1.13E+06$\pm$1.11E+06 & 1.66E+12$\pm$1.23E+12 & 2.28E+13$\pm$5.48E+13 \\ 

13.88-16.20 & 217.44$\pm$693.82 & 5.54E+04$\pm$7.08E+05 & 6.71E+11$\pm$6.43E+12 & 2.60E+12$\pm$4.99E+13 \\ 

16.20-29.99 & 148.11$\pm$32.48 & 9.77E+06$\pm$8.57E+06 & 4.95E+11$\pm$3.26E+11 & 3.54E+12$\pm$1.36E+13 \\ 

\end{tabular}
\caption{Inferred jet paramerters for GRB090719A assuming Y=10.}
\end{table}

\begin{table}[htp]
\scriptsize
\label{tab:}
\begin{tabular}{c c c c c}
Time [s] & $\Gamma$ & $r_0$ [cm] & $r_{\rm ph}$ [cm] & $r_{\rm nt}$ [cm] \\
\hline \hline\\ 

-0.20-0.20 & 1302.25$\pm$119.01 & 6.95E+05$\pm$2.54E+05 & 1.99E+11$\pm$5.45E+10 & 4.40E+13$\pm$4.21E+13 \\ 

0.20-0.80 & 985.63$\pm$18.58 & 3.92E+06$\pm$2.96E+05 & 8.76E+11$\pm$4.95E+10 & 1.10E+13$\pm$4.23E+12 \\ 

0.80-2.40 & 805.51$\pm$6.55 & 9.53E+06$\pm$3.10E+05 & 1.80E+12$\pm$4.38E+10 & 1.06E+13$\pm$2.03E+12 \\ 

2.40-3.00 & 629.75$\pm$9.99 & 1.12E+07$\pm$7.10E+05 & 2.34E+12$\pm$1.11E+11 & 6.60E+12$\pm$2.20E+12 \\ 

3.00-4.30 & 475.12$\pm$5.92 & 2.17E+07$\pm$1.08E+06 & 2.79E+12$\pm$1.04E+11 & 6.30E+12$\pm$1.48E+12 \\ 

4.30-5.70 & 393.13$\pm$6.65 & 2.03E+07$\pm$1.38E+06 & 2.87E+12$\pm$1.46E+11 & 6.33E+12$\pm$1.44E+12 \\ 

5.70-7.20 & 377.91$\pm$10.94 & 1.36E+07$\pm$1.58E+06 & 1.82E+12$\pm$1.58E+11 & 1.09E+13$\pm$3.31E+12 \\ 

7.20-12.80 & 334.58$\pm$7.01 & 1.16E+07$\pm$9.69E+05 & 1.69E+12$\pm$1.06E+11 & 9.15E+12$\pm$1.78E+12 \\ 

12.80-22.20 & 203.11$\pm$9.21 & 6.40E+06$\pm$1.16E+06 & 2.61E+12$\pm$3.55E+11 & 2.16E+12$\pm$6.94E+11 \\ 

\end{tabular}
\caption{Inferred jet paramerters for GRB100707A assuming Y=10.}
\end{table}

\begin{table}[htp]
\scriptsize
\label{tab:}
\begin{tabular}{c c c c c}
Time [s] & $\Gamma$ & $r_0$ [cm] & $r_{\rm ph}$ [cm] & $r_{\rm nt}$ [cm] \\
\hline \hline\\ 

-0.07-0.08 & 757.59$\pm$55.19 & 5.30E+04$\pm$3.33E+04 & 2.26E+12$\pm$4.90E+11 & 7.51E+12$\pm$3.47E+12 \\ 

0.08-0.48 & 546.92$\pm$18.80 & 6.47E+05$\pm$1.83E+05 & 2.34E+12$\pm$2.36E+11 & 1.03E+13$\pm$2.46E+12 \\ 

0.48-1.28 & 353.83$\pm$21.68 & 1.07E+06$\pm$6.55E+05 & 4.42E+12$\pm$7.96E+11 & 4.81E+12$\pm$2.01E+12 \\ 

1.28-2.78 & 266.32$\pm$11.20 & 2.26E+06$\pm$8.13E+05 & 6.28E+12$\pm$7.62E+11 & 7.35E+11$\pm$2.32E+11 \\ 

2.78-3.78 & 212.11$\pm$31.61 & 6.79E+05$\pm$7.58E+05 & 6.06E+12$\pm$2.65E+12 & 2.49E+11$\pm$2.39E+11 \\ 

3.78-5.88 & 228.03$\pm$96.52 & 5.44E+04$\pm$1.76E+05 & 2.76E+12$\pm$3.50E+12 & 5.34E+11$\pm$1.36E+12 \\ 

\end{tabular}
\caption{Inferred jet paramerters for GRB110721A assuming Y=10.}
\end{table}

\begin{table}[htp]
\scriptsize
\label{tab:}
\begin{tabular}{c c c c c}
Time [s] & $\Gamma$ & $r_0$ [cm] & $r_{\rm ph}$ [cm] & $r_{\rm nt}$ [cm] \\
\hline \hline\\ 

-1.60-1.10 & 856.15$\pm$249.78 & 9.44E+04$\pm$2.10E+05 & 9.34E+10$\pm$7.79E+10 & 1.37E+13$\pm$2.71E+13 \\ 

1.10-4.90 & 697.58$\pm$36.10 & 1.90E+06$\pm$7.60E+05 & 3.04E+11$\pm$3.95E+10 & 1.53E+13$\pm$8.81E+12 \\ 

4.90-6.90 & 663.82$\pm$27.17 & 4.16E+06$\pm$1.29E+06 & 4.82E+11$\pm$4.87E+10 & 1.66E+13$\pm$8.77E+12 \\ 

6.90-16.40 & 633.12$\pm$9.71 & 4.12E+06$\pm$4.42E+05 & 6.86E+11$\pm$2.54E+10 & 7.05E+12$\pm$1.17E+12 \\ 

16.40-20.20 & 559.36$\pm$13.71 & 6.68E+06$\pm$1.19E+06 & 6.87E+11$\pm$4.14E+10 & 7.49E+12$\pm$2.31E+12 \\ 

20.20-28.70 & 521.00$\pm$9.99 & 5.48E+06$\pm$7.62E+05 & 6.88E+11$\pm$3.37E+10 & 5.67E+12$\pm$1.19E+12 \\ 

28.70-37.70 & 449.07$\pm$8.82 & 1.07E+07$\pm$1.74E+06 & 6.42E+11$\pm$3.37E+10 & 9.05E+12$\pm$2.77E+12 \\ 

37.70-47.60 & 394.46$\pm$8.33 & 1.18E+07$\pm$1.93E+06 & 6.92E+11$\pm$3.92E+10 & 5.07E+12$\pm$1.50E+12 \\ 

47.60-58.50 & 361.83$\pm$8.33 & 1.47E+07$\pm$3.33E+06 & 6.17E+11$\pm$4.01E+10 & 9.74E+12$\pm$4.30E+12 \\ 

58.50-83.40 & 289.55$\pm$5.00 & 2.16E+07$\pm$3.31E+06 & 7.26E+11$\pm$3.52E+10 & 4.20E+12$\pm$1.34E+12 \\ 

83.40-105.50 & 235.68$\pm$4.85 & 3.96E+07$\pm$9.49E+06 & 8.10E+11$\pm$4.80E+10 & 4.96E+12$\pm$3.55E+12 \\ 

105.50-122.20 & 207.43$\pm$7.00 & 3.58E+07$\pm$1.16E+07 & 8.02E+11$\pm$7.82E+10 & 2.32E+12$\pm$1.63E+12 \\ 

122.20-161.00 & 170.08$\pm$4.08 & 5.33E+07$\pm$9.14E+06 & 9.10E+11$\pm$6.07E+10 & 4.52E+11$\pm$2.26E+11 \\ 

161.00-182.60 & 157.12$\pm$10.35 & 3.08E+07$\pm$1.57E+07 & 6.91E+11$\pm$1.33E+11 & 6.74E+11$\pm$5.70E+11 \\ 

182.60-236.70 & 139.90$\pm$8.29 & 4.35E+07$\pm$3.58E+07 & 5.59E+11$\pm$9.79E+10 & 1.89E+12$\pm$3.86E+12 \\ 

\end{tabular}
\caption{Inferred jet paramerters for GRB110920A assuming Y=10.}
\end{table}

\begin{table}[htp]
\scriptsize
\label{tab:}
\begin{tabular}{c c c c c}
Time [s] & $\Gamma$ & $r_0$ [cm] & $r_{\rm ph}$ [cm] & $r_{\rm nt}$ [cm] \\
\hline \hline\\ 

-0.16-0.28 & 1813.94$\pm$294.33 & 8.78E+03$\pm$1.01E+04 & 3.47E+11$\pm$1.62E+11 & 1.25E+14$\pm$1.55E+14 \\ 

0.28-0.90 & 1609.75$\pm$81.52 & 7.95E+04$\pm$3.00E+04 & 7.32E+11$\pm$9.34E+10 & 1.84E+14$\pm$1.46E+14 \\ 

0.90-1.91 & 1386.42$\pm$48.24 & 1.23E+05$\pm$3.09E+04 & 1.13E+12$\pm$9.93E+10 & 9.27E+13$\pm$4.51E+13 \\ 

1.91-4.13 & 1145.09$\pm$29.71 & 1.10E+05$\pm$2.13E+04 & 1.58E+12$\pm$1.14E+11 & 1.19E+14$\pm$2.56E+13 \\ 

4.13-6.46 & 867.62$\pm$56.22 & 4.87E+04$\pm$2.36E+04 & 1.42E+12$\pm$2.71E+11 & 7.77E+13$\pm$3.31E+13 \\ 

6.46-7.46 & 825.31$\pm$242.05 & 9.29E+03$\pm$1.97E+04 & 9.74E+11$\pm$8.55E+11 & 8.59E+13$\pm$1.53E+14 \\ 

7.46-10.77 & 654.00$\pm$95.77 & 3.21E+04$\pm$3.67E+04 & 9.30E+11$\pm$4.06E+11 & 5.71E+13$\pm$5.27E+13 \\ 

10.77-12.50 & 613.42$\pm$681.96 & 2.10E+03$\pm$1.71E+04 & 6.26E+11$\pm$2.09E+12 & 4.22E+13$\pm$2.82E+14 \\ 

12.50-18.20 & 257.60$\pm$52.81 & 9.28E+04$\pm$1.62E+05 & 4.50E+12$\pm$2.75E+12 & 2.71E+11$\pm$3.52E+11 \\ 

\end{tabular}
\caption{Inferred jet paramerters for GRB081224A assuming Y=100.}
\end{table}

\begin{table}[htp]
\scriptsize
\label{tab:}
\begin{tabular}{c c c c c}
Time [s] & $\Gamma$ & $r_0$ [cm] & $r_{\rm ph}$ [cm] & $r_{\rm nt}$ [cm] \\
\hline \hline\\ 

-0.10-0.73 & 1266.93$\pm$50.21 & 1.17E+05$\pm$1.85E+04 & 8.50E+11$\pm$1.01E+11 & 4.62E+13$\pm$1.14E+14 \\ 

0.73-3.80 & 835.94$\pm$10.32 & 3.92E+05$\pm$1.93E+04 & 2.61E+12$\pm$9.65E+10 & 2.38E+13$\pm$3.92E+12 \\ 

3.80-4.44 & 1121.29$\pm$40.39 & 1.17E+05$\pm$1.68E+04 & 1.93E+12$\pm$2.09E+11 & 8.37E+13$\pm$2.58E+13 \\ 

4.44-5.64 & 1015.04$\pm$17.59 & 2.48E+05$\pm$1.72E+04 & 3.11E+12$\pm$1.62E+11 & 6.45E+13$\pm$1.04E+13 \\ 

5.64-6.78 & 877.89$\pm$19.30 & 2.87E+05$\pm$2.52E+04 & 3.27E+12$\pm$2.16E+11 & 6.76E+13$\pm$1.25E+13 \\ 

6.78-7.46 & 785.97$\pm$25.98 & 4.11E+05$\pm$5.43E+04 & 2.81E+12$\pm$2.78E+11 & 9.22E+13$\pm$2.88E+13 \\ 

7.46-8.07 & 612.24$\pm$35.54 & 2.52E+05$\pm$5.84E+04 & 3.43E+12$\pm$5.98E+11 & 2.33E+13$\pm$9.84E+12 \\ 

8.07-10.02 & 603.46$\pm$34.57 & 1.90E+05$\pm$4.36E+04 & 1.86E+12$\pm$3.20E+11 & 1.11E+14$\pm$4.83E+13 \\ 

10.02-12.48 & 502.19$\pm$42.55 & 1.17E+05$\pm$3.97E+04 & 2.32E+12$\pm$5.90E+11 & 4.39E+13$\pm$5.14E+13 \\ 

12.48-13.88 & 455.82$\pm$112.34 & 3.56E+04$\pm$3.51E+04 & 2.95E+12$\pm$2.18E+12 & 7.86E+13$\pm$1.89E+14 \\ 

13.88-16.20 & 386.67$\pm$1233.80 & 1.75E+03$\pm$2.24E+04 & 1.19E+12$\pm$1.14E+13 & 8.95E+12$\pm$1.72E+14 \\ 

16.20-29.99 & 263.38$\pm$57.76 & 3.09E+05$\pm$2.71E+05 & 8.80E+11$\pm$5.79E+11 & 1.22E+13$\pm$4.70E+13 \\ 

\end{tabular}
\caption{Inferred jet paramerters for GRB090719A assuming Y=100.}
\end{table}

\begin{table}[htp]
\scriptsize
\label{tab:}
\begin{tabular}{c c c c c}
Time [s] & $\Gamma$ & $r_0$ [cm] & $r_{\rm ph}$ [cm] & $r_{\rm nt}$ [cm] \\
\hline \hline\\ 

-0.20-0.20 & 2315.76$\pm$211.63 & 2.20E+04$\pm$8.03E+03 & 3.53E+11$\pm$9.69E+10 & 1.52E+14$\pm$1.45E+14 \\ 

0.20-0.80 & 1752.72$\pm$33.03 & 1.24E+05$\pm$9.35E+03 & 1.56E+12$\pm$8.81E+10 & 3.80E+13$\pm$1.46E+13 \\ 

0.80-2.40 & 1432.43$\pm$11.65 & 3.01E+05$\pm$9.80E+03 & 3.19E+12$\pm$7.79E+10 & 3.67E+13$\pm$6.98E+12 \\ 

2.40-3.00 & 1119.87$\pm$17.76 & 3.54E+05$\pm$2.24E+04 & 4.15E+12$\pm$1.98E+11 & 2.27E+13$\pm$7.56E+12 \\ 

3.00-4.30 & 844.89$\pm$10.52 & 6.85E+05$\pm$3.41E+04 & 4.96E+12$\pm$1.85E+11 & 2.17E+13$\pm$5.11E+12 \\ 

4.30-5.70 & 699.10$\pm$11.83 & 6.43E+05$\pm$4.35E+04 & 5.11E+12$\pm$2.59E+11 & 2.18E+13$\pm$4.96E+12 \\ 

5.70-7.20 & 672.04$\pm$19.45 & 4.32E+05$\pm$5.00E+04 & 3.23E+12$\pm$2.81E+11 & 3.74E+13$\pm$1.14E+13 \\ 

7.20-12.80 & 594.98$\pm$12.47 & 3.65E+05$\pm$3.06E+04 & 3.01E+12$\pm$1.89E+11 & 3.15E+13$\pm$6.12E+12 \\ 

12.80-22.20 & 361.19$\pm$16.39 & 2.03E+05$\pm$3.67E+04 & 4.64E+12$\pm$6.32E+11 & 7.45E+12$\pm$2.39E+12 \\ 

\end{tabular}
\caption{Inferred jet paramerters for GRB100707A assuming Y=100.}
\end{table}

\begin{table}[htp]
\scriptsize
\label{tab:}
\begin{tabular}{c c c c c}
Time [s] & $\Gamma$ & $r_0$ [cm] & $r_{\rm ph}$ [cm] & $r_{\rm nt}$ [cm] \\
\hline \hline\\ 

-0.07-0.08 & 1347.20$\pm$98.15 & 1.68E+03$\pm$1.05E+03 & 4.02E+12$\pm$8.71E+11 & 2.59E+13$\pm$1.20E+13 \\ 

0.08-0.48 & 972.58$\pm$33.43 & 2.05E+04$\pm$5.80E+03 & 4.17E+12$\pm$4.19E+11 & 3.54E+13$\pm$8.48E+12 \\ 

0.48-1.28 & 629.21$\pm$38.55 & 3.38E+04$\pm$2.07E+04 & 7.86E+12$\pm$1.42E+12 & 1.66E+13$\pm$6.93E+12 \\ 

1.28-2.78 & 473.60$\pm$19.91 & 7.14E+04$\pm$2.57E+04 & 1.12E+13$\pm$1.36E+12 & 2.53E+12$\pm$7.98E+11 \\ 

2.78-3.78 & 377.18$\pm$56.21 & 2.15E+04$\pm$2.40E+04 & 1.08E+13$\pm$4.72E+12 & 8.57E+11$\pm$8.23E+11 \\ 

3.78-5.88 & 405.49$\pm$171.65 & 1.72E+03$\pm$5.55E+03 & 4.91E+12$\pm$6.23E+12 & 1.84E+12$\pm$4.70E+12 \\ 

\end{tabular}
\caption{Inferred jet paramerters for GRB110721A assuming Y=100.}
\end{table}

\begin{table}[htp]
\scriptsize
\label{tab:}
\begin{tabular}{c c c c c}
Time [s] & $\Gamma$ & $r_0$ [cm] & $r_{\rm ph}$ [cm] & $r_{\rm nt}$ [cm] \\
\hline \hline\\ 

-1.60-1.10 & 1522.47$\pm$444.18 & 2.98E+03$\pm$6.64E+03 & 1.66E+11$\pm$1.38E+11 & 4.71E+13$\pm$9.35E+13 \\ 

1.10-4.90 & 1240.48$\pm$64.20 & 6.01E+04$\pm$2.40E+04 & 5.40E+11$\pm$7.03E+10 & 5.27E+13$\pm$3.03E+13 \\ 

4.90-6.90 & 1180.45$\pm$48.32 & 1.31E+05$\pm$4.08E+04 & 8.56E+11$\pm$8.66E+10 & 5.73E+13$\pm$3.02E+13 \\ 

6.90-16.40 & 1125.87$\pm$17.26 & 1.30E+05$\pm$1.40E+04 & 1.22E+12$\pm$4.51E+10 & 2.43E+13$\pm$4.04E+12 \\ 

16.40-20.20 & 994.69$\pm$24.38 & 2.11E+05$\pm$3.77E+04 & 1.22E+12$\pm$7.37E+10 & 2.58E+13$\pm$7.95E+12 \\ 

20.20-28.70 & 926.49$\pm$17.77 & 1.73E+05$\pm$2.41E+04 & 1.22E+12$\pm$5.98E+10 & 1.95E+13$\pm$4.08E+12 \\ 

28.70-37.70 & 798.57$\pm$15.68 & 3.37E+05$\pm$5.51E+04 & 1.14E+12$\pm$5.99E+10 & 3.12E+13$\pm$9.55E+12 \\ 

37.70-47.60 & 701.46$\pm$14.81 & 3.73E+05$\pm$6.11E+04 & 1.23E+12$\pm$6.98E+10 & 1.74E+13$\pm$5.15E+12 \\ 

47.60-58.50 & 643.43$\pm$14.82 & 4.65E+05$\pm$1.05E+05 & 1.10E+12$\pm$7.13E+10 & 3.35E+13$\pm$1.48E+13 \\ 

58.50-83.40 & 514.90$\pm$8.88 & 6.82E+05$\pm$1.05E+05 & 1.29E+12$\pm$6.25E+10 & 1.45E+13$\pm$4.61E+12 \\ 

83.40-105.50 & 419.11$\pm$8.62 & 1.25E+06$\pm$3.00E+05 & 1.44E+12$\pm$8.53E+10 & 1.71E+13$\pm$1.22E+13 \\ 

105.50-122.20 & 368.87$\pm$12.45 & 1.13E+06$\pm$3.66E+05 & 1.43E+12$\pm$1.39E+11 & 8.00E+12$\pm$5.63E+12 \\ 

122.20-161.00 & 302.46$\pm$7.25 & 1.69E+06$\pm$2.89E+05 & 1.62E+12$\pm$1.08E+11 & 1.56E+12$\pm$7.79E+11 \\ 

161.00-182.60 & 279.41$\pm$18.41 & 9.75E+05$\pm$4.96E+05 & 1.23E+12$\pm$2.37E+11 & 2.32E+12$\pm$1.96E+12 \\ 

182.60-236.70 & 248.78$\pm$14.75 & 1.38E+06$\pm$1.13E+06 & 9.93E+11$\pm$1.74E+11 & 6.52E+12$\pm$1.33E+13 \\ 

\end{tabular}
\caption{Inferred jet paramerters for GRB110920A assuming Y=100.}
\end{table}


\section{Plots of Inferred Jet Parameters}
\begin{figure}[tp]
  \centering
  \cfig{14}{GRB081224A_ski_Y_1.pdf}{6}
  \caption{The time-resolved values of $\Gamma$, $r_0$, and $r_{\rm ph}$ for GRB 081224A assuming Y=1.}
  \label{fig:ski1}
\end{figure}


\begin{figure}[tp]
  \centering
  \cfig{14}{GRB090719A_ski_Y_1.pdf}{6}
  \caption{The time-resolved values of $\Gamma$, $r_0$, and $r_{\rm ph}$ for GRB 090719A assuming Y=1.}
  \label{fig:ski2}
\end{figure}

\begin{figure}[tp]
  \centering
  \cfig{14}{GRB100707A_ski_Y_1.pdf}{6}
  \caption{The time-resolved values of $\Gamma$, $r_0$, and $r_{\rm ph}$ for GRB 100707A assuming Y=1.}
  \label{fig:ski3}
\end{figure}


\begin{figure}[tp]
  \centering
  \cfig{14}{GRB110721A_ski_Y_1.pdf}{6}
  \caption{The time-resolved values of $\Gamma$, $r_0$, and $r_{\rm ph}$ for GRB 110721A assuming Y=1.}
  \label{fig:ski4}
\end{figure}


\begin{figure}[tp]
  \centering
  \cfig{14}{GRB110920A_ski_Y_1.pdf}{6}
  \caption{The time-resolved values of $\Gamma$, $r_0$, and $r_{\rm ph}$ for GRB 110920A assuming Y=1.}
  \label{fig:ski5}
\end{figure}

\newpage

\begin{figure}[tp]
  \centering
  \cfig{14}{GRB081224A_ski_Y_10.pdf}{6}
  \caption{The time-resolved values of $\Gamma$, $r_0$, and $r_{\rm ph}$ for GRB }
  \label{fig:ski1}
\end{figure}


\begin{figure}[tp]
  \centering
  \cfig{14}{GRB090719A_ski_Y_10.pdf}{6}
  \caption{The time-resolved values of $\Gamma$, $r_0$, and $r_{\rm ph}$ for GRB 090719A assuming Y=10.}
  \label{fig:ski2}
\end{figure}

\begin{figure}[tp]
  \centering
  \cfig{14}{GRB100707A_ski_Y_10.pdf}{6}
  \caption{The time-resolved values of $\Gamma$, $r_0$, and $r_{\rm ph}$ for GRB 100707A assuming Y=10.}
  \label{fig:ski3}
\end{figure}


\begin{figure}[tp]
  \centering
  \cfig{14}{GRB110721A_ski_Y_10.pdf}{6}
  \caption{The time-resolved values of $\Gamma$, $r_0$, and $r_{\rm ph}$ for GRB 110721A assuming Y=10.}
  \label{fig:ski4}
\end{figure}


\begin{figure}[tp]
  \centering
  \cfig{14}{GRB110920A_ski_Y_10.pdf}{6}
  \caption{The time-resolved values of $\Gamma$, $r_0$, and $r_{\rm ph}$ for GRB 110920A assuming Y=10.}
  \label{fig:ski5}
\end{figure}

\newpage
\begin{figure}[tp]
  \centering
  \cfig{14}{GRB081224A_ski_Y_100.pdf}{6}
  \caption{The time-resolved values of $\Gamma$, $r_0$, and $r_{\rm ph}$ for GRB 081224A assuming Y=100.}
  \label{fig:ski1}
\end{figure}


\begin{figure}[tp]
  \centering
  \cfig{14}{GRB090719A_ski_Y_100.pdf}{6}
  \caption{The time-resolved values of $\Gamma$, $r_0$, and $r_{\rm ph}$ for GRB 090719A assuming Y=100.}
  \label{fig:ski2}
\end{figure}

\begin{figure}[tp]
  \centering
  \cfig{14}{GRB100707A_ski_Y_100.pdf}{6}
  \caption{The time-resolved values of $\Gamma$, $r_0$, and $r_{\rm ph}$ for GRB 100707A assuming Y=100.}
  \label{fig:ski3}
\end{figure}


\begin{figure}[tp]
  \centering
  \cfig{14}{GRB110721A_ski_Y_100.pdf}{6}
  \caption{The time-resolved values of $\Gamma$, $r_0$, and $r_{\rm ph}$ for GRB 110721A assuming Y=100.}
  \label{fig:ski4}
\end{figure}


\begin{figure}[tp]
  \centering
  \cfig{14}{GRB110920A_ski_Y_100.pdf}{6}
  \caption{The time-resolved values of $\Gamma$, $r_0$, and $r_{\rm ph}$ for GRB 110920A assuming Y=100.}
  \label{fig:ski5}
\end{figure}

\section{Non-thermal Radii of GRB Sample}

\begin{figure}[tp]
  \centering
  \cfig{14}{GRB081224A_RNT_Y_1.pdf}{6}
  \caption{The time-resolved values of the maximum non-thermal radii (shaded region) compared to $r_{\rm ph}$ (\emph{red}) and $r_{0}$ (\emph{green}) for GRB 081224A assuming Y=1.}
  \label{fig:RNT1}
\end{figure}


\begin{figure}[tp]
  \centering
  \cfig{14}{GRB090719A_RNT_Y_1.pdf}{6}
  \caption{The time-resolved values of the maximum non-thermal radii (shaded region) compared to $r_{\rm ph}$ (\emph{red}) and $r_{0}$ (\emph{green}) for GRB 090719A assuming Y=1.}
  \label{fig:RNT2}
\end{figure}

\begin{figure}[tp]
  \centering
  \cfig{14}{GRB100707A_RNT_Y_1.pdf}{6}
  \caption{The time-resolved values of the maximum non-thermal radii (shaded region) compared to $r_{\rm ph}$ (\emph{red}) and $r_{0}$ (\emph{green}) for GRB 100707A assuming Y=1.}
  \label{fig:RNT3}
\end{figure}


\begin{figure}[tp]
  \centering
  \cfig{14}{GRB110721A_RNT_Y_1.pdf}{6}
  \caption{The time-resolved values of the maximum non-thermal radii (shaded region) compared to $r_{\rm ph}$ (\emph{red}) and $r_{0}$ (\emph{green}) for GRB 110721A assuming Y=1.}
  \label{fig:RNT4}
\end{figure}


\begin{figure}[tp]
  \centering
  \cfig{14}{GRB110920A_RNT_Y_1.pdf}{6}
  \caption{The time-resolved values of the maximum non-thermal radii (shaded region) compared to $r_{\rm ph}$ (\emph{red}) and $r_{0}$ (\emph{green}) for GRB 110920A assuming Y=1.}
  \label{fig:RNT5}
\end{figure}

\newpage

\begin{figure}[tp]
  \centering
  \cfig{14}{GRB081224A_RNT_Y_10.pdf}{6}
  \caption{The time-resolved values of the maximum non-thermal radii (shaded region) compared to $r_{\rm ph}$ (\emph{red}) and $r_{0}$ (\emph{green}) for GRB }
  \label{fig:RNT1}
\end{figure}


\begin{figure}[tp]
  \centering
  \cfig{14}{GRB090719A_RNT_Y_10.pdf}{6}
  \caption{The time-resolved values of the maximum non-thermal radii (shaded region) compared to $r_{\rm ph}$ (\emph{red}) and $r_{0}$ (\emph{green}) for GRB 090719A assuming Y=10.}
  \label{fig:RNT2}
\end{figure}

\begin{figure}[tp]
  \centering
  \cfig{14}{GRB100707A_RNT_Y_10.pdf}{6}
  \caption{The time-resolved values of the maximum non-thermal radii (shaded region) compared to $r_{\rm ph}$ (\emph{red}) and $r_{0}$ (\emph{green}) for GRB 100707A assuming Y=10.}
  \label{fig:RNT3}
\end{figure}


\begin{figure}[tp]
  \centering
  \cfig{14}{GRB110721A_RNT_Y_10.pdf}{6}
  \caption{The time-resolved values of the maximum non-thermal radii (shaded region) compared to $r_{\rm ph}$ (\emph{red}) and $r_{0}$ (\emph{green}) for GRB 110721A assuming Y=10.}
  \label{fig:RNT4}
\end{figure}


\begin{figure}[tp]
  \centering
  \cfig{14}{GRB110920A_RNT_Y_10.pdf}{6}
  \caption{The time-resolved values of the maximum non-thermal radii (shaded region) compared to $r_{\rm ph}$ (\emph{red}) and $r_{0}$ (\emph{green}) for GRB 110920A assuming Y=10.}
  \label{fig:RNT5}
\end{figure}

\newpage
\begin{figure}[tp]
  \centering
  \cfig{14}{GRB081224A_RNT_Y_100.pdf}{6}
  \caption{The time-resolved values of the maximum non-thermal radii (shaded region) compared to $r_{\rm ph}$ (\emph{red}) and $r_{0}$ (\emph{green}) for GRB 081224A assuming Y=100.}
  \label{fig:RNT1}
\end{figure}


\begin{figure}[tp]
  \centering
  \cfig{14}{GRB090719A_RNT_Y_100.pdf}{6}
  \caption{The time-resolved values of the maximum non-thermal radii (shaded region) compared to $r_{\rm ph}$ (\emph{red}) and $r_{0}$ (\emph{green}) for GRB 090719A assuming Y=100.}
  \label{fig:RNT2}
\end{figure}

\begin{figure}[tp]
  \centering
  \cfig{14}{GRB100707A_RNT_Y_100.pdf}{6}
  \caption{The time-resolved values of the maximum non-thermal radii (shaded region) compared to $r_{\rm ph}$ (\emph{red}) and $r_{0}$ (\emph{green}) for GRB 100707A assuming Y=100.}
  \label{fig:RNT3}
\end{figure}


\begin{figure}[tp]
  \centering
  \cfig{14}{GRB110721A_RNT_Y_100.pdf}{6}
  \caption{The time-resolved values of the maximum non-thermal radii (shaded region) compared to $r_{\rm ph}$ (\emph{red}) and $r_{0}$ (\emph{green}) for GRB 110721A assuming Y=100.}
  \label{fig:RNT4}
\end{figure}


\begin{figure}[tp]
  \centering
  \cfig{14}{GRB110920A_RNT_Y_100.pdf}{6}
  \caption{The time-resolved values of the maximum non-thermal radii (shaded region) compared to $r_{\rm ph}$ (\emph{red}) and $r_{0}$ (\emph{green}) for GRB 110920A assuming Y=100.}
  \label{fig:RNT5}
\end{figure}





%%% Local Variables: 
%%% mode: latex
%%% TeX-master: "../thesis"
%%% End: 

\chapter{Derivation of Non-Thermal Emission Radius}
\label{ch:rnt}

In order to estimate the value of $r_{nt}$, we can use the non-observation of a cooling break in the synchrotron spectrum. For the following derivation, I will use the notation $x = x_n 10^n$ to scale parameters to the relevant order of magnitude.
\begin{equation}
  \label{eq:epsyder}
  E_{\rm p,2}=\frac{3}{2}\frac{hqB}{m_e c}\gamma_{el}^2\frac{\Gamma}{1+z}=  B_3\gamma_{el,3}^2\frac{\Gamma_2}{1+z}\; {\rm keV}
\end{equation}
Synchrotron emission must occur within the dynamical time ($t_{\rm dyn} \approx \dover{R}{\Gamma c}$) of the jet. This places the
constraint that the synchrotron cooling time ($t_{\rm cool}$) can be
no longer than $t_{\rm dyn}$ where
\begin{equation}
  \label{eq:cooltime}
  t_{\rm cool} = \dover{6 \pi m_e c}{\sigma_{\rm T} B^2 \gamma_{el} (1+Y)}.
\end{equation}
Let, 
\begin{eqnarray}
  \label{eq:derv2}
 t_{\rm cool}=t_{\rm dyn} & &\\
  &\Leftrightarrow & \frac{6 \pi m_e c}{\sigma_{\rm T} B^2 \gamma_{el} (1+Y)} =  \frac{R}{\Gamma c}\\
&\Rightarrow & \gamma_{el}\equiv \gamma_{\rm cool}= \frac{6 \pi m_e c^2 \Gamma}{\sigma_{\rm T}B^2 R (1+Y)  }.
\end{eqnarray}
To calculate The peak energy of the electrons with this Lorentz factor, we substitute $\gamma_{\rm cool}$ into \equationref{eq:epsyder}:
\begin{equation}
  \label{eq:ecoolderv}
  E_{\rm cool,2} = 5.9 \times 10^{-1} \frac{\Gamma_2^3}{B_3^3 R_{12}^2 (1+Y)^2 (1+z)} \;\;{\rm keV}.
\end{equation}


Since we have so-called slow-cooled synchrotron emission with no
cooling break observed in the spectra, we will assume the $\vFv$ peak
is occurring due to electrons of at $\gamma_{el}=\gammaMin$. Then
using \equationref{eq:epsyder} we can solve for B,
\begin{equation}
  \label{eq:slvB}
  B_3=9.2\; E_{\rm p, 2} \frac{1+z}{\gamma_{\rm min,3}^2 \Gamma_2}\;\; {\rm G}.
\end{equation}
Since the cooling break is not observed, it must be greater than $\Ep$ and therefore using the value of B from \equationref{eq:slvB},
\begin{eqnarray}
  \label{eq:derv4}
 E_{\rm p,2} < E_{\rm cool,2} & & \\
 &\Leftrightarrow & E_{\rm p,2}  < 5.9 \times 10^{-1}\; \frac{\Gamma_2^3}{B_3^3 R_{12}^2 (1+Y)^2 (1+z)}\\
&\Rightarrow &  R_{12}^2 <  7.6\times 10^{-4} \; \frac{\gamma_{\rm min,3}^6 \Gamma_2^6}{E_{\rm p,2}^4 (1+Y)^2(1+z)^4}\\
&\Rightarrow & R_{12} <  2.6\times 10^{-2}\; \frac{\gamma_{\rm min}^3 \Gamma^3}{E_{\rm p}^2 (1+Y)(1+z)^2}\;\; {\rm cm}
\end{eqnarray}






%%% Local Variables: 
%%% mode: latex
%%% TeX-master: "../thesis"
%%% End: 

\chapter{Derivation of $L-\Ep$ Relation}
\label{ch:lep}

Assume that the $\vFv$ peak of the non-thermal spectrum is the $\Ep$ of the synchrotron which yields
\begin{eqnarray}
  \label{eq:ep1}
  \Ep \propto \Gamma B \gammaMin^2\\
L\propto \Gamma^2 B^2 \gammaMin^2.
\end{eqnarray}
If we consider that the adiabatic losses dominate over synchrotron due to the observed slow-cooling spectrum, then the evolution of $\gammaMin$ with radius is
\begin{equation}
  \label{eq:gr}
  \gammaMin \propto R^{-1}.
\end{equation}
The essential assumption to reproduce the observed $L-\Ep$ curve in the data is that the magnetic field is frozen into the flow, i.e.,
\begin{eqnarray}
  \label{eq:fluxfrz}
  BR^2\propto constant\;\Leftrightarrow\;B\propto R^{-2}.
\end{eqnarray}
Substituting \equationref{eq:gr,eq:fluxfrz} into \equationref{eq:ep1}
\begin{eqnarray}
  \label{eq:ep1}
  \Ep \propto \Gamma R^{-4}\\
L\propto \Gamma^2 R{-6}
\end{eqnarray}
and eliminating $R$ between them yields $L \propto \Gamma^{}
\Ep^{3/2}$ or
\begin{equation}
  \label{eq:Lep}
  L \propto \Ep^{3/2}
\end{equation}
as desired.


%%% Local Variables: 
%%% mode: latex
%%% TeX-master: "../thesis"
%%% End: 


%   Here comes the stuff after the main content
%   turn off appendix environment/package first:
\end{appendices}
\backmatter

%   Include my bibliography
%\bibliographystyle{unsrt}
%\begin{singlespace}
\bibliography{bib2}
%\end{singlespace}

%%% Local Variables: 
%%% TeX-master: "../thesis"
%%% End: 


\bibliographystyle{unsrt}
\begin{singlespace}
\bibliography{bib2}
\end{singlespace}


%   Include my biography
%\chapter{I am A Biography}

I still need to be written
\begin{singlespace}

\section*{I am Publications}

\begin{description}

\item Paper 1

\item Paper 2
\end{description}

\section*{I am Presentations}

\begin{description}

\item Presentation 1

\item Presentation 2

\end{description}

\end{singlespace}

% MWT - biography not included in UAH thesis


%   We're done.
\end{document}
