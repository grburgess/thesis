\chapter{Degeneracies in Physical Models}
\label{ch:degen}
The number of free parameters in the physical models implemented in
{\tt RMFIT} exceeds the number of parameters than can be constrained
by the fitting engine. This is due to two factors: spectral resolution
and degeneracies in the fit parameters. The spectral resolution of the
instrument is a fixed quantity and can not be altered; however, the
degeneracies in the fit parameters can be dealt with be reformulating
their numerical expressions. Here, degeneracies in the slow-cooled
synchrotron model will be dealt with.


The synchrotron spectrum has three characteristics that determine its
shape including the high-energy electron index, $\vFv$ peak position,
and the overall amplitude. However, their are six free parameters in
our formulation of the model: $n_0$, $\gammaMin$, $\gamth$, $B$,
$\delta$, $\epsilon$, and $\Gamma$. The main degeneracy exists in the determination of the of $\Ep$:
\begin{equation}
  \label{eq:epdeg}
  \Ep \propto \Gamma B \gammaMin^2.
\end{equation}
It is not possible to simply fit the value of $\Ep$ as is done with
the Band function because of the integration over the electron
distribution. The numerical integration is done in two steps for each
part of the electron distribution:
\begin{eqnarray}
  \label{eq:n}
  \Fv^{\rm thermal}=n_0\int_1^{\gammaMin} d\gamma\; n_e^{\rm thermal}(\gamma)\mathcal{F}\left(\dover{\mathcal{E}}{E_c(\gamma)} \right)\\
\Fv^{\rm power-law}=n_0\int_{\gammaMin}^{\infty} d\gamma\; n_e^{\rm power-law}(\gamma)\mathcal{F}\left(\dover{\mathcal{E}}{E_c(\gamma)} \right)\\
\Fv = \Fv^{\rm thermal} + \Fv^{\rm power-law}.
\end{eqnarray}
Therefore, the value of $\gammaMin$ must be specified even though
there is no way to determine its actual value. For this reason, $\Ep$
is broken into two parts:
\begin{equation}
  \label{eq:estar}
  \Ep = E_*\gammaMin^2
\end{equation}
and the value of $E_*$ is used to scale the energy of the $\vFv$ peak
during fitting. It is not possible to leave both $E_*$ and $\gammaMin$
free during the fit because both parameters scale $\Ep$ but do not
alter the shape of the spectrum independently (see \figureref{fig:scalep}). 
\begin{figure}[t]
  \centering
  \subfigure{
    \cfig{11}{gMin.pdf}{3.2}\cfig{11}{eCrit.pdf}{3.2}}

  
\caption{The slow-cooled synchrotron spectrum as a function of
  $\gammaMin$ (\emph{left}) and varying E$_*$ (\emph{right}). While
  the amplitude and $\vFv$ peak of the spectrum are altered, the
  overall shape remains the same.}
  \label{fig:scalep}
\end{figure}
In this work, $\gammaMin$ was chosen to be fixed and the value of
$E_*$ left free. The values of $E_*$ and $\gammaMin$ are not
independently physical for this reason, and only the value of $\Ep$
can be used to make inferences about physical parameters. Estimations
of $\gammaMin$ can be made from very general considerations about
shock acceleration theory. The total amount of energy dissipated by shocks into the electrons is
\begin{equation}
  \label{eq:gammamin}
  \int_{\gammaMin}^{\infty}d\gamma\; n_e(\gamma)(\gamma-1) = \xi \Gamma \epsilon_e \dover{m_p}{m_e}
\end{equation}
where $\xi$ is an unknown parameter characterizing the efficiency of
mildly-relativistic shocks. Solving this for $\gammaMin$ yields
\begin{equation}
  \label{eq:gminth}
  \gammaMin \approx \dover{\delta - 2}{\delta -1}\epsilon_c \dover{m_p}{m_e}.
\end{equation}
The maximum value of $\gammaMin\approx (m_p/m_e)\approx
1,800$. Therefore, we set $\gammaMin=900$ in the fits.

The value of $\gamth$ has to be fixed in the fits as well. Simulations
of relativistic shocks have shown that the ratio
$\gammaMin/\gamth\approx 3$ and therefore $\gamth=300$ is the value
chosen for the fits. In \sectionref{sec:epsdisc} the choosing of the highly
unconstrained value of $\epsilon$ is discussed. With these values set,
a tractable parametrization of the slow-cooled synchrotron model is
available.




%%% Local Variables: 
%%% mode: latex
%%% TeX-master: "../thesis"
%%% End: 
