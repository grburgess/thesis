\chapter{Discussion} 

\begin{chapterquote}{Bertrand Russell}
  We know very little, and yet it is astonishing that we know so much,
  and still more astonishing that so little knowledge can give us so
  much power.
\end{chapterquote}

\section{Physical Modeling}
It has been shown that using physical photon models to directly fit
GRB data is feasible and provides insights into the physical
mechanisms occurring in GRB outflow jets. Not only do they provide a
direct way to uncover the emission mechanisms that are responsible for
the observed GRB flux, additionally they provide better constraints on
the spectra and flux from the individual components. This leads to
smoother observed pulses in the flux evolution of the components. All
GRBs in the sample used here have their non-thermal emission best fit
with a slow-cooled synchrotron model. This is a significant step
forward in the ability to constrain the spectral models of GRBs
alleviating the degeneracies present in the use empirical photon
models.

\section{Interpretation}
There are three main interpretations that can be made from the work
here concerning the emission mechanisms and structure of the GRB
jet. These are that, at least for single pulse GRBs, the non-thermal
emission is that from electrons that have not fully cooled and piled
up at low energies. The magnetic field flux appears to be frozen into
the outflow, conserving its magnitude as a function of
radius. Additionally, GRBs range from having their internal energy
magnetically dominated to kinetically dominated. These inferences are
drawn from the spectral fits with physical models and the evolution of
the fluxes. Knowing the actual emission mechanism removes the
degeneracy of not knowing the physical form of $\Ep$ which then
enables the determination of actual physical parameters from the
collection of parameters derived from the fits.

With these considerations, the interpretation of the GRB emissions
that can be arrived at is a mixed magnetic and kinetic fireball
model. A fraction of the initial fireball energy is tied up in the
magnetic field of the jet which may serve to accelerate electrons to
high-energy via magnetic reconnection. The important function of this
magnetic energy is that the turbulence in the magnetic field serves to
re-energize the electrons as they radiate synchrotron emission. Some
of the energy must be in kinetic form due to the presence of a
photospheric component in the spectra. Additionally, the individual
indices of the kT-$\Ep$ relationship indicate that the GRBs have a
mixed amount of both energy content.

A model that predicts these features is the ICMART model mentioned
in \chapterref{ch:pap2}. The model relies heavily on the fact that at
least some of the initial explosion energy is carried in the magnetic
field of the jet. The amount of energy in the magnetic field is
quantified by a parameter
\begin{equation}
  \label{eq:sigM}
  \sigma_{\rm M}=\dover{F_P}{F_b}=\dover{B^2}{4 \pi \Gamma \rho c^2}=\dover{B^{\prime2}}{4 \pi \rho^{\prime}c^2}
\end{equation}
which is the ratio of Poynting flux ($F_P$) to baryon flux
($F_b$). The effect of $\sigma_{\rm}$ on the overall view of GRB
emission can be profound. It reduces the required brightness of the
photospheric component since the kinetic energy used to fuel that
component is placed into the magnetic field. The value of $\sigma_{\rm
  M}$ can change as magnetic energy is released during an ICMART
event. If we let $\sigma_{\rm M}^{\rm int}$ be the value before the
event and $\sigma_{\rm M}^{\rm end}$ be the value after the event then
the \equationref{eq:eff} for the efficiency is modified to the form:
\begin{equation}
  \label{eq:si}
  {\rm eff}=\dover{1}{1+\sigma_{\rm M}^{\rm end}}-\dover{\Gamma_{\rm m}(m_{\rm f}+m_{\rm s})}{(\Gamma_{\rm f}m_{\rm f}+\Gamma_{\rm s}m_{\rm s})(1+\sigma_{\rm M}^{\rm int})}.
\end{equation}
If $\sigma_{\rm M}^{\rm int}\ll 1$ then the efficiency can reach 90\%
or more. As mentioned in \chapterref{ch:pap2}, the resulting emission
from an ICMART event should be a two-component model with the
non-thermal emission in the form of synchrotron. Therefore, this model
is supported by the observations in this work.


\section{Conclusion}
In this dissertation I have attempted to show that the physical
modeling of GRB spectra is a significant and viable method for the
study of GRB prompt emission spectra. The physical parameters derived
from spectral fits using the slow-cooled synchrotron model have
allowed for an examination of GRB jet properties including the spatial
and temporal evolution of the outflow, the topology and magnitude of
the jet magnetic field, and the specific emission mechanisms by which
the electrons in the jet radiate. These insights are in many respects
deeper than those that have been derived from analysis performed with
the Band function because the physical shape of the spectrum
(slow-cooled synchrotron) is assumed a priori. The ambiguity of
interpreting Band function parameters to assess the emission
mechanisms of electrons in the outflow severely limits the extension of
spectral analysis to physical parameters because the phase space of
possibilities is large and degenerate with an empirical fitting
function. Not only are these ambiguities eliminated, but I have shown
that the use of physical photon models produces better constraints
when fitting multi-component models \figureref{fig:fluxComp}.

In particular, the 'line-of-death' problem can be resolved when using
physical models for two reasons. First, the physical interpretation of
the Band function $\alpha$ is most likely inaccurate as shown
in \sectionref{sec:results:bvs}. Only the fitting of a physical photon
model can correctly determine the physical origin of a photon
spectrum. Second, the confirmation of a photospheric blackbody
component below the non-thermal $\Ep$ softens the low-energy spectrum
and allows for the synchrotron photon model to be fit to the data. The
second 'line-of-death' problem corresponding to the fast-cooling
synchrotron is not solved but I have shown that the this model cannot
fit the data. This forces the conclusion that the electrons in the
outflow are being re-energized through some process that occurs on a
time scale comparable to $t_{\rm dyn}$. Such scenarios are a
slow-heating by magnetic turbulence via a second-order Fermi
process. While these scenarios have been theoretically discussed, the
observation made here require them for my results to be physically
viable.

The inference of internal GRB properties (e.g. $\Gamma$, r$_{\rm ph}$,
r$_0$) from the observations combined with the physical implications
of the $L-\Ep$ and $\Ep-kT$ correlations provide tools that will
ultimately lead to a deeper understanding of the evolution of the GRB
jet and its magnetic field. A larger sample of GRBs is required to
fully take advantage of these new methods. In addition, the study of
multi-episodic GRBs, i.e., those with complex lightcurves, to assess
whether or not they obey the correlations is required. Though more
complex physical models may be needed to fully explain their spectra.




%%% Local Variables: 
%%% mode: latex
%%% TeX-master: "../thesis"
%%% End: 
