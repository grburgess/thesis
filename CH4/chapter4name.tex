\chapter{Physical Model Fitting of GRB Spectra}
\label{ch:phys}
\begin{chapterquote}{Edwin Hubble}
  Equipped with his five senses,\\ man explores the universe around him\\
  and calls the adventure Science.
\end{chapterquote}
\section{Problems with the Band Function}
While the Band function has been successful at categorizing the
majority of detected GRB spectra, several limitations and problems
arise from the use of empirical models when fitting data. The Band
function has suggested a viable, comprehensive physical origin since its
canonical use in the fitting of GRBs. The first problem comes from the
fact that GRB spectra are fit with the forward-folding method. This
method assumes a model a priori and then convolves this model with the
detector response matrix to produce a count spectrum that is fit to the observed count
data. Therefore, the initial assumption of a photon model limits the
allowed shape parameters that can be tested because the photon model
is essentially imprinted on the data from the start. The Band function
approximates many non-thermal photon models, but it is not exact. The
curvature of the Band function around the $\vFv$ peak of the spectrum
is fixed by its spectral shape and differs from the curvature of
actual models such as synchrotron. The association of the Band
function parameters with physical models has typically been via the
low-energy $\alpha$ index. However, this association neglects the
curvature of the physical model. A comparison of Band shapes to
different physical models is shown in \figureref{fig:bandCompPhys}.
\begin{figure}[h]
  \centering
  \subfigure{\cfig{4}{bandMods.pdf}{2.95}}
  \subfigure{\cfig{4}{mods.pdf}{2.95}}
  \caption{A comparison a physical photon spectra (\emph{left}) and
    their associated Band function approximations (\emph{right})
    demonstrates the problems of using the Band function parameters to
    infer a physical emission model from observed spectra.}
  \label{fig:bandCompPhys}
\end{figure}
The varying curvature of these models
can result in the false association of Band $\alpha$ to a physical
origin depending on where the Band $\vFv$ peak falls with respect to
the curvature of the physical model. 

The danger of relating Band to physical models becomes very apparent
when comparing fast and slow-cooling synchrotron models. Slow-cooling
predicts $\alpha=-2/3$ while fast-cooling predicts $\alpha=-3/2$. If a
spectrum is fit with the Band function and an $\alpha\sim -3/2$ is
measured, the conclusion that the spectrum results from fast-cooling
synchrotron could be made in error. This is because the $\vFv$
curvature of the Band function is much narrower than that of the
fast-cooling synchrotron even when they possess the same low-energy
index. These issues imply that a new method for assessing the physical
origin of GRB spectra is required. The approach in this work is to fit
numerical physical models directly to the GRB count data to make a
direct association of the data to a model.

\section{Historical Fitting of Physical Models to Data}
The first attempt to fit GRB data with physical models was made by
\cite{Baring:2004} (hereafter, BB04). In this work numerical
evaluations of physical emissivities convolved with an electron
distribution resulted in photon models that were fit to GRB photon
spectra that had been deconvolved with the Band function (see
\figureref{fig:bb04}). While enlightening, this study suffered from
the limitation that the photon data that was fit had been deconvolved
with an empirical function first. This means that the data already
had the shape of the Band function imprinted on it. Therefore,
incorrect conclusions were drawn about the validity of these physical
photon models. Since the Band function's curvature was imprinted on
the data, the broader synchrotron curvature was unable to fit the data
without converging on non-physical values for the electron
distribution. The solution to this problem is to implement physical
models directly into the forward-folding scheme and is the main focus
of this work.

\begin{figure}[h]
  \centering
  \cfig{4}{bb04.pdf}{5.5}
  \caption{Examples of physical model fitting to deconvolved BATSE $\vFv$
    spectra from \cite{Baring:2004}.}
  \label{fig:bb04}
\end{figure}

\section{Numerical Physical Fitting in RMFIT}
For all spectral fitting in this work, the publicly-available analysis
package, {\tt RMFIT} \cite{rmfit} was used. {\tt RMFIT} is a
combination of IDL and FORTRAN code that is able to read GBM and LAT
data and fit photon models to the counts data. The graphical interface
is written in IDL and the fitting engine, {\tt MFIT}, is in
FORTRAN. {\tt MFIT} utilizes a Levenberg-Marquardt minimization
routine \cite{numrecipes} to minimize a chosen fitting statistic
(e.g. {\chisq}, log-likelihood) to optimize the spectral parameters of
a given photon model to the data. It is optimized to work with GBM
counts data and was chosen over the fitting package {\tt XSPEC}
\cite{xspec} because of its close association with the GBM data
types. {\tt RMFIT} contains several photon fit models but lacks the
physical models required for this study. Therefore, custom FORTRAN
modules had to be designed to enable the fitting of these models to
data.

\subsection{Physical Models in MFIT}
\label{sec:physmod:rmfit}
To enable the fitting of physical models in {\tt RMFIT}, the source
code of {\tt MFIT} was extended to contain the following models:
\begin{itemize}
\item slow-cooling synchrotron
\item fast-cooling synchrotron
\item inverse-compton from a mono-energetic seed source
\item synchrotron self-compton.
\end{itemize}
For all models, numeric integration was required. To add this
functionality into {\tt MFIT} the numeric integration routines of the
GNU Scientific Library (GSL) \cite{gsl} were added to the {\tt MFIT}
source code. These routines were written in C, and therefore, a
FORTRAN/C interface code was designed that allowed for the passing of
variables between GSL and {\tt MFIT}.

For each photon model, the emitting electron distribution must be numerically
evaluated. For all models except the slow and fast-cooled synchrotron model a
power-law was chosen for the electron distribution. For the
slow-cooled synchrotron model, the distribution of
\equationref{eq:elec_dist} was used, consisting of a relativistic
Maxwellian with a high-energy power-law tail. The fast-cooled electron distribution (\equationref{eq:necool}) was used for fast-cooled synchrotron. The electron
distribution is then numerically convolved with single particle
emissivity of the selected photon model. The functional form of the
photon models are then added to the list of the {\tt MFIT} photon
models. When {\tt MFIT} calls the models for parameter optimization, a
numerical evaluation of the model is made at each energy bin of the
counts data. The parameters of the photon model are compared with the
data and the process is iterated until the fitting statistic converges
to a minimum. This process is very CPU intensive compared to fitting
empirical models.

The fit parameters of each photon model are listed here:
\begin{itemize}
\item slow-cooled synchrotron
  \begin{itemize}
  \item total normalization (A)
  \item power-law normalization ($\epsilon$)
  \item thermal electron Lorentz factor ($\gamth$)
  \item power-law electron Lorentz factor ($\gammaMin$)
  \item electron spectral index ($\delta$)
    \item peak energy ($\Ep$)
  \end{itemize}
\item fast-cooled synchrotron
  \begin{itemize}
  \item total normalization (A)
    \item power-law electron Lorentz factor ($\gammaMin$)
  \item electron spectral index ($\delta$)
    \item peak energy ($\Ep$)
  \end{itemize}
\item mono-energetic inverse-Compton
  \begin{itemize}
  \item total normalization (A)
    \item power-law electron Lorentz factor ($\gammaMin$)
  \item electron spectral index ($\delta$)
    \item peak energy ($\Ep$)
  \end{itemize}
\item synchrotron self-Compton
  \begin{itemize}
  \item total normalization (A)
    \item power-law electron Lorentz factor ($\gammaMin$)
  \item electron spectral index ($\delta$)
    \item peak energy ($\Ep$)
  \end{itemize}

\end{itemize}
These physical models are very rigid and difficult to fit to the data
compared to the Band function. The fitting engine will often fail
completely when fitting models that are very different from the
data. This is the case for the inverse Compton and self-Compton
models. For this reason, they will be left out of the discussion and
we will focus on the synchrotron based models. 

An important difficulty arises when trying to fit these models to the
data. The number of free parameters greatly exceeds the number of
parameters that can be simultaneously constrained. To alleviate these
problems, the models have to be formulated such that the degeneracies
in the shape parameters do not exist. This reformulation is detailed
for the synchrotron model in \appendixref{ch:degen}.


%%% Local Variables: 
%%% mode: latex
%%% TeX-master: "../thesis"
%%% End: 
