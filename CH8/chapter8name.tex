
\chapter{Origin of the Hardness-Intensity Correlation via GRB 130427A}
\label{ch:130427A}
\begin{chapterquote}{Beastie Boys}
{Now here we go dropping science,\\ dropping it all over}
\end{chapterquote}


On April 27, 2013 GBM triggered on the brightest GRB ever
detected. The GRB was so intense that the data bus was overloaded
during the brightest portion of the burst. In addition to being
detected by {\it Fermi}, this burst was seen by several satellites
enabling the measurement of a redshift of z=0.34. This measurement
allows for the calculation of rest frame properties such as L and the
intrinsic $\Ep$.

For this analysis, only the first pulse will be studied due to its
non-overlapping, single pulse nature that falls in line with the
previous studies with physical modeling. The intensity of this pulse
allows for extremely fine time-resolved spectroscopy. In
\cite{preece:2013}, the first pulse was analyzed with both the
Band+blackbody and synchrotron+blackbody models. However, when using
the Band function, there was no significant detection of the
blackbody. This finding helps illustrate the importance of using
physical models for GRB spectral analysis.

\section{Spectral Analysis}
The intensity of this GRB allowed for the binning of the lightcurve at
high-resolution while maintaining enough counts to perform spectral
analysis. The time binning method chosen was constant time width bins
of $\Delta t=0.05$s. Each bin was fit with the Band+blackbody and
synchrotron+blackbody models as is described in \sectionref{sec:specan} (see
\figureref{fig:specEvo130427A}). It was found that when using the Band
function as the non-thermal component that the blackbody component was
not significant.
\begin{figure}[t]
  \centering
  \subfigure{\cfig{8}{Bandspectrum}{5.4}}
\subfigure{\cfig{8}{Syncspectrum}{5.4}}
\caption{Spectral evolution of GRB 130427A using the Band function
  (\emph{top}) and the synchrotron+blackbody model
  (\emph{bottom}). The time evolution for the non-thermal spectra is
  from cyan to blue and from yellow to red for the blackbody.}
  \label{fig:specEvo130427A}
\end{figure}
This difference in the two analysis models leads to a different
evolution of $\Ep$ in the two models. For the Band only model, $\Ep$
decays as a single power-law in time, $\Ep \propto t^{-0.97}$. The
$\Ep$ of the synchrotron model evolves as a broken power-law with the
break occurring before the peak of the emission. The index before and
after the break are $\Ep \propto t^{-0.37\pm0.24}$ and $\Ep \propto
t^{-1.17\pm0.04}$ respectively (see \figureref{fig:epevo130427A}).
\begin{figure}[t]
  \centering
  \cfig{8}{lightcurve}{5}
  \caption{The photon flux lightcurve of the first pulse of GRB
    130427A. The $\Ep$ evolution of the Band function (\emph{blue})
    and the synchrotron model (\emph{red}) is superimposed to
    demonstrate the hard-to-soft evolution.}
  \label{fig:lightcurve130427A}
\end{figure}
\begin{figure}[t]
  \centering
  \cfig{8}{multiRp}{5}
  \caption{The log-log evolution of the $\Ep$ evolution of the Band
    function (\emph{blue}) and the synchrotron model (\emph{red}).}
  \label{fig:epevo130427A}
\end{figure}
\section{The  $L-\Ep$ Plane and Magnetic Flux-Freezing}
\label{sec:fluxfrx}
The brightness of this burst allows for the computation of the HIC in
the rest-frame and therefore the $L-\Ep$ relation is shown in
\figureref{fig:feep}. The values before the peak flux are plotted in
red and those in the decay phase are in black. The break in the
relation between the rise phase and decay phase is very prominent. The
index of the relation if $L\propto \Ep^{1.4\pm 0.06}$. This clear
measurement warrants a deeper investigation into the origin of the
relation. Attempts to explain the relation (see \sectionref{sec:results:hec}) have been unsuccessful.  
\begin{figure}[t]
  \centering
  \cfig{8}{corRestSyncBBRest}{6.5}
  \caption{The $L-\Ep$ plane of GRB 130427A. The decay phase is
    plotted in black and the rise phase in red. The fit to the decay
    is plotted in green with the 1$\sigma$ error contour. The flux
    freezing model described in \appendixref{ch:lep} is plotted in
    blue.}
  \label{fig:feep}
\end{figure}
However, it can be shown that a natural explanation for the relation
can be derived if one considers magnetic flux-freezing. In this
scenario, the flux of the magnetic field is conserved with radius,
i.e., $BR^2\propto constant$. With this assumption it can be shown
(see \appendixref{ch:lep}) that $\Ep\propto L^{3/2}$, consistent with
the data of GRB 130427A.

The GRBs in \chapterref{ch:pap2} have HIC indices that range from
$\sim 1- 2.3$. Including the errors, most of these are consistent with
the flux-freezing value of $q=-3/2$ within 2$\sigma$. Therefore, it is
plausible that a similar mechanism is responsible for the observed
correlation across the sample. 

%%% Local Variables: 
%%% mode: latex
%%% TeX-master: "../thesis"
%%% End: 
