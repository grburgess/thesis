\chapter{Derived Jet Parameters}
\label{ch:jetParms}

\section{Jet Parameter Values}


\begin{table}[htp]
\scriptsize
\label{tab:}
\begin{tabular}{c c c c c}
Time [s] & $\Gamma$ & $r_0$ [cm] & $r_{\rm ph}$ [cm] & $r_{\rm nt}$ [cm] \\
\hline \hline\\ 

-0.16-0.28 & 573.62$\pm$93.07 & 8.78E+06$\pm$1.01E+07 & 1.10E+11$\pm$5.11E+10 & 6.33E+12$\pm$7.81E+12 \\ 

0.28-0.90 & 509.05$\pm$25.78 & 7.95E+07$\pm$3.00E+07 & 2.31E+11$\pm$2.95E+10 & 9.31E+12$\pm$7.35E+12 \\ 

0.90-1.91 & 438.43$\pm$15.25 & 1.23E+08$\pm$3.09E+07 & 3.58E+11$\pm$3.14E+10 & 4.68E+12$\pm$2.28E+12 \\ 

1.91-4.13 & 362.11$\pm$9.40 & 1.10E+08$\pm$2.13E+07 & 5.00E+11$\pm$3.62E+10 & 6.01E+12$\pm$1.29E+12 \\ 

4.13-6.46 & 274.37$\pm$17.78 & 4.87E+07$\pm$2.36E+07 & 4.48E+11$\pm$8.57E+10 & 3.92E+12$\pm$1.67E+12 \\ 

6.46-7.46 & 260.99$\pm$76.54 & 9.29E+06$\pm$1.97E+07 & 3.08E+11$\pm$2.70E+11 & 4.34E+12$\pm$7.74E+12 \\ 

7.46-10.77 & 206.81$\pm$30.29 & 3.21E+07$\pm$3.67E+07 & 2.94E+11$\pm$1.28E+11 & 2.89E+12$\pm$2.66E+12 \\ 

10.77-12.50 & 193.98$\pm$215.66 & 2.10E+06$\pm$1.71E+07 & 1.98E+11$\pm$6.60E+11 & 2.13E+12$\pm$1.43E+13 \\ 

12.50-18.20 & 81.46$\pm$16.70 & 9.28E+07$\pm$1.62E+08 & 1.42E+12$\pm$8.70E+11 & 1.37E+10$\pm$1.78E+10 \\ 

\end{tabular}
\caption{Inferred jet paramerters for GRB081224A assuming Y=1.}
\end{table}

\begin{table}[htp]
\scriptsize
\label{tab:}
\begin{tabular}{c c c c c}
Time [s] & $\Gamma$ & $r_0$ [cm] & $r_{\rm ph}$ [cm] & $r_{\rm nt}$ [cm] \\
\hline \hline\\ 

-0.10-0.73 & 400.64$\pm$15.88 & 1.17E+08$\pm$1.85E+07 & 2.69E+11$\pm$3.20E+10 & 2.33E+12$\pm$5.75E+12 \\ 

0.73-3.80 & 264.35$\pm$3.26 & 3.92E+08$\pm$1.93E+07 & 8.24E+11$\pm$3.05E+10 & 1.20E+12$\pm$1.98E+11 \\ 

3.80-4.44 & 354.58$\pm$12.77 & 1.17E+08$\pm$1.68E+07 & 6.12E+11$\pm$6.61E+10 & 4.23E+12$\pm$1.30E+12 \\ 

4.44-5.64 & 320.98$\pm$5.56 & 2.48E+08$\pm$1.72E+07 & 9.84E+11$\pm$5.11E+10 & 3.26E+12$\pm$5.28E+11 \\ 

5.64-6.78 & 277.61$\pm$6.10 & 2.87E+08$\pm$2.52E+07 & 1.03E+12$\pm$6.82E+10 & 3.41E+12$\pm$6.29E+11 \\ 

6.78-7.46 & 248.55$\pm$8.22 & 4.11E+08$\pm$5.43E+07 & 8.87E+11$\pm$8.80E+10 & 4.66E+12$\pm$1.46E+12 \\ 

7.46-8.07 & 193.61$\pm$11.24 & 2.52E+08$\pm$5.84E+07 & 1.09E+12$\pm$1.89E+11 & 1.18E+12$\pm$4.97E+11 \\ 

8.07-10.02 & 190.83$\pm$10.93 & 1.90E+08$\pm$4.36E+07 & 5.90E+11$\pm$1.01E+11 & 5.63E+12$\pm$2.44E+12 \\ 

10.02-12.48 & 158.81$\pm$13.46 & 1.17E+08$\pm$3.97E+07 & 7.34E+11$\pm$1.87E+11 & 2.22E+12$\pm$2.59E+12 \\ 

12.48-13.88 & 144.14$\pm$35.52 & 3.56E+07$\pm$3.51E+07 & 9.33E+11$\pm$6.90E+11 & 3.97E+12$\pm$9.54E+12 \\ 

13.88-16.20 & 122.27$\pm$390.16 & 1.75E+06$\pm$2.24E+07 & 3.78E+11$\pm$3.61E+12 & 4.52E+11$\pm$8.68E+12 \\ 

16.20-29.99 & 83.29$\pm$18.27 & 3.09E+08$\pm$2.71E+08 & 2.78E+11$\pm$1.83E+11 & 6.15E+11$\pm$2.37E+12 \\ 

\end{tabular}
\caption{Inferred jet paramerters for GRB090719A assuming Y=1.}
\end{table}

\begin{table}[htp]
\scriptsize
\label{tab:}
\begin{tabular}{c c c c c}
Time [s] & $\Gamma$ & $r_0$ [cm] & $r_{\rm ph}$ [cm] & $r_{\rm nt}$ [cm] \\
\hline \hline\\ 

-0.20-0.20 & 732.31$\pm$66.92 & 2.20E+07$\pm$8.03E+06 & 1.12E+11$\pm$3.06E+10 & 7.65E+12$\pm$7.33E+12 \\ 

0.20-0.80 & 554.26$\pm$10.45 & 1.24E+08$\pm$9.35E+06 & 4.93E+11$\pm$2.79E+10 & 1.92E+12$\pm$7.36E+11 \\ 

0.80-2.40 & 452.97$\pm$3.68 & 3.01E+08$\pm$9.80E+06 & 1.01E+12$\pm$2.46E+10 & 1.85E+12$\pm$3.53E+11 \\ 

2.40-3.00 & 354.13$\pm$5.62 & 3.54E+08$\pm$2.24E+07 & 1.31E+12$\pm$6.25E+10 & 1.15E+12$\pm$3.82E+11 \\ 

3.00-4.30 & 267.18$\pm$3.33 & 6.85E+08$\pm$3.41E+07 & 1.57E+12$\pm$5.86E+10 & 1.10E+12$\pm$2.58E+11 \\ 

4.30-5.70 & 221.07$\pm$3.74 & 6.43E+08$\pm$4.35E+07 & 1.61E+12$\pm$8.20E+10 & 1.10E+12$\pm$2.51E+11 \\ 

5.70-7.20 & 212.52$\pm$6.15 & 4.32E+08$\pm$5.00E+07 & 1.02E+12$\pm$8.87E+10 & 1.89E+12$\pm$5.76E+11 \\ 

7.20-12.80 & 188.15$\pm$3.94 & 3.65E+08$\pm$3.06E+07 & 9.51E+11$\pm$5.98E+10 & 1.59E+12$\pm$3.09E+11 \\ 

12.80-22.20 & 114.22$\pm$5.18 & 2.03E+08$\pm$3.67E+07 & 1.47E+12$\pm$2.00E+11 & 3.76E+11$\pm$1.21E+11 \\ 

\end{tabular}
\caption{Inferred jet paramerters for GRB100707A assuming Y=1.}
\end{table}

\begin{table}[htp]
\scriptsize
\label{tab:}
\begin{tabular}{c c c c c}
Time [s] & $\Gamma$ & $r_0$ [cm] & $r_{\rm ph}$ [cm] & $r_{\rm nt}$ [cm] \\
\hline \hline\\ 

-0.07-0.08 & 426.02$\pm$31.04 & 1.68E+06$\pm$1.05E+06 & 1.27E+12$\pm$2.75E+11 & 1.31E+12$\pm$6.04E+11 \\ 

0.08-0.48 & 307.56$\pm$10.57 & 2.05E+07$\pm$5.80E+06 & 1.32E+12$\pm$1.32E+11 & 1.79E+12$\pm$4.28E+11 \\ 

0.48-1.28 & 198.97$\pm$12.19 & 3.38E+07$\pm$2.07E+07 & 2.48E+12$\pm$4.48E+11 & 8.36E+11$\pm$3.50E+11 \\ 

1.28-2.78 & 149.77$\pm$6.30 & 7.14E+07$\pm$2.57E+07 & 3.53E+12$\pm$4.29E+11 & 1.28E+11$\pm$4.03E+10 \\ 

2.78-3.78 & 119.28$\pm$17.78 & 2.15E+07$\pm$2.40E+07 & 3.41E+12$\pm$1.49E+12 & 4.33E+10$\pm$4.16E+10 \\ 

3.78-5.88 & 128.23$\pm$54.28 & 1.72E+06$\pm$5.55E+06 & 1.55E+12$\pm$1.97E+12 & 9.28E+10$\pm$2.37E+11 \\ 

\end{tabular}
\caption{Inferred jet paramerters for GRB110721A assuming Y=1.}
\end{table}

\begin{table}[htp]
\scriptsize
\label{tab:}
\begin{tabular}{c c c c c}
Time [s] & $\Gamma$ & $r_0$ [cm] & $r_{\rm ph}$ [cm] & $r_{\rm nt}$ [cm] \\
\hline \hline\\ 

-1.60-1.10 & 481.45$\pm$140.46 & 2.98E+06$\pm$6.64E+06 & 5.25E+10$\pm$4.38E+10 & 2.38E+12$\pm$4.72E+12 \\ 

1.10-4.90 & 392.28$\pm$20.30 & 6.01E+07$\pm$2.40E+07 & 1.71E+11$\pm$2.22E+10 & 2.66E+12$\pm$1.53E+12 \\ 

4.90-6.90 & 373.29$\pm$15.28 & 1.31E+08$\pm$4.08E+07 & 2.71E+11$\pm$2.74E+10 & 2.89E+12$\pm$1.53E+12 \\ 

6.90-16.40 & 356.03$\pm$5.46 & 1.30E+08$\pm$1.40E+07 & 3.86E+11$\pm$1.43E+10 & 1.23E+12$\pm$2.04E+11 \\ 

16.40-20.20 & 314.55$\pm$7.71 & 2.11E+08$\pm$3.77E+07 & 3.86E+11$\pm$2.33E+10 & 1.30E+12$\pm$4.01E+11 \\ 

20.20-28.70 & 292.98$\pm$5.62 & 1.73E+08$\pm$2.41E+07 & 3.87E+11$\pm$1.89E+10 & 9.87E+11$\pm$2.06E+11 \\ 

28.70-37.70 & 252.53$\pm$4.96 & 3.37E+08$\pm$5.51E+07 & 3.61E+11$\pm$1.90E+10 & 1.57E+12$\pm$4.82E+11 \\ 

37.70-47.60 & 221.82$\pm$4.68 & 3.73E+08$\pm$6.11E+07 & 3.89E+11$\pm$2.21E+10 & 8.81E+11$\pm$2.60E+11 \\ 

47.60-58.50 & 203.47$\pm$4.69 & 4.65E+08$\pm$1.05E+08 & 3.47E+11$\pm$2.25E+10 & 1.69E+12$\pm$7.47E+11 \\ 

58.50-83.40 & 162.83$\pm$2.81 & 6.82E+08$\pm$1.05E+08 & 4.08E+11$\pm$1.98E+10 & 7.30E+11$\pm$2.33E+11 \\ 

83.40-105.50 & 132.53$\pm$2.73 & 1.25E+09$\pm$3.00E+08 & 4.56E+11$\pm$2.70E+10 & 8.62E+11$\pm$6.18E+11 \\ 

105.50-122.20 & 116.65$\pm$3.94 & 1.13E+09$\pm$3.66E+08 & 4.51E+11$\pm$4.40E+10 & 4.04E+11$\pm$2.84E+11 \\ 

122.20-161.00 & 95.64$\pm$2.29 & 1.69E+09$\pm$2.89E+08 & 5.12E+11$\pm$3.41E+10 & 7.86E+10$\pm$3.94E+10 \\ 

161.00-182.60 & 88.36$\pm$5.82 & 9.75E+08$\pm$4.96E+08 & 3.88E+11$\pm$7.49E+10 & 1.17E+11$\pm$9.92E+10 \\ 

182.60-236.70 & 78.67$\pm$4.66 & 1.38E+09$\pm$1.13E+09 & 3.14E+11$\pm$5.51E+10 & 3.29E+11$\pm$6.72E+11 \\ 

\end{tabular}
\caption{Inferred jet paramerters for GRB110920A assuming Y=1.}
\end{table}

\begin{table}[htp]
\scriptsize
\label{tab:}
\begin{tabular}{c c c c c}
Time [s] & $\Gamma$ & $r_0$ [cm] & $r_{\rm ph}$ [cm] & $r_{\rm nt}$ [cm] \\
\hline \hline\\ 

-0.16-0.28 & 1020.05$\pm$165.51 & 2.78E+05$\pm$3.20E+05 & 1.95E+11$\pm$9.09E+10 & 3.64E+13$\pm$4.49E+13 \\ 

0.28-0.90 & 905.23$\pm$45.84 & 2.51E+06$\pm$9.49E+05 & 4.11E+11$\pm$5.25E+10 & 5.35E+13$\pm$4.23E+13 \\ 

0.90-1.91 & 779.64$\pm$27.12 & 3.89E+06$\pm$9.78E+05 & 6.37E+11$\pm$5.58E+10 & 2.69E+13$\pm$1.31E+13 \\ 

1.91-4.13 & 643.93$\pm$16.71 & 3.47E+06$\pm$6.74E+05 & 8.90E+11$\pm$6.43E+10 & 3.46E+13$\pm$7.43E+12 \\ 

4.13-6.46 & 487.90$\pm$31.61 & 1.54E+06$\pm$7.45E+05 & 7.96E+11$\pm$1.52E+11 & 2.26E+13$\pm$9.60E+12 \\ 

6.46-7.46 & 464.11$\pm$136.11 & 2.94E+05$\pm$6.23E+05 & 5.48E+11$\pm$4.81E+11 & 2.49E+13$\pm$4.45E+13 \\ 

7.46-10.77 & 367.77$\pm$53.86 & 1.01E+06$\pm$1.16E+06 & 5.23E+11$\pm$2.28E+11 & 1.66E+13$\pm$1.53E+13 \\ 

10.77-12.50 & 344.95$\pm$383.50 & 6.65E+04$\pm$5.41E+05 & 3.52E+11$\pm$1.17E+12 & 1.23E+13$\pm$8.20E+13 \\ 

12.50-18.20 & 144.86$\pm$29.70 & 2.93E+06$\pm$5.11E+06 & 2.53E+12$\pm$1.55E+12 & 7.86E+10$\pm$1.02E+11 \\ 

\end{tabular}
\caption{Inferred jet paramerters for GRB081224A assuming Y=10.}
\end{table}

\begin{table}[htp]
\scriptsize
\label{tab:}
\begin{tabular}{c c c c c}
Time [s] & $\Gamma$ & $r_0$ [cm] & $r_{\rm ph}$ [cm] & $r_{\rm nt}$ [cm] \\
\hline \hline\\ 

-0.10-0.73 & 712.45$\pm$28.23 & 3.69E+06$\pm$5.86E+05 & 4.78E+11$\pm$5.68E+10 & 1.34E+13$\pm$3.31E+13 \\ 

0.73-3.80 & 470.08$\pm$5.80 & 1.24E+07$\pm$6.11E+05 & 1.47E+12$\pm$5.43E+10 & 6.90E+12$\pm$1.14E+12 \\ 

3.80-4.44 & 630.55$\pm$22.71 & 3.70E+06$\pm$5.32E+05 & 1.09E+12$\pm$1.18E+11 & 2.43E+13$\pm$7.49E+12 \\ 

4.44-5.64 & 570.80$\pm$9.89 & 7.84E+06$\pm$5.43E+05 & 1.75E+12$\pm$9.09E+10 & 1.87E+13$\pm$3.03E+12 \\ 

5.64-6.78 & 493.68$\pm$10.85 & 9.08E+06$\pm$7.98E+05 & 1.84E+12$\pm$1.21E+11 & 1.96E+13$\pm$3.62E+12 \\ 

6.78-7.46 & 441.98$\pm$14.61 & 1.30E+07$\pm$1.72E+06 & 1.58E+12$\pm$1.56E+11 & 2.68E+13$\pm$8.38E+12 \\ 

7.46-8.07 & 344.29$\pm$19.99 & 7.96E+06$\pm$1.85E+06 & 1.93E+12$\pm$3.36E+11 & 6.77E+12$\pm$2.86E+12 \\ 

8.07-10.02 & 339.35$\pm$19.44 & 6.02E+06$\pm$1.38E+06 & 1.05E+12$\pm$1.80E+11 & 3.24E+13$\pm$1.40E+13 \\ 

10.02-12.48 & 282.40$\pm$23.93 & 3.71E+06$\pm$1.26E+06 & 1.31E+12$\pm$3.32E+11 & 1.28E+13$\pm$1.49E+13 \\ 

12.48-13.88 & 256.33$\pm$63.17 & 1.13E+06$\pm$1.11E+06 & 1.66E+12$\pm$1.23E+12 & 2.28E+13$\pm$5.48E+13 \\ 

13.88-16.20 & 217.44$\pm$693.82 & 5.54E+04$\pm$7.08E+05 & 6.71E+11$\pm$6.43E+12 & 2.60E+12$\pm$4.99E+13 \\ 

16.20-29.99 & 148.11$\pm$32.48 & 9.77E+06$\pm$8.57E+06 & 4.95E+11$\pm$3.26E+11 & 3.54E+12$\pm$1.36E+13 \\ 

\end{tabular}
\caption{Inferred jet paramerters for GRB090719A assuming Y=10.}
\end{table}

\begin{table}[htp]
\scriptsize
\label{tab:}
\begin{tabular}{c c c c c}
Time [s] & $\Gamma$ & $r_0$ [cm] & $r_{\rm ph}$ [cm] & $r_{\rm nt}$ [cm] \\
\hline \hline\\ 

-0.20-0.20 & 1302.25$\pm$119.01 & 6.95E+05$\pm$2.54E+05 & 1.99E+11$\pm$5.45E+10 & 4.40E+13$\pm$4.21E+13 \\ 

0.20-0.80 & 985.63$\pm$18.58 & 3.92E+06$\pm$2.96E+05 & 8.76E+11$\pm$4.95E+10 & 1.10E+13$\pm$4.23E+12 \\ 

0.80-2.40 & 805.51$\pm$6.55 & 9.53E+06$\pm$3.10E+05 & 1.80E+12$\pm$4.38E+10 & 1.06E+13$\pm$2.03E+12 \\ 

2.40-3.00 & 629.75$\pm$9.99 & 1.12E+07$\pm$7.10E+05 & 2.34E+12$\pm$1.11E+11 & 6.60E+12$\pm$2.20E+12 \\ 

3.00-4.30 & 475.12$\pm$5.92 & 2.17E+07$\pm$1.08E+06 & 2.79E+12$\pm$1.04E+11 & 6.30E+12$\pm$1.48E+12 \\ 

4.30-5.70 & 393.13$\pm$6.65 & 2.03E+07$\pm$1.38E+06 & 2.87E+12$\pm$1.46E+11 & 6.33E+12$\pm$1.44E+12 \\ 

5.70-7.20 & 377.91$\pm$10.94 & 1.36E+07$\pm$1.58E+06 & 1.82E+12$\pm$1.58E+11 & 1.09E+13$\pm$3.31E+12 \\ 

7.20-12.80 & 334.58$\pm$7.01 & 1.16E+07$\pm$9.69E+05 & 1.69E+12$\pm$1.06E+11 & 9.15E+12$\pm$1.78E+12 \\ 

12.80-22.20 & 203.11$\pm$9.21 & 6.40E+06$\pm$1.16E+06 & 2.61E+12$\pm$3.55E+11 & 2.16E+12$\pm$6.94E+11 \\ 

\end{tabular}
\caption{Inferred jet paramerters for GRB100707A assuming Y=10.}
\end{table}

\begin{table}[htp]
\scriptsize
\label{tab:}
\begin{tabular}{c c c c c}
Time [s] & $\Gamma$ & $r_0$ [cm] & $r_{\rm ph}$ [cm] & $r_{\rm nt}$ [cm] \\
\hline \hline\\ 

-0.07-0.08 & 757.59$\pm$55.19 & 5.30E+04$\pm$3.33E+04 & 2.26E+12$\pm$4.90E+11 & 7.51E+12$\pm$3.47E+12 \\ 

0.08-0.48 & 546.92$\pm$18.80 & 6.47E+05$\pm$1.83E+05 & 2.34E+12$\pm$2.36E+11 & 1.03E+13$\pm$2.46E+12 \\ 

0.48-1.28 & 353.83$\pm$21.68 & 1.07E+06$\pm$6.55E+05 & 4.42E+12$\pm$7.96E+11 & 4.81E+12$\pm$2.01E+12 \\ 

1.28-2.78 & 266.32$\pm$11.20 & 2.26E+06$\pm$8.13E+05 & 6.28E+12$\pm$7.62E+11 & 7.35E+11$\pm$2.32E+11 \\ 

2.78-3.78 & 212.11$\pm$31.61 & 6.79E+05$\pm$7.58E+05 & 6.06E+12$\pm$2.65E+12 & 2.49E+11$\pm$2.39E+11 \\ 

3.78-5.88 & 228.03$\pm$96.52 & 5.44E+04$\pm$1.76E+05 & 2.76E+12$\pm$3.50E+12 & 5.34E+11$\pm$1.36E+12 \\ 

\end{tabular}
\caption{Inferred jet paramerters for GRB110721A assuming Y=10.}
\end{table}

\begin{table}[htp]
\scriptsize
\label{tab:}
\begin{tabular}{c c c c c}
Time [s] & $\Gamma$ & $r_0$ [cm] & $r_{\rm ph}$ [cm] & $r_{\rm nt}$ [cm] \\
\hline \hline\\ 

-1.60-1.10 & 856.15$\pm$249.78 & 9.44E+04$\pm$2.10E+05 & 9.34E+10$\pm$7.79E+10 & 1.37E+13$\pm$2.71E+13 \\ 

1.10-4.90 & 697.58$\pm$36.10 & 1.90E+06$\pm$7.60E+05 & 3.04E+11$\pm$3.95E+10 & 1.53E+13$\pm$8.81E+12 \\ 

4.90-6.90 & 663.82$\pm$27.17 & 4.16E+06$\pm$1.29E+06 & 4.82E+11$\pm$4.87E+10 & 1.66E+13$\pm$8.77E+12 \\ 

6.90-16.40 & 633.12$\pm$9.71 & 4.12E+06$\pm$4.42E+05 & 6.86E+11$\pm$2.54E+10 & 7.05E+12$\pm$1.17E+12 \\ 

16.40-20.20 & 559.36$\pm$13.71 & 6.68E+06$\pm$1.19E+06 & 6.87E+11$\pm$4.14E+10 & 7.49E+12$\pm$2.31E+12 \\ 

20.20-28.70 & 521.00$\pm$9.99 & 5.48E+06$\pm$7.62E+05 & 6.88E+11$\pm$3.37E+10 & 5.67E+12$\pm$1.19E+12 \\ 

28.70-37.70 & 449.07$\pm$8.82 & 1.07E+07$\pm$1.74E+06 & 6.42E+11$\pm$3.37E+10 & 9.05E+12$\pm$2.77E+12 \\ 

37.70-47.60 & 394.46$\pm$8.33 & 1.18E+07$\pm$1.93E+06 & 6.92E+11$\pm$3.92E+10 & 5.07E+12$\pm$1.50E+12 \\ 

47.60-58.50 & 361.83$\pm$8.33 & 1.47E+07$\pm$3.33E+06 & 6.17E+11$\pm$4.01E+10 & 9.74E+12$\pm$4.30E+12 \\ 

58.50-83.40 & 289.55$\pm$5.00 & 2.16E+07$\pm$3.31E+06 & 7.26E+11$\pm$3.52E+10 & 4.20E+12$\pm$1.34E+12 \\ 

83.40-105.50 & 235.68$\pm$4.85 & 3.96E+07$\pm$9.49E+06 & 8.10E+11$\pm$4.80E+10 & 4.96E+12$\pm$3.55E+12 \\ 

105.50-122.20 & 207.43$\pm$7.00 & 3.58E+07$\pm$1.16E+07 & 8.02E+11$\pm$7.82E+10 & 2.32E+12$\pm$1.63E+12 \\ 

122.20-161.00 & 170.08$\pm$4.08 & 5.33E+07$\pm$9.14E+06 & 9.10E+11$\pm$6.07E+10 & 4.52E+11$\pm$2.26E+11 \\ 

161.00-182.60 & 157.12$\pm$10.35 & 3.08E+07$\pm$1.57E+07 & 6.91E+11$\pm$1.33E+11 & 6.74E+11$\pm$5.70E+11 \\ 

182.60-236.70 & 139.90$\pm$8.29 & 4.35E+07$\pm$3.58E+07 & 5.59E+11$\pm$9.79E+10 & 1.89E+12$\pm$3.86E+12 \\ 

\end{tabular}
\caption{Inferred jet paramerters for GRB110920A assuming Y=10.}
\end{table}

\begin{table}[htp]
\scriptsize
\label{tab:}
\begin{tabular}{c c c c c}
Time [s] & $\Gamma$ & $r_0$ [cm] & $r_{\rm ph}$ [cm] & $r_{\rm nt}$ [cm] \\
\hline \hline\\ 

-0.16-0.28 & 1813.94$\pm$294.33 & 8.78E+03$\pm$1.01E+04 & 3.47E+11$\pm$1.62E+11 & 1.25E+14$\pm$1.55E+14 \\ 

0.28-0.90 & 1609.75$\pm$81.52 & 7.95E+04$\pm$3.00E+04 & 7.32E+11$\pm$9.34E+10 & 1.84E+14$\pm$1.46E+14 \\ 

0.90-1.91 & 1386.42$\pm$48.24 & 1.23E+05$\pm$3.09E+04 & 1.13E+12$\pm$9.93E+10 & 9.27E+13$\pm$4.51E+13 \\ 

1.91-4.13 & 1145.09$\pm$29.71 & 1.10E+05$\pm$2.13E+04 & 1.58E+12$\pm$1.14E+11 & 1.19E+14$\pm$2.56E+13 \\ 

4.13-6.46 & 867.62$\pm$56.22 & 4.87E+04$\pm$2.36E+04 & 1.42E+12$\pm$2.71E+11 & 7.77E+13$\pm$3.31E+13 \\ 

6.46-7.46 & 825.31$\pm$242.05 & 9.29E+03$\pm$1.97E+04 & 9.74E+11$\pm$8.55E+11 & 8.59E+13$\pm$1.53E+14 \\ 

7.46-10.77 & 654.00$\pm$95.77 & 3.21E+04$\pm$3.67E+04 & 9.30E+11$\pm$4.06E+11 & 5.71E+13$\pm$5.27E+13 \\ 

10.77-12.50 & 613.42$\pm$681.96 & 2.10E+03$\pm$1.71E+04 & 6.26E+11$\pm$2.09E+12 & 4.22E+13$\pm$2.82E+14 \\ 

12.50-18.20 & 257.60$\pm$52.81 & 9.28E+04$\pm$1.62E+05 & 4.50E+12$\pm$2.75E+12 & 2.71E+11$\pm$3.52E+11 \\ 

\end{tabular}
\caption{Inferred jet paramerters for GRB081224A assuming Y=100.}
\end{table}

\begin{table}[htp]
\scriptsize
\label{tab:}
\begin{tabular}{c c c c c}
Time [s] & $\Gamma$ & $r_0$ [cm] & $r_{\rm ph}$ [cm] & $r_{\rm nt}$ [cm] \\
\hline \hline\\ 

-0.10-0.73 & 1266.93$\pm$50.21 & 1.17E+05$\pm$1.85E+04 & 8.50E+11$\pm$1.01E+11 & 4.62E+13$\pm$1.14E+14 \\ 

0.73-3.80 & 835.94$\pm$10.32 & 3.92E+05$\pm$1.93E+04 & 2.61E+12$\pm$9.65E+10 & 2.38E+13$\pm$3.92E+12 \\ 

3.80-4.44 & 1121.29$\pm$40.39 & 1.17E+05$\pm$1.68E+04 & 1.93E+12$\pm$2.09E+11 & 8.37E+13$\pm$2.58E+13 \\ 

4.44-5.64 & 1015.04$\pm$17.59 & 2.48E+05$\pm$1.72E+04 & 3.11E+12$\pm$1.62E+11 & 6.45E+13$\pm$1.04E+13 \\ 

5.64-6.78 & 877.89$\pm$19.30 & 2.87E+05$\pm$2.52E+04 & 3.27E+12$\pm$2.16E+11 & 6.76E+13$\pm$1.25E+13 \\ 

6.78-7.46 & 785.97$\pm$25.98 & 4.11E+05$\pm$5.43E+04 & 2.81E+12$\pm$2.78E+11 & 9.22E+13$\pm$2.88E+13 \\ 

7.46-8.07 & 612.24$\pm$35.54 & 2.52E+05$\pm$5.84E+04 & 3.43E+12$\pm$5.98E+11 & 2.33E+13$\pm$9.84E+12 \\ 

8.07-10.02 & 603.46$\pm$34.57 & 1.90E+05$\pm$4.36E+04 & 1.86E+12$\pm$3.20E+11 & 1.11E+14$\pm$4.83E+13 \\ 

10.02-12.48 & 502.19$\pm$42.55 & 1.17E+05$\pm$3.97E+04 & 2.32E+12$\pm$5.90E+11 & 4.39E+13$\pm$5.14E+13 \\ 

12.48-13.88 & 455.82$\pm$112.34 & 3.56E+04$\pm$3.51E+04 & 2.95E+12$\pm$2.18E+12 & 7.86E+13$\pm$1.89E+14 \\ 

13.88-16.20 & 386.67$\pm$1233.80 & 1.75E+03$\pm$2.24E+04 & 1.19E+12$\pm$1.14E+13 & 8.95E+12$\pm$1.72E+14 \\ 

16.20-29.99 & 263.38$\pm$57.76 & 3.09E+05$\pm$2.71E+05 & 8.80E+11$\pm$5.79E+11 & 1.22E+13$\pm$4.70E+13 \\ 

\end{tabular}
\caption{Inferred jet paramerters for GRB090719A assuming Y=100.}
\end{table}

\begin{table}[htp]
\scriptsize
\label{tab:}
\begin{tabular}{c c c c c}
Time [s] & $\Gamma$ & $r_0$ [cm] & $r_{\rm ph}$ [cm] & $r_{\rm nt}$ [cm] \\
\hline \hline\\ 

-0.20-0.20 & 2315.76$\pm$211.63 & 2.20E+04$\pm$8.03E+03 & 3.53E+11$\pm$9.69E+10 & 1.52E+14$\pm$1.45E+14 \\ 

0.20-0.80 & 1752.72$\pm$33.03 & 1.24E+05$\pm$9.35E+03 & 1.56E+12$\pm$8.81E+10 & 3.80E+13$\pm$1.46E+13 \\ 

0.80-2.40 & 1432.43$\pm$11.65 & 3.01E+05$\pm$9.80E+03 & 3.19E+12$\pm$7.79E+10 & 3.67E+13$\pm$6.98E+12 \\ 

2.40-3.00 & 1119.87$\pm$17.76 & 3.54E+05$\pm$2.24E+04 & 4.15E+12$\pm$1.98E+11 & 2.27E+13$\pm$7.56E+12 \\ 

3.00-4.30 & 844.89$\pm$10.52 & 6.85E+05$\pm$3.41E+04 & 4.96E+12$\pm$1.85E+11 & 2.17E+13$\pm$5.11E+12 \\ 

4.30-5.70 & 699.10$\pm$11.83 & 6.43E+05$\pm$4.35E+04 & 5.11E+12$\pm$2.59E+11 & 2.18E+13$\pm$4.96E+12 \\ 

5.70-7.20 & 672.04$\pm$19.45 & 4.32E+05$\pm$5.00E+04 & 3.23E+12$\pm$2.81E+11 & 3.74E+13$\pm$1.14E+13 \\ 

7.20-12.80 & 594.98$\pm$12.47 & 3.65E+05$\pm$3.06E+04 & 3.01E+12$\pm$1.89E+11 & 3.15E+13$\pm$6.12E+12 \\ 

12.80-22.20 & 361.19$\pm$16.39 & 2.03E+05$\pm$3.67E+04 & 4.64E+12$\pm$6.32E+11 & 7.45E+12$\pm$2.39E+12 \\ 

\end{tabular}
\caption{Inferred jet paramerters for GRB100707A assuming Y=100.}
\end{table}

\begin{table}[htp]
\scriptsize
\label{tab:}
\begin{tabular}{c c c c c}
Time [s] & $\Gamma$ & $r_0$ [cm] & $r_{\rm ph}$ [cm] & $r_{\rm nt}$ [cm] \\
\hline \hline\\ 

-0.07-0.08 & 1347.20$\pm$98.15 & 1.68E+03$\pm$1.05E+03 & 4.02E+12$\pm$8.71E+11 & 2.59E+13$\pm$1.20E+13 \\ 

0.08-0.48 & 972.58$\pm$33.43 & 2.05E+04$\pm$5.80E+03 & 4.17E+12$\pm$4.19E+11 & 3.54E+13$\pm$8.48E+12 \\ 

0.48-1.28 & 629.21$\pm$38.55 & 3.38E+04$\pm$2.07E+04 & 7.86E+12$\pm$1.42E+12 & 1.66E+13$\pm$6.93E+12 \\ 

1.28-2.78 & 473.60$\pm$19.91 & 7.14E+04$\pm$2.57E+04 & 1.12E+13$\pm$1.36E+12 & 2.53E+12$\pm$7.98E+11 \\ 

2.78-3.78 & 377.18$\pm$56.21 & 2.15E+04$\pm$2.40E+04 & 1.08E+13$\pm$4.72E+12 & 8.57E+11$\pm$8.23E+11 \\ 

3.78-5.88 & 405.49$\pm$171.65 & 1.72E+03$\pm$5.55E+03 & 4.91E+12$\pm$6.23E+12 & 1.84E+12$\pm$4.70E+12 \\ 

\end{tabular}
\caption{Inferred jet paramerters for GRB110721A assuming Y=100.}
\end{table}

\begin{table}[htp]
\scriptsize
\label{tab:}
\begin{tabular}{c c c c c}
Time [s] & $\Gamma$ & $r_0$ [cm] & $r_{\rm ph}$ [cm] & $r_{\rm nt}$ [cm] \\
\hline \hline\\ 

-1.60-1.10 & 1522.47$\pm$444.18 & 2.98E+03$\pm$6.64E+03 & 1.66E+11$\pm$1.38E+11 & 4.71E+13$\pm$9.35E+13 \\ 

1.10-4.90 & 1240.48$\pm$64.20 & 6.01E+04$\pm$2.40E+04 & 5.40E+11$\pm$7.03E+10 & 5.27E+13$\pm$3.03E+13 \\ 

4.90-6.90 & 1180.45$\pm$48.32 & 1.31E+05$\pm$4.08E+04 & 8.56E+11$\pm$8.66E+10 & 5.73E+13$\pm$3.02E+13 \\ 

6.90-16.40 & 1125.87$\pm$17.26 & 1.30E+05$\pm$1.40E+04 & 1.22E+12$\pm$4.51E+10 & 2.43E+13$\pm$4.04E+12 \\ 

16.40-20.20 & 994.69$\pm$24.38 & 2.11E+05$\pm$3.77E+04 & 1.22E+12$\pm$7.37E+10 & 2.58E+13$\pm$7.95E+12 \\ 

20.20-28.70 & 926.49$\pm$17.77 & 1.73E+05$\pm$2.41E+04 & 1.22E+12$\pm$5.98E+10 & 1.95E+13$\pm$4.08E+12 \\ 

28.70-37.70 & 798.57$\pm$15.68 & 3.37E+05$\pm$5.51E+04 & 1.14E+12$\pm$5.99E+10 & 3.12E+13$\pm$9.55E+12 \\ 

37.70-47.60 & 701.46$\pm$14.81 & 3.73E+05$\pm$6.11E+04 & 1.23E+12$\pm$6.98E+10 & 1.74E+13$\pm$5.15E+12 \\ 

47.60-58.50 & 643.43$\pm$14.82 & 4.65E+05$\pm$1.05E+05 & 1.10E+12$\pm$7.13E+10 & 3.35E+13$\pm$1.48E+13 \\ 

58.50-83.40 & 514.90$\pm$8.88 & 6.82E+05$\pm$1.05E+05 & 1.29E+12$\pm$6.25E+10 & 1.45E+13$\pm$4.61E+12 \\ 

83.40-105.50 & 419.11$\pm$8.62 & 1.25E+06$\pm$3.00E+05 & 1.44E+12$\pm$8.53E+10 & 1.71E+13$\pm$1.22E+13 \\ 

105.50-122.20 & 368.87$\pm$12.45 & 1.13E+06$\pm$3.66E+05 & 1.43E+12$\pm$1.39E+11 & 8.00E+12$\pm$5.63E+12 \\ 

122.20-161.00 & 302.46$\pm$7.25 & 1.69E+06$\pm$2.89E+05 & 1.62E+12$\pm$1.08E+11 & 1.56E+12$\pm$7.79E+11 \\ 

161.00-182.60 & 279.41$\pm$18.41 & 9.75E+05$\pm$4.96E+05 & 1.23E+12$\pm$2.37E+11 & 2.32E+12$\pm$1.96E+12 \\ 

182.60-236.70 & 248.78$\pm$14.75 & 1.38E+06$\pm$1.13E+06 & 9.93E+11$\pm$1.74E+11 & 6.52E+12$\pm$1.33E+13 \\ 

\end{tabular}
\caption{Inferred jet paramerters for GRB110920A assuming Y=100.}
\end{table}


\section{Plots of Inferred Jet Parameters}
\begin{figure}[tp]
  \centering
  \cfig{14}{GRB081224A_ski_Y_1.pdf}{6}
  \caption{The time-resolved values of $\Gamma$, $r_0$, and $r_{\rm ph}$ for GRB 081224A assuming Y=1.}
  \label{fig:ski1}
\end{figure}


\begin{figure}[tp]
  \centering
  \cfig{14}{GRB090719A_ski_Y_1.pdf}{6}
  \caption{The time-resolved values of $\Gamma$, $r_0$, and $r_{\rm ph}$ for GRB 090719A assuming Y=1.}
  \label{fig:ski2}
\end{figure}

\begin{figure}[tp]
  \centering
  \cfig{14}{GRB100707A_ski_Y_1.pdf}{6}
  \caption{The time-resolved values of $\Gamma$, $r_0$, and $r_{\rm ph}$ for GRB 100707A assuming Y=1.}
  \label{fig:ski3}
\end{figure}


\begin{figure}[tp]
  \centering
  \cfig{14}{GRB110721A_ski_Y_1.pdf}{6}
  \caption{The time-resolved values of $\Gamma$, $r_0$, and $r_{\rm ph}$ for GRB 110721A assuming Y=1.}
  \label{fig:ski4}
\end{figure}


\begin{figure}[tp]
  \centering
  \cfig{14}{GRB110920A_ski_Y_1.pdf}{6}
  \caption{The time-resolved values of $\Gamma$, $r_0$, and $r_{\rm ph}$ for GRB 110920A assuming Y=1.}
  \label{fig:ski5}
\end{figure}

\newpage

\begin{figure}[tp]
  \centering
  \cfig{14}{GRB081224A_ski_Y_10.pdf}{6}
  \caption{The time-resolved values of $\Gamma$, $r_0$, and $r_{\rm ph}$ for GRB }
  \label{fig:ski1}
\end{figure}


\begin{figure}[tp]
  \centering
  \cfig{14}{GRB090719A_ski_Y_10.pdf}{6}
  \caption{The time-resolved values of $\Gamma$, $r_0$, and $r_{\rm ph}$ for GRB 090719A assuming Y=10.}
  \label{fig:ski2}
\end{figure}

\begin{figure}[tp]
  \centering
  \cfig{14}{GRB100707A_ski_Y_10.pdf}{6}
  \caption{The time-resolved values of $\Gamma$, $r_0$, and $r_{\rm ph}$ for GRB 100707A assuming Y=10.}
  \label{fig:ski3}
\end{figure}


\begin{figure}[tp]
  \centering
  \cfig{14}{GRB110721A_ski_Y_10.pdf}{6}
  \caption{The time-resolved values of $\Gamma$, $r_0$, and $r_{\rm ph}$ for GRB 110721A assuming Y=10.}
  \label{fig:ski4}
\end{figure}


\begin{figure}[tp]
  \centering
  \cfig{14}{GRB110920A_ski_Y_10.pdf}{6}
  \caption{The time-resolved values of $\Gamma$, $r_0$, and $r_{\rm ph}$ for GRB 110920A assuming Y=10.}
  \label{fig:ski5}
\end{figure}

\newpage
\begin{figure}[tp]
  \centering
  \cfig{14}{GRB081224A_ski_Y_100.pdf}{6}
  \caption{The time-resolved values of $\Gamma$, $r_0$, and $r_{\rm ph}$ for GRB 081224A assuming Y=100.}
  \label{fig:ski1}
\end{figure}


\begin{figure}[tp]
  \centering
  \cfig{14}{GRB090719A_ski_Y_100.pdf}{6}
  \caption{The time-resolved values of $\Gamma$, $r_0$, and $r_{\rm ph}$ for GRB 090719A assuming Y=100.}
  \label{fig:ski2}
\end{figure}

\begin{figure}[tp]
  \centering
  \cfig{14}{GRB100707A_ski_Y_100.pdf}{6}
  \caption{The time-resolved values of $\Gamma$, $r_0$, and $r_{\rm ph}$ for GRB 100707A assuming Y=100.}
  \label{fig:ski3}
\end{figure}


\begin{figure}[tp]
  \centering
  \cfig{14}{GRB110721A_ski_Y_100.pdf}{6}
  \caption{The time-resolved values of $\Gamma$, $r_0$, and $r_{\rm ph}$ for GRB 110721A assuming Y=100.}
  \label{fig:ski4}
\end{figure}


\begin{figure}[tp]
  \centering
  \cfig{14}{GRB110920A_ski_Y_100.pdf}{6}
  \caption{The time-resolved values of $\Gamma$, $r_0$, and $r_{\rm ph}$ for GRB 110920A assuming Y=100.}
  \label{fig:ski5}
\end{figure}

\section{Non-thermal Radii of GRB Sample}

\begin{figure}[tp]
  \centering
  \cfig{14}{GRB081224A_RNT_Y_1.pdf}{6}
  \caption{The time-resolved values of the maximum non-thermal radii (shaded region) compared to $r_{\rm ph}$ (\emph{red}) and $r_{0}$ (\emph{green}) for GRB 081224A assuming Y=1.}
  \label{fig:RNT1}
\end{figure}


\begin{figure}[tp]
  \centering
  \cfig{14}{GRB090719A_RNT_Y_1.pdf}{6}
  \caption{The time-resolved values of the maximum non-thermal radii (shaded region) compared to $r_{\rm ph}$ (\emph{red}) and $r_{0}$ (\emph{green}) for GRB 090719A assuming Y=1.}
  \label{fig:RNT2}
\end{figure}

\begin{figure}[tp]
  \centering
  \cfig{14}{GRB100707A_RNT_Y_1.pdf}{6}
  \caption{The time-resolved values of the maximum non-thermal radii (shaded region) compared to $r_{\rm ph}$ (\emph{red}) and $r_{0}$ (\emph{green}) for GRB 100707A assuming Y=1.}
  \label{fig:RNT3}
\end{figure}


\begin{figure}[tp]
  \centering
  \cfig{14}{GRB110721A_RNT_Y_1.pdf}{6}
  \caption{The time-resolved values of the maximum non-thermal radii (shaded region) compared to $r_{\rm ph}$ (\emph{red}) and $r_{0}$ (\emph{green}) for GRB 110721A assuming Y=1.}
  \label{fig:RNT4}
\end{figure}


\begin{figure}[tp]
  \centering
  \cfig{14}{GRB110920A_RNT_Y_1.pdf}{6}
  \caption{The time-resolved values of the maximum non-thermal radii (shaded region) compared to $r_{\rm ph}$ (\emph{red}) and $r_{0}$ (\emph{green}) for GRB 110920A assuming Y=1.}
  \label{fig:RNT5}
\end{figure}

\newpage

\begin{figure}[tp]
  \centering
  \cfig{14}{GRB081224A_RNT_Y_10.pdf}{6}
  \caption{The time-resolved values of the maximum non-thermal radii (shaded region) compared to $r_{\rm ph}$ (\emph{red}) and $r_{0}$ (\emph{green}) for GRB }
  \label{fig:RNT1}
\end{figure}


\begin{figure}[tp]
  \centering
  \cfig{14}{GRB090719A_RNT_Y_10.pdf}{6}
  \caption{The time-resolved values of the maximum non-thermal radii (shaded region) compared to $r_{\rm ph}$ (\emph{red}) and $r_{0}$ (\emph{green}) for GRB 090719A assuming Y=10.}
  \label{fig:RNT2}
\end{figure}

\begin{figure}[tp]
  \centering
  \cfig{14}{GRB100707A_RNT_Y_10.pdf}{6}
  \caption{The time-resolved values of the maximum non-thermal radii (shaded region) compared to $r_{\rm ph}$ (\emph{red}) and $r_{0}$ (\emph{green}) for GRB 100707A assuming Y=10.}
  \label{fig:RNT3}
\end{figure}


\begin{figure}[tp]
  \centering
  \cfig{14}{GRB110721A_RNT_Y_10.pdf}{6}
  \caption{The time-resolved values of the maximum non-thermal radii (shaded region) compared to $r_{\rm ph}$ (\emph{red}) and $r_{0}$ (\emph{green}) for GRB 110721A assuming Y=10.}
  \label{fig:RNT4}
\end{figure}


\begin{figure}[tp]
  \centering
  \cfig{14}{GRB110920A_RNT_Y_10.pdf}{6}
  \caption{The time-resolved values of the maximum non-thermal radii (shaded region) compared to $r_{\rm ph}$ (\emph{red}) and $r_{0}$ (\emph{green}) for GRB 110920A assuming Y=10.}
  \label{fig:RNT5}
\end{figure}

\newpage
\begin{figure}[tp]
  \centering
  \cfig{14}{GRB081224A_RNT_Y_100.pdf}{6}
  \caption{The time-resolved values of the maximum non-thermal radii (shaded region) compared to $r_{\rm ph}$ (\emph{red}) and $r_{0}$ (\emph{green}) for GRB 081224A assuming Y=100.}
  \label{fig:RNT1}
\end{figure}


\begin{figure}[tp]
  \centering
  \cfig{14}{GRB090719A_RNT_Y_100.pdf}{6}
  \caption{The time-resolved values of the maximum non-thermal radii (shaded region) compared to $r_{\rm ph}$ (\emph{red}) and $r_{0}$ (\emph{green}) for GRB 090719A assuming Y=100.}
  \label{fig:RNT2}
\end{figure}

\begin{figure}[tp]
  \centering
  \cfig{14}{GRB100707A_RNT_Y_100.pdf}{6}
  \caption{The time-resolved values of the maximum non-thermal radii (shaded region) compared to $r_{\rm ph}$ (\emph{red}) and $r_{0}$ (\emph{green}) for GRB 100707A assuming Y=100.}
  \label{fig:RNT3}
\end{figure}


\begin{figure}[tp]
  \centering
  \cfig{14}{GRB110721A_RNT_Y_100.pdf}{6}
  \caption{The time-resolved values of the maximum non-thermal radii (shaded region) compared to $r_{\rm ph}$ (\emph{red}) and $r_{0}$ (\emph{green}) for GRB 110721A assuming Y=100.}
  \label{fig:RNT4}
\end{figure}


\begin{figure}[tp]
  \centering
  \cfig{14}{GRB110920A_RNT_Y_100.pdf}{6}
  \caption{The time-resolved values of the maximum non-thermal radii (shaded region) compared to $r_{\rm ph}$ (\emph{red}) and $r_{0}$ (\emph{green}) for GRB 110920A assuming Y=100.}
  \label{fig:RNT5}
\end{figure}





%%% Local Variables: 
%%% mode: latex
%%% TeX-master: "../thesis"
%%% End: 
