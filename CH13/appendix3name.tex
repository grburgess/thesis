\chapter{Justification of the Synchrotron Parameters}
\label{ch:pap2app}
\section{Synchrotron-Shell-Model Constraints}
Broad ranges of parameter values are possible in a GRB colliding shell
model. Here we justify the values used to fit the {\it Fermi} GBM and
LAT GRBs, assuming that the bright keV -- MeV emission of the GRBs in
our sample is primarily nonthermal synchrotron radiation emitted by
nonthermal electrons with an isotropic pitch-angle distribution that
radiate in a spherical shell expanding at relativistic speeds, within
which is entrained randomly directed magnetic field on coherence
length scales small in comparison with the shell volume. For
additional considerations about synchrotron models, see
\cite{2013ApJ...769...69B}.


The constraints that we consider are (1) particle and magnetic-field
energetics; (2) a negligible synchrotron self-Compton (SSC) component
so that we can neglect any high-energy $\gamma$-rays that could be
absorbed through $\gamma\gamma$ pair production and make additional
radiation at energies where the data are fit; (3) small synchrotron
self-absorption; and (4) minimum bulk Lorentz factor $\Gamma_{min}$ to
avoid strong $\gamma\gamma$ opacity. We also examine (5) the criterion
for being in the strong cooling regime.  To suppress SSC, we focus on
magnetically dominated models, which are also required in some
theories of GRBs to trigger magnetic reconnection events and produce
the prompt GRB emission through synchrotron emission
\cite{2009JPhCS.189a2018G,zhang:2011}. Magnetically
dominated GRB synchrotron models are also required for efficient
acceleration of ultra-high energy cosmic rays
\cite{2010OAJ.....3..150R}.


Our fiducial parameters are: characteristic electron Lorentz factor
$\gamma^\prime = 10^3 \gamma^\prime_3$; bulk Lorentz factor $\Gamma =
300\Gamma_{300}$, and fluid magnetic field $B^\prime = 10^5
B^\prime_5$ G. Radiation with characteristic $\nu F_\nu$ peak
frequency $\nu_{obs} = m_ec^2 \epsilon /h(1+z)$ is observed during the
prompt phase of the GRB. If non-thermal lepton synchrotron radiation,
then $\epsilon \cong 3\Gamma B^\prime \gamma^{\prime 2}/2B_{cr}$, and
$z$ is the source redshift, so $B_5^\prime \cong \epsilon/\Gamma_{300}
\gamma_3^{\prime 2}$.



\subsection{Energetics}
The electron energy content $\epsilon_e^{(\prime)}$ in the source
(comoving) frame is given by $\epsilon_e^\prime = \epsilon_e/\Gamma =
N_{e0}\gamma^\prime m_ec^2$, where $N_{e0}$ is the number of
electrons, so that
\begin{equation}
\epsilon_e = {\epsilon_{par}\over 1+\zeta}
 = 
{6\pi m_ec L_{syn}\over \sigma_{\rm T} B^{\prime 2} \gamma^\prime \Gamma^2}
=
\,{27 \pi m_ec L_{syn}\over 2  \sigma_{\rm T} B_{cr}^2 \epsilon^2}
\,\Gamma\gamma^{\prime 3}
 \cong 
10^{45} \, {L_{51} \Gamma_3 \gamma_3^{\prime 3}\over \epsilon^{2}}\;\;{\rm erg}\;, 
\label{Epar}
\end{equation}
where the total particle energy is denoted $\epsilon_{par} $, and
$\zeta$ represents the additional energy in hadrons. Here the
synchrotron luminosity $L_{syn}=10^{51}L_{51}$ erg s$^{-1}$ is derived
from the synchrotron electron energy-loss rate formula, using
$L^\prime_{syn} = c \sigma_{\rm T} B^{\prime 2}\gamma^{\prime
  2}N_{e0}/6\pi$.

The magnetic-field energy density $\epsilon_B = \Gamma {\cal
  E}_B^\prime = \Gamma 4\pi r^2 \Delta r^\prime (B^{\prime 2}/8\pi)$.
The shell width $\Delta r^\prime = k r/\Gamma$, with $k$ a factor of
order unity (for details see \cite{2013ApJ...769...69B}), using the
relations $\Delta r^\prime \cong \Gamma c t_{var}$ and $r \cong
\Gamma^2 c t_{var}$, where $t_{var}$ is the measured variability time
scale in the source frame. Thus the isotropic magnetic-field energy
\begin{equation}
\epsilon_{B} = {2k \Gamma^4\over 9}\, {c^3 t_{var}^3 B_{cr}^2 \epsilon^2\over \gamma^{\prime 4}} \cong 
10^{56}  k({\Gamma_{300}\over \gamma_3^{\prime}})^4 t_{var}({\rm s})^3 \epsilon^{2}\;\;{\rm erg}.
\label{eq:EB}
\end{equation}
The absolute magnetic field energy $\epsilon_{B,abs} \cong
(\theta_j^2/2)\epsilon_{B}$ for this system greatly out of
equipartition can be reduced to acceptable values (i.e., ${\cal
  E}_{abs}\ll 10^{54}$ erg) with a sufficiently small jet opening
angle $\theta_j$ between $\approx 0.01$ and $0.1$.


\subsection{SSC Component}
The ratio of the SSC and synchrotron luminosities is related to the
ratio of the synchrotron and magnetic field energy densities through
the relation $L_{SSC}/L_{syn} \lesssim
u^\prime_{syn}/u^\prime_{B^\prime}$, with the inequality arising from
the neglect of Klein-Nishina effects on the SSC emission.  Because
$u^\prime_{syn} \cong L^\prime_{syn}/4\pi r^2 c$, we have
\begin{equation}
{L_{SSC}\over L_{syn}} \approx {2 L_{syn}\over c^3 \Gamma^6 t_{var}^2 B^{\prime 2}} \cong {10^{-5}  L_{52}\over \Gamma_{300}^6 t_{var}^2({\rm s}) B^{\prime 2}_5 } \;,
\label{SSCsynratio}
\end{equation}
and so can be safely neglected here.



\subsection{Synchrotron Self-Absorption}
For a log-parabolic description of the $\gamma^{\prime 2} N^\prime
(\gamma_p)$ electron distribution, the SSA opacity in the
$\delta$-function approximation is given by
\begin{equation}
  \tau_{\epsilon^\prime} = 2\kappa_{\epsilon^\prime} \Delta r^\prime \cong {\pi \over 9}\,{\epsilon_e^\prime \Delta r^\prime\over m_ec^2 I(b) V_b^\prime \gamma_p^{\prime 4}}\, {\lambda_{\rm C} r_e\over \epsilon^\prime} (2+ b\log x)\,x^{-(4+b\log x)}
  \equiv \tau_0 (2+ b\log x)\,x^{-(4+b\log x)}\;
\label{tauep}
\end{equation}
\cite{2009herb.book.....D,2013arXiv1304.6680D}, where
$\kappa_{\epsilon^\prime}$ is the SSA absorption coefficient (units of
inverse length), $x \equiv
\sqrt{\ep/2\varepsilon_B^\prime}/\g_p^\prime$, $\e \cong \Gamma\ep$,
shell volume $V_b^\prime = 4\pi r^2 \Delta r^\prime$, and $I(b) =
\sqrt{\pi \ln 10/b}$ normalizes the electron spectrum depending on the
value of the log-parabola width parameter $b$.  Using \equationref{eq:EB}, we
obtain
\begin{equation}
\tau_{0} \cong {\pi \over 6\e }\,{ {\lambda_{\rm C} r_e L_{syn} \over c^3 \sigma_{\rm T} B^{\prime 2} t_{var}^2 \Gamma^5 \gamma^{\prime 5} I(b)}}
\approx {10^{-16} \over \e^3 }\,{ { L_{51} \over  t_{var}^2({\rm s}) \Gamma_{300}^3 \gamma_3^\prime I(b)}}\;, 
\label{eq:tau0}
\end{equation}
using the relation $\epsilon \cong \Gamma_{300} B^\prime_5
\gamma^{\prime 2}_3$ characterizing the condition that $x \approx 1$.
Thus SSA is utterly negligible at $x \gtrsim 0.1$, where the question
of SSA opacity is most important, noting from \equationref{eq:tau0} that the
opacity can grow as fast as $x^{-4}$ at $x\approx 0.1$, when
$b\lesssim 1$.

\subsection{$\gamma$-$\gamma$ Opacity}
The minimum bulk Lorentz factor giving a $\gamma$-ray with energy
$\epsilon_\gamma = 1.96\times 10^{5} E_\gamma$(GeV) unit optical depth
for $\gamma\gamma$ absorption by the target synchrotron photons is
estimated fairly accurately by the expression
\begin{equation}
\Gamma \geq \Gamma_{min} = \left[ { \sigma_{\rm T} \hat\epsilon L(\hat \epsilon ) \epsilon_{\gamma}
\over 16 \pi m_ec^4 t_{var} }\right ]^{1/6}\;,\; \hat \epsilon \cong 2\Gamma^2/\epsilon_\gamma .
\label{tauep}
\end{equation}
Taking $\e L(\e) \cong 10^{51}L_{51}/\ln (100)$ erg s$^{-1}$, i.e., a
flat $\nu F_\nu$ spectrum over 2 decades in frequency, then the
minimum bulk Lorentz factor $\Gamma_{min} \approx 300 \,[L_{51}
E_\gamma$(100~GeV)$/t_{var}({\rm s})]^{1/6}$.  For GRB synchrotron
radiation emitted in the $0.1 \lesssim \epsilon \lesssim 10$ range,
$\gamma$-rays with energies between $\approx (0.01$ --
1)$\Gamma_{300}^2$ TeV are subject to $\gamma\gamma$ opacity.
Provided that the energy radiated at 100 GeV and TeV energies is much
smaller that the total GRB photon energy, opacity effects and
cascading can be neglected.





\subsection{Cooling Regime}

The minimum and cooling frequencies in a colliding shell are derived
in the same way as the case of a blast wave decelerating by sweeping
up external medium material at a shock \cite{sari:1998}, recognizing
that the relative Lorentz factor between two shells is more likely to
be $\Gamma_{rel}\sim 10$, compared to the external shock Lorentz
factor $\Gamma \sim 300$. The system is in the slow cooling regime
when the cooling Lorentz factor
\begin{equation}
\gamma_c^\prime \cong {6\pi m_e c\over \sigma_{\rm T} B^{\prime 2}\Gamma t_{var} }
\gtrsim
\gamma^\prime_{min} \cong \epsilon_e {m_p\over m_e} f(p) \Gamma_{rel}\;,
\label{gammaprimemin}
\end{equation}
where $\gamma^\prime_{min}$ is the minimum electron Lorentz factor,
$p$ is the injection number index of relativistic electrons,
$\epsilon_e$ is the fraction of energy dissipated at the shock that
goes into nonthermal electrons, and the factor $f(p) = (p-2)/(p-1)$
normalizes the number and energy of the energized electrons.  Solving
gives
\begin{equation}
B^{\prime}
\lesssim 
\sqrt{ 6\pi m_e c (m_e/m_p)\over \sigma_{\rm T} \Gamma t_{var}\epsilon_e f(p)\Gamma_{rel} }
\approx {120 {\rm ~G}\over \sqrt{\Gamma_{300} (\epsilon_e/0.1) t_{var}({\rm s}) f(p)\Gamma_{rel}}}\;.
\label{gammaprimemin}
\end{equation}
A system with $\sim 100$ kG fields is always in the fast cooling
regime according to this criterion.


%%% Local Variables: 
%%% mode: latex
%%% TeX-master: "../thesis"
%%% End: 
