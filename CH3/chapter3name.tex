\chapter{Gamma-Ray Burst Observations and Modeling}

\label{ch:obsGRB}
\begin{chapterquote}{Edwin Hubble}
  Equipped with his five senses,\\ man explores the universe around him\\
  and calls the adventure Science.
\end{chapterquote}

\section{The {\it Fermi} Space Telescope}
In this work, GRB data from the {\it Fermi} Space Telescope were
analyzed both temporally and spectrally. {\it Fermi} consists of two
instruments, the Gamma-Ray Burst Monitor (GBM) and the Large Area
Telescope (LAT). Together, they cover an energy range from 10 keV to
300 GeV. This extensive bandpass allows for observing the entire
$\gamma$-ray spectrum both above and below the $\vFv$ peak. The data
are readily available via the Fermi Science Support Center (FSSC)
\cite{fssc}. A description of both instruments and their utilized data
types follows.
\subsection{GBM}
The GBM consists of twelve Sodium Iodide (NaI) and two Bismuth
Germinate (BGO) detectors that cover an effective energy range of 10
keV to 40 MeV. It continuously observes the non-Earth occulted sky
except when entering the region of high charged particle activity know
as the South Atlantic Anomaly (SAA). The instrument's main goal is to
trigger when an increase in $\gamma$-ray counts is detected signaling
a GRB event. The design and placement of the detectors around the
spacecraft enables a coarse localization ($> 1^{\degree}$) of GRBs. In
the event of an extremely bright detection, GBM can trigger an
Automated Repoint Request (ARR) that serves to point the LAT into an
orientation that allows for pointed, spatially-resolved, high-energy
emission observations.

Detection of {\gray}s is possible via their conversion of into optical
wavelengths via scintillation \cite{knoll}. The {\gray} interacts with
a scintillation crystal atom primarily by the photoelectric
effect, freeing an electron. The electron moves through the crystal
and loses energy as it excites the surrounding ions that dexcite by
emitting optical photons. These optical photons are measured by a
photomultiplier tube at the base of the crystal that converts their
signal into an electrical pulse proportional to the incident \gray's
energy.

The GBM is primarily a time-domain spectrometer. Three publicly
available data types from the GBM exist. These are CTIME, CSPEC, and
TTE. CTIME is a high time-resolution (0.256 s) binned data with low
energy resolution (8 channels). CSPEC is temporally-binned high energy
resolution (128 channels) with a lower time resolution (4.096 s) than
CTIME. TTE data is event data with time-tags and a channel energy for
each individual detected count. CSPEC and CTIME are both continuously
made during the spacecraft's orbit. Until recently, TTE was only
available during GRB triggers; however, as of 2013 a flight software
upgrade allows for the continuous generation of TTE data.




\subsection{LAT}
The LAT is a pair-conversion tracker that is primarily designed to
image $\gamma$-rays with energies $>10$ MeV. A pair-conversion tracker
operates by converting an incident $\gamma$-ray into an
electron-position pair ($e^{\pm}$) via a collision with a high-Z
material. The $e^{\pm}$ then moves in the same direction through the
tracker where its trajectory is tracked with Silicon strips until it
is finally absorbed in a calorimeter at the base. By reconstructing
the path of the $e^{\pm}$ and measuring the energy deposited in the
calorimeter, the direction and energy of the incident $\gamma$-ray can
be determined.

The effective energy range of the LAT is 100 MeV - 300 GeV using the
standard event type data. However, a new data type named the LAT Low
Energy (LLE) data extends the energy range down to $\sim$30 MeV
\cite{Vero:2010}. This data is made possible by collecting all counts in the
detector that pass basic cuts instead of rejecting counts that have
poor spatial and energy information. These poorly-measured events are
typically of lower energy. The side effect of accepting these events
is a high background; therefore, the LLE data is only viable for
analyzing temporally transient events such as GRBs or pulsars.



\section{Spectrum}
\label{sec:spectrum}
This focus of this work is on the prompt emission of GRBs. It is
necessary to draw a distinction between the prompt emission and the
so-called afterglow which occurs after the primary high-energy
emission. Discussion of the afterglow is beyond the scope of this
project. To date, over 3000 GRBs have been observed by both BATSE and {\it
  Fermi}. Several catalogs detail their spectral, temporal, and
intensity properties
\cite{Goldstein:2012,Kaneko:2006,Nava:2011}. These catalogs identify
several key features of the GRB population including their subdivision
into two classes based on duration \cite{ck:1993}, the grouping of
their respective $\vFv$ peak energies \cite{Schaefer:2003}, and their
distribution of spectral indices. Key to all these findings are the
 parameters found via spectral analysis. The distribution of
photons that make up the observed spectra of GRBs is one of the main
clues to understanding the physical mechanisms behind GRB emission.
\subsection{The Band Function}
\label{sec:bandfunc}
Because of the inversion problem (see \sectionref{sec:ff}), it is impossible to directly assess
the shape of the photon spectrum. The shape of the typical
GRB spectrum is curved and asymptotically approaches a power-law at
high and low energies. A comprehensive study of empirical photon
models fit to GRBs in the BATSE data found that a specialized function
was able to fit the majority of time-integrated and time-resolved
spectra \cite{band:1993}. The Band function is a smoothly broken
power-law connected exponentially at the $\vFv$ peak.
\begin{equation}
  \label{eq:band}
   \Fv(\mathcal{E})\;=\;F_0
  \begin{cases}
    {\left(\frac{\mathcal{E}}{\rm 100\;keV}\right)^{\alpha} {\rm exp}\left(-\frac{(2+\alpha)\mathcal{E}}{E_{\rm p}}\right)} & {\mathcal{E}\le (\alpha-\beta)\frac{E_{\rm p}}{ (\alpha+2)} }\\
    {\left(\frac{\mathcal{E}}{\rm 100\;keV}\right)^{\beta}{\rm
        exp}{\left(\beta-\alpha\right)}\left[\frac{(\alpha-\beta)E_{\rm
            p}}{{\rm 100\;keV} (2+\alpha)}\right]^{\alpha-\beta}} &
    {\mathcal{E}>(\alpha-\beta)\frac{E_{\rm p}}{ (\alpha+2)} }
  \end{cases}
\end{equation}
\begin{figure}[t]
  \centering
  \cfig{3}{bandExp}{5}
  \caption{The $\vFv$ spectrum of the Band function illustrating its
    three important shape parameters: $\alpha$, $\beta$, and $\Ep$
    (see \equationref{eq:band}).}
  \label{fig:bandSpecEx}
\end{figure}




\begin{figure}[t]
  \centering
  \cfig{3}{band}{5}
  \caption{Demonstrating the variety of Band function shapes
    corresponding to differing values of $F_0$, $\alpha$, $\beta$, and
    $\Ep$ (see \equationref{eq:band}).}
  \label{fig:bandSpec}
\end{figure}
Nearly all GRBs detected by {\it Fermi} and BATSE can be fit with the
Band function. Though the function is empirical, attempts have been
made to relate its fit parameters to physical quantities. The most
important association was made by \cite{preece:1998} with the
so-called synchrotron 'line-of-death'. As noted
in \sectionref{sec:synctheory}, the low-energy slope (Band's $\alpha$
index) of a synchrotron photon spectrum is -2/3. It found that nearly
1/3 of GRB spectra had $\alpha$'s greater than -2/3 indicating that
they were inconsistent with synchrotron emission
(see \sectionref{sec:spec:lod} for a detailed discussion).

\begin{figure}[h]
  \centering 
\subfigure{\cfig{3}{epDist.pdf}{2.}}\subfigure{\cfig{3}{alpDist.pdf}{2.}}\subfigure{\cfig{3}{betaDist.pdf}{2.}}
\caption{Band function fit parameters from the first two
  years of GBM data.}
  \label{fig:catparam}
\end{figure}
Similar associations with other emission mechanisms have been
made. Much research has focused on finding an emission mechanism that
can correctly account for the Band $\alpha$ distribution
\cite{preece:1998,Beloborodov:2010,Daigne:2011,piran:2013}. To date,
no one model can account for the entire parameter space. Similarly,
E$_{\rm p}$ can be associated with radiative mechanisms through their
respective physical $\vFv$ peaks e.g. see
\equationref{eq:synchPeak}. The physical relation to $\beta$ is
typically made to the power-law index of the emitting electron
distribution.
\subsection{The Blackbody Component}
With few exceptions \cite{gonzalez:2003,abdo+GRB090902B}, the spectra
of GRBs were found to consist of only one broadband component in the
BATSE era and the early {\it Fermi} era. The overall non-thermal shape
of GRB spectra fit with the Band function left little hope for finding
the thermal signature of a blackbody predicted by the basic fireball
model. However, after the launch of {\it Fermi}, several GRBs were
satisfactorily fit with a two component model consisting of the Band
function and a blackbody
\cite{guireic:2010,Axelsson:2012,guiriec:2013}. The component was
shown to be statistically significant. When fit along with a
blackbody, the Band component of the spectra appeared to be more
consistent with synchrotron than before with Band's $\alpha$ moving
closer to the expected value of the low-energy index.

\begin{figure}[h]
  \centering
  \cfig{3}{vFv.pdf}{4}
  \caption{The BATSE
    era blackbody, which was fit to the entire spectrum is the same as
    the GBM era blackbody due to the difference in bandpass of the
    instruments.}
  \label{fig:batseBB}
\end{figure}


Prior to the discovery of the blackbody component existing in
combination with the Band function, several works studied the
existence of blackbodies in BATSE GRB spectra
\cite{Ryde:2009,Ryde:2006,Ryde:2005,Ryde:2004}. These studies found
that the entire GRB spectrum consisted of a blackbody or a blackbody
and a power-law extending into high energies. The evolution of this
blackbody component was studied extensively. It was found that the
temperature of the blackbody decayed as a broken power-law in time
with the index after the power-law break was typically $\sim-2/3$. A
post analysis of this component reveals that it is the same blackbody
found in {\it Fermi} GRBs. This is because BATSE had a limited
high-energy response above 2 MeV (see
\figureref{fig:batseBB}). Therefore, BATSE could only detect the
low-energy blackbody plus the low-energy power-law of the Band
function.

\subsection{The Synchrotron Line-of-Death and the Fast Cooling Problem}
\label{sec:spec:lod}
The distribution of Band $\alpha$ has been studied extensively in an
attempt to understand the underlying emission mechanisms. As shown
in \sectionref{sec:emissionMech}, the low-energy index of a
$\gamma$-ray spectrum is unique for many models. Synchrotron emission
from internal shocks is the simplest and most widely invoked model for
explaining GRB observations, however, the 'line-of-death' problem
presents a challenge to the theory.

\begin{figure}[h]
  \centering
  \cfig{3}{lod.pdf}{4}
  \caption{The $\alpha$ distribution
    of GBM detected GRBs illustrating the 'line-of-death' problem. The
    fast-cooling line (\emph{blue}) and slow-cooling line (\emph{red})
    are superimposed on the distribution.}
  \label{fig:lod}
\end{figure}
GRBs fit with the Band function possessing $\alpha$'s greater than
-2/3 are presumably inconsistent with synchrotron emission. Even more
problematic is that nearly \emph{all} GRBs have $\alpha$'s greater
than -3/2, the fast-cooling synchrotron index. Due to the time scales
involved, the electrons in GRBs must be in the fast-cooling regime. If
the cooling timescale argument is ignored then the problem of
efficiency forces the requirement of fast cooling for the internal
shock model. The low dissipation efficiency of internal shocks
(5-20\%) is the upper limit of the radiative efficiency, i.e., if the
radiative efficiency is 100\%, then only 5-20\% of bulk kinetic energy
of the jet can be radiated in the form of {\gray}s. Slow-cooling
electrons are much less efficient than those that are being quickly
cooled by synchrotron; therefore, the internal shock model requires
fast-cooling to be radiatively efficient. This is the so-called
fast-cooling problem.



\subsection{Spectral Evolution}
\label{sec:spec:evo}
The evolution of spectral parameters in GRBs is useful in identifying
the evolution of the jet structure and/or the evolution of the
magnetic field during the outflow. The most well studied parameter
evolution is the that of $\Ep$
\cite{Medvedev:2006,Liang:1997,Liang:1996,golenetskii:1983,Ryde:2001}. While
there is no universal evolution of $\Ep$, many GRBs exhibit the
so-called hard-to-soft evolution, i.e., the monotonic evolution of
$\Ep$ from an initially high value to a lower value. The physical
explanation of this evolution is not fully understood and will be
addressed in \sectionref{sec:results:hec,sec:fluxfrx}. Accompanying hard-to-soft evolution is the
correlation of $\Ep$ with $\Fv$ in many bursts. This correlation is
most prominent in single-pulsed GRBs. This hardness-intensity
correlation (HIC) is also not fully understood but should be related
to the radial change in the GRB jet parameters \cite{preece:2013}.

\section{Lightcurves}

\begin{figure}[h]
  \centering
  \cfig{3}{lc.pdf}{6}
  \caption{A sample of GBM lightcurves
    demonstrating the diversity of pulse shape and complexity
    \cite{valerieSite}. For this study, the concentration will be on
    single pulsed GRBs such as GRB 110605183 (above).}
  \label{fig:gbmlc}
\end{figure}

The varied nature of GRB lightcurves is impossible to describe
categorically (see \figureref{fig:gbmlc}). There are; however, a
subclass of lightcurves that have been studied intensely
\cite{Ruiz:2000,Fenimore:1996,Dermer:2004,Norris:1999}. These are
those with a fast rise and exponential decay (FREDs) in time. GRBs with well
separated FRED pulses have very similar properties in their associated
spectral evolution. The FRED shape has not been fully explained by
theory. The simplest explanation comes from special relativity. If a
spherical source emitting photons isotropically expands
relativistically, then it can be shown that the observer from pulse
resembles that of a FRED \cite{Rees:1966}. Studies of FRED pulses have
shown that this simple model cannot account for all FRED shapes
\cite{Kocevski:2003}. The relation of the rise and decay times of
pulses should give an indication of the emission model of GRB pulses
\cite{Kocevski:2003}; however, no unique solution for the association
of pulse shape has been established.

For this work, the focus will be on GRBs with single, separable pulses
that have FRED-like shapes. These GRBs have been shown to have simple
and clean spectral evolution \cite{Ryde:2009} which is imperative for the study
of physical emission models.

\section{Simulations of GRB Emission}
\label{sec:sims}
In order to understand the observations and their relation to theory,
several theoretical models have been numerically simulated and
compared to observed spectra and lightcurves
\cite{Daigne:1998,Daigne:2009,Asano:2009,Pana:1998,Chiang:1999,Cannizzo:2004,Peer:2005,Kobayashi:1997}. The
simulations typically attempt to assess either the dynamic evolution
of the jet or the emission of the electrons that have been accelerated
and their evolution. Some simulations attempt to address both
properties. The internal shock model has received the most attention
via simulation due to its relative simplicity compared to other
models.

\subsection{Simulations of the Internal Shock Model}
To simulate the internal shock model, the dynamics derived
in \sectionref{sec:nter:is} must be numerically calculated for a large
number of emitted shells. The first attempts to do so were simplistic
and attempted to simulate the shape and variability of the typical GRB
lightcurve only \cite{Kobayashi:1997}. These studies showed that the
internal shock model could reproduce the observed lightcurve shapes
but gave no information about the spectra. The next set of simulations
enhanced these results by including synchrotron emission and then
later a full radiative code that included all radiative processes
relevant to GRB emission \cite{Daigne:1998,Daigne:2009}.

These simulations consist of a large number of shells of varying mass
and $\Gamma$. They are emitted one after another with faster shells
coming after the slower shells in time. This forces a collision where the
dissipated energy is calculated from \equationref{eq:intEne}. This
dissipated energy is then used to numerically calculate an accelerated
electron distribution that cools via a radiative code that
numerically calculates:
\begin{equation}
  \label{eq:coolingcode}
  \dover{\partial n_e}{\partial t^{\prime}}= - \dover{\partial}{\partial \gamma}\left[\dover{\mathrm{d}\gamma}{\mathrm{d} t^{\prime}}\bigg|_{{\rm sync} + {\rm ic}}n_e(\gamma,t^{\prime})\right]
\end{equation}
As the electrons cool, the radiation spectrum is calculated from the
different radiative emissivities. The radiative codes used in these
simulations are simple to increase the speed of computation due to
the large number of shell collisions involved. Once the radiation
spectrum is calculated in the GRB jet rest frame, the emission is
transformed into the observer frame. The total emission from the jet
is then summed together forming a lightcurve. These lightcurves and
their associated spectra have been shown to reproduce many of the
observed features of GRBs including hard-to-soft evolution, Band-like
shape, and lightcurve shape. This success has helped to answer many
questions about viable mechanisms for producing GRBs. However, these
simulations rely on many processes that are put in by hand and not
self-consistent. These include the particle acceleration and radiative
timescales. While these simplifications aid in decreasing CPU time,
they neglect key questions which must be self-consistently validated
in order to fully assess the viability of the internal shock model.

\subsection{Full Radiative Codes and Sub-Photospheric Dissipation}
A fully self-consistent radiative code can calculate the evolution of
accelerated electrons as they cool from the various emission
mechanisms relevant to GRBs. The most complete radiative code in the
literature is that of \cite{Peer:2005}. In these simulations of GRB
emission, an injected electron distribution is tracked as it cools in
a very detailed manner. Different radiative processes occur at
different timescales making the calculations numerically challenging
and CPU intensive. Therefore, this code neglects the GRB jet dynamics
implemented in \cite{Daigne:2009}. Still, these simulations show that
the evolution of the electron distribution is highly complex and the
simplifications made in \cite{Daigne:1998,Daigne:2009} neglect
important aspects of the radiative process. Ideally, a fully detailed
radiative code should be coupled with a full jet dynamics code to
fully understand the internal shock model. The simulation collects the
associated radiation from the electrons and computes the observed
spectra and its evolution. These spectra can then be compared to data.

One key benefit of this full radiative code is that it can simulate
the evolution of electrons when the particle densities imply a high optical
depth. This allows for testing of the sub-photospheric dissipation
model. It has been shown that the spectra produced by this model can
have Band-like shapes. However, due to the limitations discussed above
no lightcurve has been produced to show that the model is fully
consistent with observations.

\subsection{Simulations of PFD and Magnetic Reconnection in GRBs}
Due to the lack of theoretical understanding of magnetic reconnection
and PFD models, very little simulation work has been done in the field
of GRBs to test their viability as an emission mechanism. No lightcurves or spectra have been numerically simulated and
therefore, no evaluation of the validity of these models can be
made. The association of these models to data has been made purely on
considerations of timescales and energy requirements that fall into
the expected GRB regime but detailed simulations must be carried out
to validate these models fully.




% \section{Tables---Kinda like Figures in Reverse}

% In the UAH format, table require ``titles'' (where figures require
% ``captions'').  So, all that really means is that you put the
% \verb|\caption{Table Title}| \textsl{before} you actually begin the
% table (but, of course, you must be in the table environment).  For
% example, this \TeX:
% \begin{verbatim}
% \begin{table}
% \begin{center}
% \caption{This is a Table Made with the Booktabs package}
% \label{tbl:table}
% \begin{tabular}{@{}lccc@{}}
% \toprule \rule[-1pt]{0pt}{14pt}Title 1-Left&Centered&Centered Again&
% Centered\\
% \midrule \rule[-1pt]{0pt}{14pt}Item 1&These are all&separated
% by&Ampersands
% \verb|&|\\
% \rule[-1pt]{0pt}{14pt}Math Works too&$E=mc^2$&$F=ma$&see?\\
% \bottomrule
% \end{tabular}
% \end{center}
% \end{table}
% \end{verbatim}

% \noindent produced the following table, \tableref{tbl:table}.
% \begin{table}
% \begin{center}
% \caption{This is a Table Made with the Booktabs package}
% \label{tbl:table}
% \begin{tabular}{@{}lccc@{}}
% \toprule
% \rule[-1pt]{0pt}{14pt}Title 1-Left&Centered&Centered Again&
% Centered\\
% \midrule
% \rule[-1pt]{0pt}{14pt}Item 1&These are all&separated by&Ampersands
% \verb|&|\\
% \rule[-1pt]{0pt}{14pt}Math Works too&$E=mc^2$&$F=ma$&see?\\
% \bottomrule
% \end{tabular}
% \end{center}
% \end{table}

% These files are set up such that \LaTeX\ will put figures and tables
% (which are collectively called ``floats'') at the top of the first
% available page (as near to where you inserted the figure or table
% environment as possible). You can change that (\ie, force \LaTeX\ to
% put the floats in different locations).  But, I prefer it simple.

%%% Local Variables: 
%%% mode: latex
%%% TeX-master: "../thesis"
%%% End: 
